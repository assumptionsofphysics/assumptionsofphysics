\documentclass[10pt,twocolumn, nofootinbib]{revtex4-2}
%\documentclass[aps,pra,10pt,twocolumn,floatfix,nofootinbib]{revtex4-1}
%\documentclass[10pt,twocolumn,letterpaper]{article}

\usepackage{amsmath}
\usepackage{mathrsfs}
\usepackage{amsfonts}

\usepackage{graphicx}
\usepackage{hyperref}
\hypersetup{
	colorlinks=true,
	citecolor=blue,
	urlcolor=blue,
	linkcolor=blue
}
\urlstyle{same}
\frenchspacing


\begin{document}

\title{The Wrong Math}
\author{Gabriele Carcassi, Christine A. Aidala}
\affiliation{Physics Department, University of Michigan, Ann Arbor, MI 48109}

\date{\today}


\begin{abstract}
It is our observations that both physicists and mathematicians do not dwell too much on the basic definitions of the mathematical tools they use. Therefore we wonder: how do we know that the mathematical structures we routinely use (e.g. Hilbert spaces, differentiable manifolds, measure theory, ...) are the ``right'' one for physics? 
In fact, what is the main feature that a mathematical structure should have to be suitable for physics? The most straight-forward answer is that a particular mathematical structure is a good match for physics if the definition corresponds to a set of physically meaningful assumptions, which then define the realm of applicability of said structure. We will see that many commonly used mathematical structures fail this test, defining a mathematical problem that is mismatched to the physical one. Our conclusion is that no significant progress can be made in fundamental physics until this situation is rectified.
\end{abstract}

\maketitle

\section{Summary}

[The whole paper in a couple of pages.]

\section{Everything you always wanted to know about Hilbert spaces (but were afraid to ask)}

Dissect Hilbert spaces and argue they are too big.

\section{The measure of a mathematical structure}

Judge by result vs judge by the definitions. No definition is always ``correct'', but some are ``wrong''.

Mathematical definitions are part of the job of the theoretical physicists. Argue that this is not the case for other tools we borrow from other fields (i.e. building codes, network standards, natural language, ...)

Proper structure (assumptions are satisfied)
Improper structure (assumptions are not satisfied)
Wrong structure (assumptions are never satistied)

Usefulness for calculation is the consequence of proper structures.

Examples: topologies as logic structures, boolean algebras for system composition.

\section{Looking for physics in all the wrong math}

Possible mistakes.

Overcomplication. If we don't have a clear idea of what we need, we are going to be pushed to use ``fancy'' tools that mathematicians/computer scientists/... like/is in their interest for us to use. Fewer things, simpler tool is better!

You model things incorrectly. Example: model the wave-function as invariant. Lack of additivity for mass.

You miss what the mathematics is capturing. Example: coordinates and units. Symplectic form and change of coordinates. Forms. Exterior derivative. Physical significance of the action principle.

\section{Finding physics}

Proper mathematical structure for generalization and comparison. Example: comparing classical and quantum mechanics.

Find the physics. Role of entropy as the definer for measures and geometry.

\section{Conclusion}


\bibliography{bibliography}


\end{document}