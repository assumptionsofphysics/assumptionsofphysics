\documentclass[review]{elsarticle}

\usepackage{lineno,hyperref}

\usepackage{amsmath, amsthm, amsfonts}
\usepackage[only,llbracket, rrbracket,llparenthesis,rrparenthesis]{stmaryrd} 
\usepackage{MnSymbol,centernot}

% Symbol for "implies" and "not implies"
\def\imp{\Rightarrow}
\def\nimp{\nRightarrow}

% Symbol for "compatibility" and "incompatibility"
\def\comp{\doublefrown}
\def\ncomp{\ndoublefrown}

\newcommand{\R}{\mathbb{R}}
\newcommand{\Q}{\mathbb{Q}}
\newcommand{\Z}{\mathbb{Z}}


\theoremstyle{plain}% default 
\newtheorem{thm}{Theorem}[section] 
\newtheorem{lem}[thm]{Lemma} 
\newtheorem{prop}[thm]{Proposition} 
\newtheorem{cor}{Corollary}

 \theoremstyle{definition}
 \newtheorem{defn}{Definition}[section]
\newtheorem{exmp}{Example}[section]
\theoremstyle{remark}
\newtheorem*{rem}{Remark}
\modulolinenumbers[5]

\journal{Topology and its Applications}


\begin{document}

\begin{frontmatter}

\title{Topology and experimental distinguishability}

%% Group authors per affiliation:

\author[]{Christine A. Aidala\corref{cor1}}
\ead{caidala@umich.edu}

\author[]{Gabriele Carcassi}
\ead{carcassi@umich.edu}

\author[]{Mark J. Greenfield}
\ead{markjg@umich.edu}

\address{University of Michigan, Ann Arbor, MI 48109, USA}

\cortext[cor1]{Corresponding author}


\begin{abstract}
We formalize the relationship between topological spaces and the ability to distinguish objects experimentally, providing understanding and justification as to why topological spaces and continuous functions are pervasive tools in science. We first define an experimental observation as a statement that can be verified using an experimental procedure and show that observations are closed under finite conjunction and countable disjunction. We then consider observations that identify elements in a set and show how they induce a Hausdorff and second-countable topology on that set, thus identifying an open set as one that can be associated with an experimental observation. We then show that continuous functions preserve experimental distinguishability and define a Hausdorff and second-countable topology for this collection. 
\end{abstract}

\begin{keyword}
topological spaces \sep scientific foundations \sep function space
\MSC[2010] 54H99 \sep 00A79
\end{keyword}

\end{frontmatter}

\linenumbers

\section{Introduction}

The successful use of mathematical ideas in experimental sciences is long established and celebrated \cite{wigner}. In many ways, topology is perhaps the most widespread, as it provides the foundation for other mathematical tools such as differential geometry and analysis. Topology is foundational in physics \cite{nakahara}, and has been the backbone of advances in many areas such as biology \cite{rashevsky} and social choice theory \cite{chichilnisky}. This leads one to ask, why is it so successful? What property is captured by topological spaces that is so fundamental for scientific investigation? 

We attempt to answer these questions in a precise manner by defining the appropriate connections. Topological spaces generalize metric spaces; they provide a notion of ``nearness" of points without necessarily quantifying it. We extend this interpretation in a precise way to cover spaces representing objects (points in space, objects needing classification, etc.) which are distinguishable by scientific experimentation. Because experimental data is universally limited by precision, it is neighborhoods of points that are associated with experimental outcomes, as opposed to the points themselves. Along the same line, experimental relationships need to preserve what can be experimentally distinguished, and they are therefore should be realized as continuous functions on the appropriate topological spaces.

In this sense, experimental distinguishability is intrinsically topological in nature. It is fitting, then, that the use of topology is so widespread, as defining what can be experimentally identified is a fundamental aspect of science. The aim of this paper is to lay down a framework that formalizes this insight. 

The work is organized as follows. We first define experimental observations as statements paired with an experimental test that is able to verify them. These will be the fundamental objects which enable us to define our universe of discourse. We study their properties under logical operations and show that they are only closed under finite conjunction and countable disjunction.  We then show that any set of objects that are experimentally distinguishable is a Hausdorff, second-countable topological space. The topology of the space is formally defined in terms of the observations themselves.  Finally, we study relationships between experimentally distinguishable objects. These are represented by continuous functions, and they can be shown to be themselves physically distinguishable.


\section{Experimental observations}

In science, a statement is considered valid for inquiry only if one can provide a way to independently test the assertion. We introduce the following definitions to capture this notion. 


\begin{defn}
	A \textbf{statement} $\mathsf{s}$ is a declarative sentence that is either true or false, as in classical logic. 
\end{defn}

Statements in this work shall have a truth value as well as \textit{logical content}, and so one can compare the meaning of statements as well as their truth values. For example, a statement can be a \textbf{contradiction}, which is stronger than being false. For example, the statement ``$\{1,2,3\}$ has four elements, and $\{1,2,3\}$ has three elements" is a contraction, which is a distinct claim from noticing it is a false statement. 

Given a statement, we wish to consider when it makes scentific sense to test. The following construction will define the next building block of this theory. 

\begin{defn}
	An \textbf{experimental test} $\mathsf{e}$ is a repeatable procedure (i.e. it can be restarted and stopped an arbitrary number of times) which may be successful, in which case it terminates in finite time, or may be unsuccessful, in which case it may or may not terminate.
\end{defn}

\begin{defn}
	An \textbf{experimental observation} is a tuple $\llparenthesis \mathsf{s}, \mathsf{e} \rrparenthesis$, normally denoted by $\mathsf{o}$, consisting of a statement $\mathsf{s}$ and an experimental test $\mathsf{e}$ such that the statement is true if and only if the  experimental test is successful. The experimental observation is \textbf{verified} if the statement is true.
\end{defn}


Experimental observations are the core object of the scientific discourse. This construction is designed to encode the forming of a claim together with a procedure to test the claim. In this work, we do not presuppose knowledge of the truth values of the statements - only by ``running" the tests can we ascertain these values. 




\section{Algebra of experimental observations}

We now want to understand how experimental observations behave under logical operations. While utilizing similar techniques, the resulting algebra will have some key differences from the standard Boolean algebra.





We define the following logical operations on experimental observations. Taken as whole, they define the \emph{algebra of experimental observations}. With each definition, we will prove that the resulting object is an experimental observation. 




\begin{defn}
	The \textbf{conjunction} or \textbf{logical AND} of a finite collection  of experimental observations $\{\mathsf{o}_i\}_{i=1}^{n}=\{\llparenthesis \mathsf{s}_i, \mathsf{e}_i\rrparenthesis\}_{i=1}^{n}$ is the experimental observation $\bigwedge\limits_{i=1}^{n} \mathsf{o}_i = \llparenthesis \mathsf{s}, \mathsf{e}\rrparenthesis$ where $\mathsf{s} = \bigwedge\limits_{i=1}^{n} \mathsf{s}_i$ is the conjunction of the respective statements and $\mathsf{e} = \mathsf{e}_\wedge(\{\mathsf{e}_i\}_{i=1}^{n})$ is the experimental test that successfully terminates if and only if all $\{\mathsf{e}_i\}_{i=1}^{n}$ successfully terminate.
\end{defn}

\begin{proof}
	We show that the conjunction exists and is well defined. Let $\mathsf{e}_\wedge=\mathsf{e}_\wedge(\{\mathsf{e}_i\}_{i=1}^{n})$ be the experimental procedure defined as follows:
	\begin{enumerate}
	\item for each $i=1,\ldots,n$ run the test $\mathsf{e}_i$
	\item if all tests $\mathsf{e}_i$ terminate successfully, terminate the test $\mathsf{e}_\wedge$ successfully.
	\end{enumerate}
	The experimental procedure so defined terminates successfully if and only if all $\mathsf{e}_i$ terminate successfully. It will do so in finite time as each of the finitely many $\mathsf{e}_i$ succeeds in finite time. Therefore $\mathsf{e}_\wedge(\{\mathsf{e}_i\}_{i=1}^{n})$ is an experimental test that is successful if and only if all statements $\{\mathsf{s}_i\}_{i=1}^{n}$ are true. So, $\bigwedge\limits_{i=1}^{n} \mathsf{o}_i = \llparenthesis\bigwedge\limits_{i=1}^{n} \mathsf{s}_i, \mathsf{e}_{\wedge}(\mathsf{e}_i)\rrparenthesis$ is an experimental observation.
\end{proof}

\begin{defn}
	The \textbf{disjunction} or \textbf{logical OR} of a countable (finite or infinite) collection of experimental observations $\{\mathsf{o}_i\}_{i=1}^{\infty}=\{\llparenthesis \mathsf{s}_i, \mathsf{e}_i\rrparenthesis\}_{i=1}^{\infty}$ is the experimental observation $\bigvee\limits_{i=1}^{\infty} \mathsf{o}_i = \llparenthesis \mathsf{s}, \mathsf{e}\rrparenthesis$ where $\mathsf{s} = \bigvee\limits_{i=1}^{\infty} \mathsf{s}_i$ is the disjunction of the respective statements and $\mathsf{e} = \mathsf{e}_\vee(\{\mathsf{e}_i\}_{i=1}^{\infty})$ is the experimental test that successfully terminates if and only if at least one experimental test in $\{\mathsf{e}_i\}_{i=1}^{\infty}$ successfully terminates.
\end{defn}

\begin{proof}
	We show that the disjunction exists and is well defined. Let $\mathsf{e}_\vee=\mathsf{e}_\vee(\{\mathsf{e}_i\}_{i=1}^{\infty})$ be the experimental procedure defined as follows:
	\begin{enumerate}
	\item initialize $n$ to 1
	\item for each $i=1,\ldots,n$:
	\begin{enumerate}
		\item run the test $\mathsf{e}_i$ for $n$ seconds
		\item if $\mathsf{e}_i$ terminated successfully, terminate $\mathsf{e}_\vee$ successfully
	\end{enumerate}
	\item increment $n$ and go to step 2
	\end{enumerate}
	Suppose there exists an $i \in \mathbb{Z}^+$ such that $\mathsf{e}_i$ will terminate successfully. Then the above procedure will eventually run that test for time sufficient for it to terminate successfully. It will do so in finite time as it will have run finitely many tests finitely many times each for a finite amount of time. Therefore $\mathsf{e}_\vee(\{\mathsf{e}_i\}_{i=1}^{\infty})$ is an experimental test that is successful if and only if at least one statement in $\{\mathsf{s}_i\}_{i=1}^{n}$ is successful. So $\bigvee\limits_{i=1}^{\infty} \mathsf{o}_i =\llparenthesis\bigvee\limits_{i=1}^{\infty} \mathsf{s}_i, \mathsf{e}_{\wedge}(\{\mathsf{e}_i\}_{i=1}^{\infty})\rrparenthesis$ is an experimental observation.
\end{proof}

Now, we define a special type of observation.

\begin{defn}
A \textbf{null observation}, denoted $\mathsf{o}_N$, is any experimental observation whose statement is a contradiction.
\end{defn}

We will need to compare certain observations with each other, and an important relationship is the following:
\begin{defn}
Two experimental observations $\mathsf{o}_1$ and $\mathsf{o}_2$ are said to be \textbf{incompatible} or \textbf{contradictory} if the conjunction $\mathsf{o}_1\wedge\mathsf{o}_2$ is a null observation.
\end{defn}


\section{Experimental domain}

Given a collection of observations $\mathcal{X}$, we wish to understand both the largest collection $\mathcal{Y}\supset\mathcal{X}$ of observations which can be constructed from $\mathcal{X}$ by our algebra of observations, and similarly, we wish to find smaller collections $\mathcal{Z}\subset\mathcal{X}$ from which we can construct all of $\mathcal{X}$. With this in mind, we introduce the following definitions.


\begin{defn}
	An \textbf{experimental domain} is a set of observations closed under finite conjunction and countable disjunction, such that all observations can be tested in infinite time. 
\end{defn}


\begin{defn}
	A \textbf{sub-basis} of an experimental domain is any subset that can generate all others via finite conjunction and countable disjunction. A \textbf{basis} of an experimental domain is any subset that can generate all others by countable disjunctions. 
\end{defn}

Clearly, given a sub-basis one can generate a basis by taking all finite conjunctions of the observations. Note that any infinite sub-basis will generate a basis of the same cardinality. We claim that it is only possible to verify all members of an experimental domain with a (sub-)basis of at most countably infinite cardinality, even given infinite time. 

\begin{prop}
Let $\mathcal{D}$ be an experimental domain. Then there exists a countable basis (equivalently, sub-basis) $\mathcal{B}$ of $\mathcal{S}$.
\end{prop}

\begin{proof}
If there exists a countable basis $\mathcal{B}$, then given infinite time one can test all observations in $\mathcal{B}$. Given which observations of the basis are verified, one can deduce which other observations in $\mathcal{D}$ are verified (again using infinite time) by computing the appropriate disjunctions. 

If there does not exist a countable basis, then by definition there does not exist a sequence of experimental observations in $\mathcal{D}$ from which one can deduce all other observations in $\mathcal{D}$. Hence it is impossible to test all members of $\mathcal{S}$.
\end{proof}

\begin{rem}
If the (countable) disjunction of an entire basis produces an experimental observation which is always verified, we call this the \textbf{trivial observation}, denoted $\mathsf{o}_T$. This may not always be the case: consider, for example, the rather uninteresting case of an experimental domain consisting only of the null observation. 
\end{rem}




\section{Experimental distinguishability}

Now that we have a framework for testing a system, we will define a class of systems we can understand with these tests. The framework will be general enough that one can readily apply it to many real-world experimental scenarios. In this construction, we will begin with a space we wish to understand - in this case, finding the appropriate element - and then define the appropriate collection of experimental observations to do so. In what follows, let $X$ be a set. 

\begin{defn}
For $x^*\in X$ and $U\subseteq X$, an \textbf{experimental identification} is any experimental observation of the form $\mathsf{o} = (x^*\in U, \mathsf{e}_\in(U))$ where $\mathsf{e}_\in(U)$ is an experimental test that succeeds if and only if $x^*\in U$. Any subset $U\subseteq X$ for which such an observation exists is said to be a \textbf{verifiable subset}. 
\end{defn} 


\begin{defn}
A \textbf{Experimental Identification Problem} $(X,\mathcal{D})$ is a set $X$ of cardinality $|X|\geq2$ together with a collection $\mathcal{D}$ of experimental identifications for the same element $x^*\in X$. 
\end{defn}

Next, we will explore the important properties of the collection of verifiable subsets. 


\begin{lem}
\label{setbehavior}
	Let $U_1, U_2, ... , U_n, ...$ be a countable infinite sequence of verifiable sets. The finite intersection $\bigcap\limits_{i=1}^{n} U_i$ and the countable union $\bigcup\limits_{i=1}^{\infty} U_i$ are verifiable sets.
\end{lem}

\begin{proof}
	We claim the finite intersection of verifiable sets is a verifiable set. Let $U_1, U_2, ... , U_n \subseteq X$ be n verifiable sets. For each $U_i$ there exists an experimental observation $\mathsf{o}_i = (x^*\in U_i, \mathsf{e}_\in(U_i))$. Consider $\mathsf{o} = \bigwedge\limits_{i=1}^{n} \mathsf{o}_i = (\bigwedge\limits_{i=1}^{n} x^*\in U_i , \mathsf{e}_{\wedge}(\mathsf{e}_\in(U_i)))=( x^*\in \bigcap\limits_{i=1}^{n} U_i, \mathsf{e}_\in(\bigcap\limits_{i=1}^{n} U_i))$ is an experimental observation. We conclude $\bigcap\limits_{i=1}^{n} U_i$ is a verifiable set.
	
	We claim the countable union of verifiable sets is a verifiable set. Let $U_1, U_2, ... , U_n, ... \subseteq X$ be an infinite sequence of verifiable sets. For each $U_i$ there exists an experimental observation $\mathsf{o}_i = (x^*\in U_i, \mathsf{e}_\in(U_i))$. Consider $\mathsf{o} = \bigvee\limits_{i=1}^{\infty} \mathsf{o}_i = (\bigvee\limits_{i=1}^{\infty} x^*\in U_i, \mathsf{e}_{\vee}(\mathsf{e}_\in(U_i)))=( x^*\in \bigcup\limits_{i=1}^{\infty} U_i, \mathsf{e}_\in(\bigcup\limits_{i=1}^{\infty} U_i))$ is an experimental observation. Thus $\bigcup\limits_{i=1}^{\infty} U_i$ is a verifiable set.
\end{proof}

Next, we wish to isolate the experimental identification problems for which there exists an appropriate (for applications) collection of experimental observations. 

\begin{defn}
An experimental identification problem $(X,\mathcal{D})$ is called \textbf{experimentally distinguishable} if $\mathcal{D}$ is an experimental domain such that given any two possibilities $x_1, x_2 \in X$ we can find two experimental observations $(x^*\in U_1, \mathsf{e}_\in(U_1)), (x^*\in U_2, \mathsf{e}_\in(U_2))\in\mathcal{D}$ such that $x_i\in U_i$ for $i=1,2$, and $U_1\cap U_2 = \emptyset$.
\end{defn}

From the work above, we will quickly arrive at the following.

\begin{thm}
An experimentally distinguishable identification problem $(X,\mathcal{D})$ forms a Hausdorff, second-countable topology $(X,\mathsf{T})$ with open sets given by the verifiable sets. 
\end{thm}
\begin{proof}
First, from the definition one can see that for all $x\in X$, there exists a verifiable set $U$ with $x\in U$, so their union $X$ is also a verifiable set. Further, there is a null observation by taking the conjunction of two identifications corresponding to disjoint sets, so the empty set is a verifiable set. Now, Proposition \ref{setbehavior} shows that the collection $\mathsf{T}$ is closed under finite intersection and countable union. Because $\mathsf{T}$ is determined by an experimental domain of observations, there is a countable basis of operations which translates to a countable basis of open (verifiable) sets, so it is second-countable. To show it is closed under arbitrary union, notice that an arbitrary union may be rewritten as a union of basis elements, which is then a countable union, and so it remains in $\mathsf{T}$. That the topology is Hausdorff is immediate from the last part of the definition. 
\end{proof}


And so we have found that all experimentally distinguishable spaces may be given by Hausdorff, second-countable topological spaces. 


\begin{rem}
It is an elementary exercise to show that any Hausdorff, second-countable topological space has cardinality at most that of the continuum. This means that, no matter what technique we use now or in the future, the collections of elements that we can properly define experimentally are at most the cardinality of the continuum. This already gives us a first simple and basic requirement a set of mathematical objects need to pass to be of scientific interest.

For example, these objects have cardinality of continuum, and therefore are good candidates:
\begin{itemize}
	\item Euclidean space $\mathbb{R}^n$
	\item all continuous functions from $\mathbb{R}$ to $\mathbb{R}$
	\item all open sets in $\mathbb{R}^n$
	\item all subsets of $\mathbb{N}$
\end{itemize}

These, instead, have cardinality greater then continuum, and therefore are not good candidates:
\begin{itemize}
	\item all functions from $\mathbb{R}$ to $\mathbb{R}$
	\item all subsets of $\mathbb{R}$
\end{itemize}
\end{rem}



\section{Experimental relationships}

TODO

Now that we have defined the possibilities of our domains and the topology associated with them, we want to establish relationships between them. The aim of scientific investigation is studying and characterizing these relationships. 



\begin{prop}
	\label{setfunctions}
	Let $X$ and $Y$ be two Hausdorff topological spaces, and let $g: \mathsf{T}(Y) \rightarrow \mathsf{T}(X)$ be a mapping such that:
	\begin{enumerate}
		\item It is compatible with union and intersection $\forall V_1, V_2 \in Y$ $g(V_1 \cup V_2)=g(V_1)\cup g(V_2)$ and $g(V_1 \cap V_2)=g(V_1)\cap g(V_2)$
		\item $g(\emptyset) = \emptyset$
		\item $g(Y) = X$
	\end{enumerate}
	Then there exists a unique continuous function $f: X \rightarrow Y$ such that $g = f^{-1} |_{\mathsf{T}(Y)}$.
\end{prop}

\begin{proof}
	We claim there exists a unique extension $\bar{g}:\sigma(Y)\to\sigma(X)$ to the Borel $\sigma$-algebras of $X$ and $Y$, respectively $\sigma(X)$ and $\sigma(Y)$, such that $\bar{g}|_{\mathsf{T}(Y)}=g$ and $\bar{g}$ is compatible with union, intersection and complements. Let $\bar{g}(V) = g(V)$ for all open sets $V \in \mathsf{T}(Y)$. Let $A \in \sigma(Y)$ (not necessarily open) and $A^C$ be its complement. We must have $\bar{g}(A^C) = \bar{g}(A)^C = X\setminus \bar{g}(A)$ for $\bar{g}$ to be compatible with complements. Recall that all Borel sets in $\sigma(Y)$ and $\sigma(X)$ may be written as some combination of unions, intersections, and complements of open sets. Thus, the construction uniquely determine what $\bar{g}$ should output on any Borel set. We need only check that the output is still a Borel set. But by definition of $\bar{g}$, the outputs will be given as unions, intersections, and complements of outputs of $g$, which are open sets, and so the image of $\bar{g}$ is contained in $\sigma(X)$.  $\bar{g}$ is well defined. The function $\bar{g}$ in a sense represents extracting the maximum amount of information possible out of $g$.
	
	We claim we can define $\hat{g}:Y\to\sigma(X)$ such that $\hat{g}(y) = \bar{g}(\{y\})$. Since $Y$ is Hausdorff, every singleton $\{y\}$ is closed and is therefore a Borel set. $\bar{g}(\{y\})$ is well defined and so is $\hat{g}(y)$.
	
	We claim  $\hat{g}(y_1)\cap\hat{g}(y_2) = \emptyset$ if and only if $y_1\neq y_2$ for all $y_1,y_2\in Y$ such that $\hat{g}(y_i)\neq\emptyset$ for $i=1,2$. If $y_1\neq y_2$ we have
	$$
	\hat{g}(y_1)\cap\hat{g}(y_2) = \bar{g}(\{y_1\})\cap\bar{g}(\{y_2\}) = \bar{g}(\{y_1\}\cap\{y_2\}) = \bar{g}(\emptyset) = \emptyset.
	$$
	Conversely, if $y_1 = y_2$ we have
	$$
	\hat{g}(y_1)\cap\hat{g}(y_2) = 	\hat{g}(y_1)\cap\hat{g}(y_1) = 
	\hat{g}(y_1) \neq \emptyset.
	$$
	
	We claim we can define $f: X\to Y$ such that $f(x) = y$ if and only if $x\in \hat{g}(y)$. Since $g(Y)=X$, there exists $y\in Y$ such that $x\in\hat{g}(y)$. By the preceding claim, this $y$ is unique. $f: X\to Y$ is well defined. Note that no arbitrary choice where made so far that lead to the construction of $f$, which is therefore determined uniquely by $g$. 
	
	We claim $g = f^{-1} |_{\mathsf{T}(Y)}$. Let $V\in\mathsf{T}(Y)$. We want to show $f^{-1}(V) = g(V)$. Let $x\in f^{-1}(V)$. Then for some $y \in V$ we have $f(x)=y$. $x\in \hat{g}(y)$ by construction of $f$. $\hat{g}(y) \subset g(V)$ since $\{y\}\subset V$, so $x\in g(V)$. $f^{-1}(V) \subseteq g(V)$. Conversely, let $x\in g(V)=\bar{g}(V)$. Then for some $y\in V$, we have $x\in\bar{g}(\{y\})\subset\bar{g}(V)$. But then by definition we have $f(x)=y$, so $x\in f^{-1}(V)$. $f^{-1}(V) \supseteq g(V)$. $f^{-1}(V) = g(V)$ for all $V\in\mathsf{T}(Y)$ and therefore $f^{-1}|_{\mathsf{T}(Y)}=g$.
	
	We claim $f$ is continuous. It is so since $g = f^{-1} |_{\mathsf{T}(Y)}$ takes open sets to open sets. 
\end{proof}

The above work gives us an approach to reconstruct continuous functions given the behavior of the inverse on open sets. In the context of collecting information on observable phenomena, the function $g$ represents the total of all information possible to gather on the correlation between two variables. The fact that from $g$ we can construct a unique continuous $f$ shows us that in the infinite-resource ideal, we fully obtain the information to give exact relations between variables. Further, the result itself tells us that the interesting phenomena in our framework of experimental observations will be continuous, which is often a baseline assumption in physics. 


For continuous functions to be physically distinguishable, we need to show they can always be given, as a set, given a topology that is Hausdorff and second-countable.

To that end, we introduce the basis-to-basis topology on the set of continuous functions from two topological spaces. This is the topology generated by the sets of functions that map a basis element of one element inside an element of the other space. 

\begin{defn} Let $X$ and $Y$ be two topological spaces. Let $C(X,Y)$ denote the set of all continuous functions from $X$ to $Y$. Let $\mathcal{B}_X$ and $\mathcal{B}_Y$ be two bases for $X$ and $Y$ respectively. The basis-to-basis topology $\mathsf{T}(C(X,Y), \mathcal{B}_X, \mathcal{B}_Y)$ on $C(X,Y)$ with respect to the basis $\mathcal{B}_X$ and $\mathcal{B}_Y$ is the topology generated by all sets of the form 
	$$
	V(U_X, U_Y) = \{f\in C(X,Y) : f(U_X)\subset U_Y\}
	$$
where $U_X \in \mathcal{B}_X$ and $U_Y \in \mathcal{B}_Y$.
\end{defn}

Now we need to show that if the two spaces $X$ and $Y$ are Hausdorff and second-countable, the basis-to-basis topology is Hausdorff and second-countable for a suitable choice of basis.

\begin{prop}
	Let $X$ and $Y$ be two Hausdorff and second-countable topological spaces. Let $C(X,Y)$ denote the set of all continuous functions from $X$ to $Y$. Let $\mathcal{B}_X$ and $\mathcal{B}_Y$ be two countable bases for $X$ and $Y$ respectively. The basis-to-basis topology $\mathsf{T}(C(X,Y), \mathcal{B}_X, \mathcal{B}_Y)$ on $C(X,Y)$ with respect to the basis $\mathcal{B}_X$ and $\mathcal{B}_Y$ is Hausdorff and second-countable. 
\end{prop}
\begin{proof}
	We claim $\mathsf{T}(C(X,Y), \mathcal{B}_X, \mathcal{B}_Y)$ is second-countable. We first note that the sub-basis $\{V(U_X, U_Y) \, |\,   U_X \in \mathcal{B}_X , U_Y \in \mathcal{B}_Y \}$ is countable since $\mathcal{B}_X$ and $\mathcal{B}_Y$ are countable. The collection $\mathcal{B}$ of all finite intersections is still countable, since it is in one-to-one correspondence with the collection of finite subsets of a countable set, which is still countable. Therefore $\mathcal{B}$ is a countable basis, which means $\mathsf{T}(C(X,Y), \mathcal{B}_X, \mathcal{B}_Y)$ is second-countable.
	
	We claim $\mathsf{T}(C(X,Y), \mathcal{B}_X, \mathcal{B}_Y)$ is Hausdorff. Let $f,g:X\to Y$ be distinct continuous functions. Then for some $x\in X$, we have $f(x)\neq g(x)$. Pick $V_1, V_2$ disjoint open subsets of $Y$ with $f(x)\in V_1$ and $g(x)\in V_2$. We may assume (possibly by shrinking $V_1$ or $V_2$) that both are basis elements for the topology of $Y$. Let $U=f^{-1}(V_1)\cap g^{-1}(V_2)$. Then $U$ is an open neighborhood of $x$. We may assume again that $U$ is a basis element for the topology on $X$ by shrinking it if necessary. Now, let $T_1$ to be the (sub-)basis element for basis-to-basis topology corresponding to $U$ and $V_1$. By construction, $f\in T_1$. Similarly, let $T_2$ to be the basis element for the basis-to-basis topology corresponding to $U$ and $V_2$ and containing $g$. Since $V_1$ and $V_2$ are disjoint, so are $T_1$ and $T_2$. $\mathsf{T}(\mathcal{C})$ is Hausdorff.
\end{proof}

We will also show that this is not in general equal to the open-open topology. 

\begin{prop}
	In general, the topology $\mathsf{T}(\mathcal{C})$ defined above depends on the bases for $X$ and $Y$. 
\end{prop}
\begin{proof}
	We will give an example of such a case. Let $X=Y=\mathbb{R}$ with the usual topology on $\mathbb{R}$. Let $\mathcal{B}_1$ be the basis for $\mathbb{R}$ consisting of all open intervals with rational endpoints, and let $\mathcal{B}_2 = \mathcal{B}_1\cup\{(0,\pi)\}$. We find a set open in our topology $\mathsf{T}(\mathcal{C})$ using $\mathcal{B}_2$ as our basis which is not open when we use $\mathcal{B}_1$. Consider the following set, open in our topology when using $\mathcal{B}_2$:
	$$
	V = \{f\in\mathcal{C}| f((0,\pi))\subset(0,1)\}
	$$
	This is clearly open because $(0,\pi)$ and $(0,1)$ are basis elements in $\mathcal{B}_2$. 
	
	Next, we show that when using $\mathcal{B}_1$ as our basis for $X=Y=\mathbb{R}$, no finite intersection and/or infinite union of sub-basis elements for $\mathsf{T}(\mathcal{C})$ will equal $V$. Henceforth, when we say ``open" subsets, unless otherwise stated, we will refer exclusively to the topology $\mathsf{T}(\mathcal{C})$ when using $\mathcal{B}_1$ for $\mathbb{R}$. The smallest sub-basis subsets containing $V$ are those of the form:
	$$
	B_r = \{f\in\mathcal{C} | f((0,r))\subset(0,1)\}
	$$
	for each $r\in\mathbb{Q}$ with $r<\pi$. By ``smallest" we mean any sub-basis elements containing $V$ which are not in the form of $B_r$ will contain some $B_r$. Now, any finite intersection of the $B_r$'s will be another $B_r$, but these strictly contain $V$, so we can never obtain $V$ from this sub-basis. 
	
	Similarly, the largest sub-basis sets contained within $V$ are of the form: 
	$$
	S_r = \{f\in\mathcal{C} | f((0,r))\subset(0,1)\}
	$$
	for each $r\in \mathbb{Q}$ with $r>\pi$. Let $S = \cup_{r>\pi}S_r$. This is the largest open set in our topology which is contained in $V$. Consider the function $f(x) = x/\pi$. This is continuous and is an element of $V$. But notice $f(\pi)=1$, so $f\notin S_r$ for all $r>\pi$, hence $f\notin S$. Thus, $V$ is a set which is open in $\mathsf{T}(\mathcal{C})$ when we use $\mathcal{B}_1$ for $X$ and $Y$, but not open when we use $\mathcal{B}_2$ for $X$ and $Y$. 
\end{proof}


\begin{cor}
The topology $\mathsf{T}(\mathcal{C})$ on the set of continuous functions $C(\mathbb{R},\mathbb{R})$ from $\mathbb{R}$ to itself is in general not equal to the open-open topology on $C(\mathbb{R},\mathbb{R})$.
\end{cor}
\begin{proof}
In the proof of the previous proposition, note that the basis for the open-open topology is independent of the basis used for the underlying spaces, and in particular is a superset of the basis for $\mathsf{T}(\mathcal{C})$ when using $\mathbb{B}_2$ to construct it. Hence the open-open topology is distinct from $\mathsf{T}(\mathcal{C})$ when using the basis $\mathcal{B}_1$ on $\mathbb{R}$. 
\end{proof}



Thus we have shown one can start with two Hausdorff, second-countable spaces, and generate a new Hausdorff second-countable space consisting of all functions between them. Topological spaces with these properties make up our class of scientifically interesting spaces, so the space of continuous (distinguishability-preserving) functions between spaces of distinguishable quantities is again a physically distinguishable space. Because one can iterate this construction (since the relevant properties are preserved), this is a way to generate arbitrarily many new physically distinguishable spaces encoding important information about the ``lower-order" spaces (i.e. they consist of functions which encode a conversion between verifiable sets in different spaces).

\begin{table}[h]
	\centering
	\begin{tabular}{p{0.20\textwidth} p{0.7\textwidth}}
		Math/Topology & Science/Physics \\ 
		\hline 
		$T_2$, 2nd ctbl space. & Experimentally distinguishable space, whose points are the possible values and whose open sets represent the experimentally attainable levels of precision \\
		Open set & Verifiable set. We can verify experimentally that an element is within the set  \\ 
		Closed set & Refutable set. We can verify experimentally that an element is not in the set \\ 
		Basis of top. & A collection of verifiable sets such that any verifiable set is determined by the basis sets\\
		Continuous \newline function &  A function between two sets of experimental distinguishable elements that preserves distinguishability \\
		Homeomorphism &  A perfect equivalence between experimentally distinguishable spaces. \\
	\end{tabular} 
	\caption{Topology to physics dictionary}
\end{table}


\section{Conclusion}

We have shown that it is possible to formally capture the requirements of experimental sciences using experimental observations and their properties. These definitions lead naturally to topological spaces, giving the result that any set of experimentally distinguishable possibilities is a Hausdorff and second-countable topological space. In the same framework, experimental relationships between these objects are represented mathematically by continuous functions and are themselves experimentally distinguishable objects, as they too can be given a Hausdorff and secound-countable topology. The ability to create functions of any order confirms that the universe of discourse provided by these definitions is closed and self-consistent.

We hope that this work, because of both its method and result, can provide a more solid foundation to formalize experimental sciences.


\section*{Acknowledgments}
Funding for this work was provided in part by the MCubed program of the University of Michigan. The third author is supported by the National Science Foundation Graduate Research Fellowship Program under Grant No. DGE\#1256260. Any opinions, findings, and conclusions or recommendations expressed in this material are those of the author(s) and do not necessarily reflect the views of the National Science Foundation.

\bibliographystyle{elsarticle-num}
\bibliography{bibliography}

\end{document}