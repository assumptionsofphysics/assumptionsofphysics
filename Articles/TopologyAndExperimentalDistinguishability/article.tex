\documentclass[review]{elsarticle}

\usepackage{lineno,hyperref}

\usepackage{amsmath, amsthm, amsfonts}
\usepackage[only,llbracket, rrbracket,llparenthesis,rrparenthesis]{stmaryrd} 
\usepackage{MnSymbol,centernot}

% Symbol for "implies" and "not implies"
\def\imp{\Rightarrow}
\def\nimp{\nRightarrow}

% Symbol for "compatibility" and "incompatibility"
\def\comp{\doublefrown}
\def\ncomp{\ndoublefrown}

\newcommand{\R}{\mathbb{R}}
\newcommand{\Q}{\mathbb{Q}}
\newcommand{\Z}{\mathbb{Z}}


\theoremstyle{plain}% default 
\newtheorem{thm}{Theorem}[section] 
\newtheorem{lem}[thm]{Lemma} 
\newtheorem{prop}[thm]{Proposition} 
\newtheorem{cor}{Corollary}

 \theoremstyle{definition}
 \newtheorem{defn}{Definition}[section]
\newtheorem{exmp}{Example}[section]
\theoremstyle{remark}
\newtheorem*{rem}{Remark}
\modulolinenumbers[5]

\journal{Topology and its Applications}


\begin{document}

\begin{frontmatter}

\title{Topology and experimental distinguishability}

%% Group authors per affiliation:

\author[]{Christine A. Aidala\corref{cor1}}
\ead{caidala@umich.edu}

\author[]{Gabriele Carcassi}
\ead{carcassi@umich.edu}

\author[]{Mark J. Greenfield}
\ead{markjg@umich.edu}

\address{University of Michigan, Ann Arbor, MI 48109, USA}

\cortext[cor1]{Corresponding author}


\begin{abstract}
We formalize the relationship between topological spaces and the ability to distinguish objects experimentally, providing understanding and justification as to why topological spaces and continuous functions are pervasive tools in science. We first define an experimental observation as a statement that can be verified using an experimental procedure and show that observations are closed under finite conjunction and countable disjunction. We then consider observations that identify elements in a set and show how they induce a Hausdorff and second-countable topology on that set, thus identifying an open set as one that can be associated with an experimental observation. We then show that continuous functions preserve experimental distinguishability and define a Hausdorff and second-countable topology for this collection. 
\end{abstract}

\begin{keyword}
topological spaces \sep scientific foundations \sep function space
\MSC[2010] 54H99 \sep 00A79
\end{keyword}

\end{frontmatter}

\linenumbers

\section{Introduction}

The successful use of mathematical ideas in experimental sciences is long established and celebrated \cite{wigner}. In many ways, topology is perhaps the most widespread, as it provides the foundation for other mathematical tools such as differential geometry and analysis. Topology is foundational in physics \cite{nakahara}, and has been the backbone of advances in many areas such as biology \cite{rashevsky} and social choice theory \cite{chichilnisky}. This leads one to ask, why is it so successful? What property is captured by topological spaces that is so fundamental for scientific investigation? 

We attempt to answer these questions in a precise manner by defining the appropriate connections. Topological spaces generalize metric spaces; they provide a notion of ``nearness" of points without necessarily quantifying it. We extend this interpretation in a precise way to cover spaces representing objects (points in space, objects needing classification, etc.) which are distinguishable by scientific experimentation. Because experimental data is universally limited by precision, it is neighborhoods of points that are associated with experimental outcomes, as opposed to the points themselves. Along the same line, experimental relationships need to preserve what can be experimentally distinguished, and they are therefore should be realized as continuous functions on the appropriate topological spaces.

In this sense, experimental distinguishability is intrinsically topological in nature. It is fitting, then, that the use of topology is so widespread, as defining what can be experimentally identified is a fundamental aspect of science. The aim of this paper is to lay down a framework that formalizes this insight. 

The work is organized as follows. We first define experimental observations as statements paired with an experimental test that is able to verify them. These will be the fundamental objects which enable us to define our universe of discourse. We study their properties under logical operations and show that they are only closed under finite conjunction and countable disjunction.  We then show that any set of objects that are experimentally distinguishable is a Hausdorff, second-countable topological space. The topology of the space is formally defined in terms of the observations themselves.  Finally, we study relationships between experimentally distinguishable objects. These are represented by continuous functions, and they can be shown to be themselves physically distinguishable.


\section{Experimental observations}

In science, a statement can be accepted as true only if there exists a way to independently verify it. To capture this notion we introduce the following definitions.

\begin{defn}
	A \textbf{statement} $\mathsf{s}$ is a declarative sentence that is either true or false, as in classical logic. 
\end{defn}

\begin{defn}
	An \textbf{experimental test} $\mathsf{e}$ is a repeatable procedure (i.e. it can be restarted and stopped an arbitrary number of times) which may be successful, in which case it terminates in finite time, or may be not successful, in which case it may or may not terminate.
\end{defn}

% https://ebooks.adelaide.edu.au/m/mill/john_stuart/system_of_logic/chapter18.html for black swan reference

\begin{defn}
	An \textbf{experimental observation} $\mathsf{o}$ is a tuple $\llparenthesis \mathsf{s}, \mathsf{e} \rrparenthesis$ consisting of a statement $\mathsf{s}$ and an experimental test $\mathsf{e}$ such that the statement is true if and only if the  experimental test is successful. The experimental observation is \textbf{verified} if the statement is true.
\end{defn}

\section{Algebra of experimental observations}

We now want to understand how the standard Boolean algebra defined on statements carries over to experimental observations.

\begin{rem}
	Experimental observations are not closed under negation. The existence of an experimental test to verify a statement does not imply the existence of an experimental test to verify its negation.
\end{rem}

\begin{defn}
	The \textbf{conjunction} or \textbf{logical AND} of a finite collection  of experimental observations $\{\mathsf{o}_i\}_{i=1}^{n}=\{\llparenthesis \mathsf{s}_i, \mathsf{e}_i\rrparenthesis\}_{i=1}^{n}$ is the experimental observation $\bigwedge\limits_{i=1}^{n} \mathsf{o}_i = \llparenthesis \mathsf{s}, \mathsf{e}\rrparenthesis$ where $\mathsf{s} = \bigwedge\limits_{i=1}^{n} \mathsf{s}_i$ is the conjunction of the respective statements and $\mathsf{e} = \mathsf{e}_\wedge(\{\mathsf{e}_i\}_{i=1}^{n})$ is the experimental test that successfully terminates if and only if all $\{\mathsf{e}_i\}_{i=1}^{n}$ successfully terminate.
\end{defn}

\begin{proof}
	We show that the conjunction exists and is well defined. Let $\mathsf{e}_\wedge=\mathsf{e}_\wedge(\{\mathsf{e}_i\}_{i=1}^{n})$ be the experimental procedure defined as follows:
	\begin{enumerate}
	\item for each $i=1,\ldots,n$ run the test $\mathsf{e}_i$
	\item if all tests $\mathsf{e}_i$ terminate successfully, terminate the test $\mathsf{e}_\wedge$ successfully.
	\end{enumerate}
	The experimental procedure so defined terminates successfully if and only if all $\mathsf{e}_i$ terminate successfully. It will do so in finite time as each of the finitely many $\mathsf{e}_i$ succeeds in finite time. Therefore $\mathsf{e}_\wedge(\{\mathsf{e}_i\}_{i=1}^{n})$ is an experimental test that is successful if and only if all statements $\{\mathsf{s}_i\}_{i=1}^{n}$ are true. So, $\bigwedge\limits_{i=1}^{n} \mathsf{o}_i = \llparenthesis\bigwedge\limits_{i=1}^{n} \mathsf{s}_i, \mathsf{e}_{\wedge}(\mathsf{e}_i)\rrparenthesis$ is an experimental observation.
\end{proof}

\begin{rem}
	Conjunction cannot be extended to a countable collection as verification would require infinite time.
\end{rem}

\begin{defn}
	The \textbf{disjunction} or \textbf{logical OR} of a countable (finite or infinite) collection of experimental observations $\{\mathsf{o}_i\}_{i=1}^{\infty}=\{\llparenthesis \mathsf{s}_i, \mathsf{e}_i\rrparenthesis\}_{i=1}^{\infty}$ is the experimental observation $\bigvee\limits_{i=1}^{\infty} \mathsf{o}_i = \llparenthesis \mathsf{s}, \mathsf{e}\rrparenthesis$ where $\mathsf{s} = \bigvee\limits_{i=1}^{\infty} \mathsf{s}_i$ is the disjunction of the respective statements and $\mathsf{e} = \mathsf{e}_\vee(\{\mathsf{e}_i\}_{i=1}^{\infty})$ is the experimental test that successfully terminates if and only if at least one experimental test in $\{\mathsf{e}_i\}_{i=1}^{\infty}$ successfully terminates.
\end{defn}

\begin{proof}
	We show that the disjunction exists and is well defined. Let $\mathsf{e}_\vee=\mathsf{e}_\vee(\{\mathsf{e}_i\}_{i=1}^{\infty})$ be the experimental procedure defined as follows:
	\begin{enumerate}
	\item initialize $n$ to 1
	\item for each $i=1,\ldots,n$:
	\begin{enumerate}
		\item run the test $\mathsf{e}_i$ for $n$ seconds
		\item if $\mathsf{e}_i$ terminated successfully, terminate $\mathsf{e}_\vee$ successfully
	\end{enumerate}
	\item increment $n$ and go to step 2
	\end{enumerate}
	Suppose there exists an $i \in \mathbb{Z}^+$ such that $\mathsf{e}_i$ will terminate successfully. Then the above procedure will eventually run that test for time sufficient for it to terminate successfully. It will do so in finite time as it will have run finitely many tests finitely many times each for a finite amount of time. Therefore $\mathsf{e}_\vee(\{\mathsf{e}_i\}_{i=1}^{\infty})$ is an experimental test that is successful if and only if at least one statement in $\{\mathsf{s}_i\}_{i=1}^{n}$ is successful. So $\bigvee\limits_{i=1}^{\infty} \mathsf{o}_i =\llparenthesis\bigvee\limits_{i=1}^{\infty} \mathsf{s}_i, \mathsf{e}_{\wedge}(\{\mathsf{e}_i\}_{i=1}^{\infty})\rrparenthesis$ is an experimental observation.
\end{proof}

Taken together, finite conjunction and countable disjunction form the algebra of experimental observations. We also introduce the following special case, which will be useful later.

\begin{defn}
Any experimental observation whose statement is a contradiction is also called \textbf{contradiction} and is noted by $\bot$.
\end{defn}

\begin{defn}
Two experimental observations $\mathsf{o}_1$ and $\mathsf{o}_2$ are said to be \textbf{incompatible} if the conjunction $\mathsf{o}_1\wedge\mathsf{o}_2$ is a contradiction.
\end{defn}


\section{Experimental domain}

We now want to characterize the sets of observations for which it is feasible to experimentally verify all true statements.

\begin{defn}
	An \textbf{experimental domain} is a set of observations closed under finite conjunction and countable disjunction, such that all observations can be tested in infinite time. 
\end{defn}

\begin{rem}
	We do allow infinite time for the verification of a domain with the understanding that some domains will only be approximately verified in finite time. As we have, so to speak, only one infinity to spend, we spend it here to maximize its usefulness.
\end{rem}

At this point, the similarities between this mathematical structure and topologies are starting to emerge. In analogy to the latter, we define the following.

\begin{defn}
	A \textbf{sub-basis} of an experimental domain is any subset that can generate all others via finite conjunction and countable disjunction. A \textbf{basis} of an experimental domain is any subset that can generate all others by countable disjunctions. 
\end{defn}

\begin{rem}
	As for topologies, given a sub-basis one can generate a basis by taking all finite conjunctions of the observations. Any infinite sub-basis will generate a basis of the same cardinality.
\end{rem}

\begin{prop}
Let $\mathcal{D}$ be an experimental domain. Then there exists a countable basis (equivalently, sub-basis) $\mathcal{B}$ of $\mathcal{S}$.
\end{prop}

\begin{proof}
If there exists a countable basis $\mathcal{B}$, then given infinite time one can test all observations in $\mathcal{B}$. Given which observations of the basis are verified, one can deduce which other observations in $\mathcal{D}$ are verified (again using infinite time) by computing the appropriate disjunctions. 

If there does not exist a countable basis, then by definition there does not exist a sequence of experimental observations in $\mathcal{D}$ from which one can deduce all other observations in $\mathcal{D}$. Hence it is impossible to test all members of $\mathcal{S}$.
\end{proof}

\section{Experimental distinguishability}

We now turn our attention to a more specific case. We want to characterize an experimental domain whose function is to identify the value of an attribute among all possible values.

\begin{defn}
	An \textbf{attribute} is the triple $(X, x, \mathcal{D}_x)$ where:
	\begin{itemize}
		\item $X$ is the set of all possible values, which we call \textbf{possibilities}, and satisfies $|X|>1$
		\item $x$ is the value of the attribute, therefore $x \in X$
		\item $\mathcal{D}_x$ is an experimental domain containing all possible experimental observations of the form $\mathsf{o} = (x\in U, \mathsf{e}_\in(U))$ where $U \subseteq X$ is a set of possibilities and $\mathsf{e}_\in(U)$ is an experimental test that succeeds if and only if $x \in U$
	\end{itemize}
	Any subset $U\subseteq X$ for which such an observation exists is said to be a \textbf{verifiable set}.
\end{defn} 

\begin{lem}
\label{setbehavior}
	Let $U_1, U_2, ... , U_n, ...$ be a countable infinite sequence of verifiable sets. The finite intersection $\bigcap\limits_{i=1}^{n} U_i$ and the countable union $\bigcup\limits_{i=1}^{\infty} U_i$ are verifiable sets.
\end{lem}

\begin{proof}
	We claim the finite intersection of verifiable sets is a verifiable set. Let $U_1, U_2, ... , U_n \subseteq X$ be n verifiable sets. For each $U_i$ there exists an experimental observation $\mathsf{o}_i = (x\in U_i, \mathsf{e}_\in(U_i))$. Consider $\mathsf{o} = \bigwedge\limits_{i=1}^{n} \mathsf{o}_i = (\bigwedge\limits_{i=1}^{n} x\in U_i , \mathsf{e}_{\wedge}(\mathsf{e}_\in(U_i)))=( x\in \bigcap\limits_{i=1}^{n} U_i, \mathsf{e}_\in(\bigcap\limits_{i=1}^{n} U_i))$ is an experimental observation. We conclude $\bigcap\limits_{i=1}^{n} U_i$ is a verifiable set.
	
	We claim the countable union of verifiable sets is a verifiable set. Let $U_1, U_2, ... , U_n, ... \subseteq X$ be an infinite sequence of verifiable sets. For each $U_i$ there exists an experimental observation $\mathsf{o}_i = (x\in U_i, \mathsf{e}_\in(U_i))$. Consider $\mathsf{o} = \bigvee\limits_{i=1}^{\infty} \mathsf{o}_i = (\bigvee\limits_{i=1}^{\infty} x\in U_i, \mathsf{e}_{\vee}(\mathsf{e}_\in(U_i)))=( x\in \bigcup\limits_{i=1}^{\infty} U_i, \mathsf{e}_\in(\bigcup\limits_{i=1}^{\infty} U_i))$ is an experimental observation. Thus $\bigcup\limits_{i=1}^{\infty} U_i$ is a verifiable set.
\end{proof}

Next, we wish to isolate the experimental identification problems for which there exists an appropriate (for applications) collection of experimental observations. 

\begin{defn}
The possibilities of an attribute $(X, x, \mathcal{D}_x)$, and by extension the attribute itself, are said \textbf{experimentally distinguishable} if $\mathcal{D}_x$ is such that given any two possibilities $x_1, x_2 \in X$ we can find two incompatible experimental observations $(x\in U_1, \mathsf{e}_\in(U_1)), (x\in U_2, \mathsf{e}_\in(U_2))\in\mathcal{D}$ such that $x_i\in U_i$ for $i=1,2$. 
\end{defn}

We are now ready to prove the first main result of this work.

\begin{thm}
The set of experimentally distinguishable possibilities for an attribute $(X, x, \mathcal{D}_x)$ is a Hausdorff, second-countable topological space $(X,\mathsf{T})$ where the open sets are given by the verifiable sets. 
\end{thm}
\begin{proof}
First, from the definition one can see that for all $x\in X$, there exists a verifiable set $U$ with $x\in U$, so their union $X$ is also a verifiable set. Further, there is a null observation by taking the conjunction of two identifications corresponding to disjoint sets, so the empty set is a verifiable set. Now, Proposition \ref{setbehavior} shows that the collection $\mathsf{T}$ is closed under finite intersection and countable union. Because $\mathsf{T}$ is determined by an experimental domain of observations, there is a countable basis of operations which translates to a countable basis of open (verifiable) sets, so it is second-countable. To show it is closed under arbitrary union, notice that an arbitrary union may be rewritten as a union of basis elements, which is then a countable union, and so it remains in $\mathsf{T}$. That the topology is Hausdorff is immediate from the last part of the definition. 
\end{proof}

\begin{rem}
As any Hausdorff, second-countable topological space has at most cardinality of the continuum, that is also the greatest cardinality that a set of experimentally distinguishable objects can have. We can conclude that set of mathematical objects, such as all functions from $\mathbb{R}$ to $\mathbb{R}$, that do not satisfy this requirement are not good candidates to represent scientific concepts.
\end{rem}

\section{Experimental relationships}

We now want to characterize relationships between two attributes. Such relationships can be defined either on the values (i.e. the possibilities) or on the observations (i.e. the experimental domain). We need to show that both definitions lead to the same mathematical object.

\begin{defn}[Experimental relationship on possibilities]
	Let $(X, x, \mathcal{D}_x)$ and $(Y, y, \mathcal{D}_y)$ be two attributes. An \textbf{experimental relationship} is a map $f : X \rightarrow Y$ that can be used within an experimental test.
\end{defn}

\begin{prop}
	The experimental relationship $f$ defined above is a continuous function.
\end{prop}
\begin{proof}
Let $o_y = \llparenthesis y\in U_Y,\mathsf{e}_{\in}(U_Y)\rrparenthesis \in \mathcal{D}_y$.  Consider the following experimental procedure:
\begin{enumerate}
	\item map $x$ to $y=f(x)$
	\item run the test $\mathsf{e}_{\in}(U_Y)$
\end{enumerate}
It will be successful if and only if $y \in U_Y$. Since $y=f(x)$, it is successful if and only if $x \in f^{-1}(U_Y)$: the procedure is the test $\mathsf{e}_{\in}(f^{-1}(U_Y))$. This means that $o_x = \llparenthesis x \in f^{-1}(U_Y),\mathsf{e}_{\in}(f^{-1}(U_Y))\rrparenthesis$ is an experimental observation and it must be in $\mathcal{D}_x$ since $\mathcal{D}_x$ contains all possible experimental observations of that form. It follows that $f^{-1}(U_Y)$ is a verifiable set and must be part of the topology $\mathsf{T}_X$.
\end{proof}


Next, we consider the opposite problem. Suppose we have two experimentally distinguishable spaces $(X,\mathsf{T}_X)$ and $(Y,\mathsf{T}_Y)$, and that we have a well-behaved (with respect to conjunction and disjunction) map on the experimental domains. Then this translates to a map of topologies, taking the verifiable (open) sets of $Y$ to those of $X$, which respects (arbitrary) union and (finite) intersection. In this case, we will show that one can find a continuous map which induces this map of topologies. 


\begin{defn}[Experimental relationship on observations]
	Let $(X, x, \mathcal{D}_x)$ and $(Y, y, \mathcal{D}_y)$ be two attributes. An \textbf{experimental relationship} is a map $g : \mathcal{D}_y \rightarrow \mathcal{D}_x$ such that if $\mathsf{o}_y \in \mathcal{D}_y$ is verified then $\mathsf{o}_x \in \mathcal{D}_x$ must also be verified. To be consistent, such relationship must have these properties:
	\begin{enumerate}
	\item it is compatible with conjunction and disjunction: for any $\mathsf{o}_1, \mathsf{o}_2 \in \mathcal{D}_y$, we have $g(\mathsf{o}_1 \wedge \mathsf{o}_2)=g(\mathsf{o}_1)\wedge g(\mathsf{o}_2)$ and $g(\mathsf{o}_1 \vee \mathsf{o}_2)=g(\mathsf{o}_1)\vee g(\mathsf{o}_2)$
	\item contradiction leads to contradiction: $g(\bot) = \bot$
	\item no knowledge leads to no knowledge: if $Y$ is the verifiable set associated with $\mathsf{o}_y$ then $X$ is the verifiable set associated with  $g(\mathsf{o}_y)$
	\end{enumerate}
\end{defn}

\begin{prop}
	Each experimental relationship $g$ defined above is associated with a unique continuous function $f: X \rightarrow Y$ such that $g(\llparenthesis y\in U_Y,\mathsf{e}_{\in}(U_Y)\rrparenthesis) = \llparenthesis x\in f^{-1}(U_Y),\mathsf{e}_{\in}(f^{-1}(U_Y))\rrparenthesis$.
\end{prop}

\begin{proof}
TODO: 	Let $X$ and $Y$ be two Hausdorff topological spaces, and let $g: \mathsf{T}_Y \rightarrow \mathsf{T}_X$ be a mapping such that:
\begin{enumerate}
	\item It is compatible with union and intersection, i.e. for any subsets $V_1, V_2 \subseteq Y$, we have $g(V_1 \cup V_2)=g(V_1)\cup g(V_2)$ and $g(V_1 \cap V_2)=g(V_1)\cap g(V_2)$
	\item $g(\emptyset) = \emptyset$
	\item $g(Y) = X$
\end{enumerate}


	We claim there exists a unique extension $\bar{g}:\sigma(Y)\to\sigma(X)$ to the Borel $\sigma$-algebras of $X$ and $Y$, respectively $\sigma(X)$ and $\sigma(Y)$, such that $\bar{g}|_{\mathsf{T}_Y}=g$ and $\bar{g}$ is compatible with union, intersection and complements. Let $\bar{g}(V) = g(V)$ for all open sets $V \in \mathsf{T}_Y$. Let $A \in \sigma(Y)$ (not necessarily open) and $A^C$ be its complement. We must have $\bar{g}(A^C) = \bar{g}(A)^C = X\setminus \bar{g}(A)$ for $\bar{g}$ to be compatible with complements. Recall that all Borel sets in $\sigma(Y)$ and $\sigma(X)$ may be written as some combination of unions, intersections, and complements of open sets. Thus, the construction uniquely determine what $\bar{g}$ should output on any Borel set. We need only check that the output is still a Borel set. But by definition of $\bar{g}$, the outputs will be given as unions, intersections, and complements of outputs of $g$, which are open sets, and so the image of $\bar{g}$ is contained in $\sigma(X)$.  $\bar{g}$ is well defined. The function $\bar{g}$ in a sense represents extracting the maximum amount of information possible out of $g$.
	
	We claim we can define $\hat{g}:Y\to\sigma(X)$ such that $\hat{g}(y) = \bar{g}(\{y\})$. Since $Y$ is Hausdorff, every singleton $\{y\}$ is closed and is therefore a Borel set. $\bar{g}(\{y\})$ is well defined and so is $\hat{g}(y)$.
	
	We claim  $\hat{g}(y_1)\cap\hat{g}(y_2) = \emptyset$ if and only if $y_1\neq y_2$ for all $y_1,y_2\in Y$ such that $\hat{g}(y_i)\neq\emptyset$ for $i=1,2$. If $y_1\neq y_2$ we have
	$$
	\hat{g}(y_1)\cap\hat{g}(y_2) = \bar{g}(\{y_1\})\cap\bar{g}(\{y_2\}) = \bar{g}(\{y_1\}\cap\{y_2\}) = \bar{g}(\emptyset) = \emptyset.
	$$
	Conversely, if $y_1 = y_2$ we have
	$$
	\hat{g}(y_1)\cap\hat{g}(y_2) = 	\hat{g}(y_1)\cap\hat{g}(y_1) = 
	\hat{g}(y_1) \neq \emptyset.
	$$
	
	We claim we can define $f: X\to Y$ such that $f(x) = y$ if and only if $x\in \hat{g}(y)$. Since $g(Y)=X$, there exists $y\in Y$ such that $x\in\hat{g}(y)$. By the preceding claim, this $y$ is unique. $f: X\to Y$ is well defined. Note that no arbitrary choice where made so far that lead to the construction of $f$, which is therefore determined uniquely by $g$. 
	
	We claim $g = f^{-1} |_{\mathsf{T}(Y)}$. Let $V\in\mathsf{T}(Y)$. We want to show $f^{-1}(V) = g(V)$. Let $x\in f^{-1}(V)$. Then for some $y \in V$ we have $f(x)=y$. $x\in \hat{g}(y)$ by construction of $f$. $\hat{g}(y) \subset g(V)$ since $\{y\}\subset V$, so $x\in g(V)$. $f^{-1}(V) \subseteq g(V)$. Conversely, let $x\in g(V)=\bar{g}(V)$. Then for some $y\in V$, we have $x\in\bar{g}(\{y\})\subset\bar{g}(V)$. But then by definition we have $f(x)=y$, so $x\in f^{-1}(V)$. $f^{-1}(V) \supseteq g(V)$. $f^{-1}(V) = g(V)$ for all $V\in\mathsf{T}(Y)$ and therefore $f^{-1}|_{\mathsf{T}(Y)}=g$.
	
	We claim $f$ is continuous. It is so since $g = f^{-1} |_{\mathsf{T}(Y)}$ takes open sets to open sets. 
\end{proof}

We can now state the second main result of this work.

\begin{thm}
	An experimental relationship between two sets $X$ and $Y$ of experimentally distinguishable possibilities is a continuous function $f : X \rightarrow Y$ with respect to the between the respective topological spaces $(X,\mathsf{T}_X)$ and $(Y,\mathsf{T}_Y)$ given by the verifiable sets. 
\end{thm}

\begin{rem}
	This result gives a formal justification as to why continuous functions are prevalent in science in general and in physics in particular. As topologies capture experimental distinguishability, continuous function preserve it.
\end{rem}


\section{Distinguishability of experimental relationships}

To conclude we want to make sure that experimental relationships are themselves experimentally distinguishable. To do so it suffices to show that the set of continuous functions between two Hausdorff, second-countable topological spaces can be given a topology that is Hausdorff and second-countable.

\begin{defn} Let $X$ and $Y$ be two topological spaces. Let $C(X,Y)$ denote the set of all continuous functions from $X$ to $Y$. Let $\mathcal{B}_X$ and $\mathcal{B}_Y$ be two bases for $X$ and $Y$ respectively. The basis-to-basis topology $\mathsf{T}(C(X,Y), \mathcal{B}_X, \mathcal{B}_Y)$ on $C(X,Y)$ with respect to the basis $\mathcal{B}_X$ and $\mathcal{B}_Y$ is the topology generated by all sets of the form 
	$$
	V(U_X, U_Y) = \{f\in C(X,Y) : f(U_X)\subset U_Y\}
	$$
where $U_X \in \mathcal{B}_X$ and $U_Y \in \mathcal{B}_Y$.
\end{defn}

Now we need to show that if the two spaces $X$ and $Y$ are Hausdorff and second-countable, the basis-to-basis topology is Hausdorff and second-countable for a suitable choice of basis.

\begin{prop}
	Let $X$ and $Y$ be two Hausdorff and second-countable topological spaces. Let $C(X,Y)$ denote the set of all continuous functions from $X$ to $Y$. Let $\mathcal{B}_X$ and $\mathcal{B}_Y$ be two countable bases for $X$ and $Y$ respectively. The basis-to-basis topology $\mathsf{T}(C(X,Y), \mathcal{B}_X, \mathcal{B}_Y)$ on $C(X,Y)$ with respect to the basis $\mathcal{B}_X$ and $\mathcal{B}_Y$ is Hausdorff and second-countable. 
\end{prop}
\begin{proof}
	We claim $\mathsf{T}(C(X,Y), \mathcal{B}_X, \mathcal{B}_Y)$ is second-countable. We first note that the sub-basis $\{V(U_X, U_Y) \, |\,   U_X \in \mathcal{B}_X , U_Y \in \mathcal{B}_Y \}$ is countable since $\mathcal{B}_X$ and $\mathcal{B}_Y$ are countable. The collection $\mathcal{B}$ of all finite intersections is still countable, since it is in one-to-one correspondence with the collection of finite subsets of a countable set, which is still countable. Therefore $\mathcal{B}$ is a countable basis, which means $\mathsf{T}(C(X,Y), \mathcal{B}_X, \mathcal{B}_Y)$ is second-countable.
	
	We claim $\mathsf{T}(C(X,Y), \mathcal{B}_X, \mathcal{B}_Y)$ is Hausdorff. Let $f,g:X\to Y$ be distinct continuous functions. Then for some $x\in X$, we have $f(x)\neq g(x)$. Pick $V_1, V_2$ disjoint open subsets of $Y$ with $f(x)\in V_1$ and $g(x)\in V_2$. We may assume (possibly by shrinking $V_1$ or $V_2$) that both are basis elements for the topology of $Y$. Let $U=f^{-1}(V_1)\cap g^{-1}(V_2)$. Then $U$ is an open neighborhood of $x$. We may assume again that $U$ is a basis element for the topology on $X$ by shrinking it if necessary. Now, let $T_1$ to be the (sub-)basis element for basis-to-basis topology corresponding to $U$ and $V_1$. By construction, $f\in T_1$. Similarly, let $T_2$ to be the basis element for the basis-to-basis topology corresponding to $U$ and $V_2$ and containing $g$. Since $V_1$ and $V_2$ are disjoint, so are $T_1$ and $T_2$. $\mathsf{T}(\mathcal{C})$ is Hausdorff.
\end{proof}

\begin{rem}
	Note that the basis-to-basis topology is not in general equal to the open-open topology. The former may depend on the basis chosen while the second is uniquely defined by the topology.
\end{rem}

As experimental relationships are themselves distinguishable, we can recursively form experimental relationships between experimental relationship leading to functions of arbitrary order while remaining within the definition provided. The framework is therefore complete.

\begin{table}[h]
	\centering
	\begin{tabular}{p{0.20\textwidth} p{0.7\textwidth}}
		Math/Topology & Science/Physics \\ 
		\hline 
		$T_2$, 2nd ctbl space. & Experimentally distinguishable space, whose points are the possible values and whose open sets represent the experimentally attainable levels of precision \\
		Open set & Verifiable set. We can verify experimentally that an element is within the set  \\ 
		Closed set & Refutable set. We can verify experimentally that an element is not in the set \\ 
		Basis of top. & A collection of verifiable sets such that any verifiable set is determined by the basis sets\\
		Continuous \newline function &  A function between two sets of experimental distinguishable elements that preserves distinguishability \\
		Homeomorphism &  A perfect equivalence between experimentally distinguishable spaces. \\
	\end{tabular} 
	\caption{Topology to physics dictionary. This table sums up the relationship established between math and scientific concepts.}
\end{table}


\section{Conclusion}

We have shown that it is possible to formally capture the requirements of experimental sciences using experimental observations and their properties. These definitions lead naturally to topological spaces, giving the result that any set of experimentally distinguishable possibilities is a Hausdorff and second-countable topological space. In the same framework, experimental relationships between these objects are represented mathematically by continuous functions and are themselves experimentally distinguishable objects, as they too can be given a Hausdorff and secound-countable topology. The ability to create functions of any order confirms that the universe of discourse provided by these definitions is closed and self-consistent.

We hope that this work, because of both its method and result, can provide a more solid foundation to formalize experimental sciences.


\section*{Acknowledgments}
Funding for this work was provided in part by the MCubed program of the University of Michigan. The third author is supported by the National Science Foundation Graduate Research Fellowship Program under Grant No. DGE\#1256260. Any opinions, findings, and conclusions or recommendations expressed in this material are those of the author(s) and do not necessarily reflect the views of the National Science Foundation.

\bibliographystyle{elsarticle-num}
\bibliography{bibliography}

\end{document}