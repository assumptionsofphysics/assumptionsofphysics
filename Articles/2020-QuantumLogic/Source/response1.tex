\documentclass[11pt]{article}

\usepackage[margin=1.25in]{geometry}

\usepackage{amsmath, amsthm, amssymb}
\usepackage{amsfonts}
\usepackage{MnSymbol}
\usepackage{mathtools}
\usepackage{xcolor}
\usepackage{graphicx}
\usepackage{enumitem}

\newcommand\change[1]{{\color{red!75!black} #1}}


\begin{document}

TO THE EDITOR

thank you very much for having provided us
good quality referee reports for our paper. We appreciate that this is
not an easy task.



\section*{Referee 1}

\begin{quote}
The topic of this paper is the standing of "quantum logic" vis-á-vis "classical logic", and in particular whether quantum mechanics demands either a rejection of or adjustment to classical logic. The answer, in a nutshell, is "no". The burden of the paper is to explain exactly why the standard arguments to the opposite conclusion are incorrect, and incidentally how both distributive Boolean lattices of propositions can be defined in quantum mechanics and non-distributive, non-Boolean lattices of propositions can be defined in the theory of probability distributions over classical phase space. These observations serve to show how the non-distributivity is not connected either to quantum mechanics or to the use of complex (rather than real) numbers.

The paper is clear and decisive. There are sufficiently many citations to justify publications, for even though the star of quantum logic has faded it has not entirely gone out. It is worthwhile to have all of this laid out in a single, compact presentation.

I have only a couple of comments/ suggestions.

p. 2: The quotation from Holik seems quite reasonable. All that Holik claims is that the connectives he is using in the context of quantum theory are "analogues" to the classical truth-functional connectives, and immediately mentions that they cannot be taken to be the classical connectives. Unless Holik adds that the quantum ones are meant to replace or correct classical logic, the statement seems fine. It is certainly much better than the Stefanovich talk of "replacement".

p. 9: "More precisely, we are going to argue that (i) it is not necessarily the case that a quantum
disjunction must be true when neither of its members is". But surely that is attacking a straw man. No one every claimed that a "quantum disjunction" (join) must be true when the propositions joined fail to be true! For example, when they are both false.

\end{quote}

TODO

\begin{quote}
p. 1 1: "The solution is simple: while
the value of an observable is not always defined, its expectation value always is. The idea is
then to translate qi and ri into propositions about expectation values, which we already know
how to handle, and apply the standard rules. Consider the following statements:

q'i: "before the measurement, the expectation value for y-spin is ½ h.

r'i: "before the measurement, the expectation value for y-spin is -½ h.

Because the spin of the electron along any axis is bounded by ±1/2 h, the only way that q'I is
true is if qi is true. Therefore q'I and r'i are respectively equivalent to qi and ri."

I can't follow this claim. It is true that the only way that the expectation value can be, say, ½ h is if the system is in an eigenstate of y-spin with the value up, but one could certainly imagine a situation (according to a theory) in which the particle does indeed have y-spin up but the expectation value associated with the system is not +1/2 h. This could be true on a statistical understanding of the expectation value: even though this system has y-spin up, the expectation value is calculated with respect to an ensemble of systems, not all of which have y-spin up. So the inference certainly goes through in one direction, but whether it does in the other depends on how the expectation values are understood. But "equivalent" suggests equivalence in both directions. So I think more would have to be said here.
\end{quote}

To remove ambiguity, we expanded the example to take the most general case, that of a density matrix, and show that fixing the expectation value to 1/2 h means the density matrix must correspond to the pure state of spin up. That way it is clear what we mean by expectation, and we cover all possible cases that are present in quantum mechanics (i.e. there is no mixed state that could give the same result). It reads:

\change{If $\rho_i$ is the (potentially) mixed state before the measurement, the expectation value for $y$-spin is given by $\text{Tr}(\rho_i S_y) = \frac{1}{2} \hbar \langle \psi_y^+ | \rho_i | \psi_y^+ \rangle - \frac{1}{2} \hbar \langle \psi_y^- | \rho_i | \psi_y^- \rangle$. Therefore $q'_i$ is true if and only if $\text{Tr}(\rho_i S_y) = \frac{1}{2} \hbar$. Because $\rho_i$ is positive definite, this can only happen if $\langle \psi_y^+ | \rho_i | \psi_y^+ \rangle = 1$ and $\langle \psi_y^- | \rho_i | \psi_y^- \rangle = 0$, which means $\rho_i = | \psi_y^+ \rangle \langle \psi_y^+ |$. This means that the expectation value for $y$-spin is $\frac{1}{2} \hbar$ if and only if the electron was prepared in a pure state of $y$-spin up. In other words, because the spin of the electron along any axis is bounded by $\pm \frac{1}{2} \hbar$, the only way that $q'_i$ is true is if $q_i$ is true. We have that $q'_i$ and $r'_i$ are respectively equivalent to $q_i$ and $r_i$.}

Note that the argument holds regardless of how one interprets $\rho$ (e.g. state of knowledge, fraction of repeated experiments, ...).

\begin{quote}
I can see why there is a desire to introduce claims about expectation values: that make the transition to the discussion of classical physics smoother. But I'm not sure that this is the best way to do it. The main point about being careful with the time indices is the key to resolving the non-distributivity issue, and that can be does without bringing in expectation values. So this passage might be made clearer.
\end{quote}
Actually, the smoother transition is a bonus. The main reason to introduce expectation values here is to settle the question of how to evaluate statements of over one observable when an incompatible observable is prepared. As the text reads ``The difficulty here is that before the measurement of $S_y$ only $x$-spin is defined, so how are we going to compute statements about the value of $y$-spin?'' We suspect that, given the length of the paragraph, the point is lost by the end, so we close the paragraph with:

\change{The overall point is that the incompatibility of the observables does indeed introduce a problem because it is not clear how to evaluate propositions that mix different variables. The use of equivalent propositions over expectation values helps us solve the ambiguities.}

So the idea is that it is useful to add these propositions anyway, so it already gives us a hint to compare to classical statistical propositions.

\section*{Referee 2}

\begin{quote}
This is a very interesting article, and it puts back on the table an older issue that had been dormant for a while. As such, it is a helpful contribution. However, it does have several shortcomings that prevent me from endorsing for publication in its present form. 
\end{quote}

\begin{quote}
1. The primary defect of the paper is that it greatly overstates the degree to which people are convinced that QM demands a revision of our logic. I would say that 90\% of philosophers of physics believe that it does *not* demand a revision of our logic, and that 90\% of physicists would say that they do not know what it means to say that QM demands a revision of logic. The citations that the authors use represent only a small minority of philosophers and physicists who were trying to sell this idea. 

The main problem here is not that the authors misdescribe the situation. The main problem is that their arguments are largely aimed at a straw person -- and hence it is not so likely that the paper would receive a broad readership. 
\end{quote}

Looking again at the written prose, the stated motivation of the paper (particularly the abstract) is indeed just an argument against Quantum Logic. In that sense, the referee is correct. On one hand, QL still attracts funding and interest and, as referee 1 puts it:
\begin{quote}
	... for even though the star of quantum logic has faded it has not entirely gone out. It is worthwhile to have all of this laid out in a single, compact presentation.
\end{quote}
So we still think talking about QL is relevant, However, as the referee 2 notes later, the article goes past QL and compares at a deeper level the logical structures of classical and quantum mechanics. This, we believe, is of general interest regardless of how one may feel about QL. Given  comments from both referees, we propose to keep the arguments against QL, but to make it clear that our discussion goes deeper.

We revised the abstract as follows:

\change{At the onset of quantum mechanics, it was speculated that the new theory would entail a rejection of classical logic. The main arguments to support this claim come from the non-commutativity of quantum observables, which allegedly would generate a non-distributive lattice of propositions, and from quantum superpositions, which would entail new rules for quantum disjunctions. While the quantum logic program is not as popular as it once was, the core question remains unsettled: what is the relationship between the logical structures of the two theories? In this essay we show that the original arguments promoting quantum logic contain serious flaws, and that quantum theory does satisfy the classical distributivity law once the full meaning of quantum propositions is properly taken into account. Moreover, we show that quantum mechanics can generate a distributive lattice of proposition, which, unlike the one of quantum logic, includes statements about expectation values which are of undoubtable physical interest. Lastly, we show that the lattice of statistical propositions in classical mechanics follows the same structure, yielding an analogue non-commutative sublattice of classical propositions. This shows that the purported difference between classical and quantum logic stems from a misconstructed parallel between the two theories.}

We revised part of the introduction as follows:

\change{While the approach is still being actively pursued by a sized community, and despite the many interesting applications in the fields of quantum information, computation and cryptography that are being investigated to the present day, QL is no longer viewed as a solution to the foundational problems affecting quantum theory.\footnote{This conclusion is now accepted among experts; for details see e.g.\ \cite{Bacciagaluppi:2009} and \cite{Giuntini:2002}. Referring to this, the latter authors stated explicitly that ``quantum logics are not to be regarded as a kind of ``clue'', capable of solving the main physical and epistemological difficulties of QT [quantum theory]. This was perhaps an illusion of some pioneering workers in quantum logic'' (\cite{Giuntini:2002}, p.\ 225).} Nevertheless, while it may be reasonable for a physicist to simply discard the approach over practical considerations, such as the lack of a desired result, on philosophical grounds it begs the question: what is the relationship between the logical structures of classical and quantum mechanics? How are they different and how are they similar? What should we think about the core insights that lead to the development of QL? Were they substantially correct, or should they be rejected? Given the fundamental role that logic plays in philosophy, we feel that these questions still must be answered regardless of how one may feel towards QL, or we are bound to make the same mistakes.}

TODO: Maybe review the conclusions as well?


\begin{quote}
2. In Section 3, the authors argue the logic of quantum propositions is distributive ... if one indexes the propositions so that they refer to times (either before or after measurement). This point is not novel. It was already pointed out (although vaguely) by Niels Bohr in reply to John von Neumann. It was again pointed out by Patrick Heelan in the 1970s, and by Stanley Gudder shortly thereafter. The point of all these suggestions is roughly that if one indexes propositions by their context, then a non-distributive logic can be represented as a contextual distributive logic. 
\end{quote}


\begin{quote}
3. Section 4 is actually innovative, and could well be developed into a novel point. Especially interesting is the argument (pages 16-18) that all Borel subsets of Hilbert space represent physically interesting propositions about the system. But let me note a few qualifications. (a) The authors do not show that *every* Borel subset of H is of the form described in Proposition 2. That would be a very interesting result. (b) The authors do not cite others who have argued that all Borel subsets of H are physically relevant. See especially David Wallace's book on the quantum multiverse. (c) The authors do not consider any of the many objections that could be raised against their proposal. For example, one would not normally say that "quantity A having expectation value r" is a property of a system, but that it is a meta-property representing our ignorance of the actual value of A.
\end{quote}

For (a), the situation is a bit more complicated. Let us mention a few problems.

Not all Borel subsets of H are of the form described in Proposition 2. In fact, they are not all physically relevant. The issue is that a state is not a vector in the Hilbert space, but a ray (i.e. a subspace of dimension 1). The absolute phase and absolute norm are not physically observable per se, though these can be differentiated with Borel sets. The proper physically meaningful state space would be the projective space, whose Borel sets are a subset of the Borel sets of the Hilbert space.

Additionally: not all rays can realistically be considered physical. Some states, for example, correspond to infinite expectation for position, momentum, energy. One can, however, show that the set of states with finite expectation values for all polynomial of position and momentum corresponds to the Schwartz space, which is dense in the Hilbert space. One can then show that a basis for the topology of the projective space of the Schwartz space can be constructed with sets of the form described in Proposition 2.

More in general, to make a solid map between statements and experimental verification, the best way would be to use the results of Kelly or Carcassi/Aidala that link the open set of a topology with experimentally verifiable. This would also allow to go in further details. For example, for the space to be (at least in the limit) explorable experimentally the topology must be second countable and $T_0$. Outside of those restriction, one can show that the points of the space do not correspond to cases that can be experimentally distinguished.

Note that all these issues apply in an analogous way for the lattice of classical statistical propositions we consider.

The point here is that we would need the appropriate space to go through all of these details, which would probably double the size of the paper as we would have to introduce a lot of previous work that we cannot expect the reader to be familiar with. Our purpose therefore is limited to only show that the Borel sets contain a big class of experimentally interesting propositions that are not contained in the QL logic lattice.

We added the following footnote to both clarify and give some more information about the issue:

\change{Note that we are not claiming that all Borel sets can be constructed in this way or that all Borel sets correspond to physically interesting propositions. First of all, it is not true: $\{ \psi \}$ and $\{ e^{\imath \theta} \psi\}$ are different Borel sets, yet they are not physically distinguishable since a difference in absolute phase is not physically relevant. This can be fixed by considering the Borel sets of the projective space. Second, to really examine the physicality of all Borel sets we need a ``general theory of experimental logic'', of the type provided by [cite kevin kelly] or [cite carcassi aidala], which would go well beyond the present discussion. With appropriate caveats (e.g. the Hilbert space must be separable), one can say that the Borel sets corresponds to propositions that can be associated to an experimental test (what [carcassi aidala] call theoretical statements) which may or may not terminate in any or all cases. We leave this discussion for another work.}

For (b), TODO

For (c), each interpretation of quantum mechanics will have a competing view on the meaning of the expectation values and on the meaning of the propositions. For example, for some (e.g. Qubism, TODO) all expectations will indeed provide only information about the state of knowledge of the agent because, in their view, that is all that quantum states provide. For other (e.g. Bohmian, TODO) they will represent expectation over hidden variables.

Regardless of the interpretation, the framework of quantum mechanics provides a well-defined expectation value operation for either a  pure state $\psi$ and a density matrix $\rho$. Therefore the statement ``quantity A has expectation value r'' will correspond to the mathematical expression $\langle \psi | A | \psi \rangle = r$ (or $\text{Tr}(\rho) = r$). It is up to the particular interpretation to make it clear what those expectation mean. The logical structure (i.e. the Borel sets) will be the same regardless.

TODO: add a footnote somewhere?


--------------------


In closing, we were very pleasantly surprised by the quality of both
reports and we thank both Referees for the time they spent in
evaluating our paper.

Sincerely,

The Authors

\end{document}