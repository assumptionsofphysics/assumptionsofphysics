\documentclass[10pt,twocolumn, nofootinbib]{revtex4-2}
%\documentclass[aps,pra,10pt,twocolumn,floatfix,nofootinbib]{revtex4-1}
%\documentclass[10pt,twocolumn,letterpaper]{article}

\usepackage{amsmath}
\usepackage{mathrsfs}
\usepackage{amsfonts}

\usepackage{graphicx}
\usepackage{hyperref}
\hypersetup{
	colorlinks=true,
	citecolor=blue,
	urlcolor=blue,
	linkcolor=blue
}
\urlstyle{same}
\frenchspacing


\begin{document}

\title{A physical foundation of Differential Topology}
\author{Gabriele Carcassi, Christine A. Aidala}
\affiliation{Physics Department, University of Michigan, Ann Arbor, MI 48109}

\date{\today}


\begin{abstract}

\end{abstract}

\maketitle

\section{Introduction}

\section{Current problems}

Before proposing an alternative set of core definitions for differential topology, we should discuss what are the problems with the current ones. As with any choice, there is a certain element of personal preference, which we acknowledge. However, given that physics and mathematics have different goals, it should not be too surprising that they have different requirements. In particular, in physics formal objects have to model the real world, therefore definitions that map better to physical entities and make explicit the idealizations assumed by the model are objectively preferable. Such tools foster a mathematical intuition connected to the physical intuition, and considerably lowers the cognitive load. Let us explore, then, some areas in which the current tools are less than optimal.

In differential topology, one defines local objects (e.g. vectors and forms) and then defines finite quantities associated to regions through integration. For example, one defines volume forms and then defines its integral over a region. Physics proceeds the opposite way. Experimentally, we do not measure directly a mass density, but rather we measure mass and volume of a finite region and take the ratio. We then imagine to take smaller and smaller regions, and the mass density is ratio in the limit.\footnote{The Radon–Nikodym derivative, for example, follows this approach.} Ideally, our definitions should mirror this process to which there are at least to major advantages. First, it is a lot easier to build physical intuition on the finite objects than on the local ones. Second, we can understand more clearly what physical requirements and idealizations are embedded in the definitions. For example, one basic assumption to define the mass density is that if we partition the volume, the total mass is the sum of the one found in each part. Given the equivalence of mass and energy, this is strictly true only if there is no interaction energy between the parts. Exactly what happens is those cases, then, is not clear, and starting directly with local objects may hide actual conceptual problems.

Differentiability of a manifold is defined by requiring that the coordinate transformation among all charts are differentiable. All other notions, such as vectors as directional derivations or differentials as linear map, are therefore built on top. Therefore, we conceptually have two layers of definitions, one for ``analytical'' differentiability and one for ``topological'' differentiability. We would like to have a single unified notion that work directly at both levels. Ideally, the same definition should be generalizable to the infinite dimensional case, so it could be directly applied to field theories.

While units and physical dimensions play a crucial role in physics, these ignored by the mathematics with important consequences. First of all, most people, physicists and mathematicians alike, miss that transformation rules of tensor components (i.e. covariance and contravariance) are primarily about units. That is, some quantities are expressed in base units (e.g. position and velocity) and others in derived units (e.g. velocity), therefore a change of definition in the first induces a change in the others. Since mathematics does not explicitly formalize units, it can do two things: either just define objects based on their transformation rules or look for objects that, by definition, follow the same transformation rules.\footnote{It seems that,in the mathematical world, the latter approach has been gaining favor with respect to the second.} To give an example of why this is problematic, in differential geometry one talks about tangent and cotangent bundle. Roughly speaking, the first pairs a point with a vector and the second pairs a point with a one form. It is often said that Lagrangian mechanics works on the tangent bundle, while Hamiltonian mechanics work on the cotangent bundle since one identifies velocity with a vector and momentum with a one form. Conceptually, however, the space in both cases represents all possible states of point particles. It is a manifold and the state is identified by a set of six variables. The fact that we choose position/velocity or position/momentum does not change the nature of the state space. Moreover, nothing prevents us to use units of velocity that are different from the one derived from the units of position.\footnote{This is actually often done for polar and cylindrical coordinates. The basis induced by the coordinates is not orthonormal, and is often replaced by one that is. It is not rare for people to not be aware of this and getting confused.} Therefore the notion of tangent/cotangent bundle is there to capture unit relationships between state variables. We can now ask: are there similar relationships that we may want to capture? If we have multiple particles, we will typically want to express their position in the same units. If we have the state space for $N$ particles, then, the transformation rules would be more restrictive than those of a tangent or cotangent bundle.

The issue of units is even more crucial, as it breaks the notion of tangent space as it is typically defined in differential topology. There are two standard ways to define the tangent space: either as a variation of the point under a change of coordinates (i.e. the basis becomes $e_i = \partial_i P$) or as a directional derivative of a scalar function (i.e. the basis becomes $e_i = \partial_i$). Both use the notion of a derivative, which requires a parameter that, ultimately, defines the speed of the change. Physically, that parameters will have a unit. This presents two problems. First, the unit is an extra choice that cannot be uniquely defined by the choice of manifold. Second, different coordinates can be expressed as different units and, since we can only sum quantities with the same units, we would have a linear space for each choice of units/physical dimension. For example, suppose $M$ is a two dimensional manifold, and suppose we use polar coordinates $[r, \theta]$ with $r$ is expressed in meters and $\theta$ is expressed in radians. Let the directional derivatives $\partial_r$ and $\partial_\theta$ represent a basis of the tangent space. The former has units of inverse meters while the latter has units of inverse radians. The sum $\partial_r + \partial_\theta$ is ill-defined. 


\section{Differentials as sequences}

\section{Differentiability as reducibility}

\section{Tensors as maps between differentials}

\section{Reducible functionals and forms}

\bibliography{bibliography}


\end{document}