\documentclass[letterpaper]{article}

% Math packages
\usepackage{amsthm}
\usepackage{amsmath}
\usepackage{amssymb}

% For more complicated table formatting
\usepackage{multirow}
\usepackage{tabularx}
\newcolumntype{L}[1]{>{\raggedright\let\newline\\\arraybackslash\hspace{0pt}}m{#1}}
\newcolumntype{C}[1]{>{\centering\let\newline\\\arraybackslash\hspace{0pt}}m{#1}}
\newcolumntype{R}[1]{>{\raggedleft\let\newline\\\arraybackslash\hspace{0pt}}m{#1}}

% Remove line spaces between items of enumerate and itemize
\usepackage{enumitem}
\setlist{noitemsep}

% Adds double bracket symbols
\usepackage{stmaryrd}

% Allows to create negation symbols
\usepackage{MnSymbol}

% General math symbols
\DeclareMathOperator{\Id}{Id} % Identity function

% Theorem styles

%\renewcommand\thesubsection{\thesection.\Alph{subsection}}
%\renewcommand{\theequation}{\thechapter.\arabic{equation}}

\newtheorem{assump}{Assumption}
\renewcommand*{\theassump}{\Roman{assump}}

\newtheorem{axiom}[equation]{Axiom}
\newtheorem{defn}[equation]{Definition}
\newtheorem{prop}[equation]{Proposition}
\newtheorem{coro}[equation]{Corollary}
\newtheorem{thrm}[equation]{Theorem}

%\theoremstyle{definition}

\newenvironment{remark}{\emph{Remark}.}{}
\newenvironment{rationale}{\emph{Rationale}.}{\qed}
\newenvironment{justification}{\emph{Justification}.}{\qed}
\renewenvironment{proof}{\emph{Proof}.}{\qed}


% LOGIC symbols
% -------------

% Boolean symbols and algebra
\def\Bool{\mathbb{B}}
\def\TRUE{\textsc{true}}
\def\FALSE{\textsc{false}}
\def\AND{\wedge}
\def\bigAND{\bigwedge}
\def\OR{\vee}
\def\bigOR{\bigvee}
\def\NOT{\neg}

% Logical context and related symbols
\def\logCtx{\mathcal{S}}
\def\vstmtSet{\mathcal{S}_\textsf{v}}
\def\dstmtSet{\mathcal{S}_\textsf{d}}
\newcommand{\pAss}[1][\mathcal{S}] {\mathcal{A}_{#1}}
\DeclareMathOperator{\truth}{truth}

% Experimental test symbols
\newcommand{\exptSet}{\mathcal{E}}
\newcommand{\expt}[1][e] {\mathsf{#1}}
\DeclareMathOperator{\result}{result}
\def\SUCCESS{\textsc{success}}
\def\FAILURE{\textsc{failure}}
\def\UNDEF{\textsc{undefined}}

% Statements
\def\tautology{\top} % Tautology
\def\contradiction{\bot} % Contradiction
\newcommand{\stmt}[1][s] {\mathsf{#1}} % Statement
\newcommand{\tstmt}[1][s] {\bar{\mathsf{#1}}} % Theoretical statement

% Relationships between statements
\def\comp{\doublefrown} % Compatibility
\def\ncomp{\ndoublefrown} 
\def\narrower{\preccurlyeq} % Narrowness
\def\nnarrower{\npreccurlyeq}
\def\snarrower{\prec}
\def\nsnarrower{\nprec}
\def\broader{\succcurlyeq} % Broadness
\def\nbroader{\nsucccurlyeq}
\def\sbroader{\succ}
\def\nsbroader{\nsucc}
\def\indep{\upmodels} % Independent
\def\nindep{\nupmodels}

% Experimental domains and related symbols
\newcommand{\edomain}[1][D] {\mathcal{#1}} % Experimental domain
\newcommand{\tdomain}[1][D] {\bar{\mathcal{#1}}} % Theoretical domain
\newcommand{\basis}[1][B] {\mathcal{#1}} % Basis
\newcommand{\resPoss}[1][x] {\mathring{#1}} % Residual possibility
\newcommand{\estPoss}[1][x] {\dot{#1}} % Established possibility

% Formatting for experimental relationships
\newcommand{\erel}[1][r] {#1}

% Formatting for sentence statements
\newcommand{\statement}[1] {\emph{``#1"}}

% Formatting for reference
\newcommand{\refStmt}[1][r]{\textbf{#1}}

\DeclareMathOperator{\ver}{ver}
\DeclareMathOperator{\fal}{fal}
\DeclareMathOperator{\und}{und}

\DeclareMathOperator{\interior}{int}
\DeclareMathOperator{\exterior}{ext}

% Level of detail
\def\eqgran{\doteq}
\def\finer{\leqdot}
\def\nfiner{\nleqdot}
\def\coarser{\geqdot}
\def\sfiner{\lessdot}
\def\scoarser{\gtrdot}


\begin{document}

\title{TBD}
\author{Gabriele Carcassi, Christine A. Aidala \\ University of Michigan}

\date{\today}

\maketitle

\begin{abstract}
	TBD
\end{abstract}


\section{Introduction}

------------------
Cite other axiomatic works.

Single state space to capture:
* descriptions at all levels
* descriptions at different times
* correlations between descriptions at equal time (equations of state)
* correlations between descriptions at different times (evolution laws)
* different correlations at different description
* granularity/precision of such descriptions

We can't have:
* probability as a native concept
------------------

In the intro, describe only the discrete case as it simpler to understand and already contains all the physics. The continuous case adds the complication of 

Evolution is a temporal sequence of states. Process is a set of possible evolutions. What we want to study is how a statement about a state at a particular time selects a particular set of possible evolutions and therefore selects a set of possible states at a previous or past time.

Define determinism and reversibility in terms of what evolutions exist. Note that a dynamical system is a deterministic system.

Note how the number of states does not always match the number evolutions it is compatible with. Note that under deterministic evolution the number of compatible evolutions can only increase. For dynamical systems, then, the number of evolutions associated to a state can only increase in time and it is stationary for equilibria.

If we compose independent systems, the number of possible states and trajectories is the product. If we take the logarithm, then those numbers are additive. Conceptually the entropy is the logarithm of the number of evolutions compatible with a given state. This article shows how this simpler and more general concept leads, in special cases, to the Boltzman entropy and the Shannon entropy. We will also see how this same concepts is applicable more in general, including the case of non-Hamiltonian systems.

\section{Main points}
\begin{enumerate}
	\item Entropy assigns a size to points. Shannon entropy works on a distribution where each point has the same entropy.
	\item Measure theory differentiate between points and sets. This does not work when combining different level of granularity.
\end{enumerate}

\section{Main section}
\begin{enumerate}
	\item \emph{Experimentally testable descriptions and their logic relationship.} Here the goal is to formalize the basic structure required by a set of experimentally verifiable assertions. Some descriptions are more specific than others, some are compatible with each other, and so on. An evolution is the most specific description, one that tells us what happens at all times.
	\item \emph{Group descriptions by time.} Here the goal is to group description at ``instants of time'', which are taken to be small intervals. A snapshot is the most specific description at an instant, on that tells us everything there is to know about the system at the given time. Determinism and reversibility are defined at this level (i.e. deterministic process is one where the snapshot at one time tells us the snapshots at future times).
	\item \emph{Granularity of the description and counting evolutions.} Here the goal is to quantify how much a description is more refined than another. This can be done by quantifying the possible evolutions that are compatible with each (i.e. the more specific description will be compatible with fewer evolutions). The evolution entropy quantifies the difference in granularity in terms of how many questions (i.e. bits) are needed to go from one description to another. The evolution entropy cannot decrease for a deterministic process and is stationary if the process is also reversible.
	\item Define deterministic evolution: can predict future state. Past state narrower. Number of evolutions increases. Define reversible. Future state narrower. Number of evolution decreases. Det/rev remains the same. How do we quantify that?
	\item Define way to compare granularity. Not a single measure, not everything comparable or finitely comparable. Define probability in terms of measure.
	\item States of the same system must be finitely comparable. Can put a measure on states themselves, which is a count of evolutions. Examples to show that: the measure and state space, in general, change over time. State spaces are stable and can be assigned a measure only at equilibrium.
	\item Define independent systems as those where the state of one does not constrain the state of the other. The measure must be the product.
	
\end{enumerate}

\begin{enumerate}
	\item A thermodynamic system is one that undergoes deterministic evolution with equilibria. There are variables that  
	\item Extensive quantities are somewhat postulated at this point. So is the existence of energy.
	\item Second law come from deterministic system.  Define entropy as the log of the measure, so that it is additive under independent systems. We have: entropy always increases under deterministic evolution; entropy is maximal at equilibrium; entropy is additive for independent system. Entropy is a function of the equilibrium, which is identified by state variables.
	\item Third law comes from states of the same system being finitely comparable. Lowest entropy is "one evolution" in terms of system granularity.
	
\end{enumerate}


\section{Evolutions, states and levels of description}

The overall goal is to describe a generic system as it evolves in time by defining a minimal set of concepts that such a setup must define. We want to find notions that can be applied, for example, to thermodyamic and dynamical systems alike. Then we want to derive general results based on such definitions. Here we provide an overview of the main features, leaving the formal definitions in the (yet to be written) appendix.

\subsection{Descriptions and logical relationships}

Our starting point is the \textbf{process domain} $\tdomain[P]$, which is the set that contains all possible descriptions of the system at all levels of detail at all possible times. Depending on the process, it will contain statements like \statement{the average volume between 1 and 2 seconds is between 3 and 4 liters} or \statement{the trajectory of the particle is $y=10 \, m - t^2 \cdot 9.80665 \, m/s^2 $}. It does not matter at this point what the statements themselves are, except they must be, at least in line of principle, defined from experimentally well-defined starting points. Mathematically, $\tdomain[P]$ is a $\sigma$-complete Boolean algebra\footnote{A set of statements that is closed under negation (NOT), countable disjunction (OR) and countable conjunction (AND).} and it is the closure of the set of experimentally verifiable statements $\edomain[P] \subseteq \tdomain[P]$.\footnote{The set $\edomain[P]$ will form a Heyting algebra which, to be experimentally reachable, must allow a countable basis.}

As $\tdomain[P]$ contains different levels of description, one statement may give a more specific description than another. For example, we say that \statement{the horizontal position is between 2.5 and 3 meters} is \textbf{narrower} than \statement{the horizontal position is between 2 and 3.5 meters}. Note that whenever the first statement is true, the second one must be true as well. Given two statements $\stmt_1, \stmt_2 \in \tdomain[P]$, we formally capture this relationship by noting $\stmt_1 \narrower \stmt_2$. Narrowness can describe relationships at different scales, on different quantities (e.g. \statement{the horizontal position is between 2.5 and 3 meters and the vertical position is between 1 and 1.5 meters} $\narrower$ \statement{the horizontal position is between 2.5 and 3 meters}), constraints between variables at the same time (e.g. \statement{the temperature of the water in the glass is 3.98 C} $\narrower$ \statement{the density of the water in the glass is 1 $g/cm^3$}) or at different times (e.g. \statement{at time 0 s the position is 1 m and the velocity is 1 m/s} $\narrower$ \statement{at time 1 s the position is 2 m}). It is a general tool to capture relationships within the theory which intuitively can also be thought as implication.\footnote{Technically, implication in classical logic is a truth function as classical logic does not incorporate the idea of different ``possible cases''. What we have is more similar in spirit to semantic consequence in modal logic, though modal logic brings in a lot of undesirable elements we do not want.}

Within our process domain $\tdomain[P]$ we can define the set $E$ of the narrowest possible statements. Its elements give a complete description of our system at all times and therefore we call them \textbf{evolutions} while we call $E \subset \tdomain[P]$ the set of all possible evolutions.

Another logical relationship we want to capture is the idea of whether two statements can be true at the same time. For example, we want to say that \statement{the horizontal position is between 2.5 and 3.5 meters} is \textbf{compatible} with \statement{the horizontal position is between 2 and 3 meters}. In general, given two statements $\stmt_1, \stmt_2 \in \tdomain[P]$ we note $\stmt_1 \comp \stmt_2$ if it is possible for them to be both true. As with narrowness, compatibility can be in terms of different ranges of the same quantities, of different quantities and of constraints at equal or different times.

Every statement $\stmt \in \tdomain[P]$ can now be characterized by the set $A(\stmt) \subseteq E$ of all evolutions compatible with $\stmt$, that is all the evolutions for which $\stmt$ will be true. Mathematically,  $A(\edomain[P])$ is a topology over $E$ while $A(\tdomain[P])$ forms a $\sigma$-algebra over $E$, the Borel algebra of $A(\edomain[P])$. Narrowness and compatibility become set realtionships in the $\sigma$-algebra: $\stmt_1 \narrower \stmt_2$ if and only if $A(\stmt_1) \subseteq A(\stmt_2)$ (i.e.~all trajectories for which $\stmt_1$ is true are such that $\stmt_2$ is also true) and $\stmt_1 \comp \stmt_2$ if and only if $A(\stmt_1) \cap A(\stmt_2) \neq \emptyset$ (i.e.~two descriptions are compatible if there is an evolution that is compatbile with both).

\begin{table}[h!]
	\centering
\begin{tabular}[h]{|c|c|p{6cm}|}
	\hline 
	Symbol & Name & Meaning \\ 
	\hline 
	$\tdomain[P]$ & Process domain & The set of all possible descriptions at all times and at all levels of granularity \\ 
	\hline 
	$E \subset \tdomain[P]$ & Possible evolutions & The set of descriptions that give a full account of the process; they correspond to the narrowest statements within $\tdomain[P]$ \\ 
	\hline 
	$A(\stmt)$ & Compatible evolutions & The set of evolutions that are compatible with the statement $\stmt$ \\ 
	\hline 
	$\narrower$ & narrower & A statement is narrower than another $\stmt_1 \narrower \stmt_2$ if the second one is true whenever the first one is \\ 
	\hline 
	$\comp$ & compatible & A statement is compatible with another $\stmt_1 \comp \stmt_2$ if they can be true at the same time \\ 
	\hline 
\end{tabular} 
	\caption{Process descriptions and their logical relationships}
	\label{table:logic}
\end{table}


\subsection{Time and snapshots}

As the process extends over time, we want to organize the statements in our process domain into partial domains, one for each moment in time. Given a \textbf{time parameter} $t \in T \subseteq \mathbb{R}$, we can imagine to carve a subdomain $\tdomain_t$ that would correspond to all the statements about the system one can make at the given time. Note that $\tdomain_t$ is a $\sigma$-complete Boolean sub-algebra of $\tdomain[P]$ since any logical operation on statements at fixed $t$ is also another statement at fixed $t$. The set $X_t \subset \tdomain_t$ of the narrowest statements will give the most precise description at that time, we call these the  possible \textbf{snapshots} of our system at that time.

We should note that, technically, the statements are not defined at an infinitesimal moment $t$, rather within a finite interval $[t, t + \Delta t]$. This represents the time-scale at which the process will be described, meaning the description within $\tdomain_t$ should not be sensitive to what happens at faster scales. The quantities used to represent states at time $t$, then, should really be thought as averages within $\Delta t$. In the same way, $\tdomain_t$ will in general represent the system at the chosen level of description, not the narrowest level possible. Statements about the environment or at a finer level of description (i.e. the positions of all molecules for a gas) will not be included.

We should also note that the set of possible snapshots $X_t$ may not be the same at all times and may not correspond to the state of the system. First of all $X_t$ will only include those configurations that are compatible with least by one possible evolution (i.e. $A(x) \neq \emptyset$ for all $x \in X_t$), which may change in time. For example, the set of possible configurations for a system under dissaptive processes will shrink as they converge to equilibrium. In quantum mechanics, the possible configurations after a measurement are restricted to the eigenestates, and a different choice of measurement will lead to a different process with different possible evolutions. Secondly, the best possible description may be broader than the full state of the system (e.g. when external interference forces us to give only a statistical account within our $\Delta t$) or narrower (e.g. when there are known correlations to the other degrees of freedom of the environment or of the system itself). The idea of a state space of the system, independent of the process, is something we need to construct. Finding the requirements needed for that construction, the hidden assumptions, is precisely the issue we need to address.

We say a process is \textbf{deterministic} over the time domains $\{X_t\}_{t \in T}$ if given the snapshot at one time we can always predict the snapshot at all future times. Formally, let $t_1, t_2 \in T$ such that $t_1 \leq t_2$, for any $x_1 \in X_{t_1}$ we can find an $x_2 \in X_{t_2}$ such that $x_1 \narrower x_2$.

Intuitively, all the evolutions that pass through $x_1$ will also pass through $x_2$. If $x(t)$ is the snapshot trajectory in time, we have $A(x(t)) \subseteq A(x(t + \Delta t))$: the set of compatible evolutions must stay the same or become larger.

Conversely, we say a process is \textbf{reversible} over the time domains $\{X_t\}_{t \in T}$ if given the snapshot at one time we can always reconstruct the snapshot at all past times. Formally, let $t_1, t_2 \in T$ such that $t_1 \leq t_2$, for any $x_2 \in X_{t_2}$ we can find a $x_1 \in X_{t_1}$ such that $x_2 \narrower x_1$.

In this case, all the evolutions that pass through $x_2$ must have passed through $x_1$. If $x(t)$ is the snapshot trajectory in time, we have $A(x(t)) \subseteq A(x(t - \Delta t))$: the set of compatible evolutions must stay the same or become smaller.

A deterministic and reversible process, then, will mean $x_1$ and $x_2$ are equivalent and that  $A(x(t)) = A(x(t + \Delta t))$: the set of evolutions remains the same. The fact that past and future snapshots are equivalent $x(t) \equiv x(t + \Delta t)$ does not mean they are the same statement in terms of the description at each time: it means there is an if-and-only-if relationship between the descriptions at different times. For example, \statement{at time $t$ the position is $q$ and the velocity is $v$} if and only if \statement{at time $t + \Delta t$ the position is $q + v \Delta t$ and the velocity is $v$}. The differential equation $\dot{v} = 0$ is a short hand for all such relationships.

If we limit ourselves to deterministic processes, with these few definitions, we have already found something that ``grows bigger'' under irreversible evolution while it remains the ``the same size'' under reversible evolution.

TODO: picture with deterministic evolution merging "streams" of evolutions, and detrev evolutions not merging

\begin{table}[h!]
	\centering
	\begin{tabular}[h]{|c|c|p{6cm}|}
		\hline 
		Symbol & Name & Meaning \\ 
		\hline 
		$t \in T$ & Time parameter & Time is treated as a real valued parameter within the range $T \subseteq \mathbb{R}$ \\ 
		\hline 
		$\tdomain_t \subset \tdomain[P]$ & Time domain & The set of descriptions at a given time \\ 
		\hline 
		$X_t \subset \tdomain_t$ & Possible states & The set of descriptions that give a full account at time $t$; they correspond to the narrowest statements within $\tdomain_t$ \\ 
		\hline 
		$x(t) \narrower x(t + \Delta t)$ & Determinism & Given the state at one time we can predict the state at future times \\ 
		\hline 
		$x(t) \narrower x(t - \Delta t)$ & Reversibility & Given the state at one time we can reconstruct the state at past times \\ 
		\hline 
	\end{tabular} 
	\caption{Time domain, states and determinism}
	\label{table:states}
\end{table}

\subsection{Granularity and quantifying evolutions}

Next step is to quantify the size of the sets of evolutions, specifically those that correspond to each statement. Since statements at a finer level of description will put a greater constrain to the set of evolutions, quantifying the size of $A(\stmt)$ means quantifying the granularity of the description of $\stmt$. The statement \statement{at time 0 s the particle is between 0 m and 1 m} is doubly more specific than \statement{at time 0 s the particle is between 0 m and 2 m} precisely because it correspond to half the evolutions of the second statement.

The naive approach would be to assign a measure over $A(\tdomain[P])$: since this is is a $\sigma$-algebra over $E$, it would seem we are done. Unfortunately it is not that simple: a single measure can only compare objects of the same dimensionality. If we imagine all evolutions to be points on an $n$ dimensional manifold, if the measure gives finite values for $k$-dimensional regions it will necessarily give measure zero for all regions with lesser dimensionality and infinite measure for those with greater dimensionality. Coarse grained and subsystem descriptions will have different dimensionality as they describe different numbers of degrees of freedom, therefore we need a way to quantify objects at all levels.\footnote{Typically one uses geometrical structure for his purpose. In phase space, the symplectic form allows us to define areas of two-dimensional surfaces, and use those to define volumes or even-dimensional areas. Note that it does not quantify the size of the points, which are implicitly assumed to be of equal size. This is a special case and cannot be the starting point.}

Therefore our starting point will be the ability to compare two statements and decide which one gives a refined description of the process. Formally, we have partial order $\finer$ on $\tdomain[P]$ which, given two statements, it tells us whether the description of one is \textbf{finer}, more refined, than the other. Two statements are \textbf{equigranualar}, noted $\stmt_1 \eqgran \stmt_2$, if they provide the same level of description. For example, we can say that \statement{the trajectory of the particle is $y=t \cdot 1 \, m/s$} $\eqgran$ \statement{the trajectory of the particle is $y=t \cdot 2 \, m/s$} and that \statement{the position at time 0 sec is between 0 and 1 m} $\eqgran$ \statement{the position at time 0 sec is between 1 and 2 m}, even though the second pair is infinitely less discerning than the first. Note that if one statement is narrower than the other (i.e. $\stmt_1 \narrower \stmt_2$) then it is also finer than the other (i.e. $\stmt_1 \finer \stmt_2$) while the converse is not necessarily true. 


We say two statements are \textbf{comparable} if one is finer than the other. The order is partial because not all statements are comparable to each other. For example, consider the pressure/volume state space for an ideal gas. Comparing \statement{the pressure is 1 kPa and the volume is between 1 and 2 liters} and \statement{the pressure is between 1 and 2 kPa and the volume is 1 liter} would mean saying where one unit liter is bigger or smaller than 1 kPa.\footnote{In phase space, for example, we cannot in general compare $q^i$ and $p_i$, yet we can always compare areas of phase space.}

To recover measures, we pick a statement $\stmt[u] \in \tdomain[P]$ and construct a \textbf{measure} $\mu_{\stmt[u]}$ such that $\mu_{\stmt[u]}(\stmt[u]) = 1$ and $\stmt_1 \finer \stmt_2$ will mean $\mu_{\stmt[u]}(\stmt_1) \leq \mu_{\stmt[u]}(\stmt_2)$. That is, we can quantity how more or less precise a statement is compared to a fixed statement taken as a unit. If two statements are such that the measure of one with respect to the other is finite and non-zero, we say they are \textbf{finitely comparable}. Furthermore, since all evolutions give a complete description of the whole process, we will assume they are all equigranular. Therefore if we pick $\stmt[u] \in E$, then $\mu_{\stmt[u]}(\stmt)=\#(A(\stmt))$ would be the counting measure of evolutions compatible with $\stmt$.\footnote{We still needs to understand the necessary and sufficient conditions under which such measures would exist and be unique. For the present work, we assume it can be done. We know that it is a necessary condition that all evolution must be equigranular. It remains an open question is whether it is sufficient.}

We can also define the \textbf{evolution entropy} with respect to $\stmt[u]$ the quantity $S_{\stmt[u]}(\stmt) = \log \mu_{\stmt[u]}(\stmt)$. This can be thought as the number of bits, yes/no questions, that separate the level of description of $\stmt[u]$ from the one provided by $\stmt$. Connecting the definitions from above, under a deterministic process we have $S_{\stmt[u]}(x(t)) \leq S_{\stmt[u]}(x(t + \Delta t))$ since $x(t)\narrower x(t+\Delta t)$ means $x(t) \finer x(t+\Delta t)$ and $\mu_{\stmt[u]}(x(t)) \leq \mu_{\stmt[u]}(x(t+\Delta t))$. If the process is determistic and reversible, we have $S_{\stmt[u]}(x(t)) = S_{\stmt[u]}(x(t + \Delta t))$. In other words, \textbf{evolution entropy cannot decrease for a deterministic process and is stationary if the process is reversible as well}.

Quantifying evolutions allows us to define probability as well. If we consider all the evolutions compatible with \statement{the coin is fair}, we must have that half of them are compatible with \statement{the coin is head} while the other are compatible with \statement{the coin is tails}.  Therefore we have $P(\stmt_2 | \stmt_1) = \mu_{\stmt_1}(\stmt_1 \AND \stmt_2)$ where $\stmt_1 \AND \stmt_2$ is the logical conjunction (AND) between the two statements. In other words, the probability quantifies the ratio of evolutions compatible with $\stmt_1$ that are also compatible with $\stmt_2$.

%Note that this approach resolves a problem that causes friction between quantum mechanics and standard probability theory. Given two incompatible states $\psi$ and $\phi$, that is $0 < | \left<\psi | \phi \right> |^2 < 1$, we would like to say that that the probability of one given the other is $P(\psi | \phi) = | \left<\psi | \phi \right> |^2$.

\begin{table}[h!]
	\centering
	\begin{tabular}[h]{|c|c|p{6cm}|}
		\hline 
		Symbol & Name & Meaning \\ 
		\hline 
		$\finer$ & finer & A statement is finer than another $\stmt_1 \finer \stmt_2$ if it provides a more refined description than the second \\ 
		\hline 
		$\eqgran$ & equigranular & A statement is equigranular to another $\stmt_1 \eqgran \stmt_2$ if they provide the same level of detail \\ 
		\hline 
		$\mu_{\stmt[u]}(\stmt)$ & Measure & Quantifies the precision of a statement $\stmt$ in terms of a reference unit statement $\stmt[u]$; it quantifies how many evolutions are compatible with $\stmt$ compared to the evolutions compatible with $\stmt[u]$  \\ 
		\hline 
		$S_{\stmt[u]}(\stmt)$ & Evolution entropy & The number of bits that separate the level of description of $\stmt$ from the one provided by the unit statement $\stmt[u]$; it quantifies the number of questions needed to go from the level of description of $\stmt$ to the level of description of $\stmt[u]$  \\ 
		\hline 
		$P(\stmt_2 | \stmt_1)$ & Probability & Quantifies quantifies the ratio of evolutions compatible with $\stmt_1$ that are also compatible with $\stmt_2$  \\ 
		\hline 
		 & Comparable & Two statements are comparable if one is finer than the other \\ 
		\hline 
		 & Finitely comparable & Two statements are finitely if the measure of one in terms of the other is finite and non-zero \\ 
		\hline 
	\end{tabular} 
	\caption{Granularity, measures and probability}
	\label{table:states}
\end{table}

\subsection{Systems, equilibria and state spaces}

So far we have made no guarantees as to whether each $X_t$ contains descriptions that refer to the same physical object. So we need to understand what is needed to be able to define a system and its state space. Intuitively, the state space should provide us with all the descriptions of the system at any point in time for any process. We are therefore asked to create an abstraction that stitches together all possible description, all possible snapshots, about the system and only about the system. There are therefore a few issues we need to consider.

First of all, when we say ``all process'' we can't really mean absolutely all of them, only the ones that somehow do not disrupt the system. For example, we can study the motion of a balloon on earth at 20 Celsius. This would not be possible on the surface of the sun as our balloon would not exist. The process has to be such that the system is recognizable to be the same, our descriptions must still make sense. We can imagine a space $X$ of all statement fragments like \statement{the position of the ball is between 2 and 3 meters} and a surjective map $\iota : X \times T \twoheadrightarrow X_t$ that adds \statement{at time t} for each possible time. The map $\iota$ must be a surjection that preserves the logical structure (i.e. logical operations, narrowness, compatibility, ...) so that all descriptions are only about the system and the system determines the basic logical relationships that are valid at all times (i.e. \statement{the position of the ball is between 2 and 3 meters} $\narrower$ \statement{the position of the ball is between 0 and 300 meters} at every $t$).

Note, though, that $\iota( \cdot, t_0)$ in general is not an isomorphism (i.e. a bijection that preserves the logical structure) for two reasons. The first is that not all states will be available at every moment in every process. As we said before, if the process is dissipative the set of possible states will become smaller. In that case, $\iota$ will map all the inaccessible states to $\contradiction$, as they will be impossible. Therefore $\iota$ is not in general a bijection. The second reason is that the statements in $X$ represent only descriptions about the system itself and nothing else. In a particular process, however, knowing the state of one system might tell us something about other system as well. The presence of correlations/coupling may mean that the precise knowledge of the degrees of freedom of the system may allow us to tell something about the environment or the internal structure (i.e. microstate). Mathematically, $\iota$ may map to statements that are narrower than the original (e.g. we can have $\iota($\statement{position of A is 1 meter}$, t_0) \equiv$ \statement{at time $t_0$ the position of A is 1 meter and the position of B is -1 meter}).

States coincide with snapshots of the system only in the case that the system is uncorrelated with environment and it does constrain the internal dynamics. This is also the case when the statements are broadest, which means it the case when the process entropy is maximized. We therefore define the \textbf{state entropy} as the maximum of all possible evolution entropies one can find for the given description of the system in all possible processes at all times.

In this context, entropy maximization is a necessary step to be able to define a system. If we are not able to find a process that, at least in some cases with good approximation, makes the evolution of the system independent from the environment and from the internal dynamics, we cannot talk about a system to begin with. In other words, if we are not able to isolate the system so that it can be studied independently, we are not going to be able to assign it a state, we are not able to construct a state space.

It is then useful to start studying pro


The statements present in our state space $X$, then, represent theIf $\iota$ can map to statements of different granularity, we cannot assign a unique evolution entropy to the statements of $X$. Yet, 

 In our framework, we need some type of homomorphism between $\tdomain_t$ and $\tdomain_{t + \Delta t}$ that preserves narrowness and fineness, that tells us the system is the same.



So far we have only characterized how descriptions at different times can be related to each other. We now want to group descriptions that belong together, that describe a single system as it is subjected to different processes. In other words: we want to start talking about \textit{states of the system}, not just descriptions or snapshots. In the simplest form, we prepare the system in an initial state, let it evolve and look at the final state. We want to characterize the state space of all possible initial and final states under all process and characterize the process. Our framework already tells us that if the process is deterministic, the process can be characterized as a map $x(t+\Delta t) = f(x(t))$. If the process is non-deterministic, the evolutions for each state may split, and the probability $P(x(t + \Delta t) | x(t))$ can be used to characterize the fraction of evolutions that start in $x(t)$ and end in $x(t+\Delta t)$. Yet, it also tells us there are a few things that we have to worry about.

Second we want to make sure that the state space of the system really tells us only about the system itself. Within a specific process, for example, saying \statement{at time $t$, the velocity of object $A$ is $v$} may also mean that \statement{at time $t$, the velocity of object $B$ is $v$} because they are tied together. We say two systems are \textbf{independent} at a particular time if the description of one system does not constrain the description of the other. In terms of measure $\mu_{\stmt[u]_a \AND \stmt[u]_b}(x_a \AND x_b) = \mu_{\stmt[u]_a}(x_a) \mu_{\stmt[u]_b}(x_b)$ must factorize. If we want our state space to describe only the system at hand, then we are assuming that states are defined on the assumption that the system is independent at least from all other systems with the chosen processes.

On the other side, a state of a system is never really the description of the whole system, but of the part that is relevant to the process at hand. If we study the motion of the balloon, position and velocity may be enough; if we study it as a thermodynamic system, pressure and volume may be enough. The idea is that the state captures the system at a particular level of granularity and the process does not depend, is not sensitive, to the finer dynamics. We say a system is in \textbf{equilibrium} if the state tells us nothing about the 

We stress that the independence cannot be, in general, independent throughout the whole process or there cannot be interactions between the systems. By the same token, the finer dynamics may play a part throughout the process. The constraint is only on the initial and final states, where the state space is being constructed. We also stress that these constraints make the state definition contingent upon the type of processes one studies. If we want to study a process where two systems are not independent, one that depends on the finer dynamics, our state definition will have to change to accommodate.






 In all processes we are considering, then, the system must be indAny other system would be independent If we found other systems that were always correlated For example, if  There can't be other systems that participate in the process whose state 



We should caution that 

The simplest setting is the follow

 into Our goal now is to construct state spaces for a system.


 Consequently any statement that corresponds to finitely many evolutions (e.g. \statement{the full trajectories is either $y=t \cdot 1 m/s$ or $y=-t \cdot 1 m/s$}) will be comparable to all other statements.

Also note that in our ideal gas example one of the quantities was determined with infinite precision. If we gave a range for all quantities, we would compare areas in that space, which we can always do. Statements with finite precision measurements are the only ones we can experimentally verify. So, we will require that all experimentally verifiable statements are comparable. We call \textbf{experimental domain} for the process the subset $\edomain[P] \subseteq \tdomain[P]$ of statements that can be, at least in line of principle, experimentally verified.\footnote{Yet, as we will see later, the precision of measurements will be quantified in terms of number of evolutions, not the range/uncertainty on the variables.}

While we leave the full mathematical characterization in the mathematical section, there are a few things of note. Verifiable statements will correspond, in terms of sets of $E$, to finite and infinite volumes. Requiring those to be comparable means that $E$ is a measurable space and in phase space, for example, the standard measure will be recovered. As the process domain corresponds to a $\sigma$-algebra on $E$, the experimental domain corresponds to a topology on $E$. Therefore notions of continuity can be recovered as well. The process domain is the closure of the experimental domain under standard Boolean operations. This means that all the process domain is fully specify by, at least in principle, experimentally verifiable statements. The space does not contain any assertion that cannot ultimately be defined in terms of experiments. Lastly, requiring that, given enough time, each statement will be verified means the experimental domain must be generated by countably many statements. This means the topology is second countable and one is limited to work with ``nice'' spaces (e.g. spaces of continuous functions, separable Hilbert spaces, ...) not because they are ``nice'', but because they are the only physically meaningful ones.

To sum up, we have two foundational object. The first object is the process domain $\tdomain[P]$ which is a Boolean algebra that contain all the possible description of our process. A subset $\edomain[P]$ contains those that are experimentally verifiable, which contain all the information in the full domain. The narrowest statements in $\tdomain[P]$ correspond to the evolutions $E$, that provide the full description at all times. The second object is the operator $\finer$ which gives us a way to compare two statements and decide which one gives a more refined description. Though two statements will not in general be comparable, all verifiable statements are comparable and all evolutions provide an equally refined description.

This framework imposes very few requirements which are of very general applicability. It does not matter what specific quantities are used, or whether the process is from physics, chemistry, control systems, ecology, economics, sociology, or some other realm we have yet to discover. In the end, if we have a process we need to model, our theory will need at least to provide those objects. This means that the results we find will have broad applicability as well.

\section{States}

As the process extends over time, it is natural to want to organize the statements in our process domain into partial domains, one for each moment in time. Given our time parameter $t \in T \subseteq \mathbb{R}$, we can imagine to carve a subdomain $\tdomain_t$ that would correspond to all the statements about the system one can make at the given time. Any logical operation on statements at fixed $t$ is also another statement at fixed $t$ therefore $\tdomain_t$ is a $\sigma$-complete Boolean sub-algebra of $\tdomain[P]$. The set $X_t \subset \tdomain_t$ of the narrowest statements will give the most precise description at that time: these are the possible states of our system.

We should note that, technically, the statements are not defined at an infinitesimal moment $t$, rather within a finite interval $[t, t + \Delta t]$. This represent the time-scale at which the process will be described, meaning the description within $\tdomain_t$ should not be sensitive to what happens at faster scales. The quantities used to represent states at time $t$, then, should really be thought as averages within $\Delta t$.

We should also note that there is nothing at this point requiring us that the set of descriptions $\tdomain_t$ and states $X_t$ will be the same at all times.

What we want to do now is relate states at different times. In particular we want to formally define the notion of deterministic process: one in which, given the state at a given time, we can always predict the state at all future times. This can be done through the use of the narrowness operator: $\stmt_{t_1} \narrower \stmt_{t_2}$ means that all evolutions compatible with $\stmt_{t_1}$ are also compatible with $\stmt_{t_2}$. In other words, if we find $\stmt_{t_1}$ to be true we will also find $\stmt_{t_2}$ to be true as well. Therefore a process is deterministic if, for all $t_1, t_2 \in T$, given $x_{t_1} \in X_{t_1}$ we can find $x_{t_2} \in X_{t_2}$ such that $x_{t_1} \narrower x_{t_2}$.



At each moment, we have a set of possible statements about the system corresponding to physically measurable properties. We call state the maximal description that can be given at a particular time. Given the state, we know all that can be said about the system in that moment. We define an evolution of the system as the maximal description of the system at all times. If we knew the exact evolution, we know all possible descriptions of the system at all times. Therefore picking an evolution is equivalent to picking a state at each time. A particular process will constrain how the system can evolve, it will constrain the possible evolutions. Therefore we formally define a process as a set of possible evolutions, the ones that are allowed by the process.

As each evolution gives the total description of the system at all times, we assume that all evolutions give the same level of description. We also assume we have a way to compare sets of possible evolutions and determine which is bigger and which is smaller. Mathematically, we have a measure defined on the set of possible evolutions. Each description, if verified experimentally, will narrow the set of possible evolutions to the ones that are compatible with said description. Therefore each description, and in particular each state, can be assigned a value that corresponds to the number of evolutions compatible with it. We call the logarithm of the number of evolution the entropy associated to the description.

We define a process to be deterministic if knowing the state at one time means we can predict with absolute certainty the state at a future time. If this is the case, all evolutions compatible with one state cannot split and must all pass through the same states at all times. During a deterministic process, then, the number of evolutions per state can only increase over time. The entropy associated to each state is a monotonically increasing function.

Conversely, we define a process to be reversible if knowing the state at one time means we can reconstruct with absolute certainty the state a previous time.\footnote{Note that this notion of reversibility is not identical to the one usually found in thermodynamics, which corresponds to the ``ability to undo the change''. We will later find that the notion of thermodynamic reversibility corresponds to the case where system and environment together undergo a deterministic and reversible process.} Reasoning along the same lines, during a reversible process the entropy associated to each state can only decrease over time, as evolutions cannot merge. A deterministic and reversible process will necessarily see the number of states, and therefore the entropy, remain the same.

In a deterministic process, a state will be an equilibrium if, roughly speaking, the evolutions have fully merged and they will all give the same future state. In other words, after the equilibria the evolution is effectively reversible. This means that, at equilibria, the entropy will have reached a maximum. Each equilibria will be associated with a bundle of evolutions which will define, at times before the equilibrium, the set of states that will merge. For each equilibria one can construct a constraint, an assertion that remains true for each equilibria. Therefore each equilibria can be seen as the maximizing the entropy under the constraints imposed by the process.

Finally, a process over a system can be broken into two processes over two systems if and only if they are independent. That is, if the evolution over one subsystem tells us nothing about the evolution of the other or, equivalently, the measure on the evolutions on the whole is the product of the measure on the subsystems. In those cases, the entropy of the system is equal to the sum of the entropy of the subsystems. However, note that if the systems are independent throughout the whole evolution, no coupling can be defined between the subsystem.

A useful scenario is when the systems can be considered independent at the beginning of the evolution, couple, and then return to be independent at equilibrium. The entropy can then be considered additive at the beginning and at the end of the evolution, but not in between. Equilibria on the whole will necessarily correspond to equilibria on each subsystem. Note how this conceptually corresponds to the scenarios studied in thermodynamics.


NOTE: measure is invariant on Hamiltonian evolution, but not under irreversible evolution. Can't have the same measure at all times.


\section{Math section}

Let $\tdomain$ be the set of all statements about the evolution of the system. This is a $\sigma$-complete Boolean algebra. Let $E \subset \tdomain$ the set of all evolutions, these are the narrowest statements. To each statement $\stmt \in \tdomain$ corresponds a set of evolutions $A(s)$ that are compatible with the statement.

Given two statements $s_1, s_2 \in \mathcal{D}$ we can say whether one is finer than the other $s_1 \finer s_2$. Given any statement $\stmt[u] \in \mathcal{D}$, we have a function $\mu_u : \mathcal{D} \to \mathbb{R}^+\cup \{+\infty \}$ such that:
\begin{enumerate}
	\item $\mu_u$ is a measure
	\item $\mu_u(u) = 1$
	\item $\mu_u(s_1) \leq \mu_u(s_2)$ if $s_1 \finer s_2$
\end{enumerate}
The value $\mu_u(\stmt)$ tells us how many trajectories are compatible with $\stmt$ in terms of how many trajectories are compatible with the unit statement $\stmt[u]$.

The conditional probability $P(\cdot | \cdot) : \tdomain \times \tdomain \to [0,1]$ is defined as $P(\stmt_1 | \stmt_2) = \mu_{\stmt_2}(\stmt_1 \AND \stmt_2)$ and represents the fraction of trajectories that are compatible with $\stmt_2$ that are also with $\stmt_1$.

We define $T \subseteq \mathbb{R}$ to be the range of the time parameter. We define $\tdomain_{t} \subseteq \tdomain$ as the set of statements defined at the time t. Note this does not mean they are statement of only what happens at an instant in time. They can be averages around. It's how we decompose the problem at the desired time resolution. The states at time $t$ are the possibilities of $\tdomain_t$. The process is deterministic over the chosen states if for each $x_0 \in \tdomain_{t_0}$ we can find an $x_1 \in \tdomain_{t_1}$ such that $x_0 \narrower x_1$.  The process is reversible over the chosen states if for each $x_1 \in \tdomain_{t_1}$ we can find an $x_0 \in \tdomain_{t_0}$ such that $x_1 \narrower x_0$.

Things I want to prove:

\begin{prop}
	Let $\tdomain[P]$ be a deterministic process over $\{X_t\}_{t \in T}$. Then $S(x(t))$ is a monotonically increasing function.
\end{prop}

\begin{prop}
	Let $\tdomain[P]$ be a reversible process over $\{X_t\}_{t \in T}$. Then $S(x(t))$ is a monotonically decreasing function.
\end{prop}

\begin{prop}
	Let $\tdomain[P]$ be a deterministic and reversible process over $\{X_t\}_{t \in T}$. Then $S(x(t))$ is constant.
\end{prop}

\begin{prop}
	Let $\tdomain[P]$ be a deterministic process over $\{X_t\}_{t \in T}$. Then $S(x(t))$ is a monotonically increasing function.
\end{prop}

\bibliography{bibliography}


\end{document}