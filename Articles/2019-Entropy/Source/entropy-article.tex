\documentclass[letterpaper]{article}

\begin{document}

\title{TBD}
\author{Gabriele Carcassi, Christine A. Aidala \\ University of Michigan}

\date{\today}

\maketitle

\begin{abstract}
	TBD
\end{abstract}


\section{Introduction}

In the intro, describe only the discrete case as it simpler to understand and already contains all the physics. The continuous case adds the complication of 

Evolution is a temporal sequence of states. Process is a set of possible evolutions. What we want to study is how a statement about a state at a particular time selects a particular set of possible evolutions and therefore selects a set of possible states at a previous or past time.

Define determinism and reversibility in terms of what evolutions exist. Note that a dynamical system is a deterministic system.

Note how the number of states does not always match the number evolutions it is compatible with. Note that under deterministic evolution the number of compatible evolutions can only increase. For dynamical systems, then, the number of evolutions associated to a state can only increase in time and it is stationary for equilibria.

If we compose independent systems, the number of possible states and trajectories is the product. If we take the logarithm, then those numbers are additive. Conceptually the entropy is the logarithm of the number of evolutions compatible with a given state. This article shows how this simpler and more general concept leads, in special cases, to the Boltzman entropy and the Shannon entropy. We will also see how this same concepts is applicable more in general, including the case of non-Hamiltonian systems.

\section{Different level of precision}


We regard states as points in a space. Conceptually, this presents problems. Mathematically, what we care about is the sigma algebra.

Introduce theoretical domains as point-less spaces, descriptions, that follow a countable boolean algebra. A size is given by a measure, additive map from the theoretical domain to the reals.

To define a unit system, we take one statement to be the unit, the size of a set can be defined as the ratio of the measure of the set and the measure of the set of the unit.

Two statements are independent if all minterms are possible, are not contradictions.

TODO: Combine with

\section{Introduction}

Cite other axiomatic works.

Single state space to capture:
* descriptions at all levels
* descriptions at different times
* correlations between descriptions at equal time (equations of state)
* correlations between descriptions at different times (evolution laws)
* different correlations at different description
* granularity/precision of such descriptions

We can't have:
* probability as a native concept

\section{Experimental domains}

First introduce tabular versions of contexts. Introduce equivalence, narrowness, independence and incompatibility on tables.

Introduce verifiable statements and domains. States as possibilities.

Precision over statements.


\bibliography{bibliography}


\end{document}