\documentclass{article}
\usepackage{authblk}

\title{Title}
\author[1]{Gabriele Carcassi}
\author[1]{Christine A. Aidala}
\affil[1]{
	Physics Department\\
	University of Michigan \\
	Ann Arbor, MI 48109
}

\date{\today}
\begin{document}
	
%	\maketitle
	
	
	
	
	\section{Notes when reading Uffink}
	


\begin{quote}
	Another point to be noted is that Clausius’ result that the entropy in an adiabatically isolated
	system can never decrease is derived from the assumption that one can find a quasistatic process
	that connects the final to the initial state, in order to complete a cycle. Indeed, if such a process
	did not exist, the entropy difference of these two states would not be defined. The existence of
	such quasistatic processes is not problematic in many intended applications (e.g. if sf and si are
	equilibrium states of a fluid); but it may be far from obvious in more general settings (for instance
	if one considers processes far from equilibrium in a complex system, such as a living cell). This
	warning that the increase of entropy is thus conditional on the existence of quasistatic transitions has
	been pointed out already by Kirchhoff (1894, p. 69).
\end{quote}
We require being able to isolate the system.

An important difference is in the notion of reversibility. Thermodynamics reversibility requires the existence of a process that can ``run the system in reverse''. Our starting point, which we may call dynamical reversibility, is simply the ability to reconstruct the initial state given the final, the dual of determinism. The key insight is that a system undergoes a thermodynamically reversible process if and only if the whole (i.e. system plus couple system) undergo a dynamically reversible process.

impinges on the ability that can ``run the system in reverse'' or, at the very least, quasistatic transitions. The second law is often defined in terms of this concept and therefore the very notion of entropy as a state variable relies on this notion (CITE Kirchoff 1894).


In standard thermodynamics, the notion of entropy as a state variable is conditional on the existence of processes that can ``run the system in reverse'' or that, at the very least, quasistatic transitions (CITE something) which is, in many cases, a problematic assumption .  In our work, we d the notion of reversibility is simply the dual of determinism. The  (and therefore indirectly ) is conditional on the existence of a process that can ``undo'' the work. In our framework, 

One advantage of this approach is substituting the In standard thermodynamics the 

\begin{quote}
	This classical conception of probability would, of course, remain a view without any bite, if it
	were not accompanied by some rule for assigning values to probabilities in specific cases. The only
	such available rule is the so-called ‘principle of insufficient reason’: whenever we have no reason to
	believe that one case rather than another is realized, we should assign them equal probabilities (cf.
	Uffink 1995). A closely related version is the rule that two or more variables should be independent
	whenever we have no reason to believe that they influence each other.
\end{quote}
We get equal probability from equal description.

\begin{quote}
	In this respect, his position stands in sharp contrast to that of Boltzmann, who made the project of	finding this unified basis [between statistical and mechanical] his lifework.
\end{quote}

\begin{quote}
	Modern commentators are utterly divided in the search for a direction in which a motivation
	for the choice of the size of these cells can be found. Some argue that the choice should be made
	in accordance with the actual finite resolution of measuring instruments or human observation capabilities.
	The question whether these do in fact favour a partition into cells of equal phase space
	volume has hardly been touched upon.
\end{quote}
We take invariance of densities as giving the correct measure.

Note somewhere: we are not interested in demonstrating that processes go to equilibria. This does not always happen anyway, so we can't demostrate it in general. What we want to show that, if we have a process that tends to an equilibrium, the entropy is maximized at equilibrium. And that, in a thermodynamic context, that entropy corresponds to the thermodynamic entropy.

\begin{quote}
	He approaches this issue quite cautiously, by pointing out certain analogies
	between relations holding for the canonical and microcanonical ensembles and results of thermodynamics.
	At no point does Gibbs claim to have reduced thermodynamics to statistical mechanics.
\end{quote}
	\bibliographystyle{plain}
\bibliography{bibliography}

\end{document}
