\documentclass[11pt]{article}

\usepackage[margin=1.25in]{geometry}

\usepackage{amssymb}
\usepackage{color}
\usepackage{graphicx}
\usepackage{epsfig,amssymb,amsmath,amsthm}
%\usepackage[active]{srcltx}
%\usepackage[hypertex,linkcolor=red]{hyperref}

\usepackage{color}
\usepackage{cancel}
\usepackage{tikz-cd}

\DeclareMathOperator{\spn}{span}

\newcommand{\pj}[1] {\underbar{$#1$}}
%\newcommand{\pj}[1] {\overline{#1}}


\def\>{\rangle}
\def\<{\langle}
\def\ca{_{\cal A}}
\def\cb{_{\cal B}}
\def\cc{_{\cal C}}
\def\comment#1{}
%\def\comment#1{ [{\bf Comment Lor:} {\sf #1}]}
\def\commentg#1{ [{\bf Comment Gabriele:} {\sf #1}]}
\def\labell#1{\label{#1}}
%\def\labell#1{\label{#1}{\mbox{{\tiny #1}}}}
%\def\section#1{{\par\em #1:--- }}
\def\togli#1{}
\def\sh{\mbox{sh}}
\def\iden{\openone}



\begin{document}

Dear Dr. Robert Garisto, 

thank you very much for having provided us
good quality referee reports for our paper. We appreciate that this is
not an easy task.

We believe we have overcome the criticisms pointed out by the referees
and we detail below our point-by-point response. Referee A suggested
that our paper ``could be published in PRL after substantial re-writing
and polishing of the arguments", which we have carefully done
following his/her useful suggestions, even though it took a
substantial effort to rewrite most of the paper. Referee B appreciates
our work (``There is no doubt that this paper represents some fine
mathematical work"), but questions its suitability for PRL. However,
the Referee has essentially focused on the supplementary material, in
his/her words: ``The essence of the paper, to my mind, lies in the
supplementary information." We agree with the Referee that the
supplementary information is not suitable as a paper for PRL. However,
the paper itself was carefully written to emphasize the PHYSICAL
arguments whereas the supplementary material just contains the
(nontrivial) mathematical details behind the physics. We urge the
Referee to take a fresh look at the paper with this in mind. Of course
we take full responsibility for the Referee's misconception: the
Referee is clearly very qualified, and in our previous version of the
paper we had failed to make this distinction clear. (Indeed we had
inaccurately claimed that the supplementary material contained the same
material formulated in a different manner.)

Thank you for your time. 

In the following we include our detailed answers to the Referees. Note 
that we will not enclose the changes made to
the paper, as we have modified most of it as requested by Referee A (removing some parts to make
space).


\section*{Referee A}

\begin{quote}
The authors address a very interesting problem that has
not received much attention, and they present an important result
about the internal logical structure of quantum theory.
\end{quote}

We thank the Referee for the positive assessment of our work.
\begin{quote}
I would say that the result is correct, but I could not
completely follow the argumentation of the article. I found the text
quite confusing and I think that it does not meet the writing
standards of PRL. In my opinion the arguments could be simplified and
expressed with simpler mathematics (see below). I think that the paper
could be published in PRL after substantial re-writing and polishing
of the arguments.
\end{quote}

We appreciate the suggestion that our paper can be published in
PRL. As suggested, we have substantially rewritten our paper following
the useful and insightful suggestions by the Referee, which we
sincerely thank for the feedback. We should remark that this was not
an easy undertaking, and we had to devote substantial effort in
rewording most of the paper. Nonetheless, we believe it was worth it,
because the argument is much clearer now.

\begin{quote}
1. Regarding 'statistical independence is not an
additional requirement: it is already contained in the measurement
postulate.' I think that this is a bit more subtle. [...]
\end{quote}

This is a great argument! We had not thought of that. The Referee is
correct, one can indeed prove the statistical independence of the
outcomes as the Referee does. And indeed the Referee is correct that
we had not considered that.

However, upon re-checking our proof very carefully, we realized that
we do not need the statistical independence of the outcomes. We just
need the statistical independence of the preparations. Nonetheless, we
believe that the argument the Referee proposes is really nice and we
respectfully ask the Referee if we can add it to our paper (citing an
anonymous Referee, or the Referee may contact us separately if he/she
wants to be properly acknowledged). For now we have added this at the end of the
supplementary material, but, if the Referee disagrees, we can remove
it and replace it with a sentence of the type: ``The statistical
independence of the outcomes is not an additional assumption, as it
can be proved independently."

The Referee's comment made us realize that we didn't discuss the
statistical independence in sufficient detail, and that is actually
one of the critical steps in the demonstration (i.e. making explicit an
implicit assumption). We rewrote and expanded the part of the paper
(and the supplementary material) where we discuss this, mainly in the
first column of page 3.

\begin{quote}
2. I would say that physicists are familiar with the
notion of `colinear'.  Hence, I would either (i) provide a definition
or (ii) express the argument with the more familiar notions `linear'
and `anti-linear' I would be inclined for the second (if possible).
\end{quote}

We presume that the Referee intended ``physicists are UNfamiliar".  We
agree and we also agree that we have not done enough in the text to
guide the reader in that respect. As suggested, we have explained what
``colinear" means. As far as we understand, we cannot remove this
notion from our proof, unfortunately. Indeed, the use of projective
geometry is necessary. We now have a paragraph that clearly says that
and why we need projective geometry.

This is necessary to understand the properties of the maps M and m
that we need to derive the tensor products.

\begin{quote}
3. Regarding `the dual-space description of a quantum
system through bras is equivalent to the description using kets', what
means `equivalent' here? Can you justify it with an argument instead
of the anecdote that `in his first papers Schroedinger used both signs
in his equation, effectively writing two equivalent equations with
complex-conjugate solutions'?
\end{quote}

We simply meant the following. Given any equation in quantum mechanics
in terms of operators acting on ket vectors, one can write an
EQUIVALENT equation in terms of bra vectors by simply taking the
Hermitian conjugate of both sides of the first equation. Namely, the
equation $A|\psi\>=B|\phi\>$ is equivalent to $\<\psi|A^\dag=\<\phi|B^\dag$. In
terms of wave functions, it means taking the complex conjugate (which
entails a change of sign in the case of the Schroedinger equation). We
apologize this was unclear (upon re-reading that part, we agree) and
we have now specified it explicitly. We rewrote that part completely,
and we added the previous equivalence explicitly.

\begin{quote}
4. Regarding `The function m, then, is an isomorphism over
all elements of the basis $|a_i\>|b_j\>$'. What it means to be an
`isomorphism over' some vectors?
\end{quote}

We mean that the function m is a one-to-one mapping, that maps the
elements of the basis into elements of a basis of a tensor product
space. We understand our notation could lead to confusion. We rewrote that part, and also changed the (admittedly confusing)
notation to clarify the whole thing.

\begin{quote}
5. Regarding `other linear products of linear spaces
exist'. Is the direct product of two vector spaces a vector space? (I
am assuming that a composite system is also a system and hence it has
an associated Hilbert space.)
\end{quote}

Yes, a direct product of vector spaces is a vector space. You can
certainly define a Hilbert space using direct products of vectors of
two Hilbert spaces (you just need to make sure the norm doesn't
diverge). Of course, that is not the correct thing to do in quantum
mechanics. We are simply pointing out that there are many ways to map
two linear spaces to another linear space, so requiring linearity on
the spaces is not sufficient to single out the tensor product. In
fact, linearity by itself is not sufficient to warrant use of a
product to create the joint space.

We have changed that sentence to make it clearer, and have also added
a footnote to further clarify the distinction.

\begin{quote}
How you define multiplication by scalars? If it is defined
distributively then this direct product becomes the direct sum of the
two spaces. And similarly with the topological product. Therefore,
these three examples of product of linear spaces are the same one.
\end{quote}

Each linear space comes with its own sum (e.g. $+ : A \times A \to A$)
and its own scalar multiplication (e.g. $* : \mathbb{R} \times A \to
A$). Distributivity is $r * (a + b) = (r*a) + (r*b)$. Each space
individually will retain that property.

If we imagine a map $m : A \times B \to C$, we can ask how can we
express $r*m(a,b)$ as a function of $m(r*a)$ and $m(r*b)$. This
differs from product to product. For the direct product, we have $r *
(a \times b) = r*a \times r*b$ which is distributive. For the tensor
product, we have $r * (a \otimes b) = r*a \otimes b = a \otimes r*b$
which makes it not distributive.

We can also ask how we can express $m(a,b+c)$ in terms of $m(a,b)$ and
$m(a,c)$. For the tensor product we have $a \otimes (b+c) = a \otimes
b + a \otimes c$ which is distributive. For the direct product we have
$a \times (b+c) = a \times b + 0 \times c$ where $0$ is the zero
vector. So the direct product is not distributive.

The direct sum and direct product are the same in the finite case but
not in the infinite case. In that case, one can only take finitely
many non-zero vectors for the direct sum, while this is not required
in the direct product.

The overall point is: there are many choices. For distributivity, one
has to decide if it's distributivity over the addition or the scalar
multiplication. That choice needs to be motivated. The exterior
product is also distributive over the addition, like the tensor
product (as it is a quotient of the tensor product). So, why the
tensor product and not the exterior product?

We added a footnote which succinctly highlights
the issue. 

\begin{quote}
6. Regarding `(Dirac) adds the seemingly innocuous
request'. I understand that requiring the product of vectors to be
distributive is the same as requiring it to be bilinear, that is $|a\>
(|b\> + |c\>) = |a\>|b\> + |a\>|c\>$. Am I missing something?
\end{quote}

No, what the Referee states is correct. Our point is that such request
is far from innocuous, as for example it excludes (without any
physical justification whatsoever) the direct product. The requirement
of distributivity (or bilinearity) introduced by Dirac is an
ADDITIONAL assumption that is certainly not warranted and it is even
NOT natural. For example, it does not hold in classical mechanics,
where the state space of composite systems is obtained through the
direct product, and (a,b+c) is DIFFERENT from (a,b)+(a,c). We explained it better in the new version of the paper.

\begin{quote}
Supplementary information 1. Definition of composite
system. Why do you say `for any compatible pair' instead of `for any
pair'? Can we have a pair of systems where not all pair of states are
compatible?
\end{quote}

We can have a pair of systems where not all pairs of states are
compatible. Consider two electrons. They cannot be prepared in the
same state. Therefore the state space of the composite system should
exclude those cases. We use the exterior product (i.e. the
anti-symmetrized tensor product) in those cases. As we explicitly say
elsewhere, electrons cannot be properly considered as quantum
subsystems (at least in the case in which indistinguishability plays a
role): the quantum system is the field, and electrons are just field
excitations (quanta).

We now made this more explicit after the definition, with a more detailed physical motivation of the requirements. 

\begin{quote}
``I would say that this high degree of generality obscures
the argument. Perhaps this case of systems with non-compatible pairs
of states can be left for a discussion after the main proof."
\end{quote}
We agree with the Referee, but we suppose that the Supplementary
information section will be only read by people who want to understand
the full details of our argument. This type of reader will not be
scared by the full generality of the argument. Instead in the main
paper we were more careful to consider first the compatible states and
then the general issues. In the Supplementary information we need to
consider the situation in full generality, e.g.  systems of multiple
fermions and bosons when considered as systems as is sometimes done
improperly (a common enough case that we have to justify why, in those
cases, the tensor product is not used directly). In the proof, we
already concentrate on the preparation independent case as we also
didn't want to make the derivation too heavy. We hope that the above
clarification explains why we believe we have to make that distinction
mathematically and why it is also interesting physically. In any case,
we have clarified this point both in the paper and the supplementary
material.

\begin{quote}
2. Regarding 'the proposition a wedge b is equivalent to
the (pure) state $M (a, b)$'. The authors do not define what it means
for a logical proposition to be equivalent to a ray? Is it necessary
to use the formalism of logical propositions to derive the tensor
product? If so, some definitions should be introduced.  Similarly,
what it means to say that a state $M (a, b)$ corresponds to an event in
probability space. I have the feeling that the same can be said in
more plain words.
\end{quote}

We agree that we hadn't clearly explained the formalism and we have
now done so. We do not really need the formalism of logical
propositions, but just the logical connective ``AND" that we indicated
with the wedge symbol. That's all we need, and it's necessary for
defining the JOINT probability. We could remove the wedge symbol and
replace it with the connective ``AND". We tried, but the formulas
become quite heavy, so we believe that this (minimal) concession to
the formalism is useful. 
	
We have modified the paper (and the supplementary material) in
multiple places to clarify the formalism: not only the formalism of
logical propositions, but also how these are mapped in the formalism of quantum mechanics
(e.g. rays and Hilbert subspaces).

\begin{quote}
3. Regarding `Every preparation of the composite gives a
nontrivial measurement on the components.' Do you mean that, for any
state of the composite system, local projective measurements must have
at least one outcome with non-zero probability? I think that this
phrasing is easier to accept.
\end{quote}
Yes. That is exactly what we meant. We agree that the Referee's
wording is clearer and we have replaced it in the new version of our
paper. 

In conclusion, we thank the Referee for the careful reading and the
thoughtful suggestions. They have been all very valuable in enhancing
the paper.

\section*{Referee B}

\begin{quote}
This paper shows how the tensor-product assumption of
quantum theory can be derived from two standard postulates of quantum
theory--specifically, the postulates about states and
measurements--together with a set of assumptions reflecting reasonable
interpretations of the expressions `composite system' and `independent
systems'. The essence of the paper, to my mind, lies in the
supplementary information.
\end{quote}

We respectfully disagree with this assessment: while the supplementary
information contains the nontrivial mathematical formulation of the
ideas presented in our paper, we believe that our main contribution is
in the PHYSICS, rather than in the MATHEMATICS. Namely, we show how
some PHYSICAL ASSUMPTIONS (contained in the other postulates of
quantum mechanics) must give rise to tensor products. We have
carefully written our paper on two separate levels (actually three):
at a first level, a physicist interested in the main physical ideas
will only read the paper, and can easily skip the sketch of the proof
presented there, while still understanding the main results. At a
second level, a reader that wants more details can also follow the
sketch of the proof that we put in the main paper. Finally, a
specialist of quantum mechanics (such as, clearly, the Referee) can
get into the fine details of the argument by looking at the physical
considerations in the paper and the mathematical details in the
supplementary information. We understand that this distinction was not
clear in the first version of the paper, and we take
responsibility for this (we ourselves had mischaracterized our
supplementary information). We urge the Referee to re-read the paper
keeping in mind that the main paper is written for a more general
audience and mostly contains the physics, and the supplementary
information only contains the details of the mathematical proofs.
\begin{quote}
This is where one finds a precise statement of each
assumption and a precise articulation of the logical connections
between statements. Indeed, the authors' main contribution is the
articulation of this logical argument. A number of physicists may
already hold the intuition that the tensor product rule is the only
composition rule one can imagine in light of the rest of the theory;
however, making this intuition precise, as the authors have done, is
another matter.\end{quote}

We thank the Referee for this compliment. We agree: most physicists do
hold the intuition that the tensor product is the natural composition
rule for quantum mechanics, but, as the Referee suggests, this had
indeed never been shown previously, at least using the conventional
formulation of quantum mechanics as we do.

\begin{quote}There is no doubt that this paper represents some fine
mathematical work.\end{quote}

Thank you!

\begin{quote}The question is whether it meets the criteria for
publication in PRL.  As the authors point out, the tensor product
postulate has been addressed by other authors. Some authors explain
why this postulate is reasonable but do not prove in detail how it
follows from other assumptions. Within the quantum logic community,
the tensor product rule has been derived rigorously from simpler
axioms. The authors' approach is different in that they are starting
with two other standard postulates of quantum theory and thereby aim
to show that there is some redundancy in the standard postulates.
\end{quote}
We agree: this is a good characterization of our work.

\begin{quote}In this respect, the authors' approach could be compared
to that of Ref. [8], which derives the measurement postulate from
other standard postulates (including the tensor product
postulate). Ref. [8] derives not only the mathematical representation
of measurement outcomes, but also the Born rule and the general form
of the rule for determining the post-measurement state of the
system. That paper was published in Nature Communications. Another
paper, Ref. [28], which also derives the Born rule from the postulate
about states and the tensor product rule, was published many years ago
in American Journal of Physics.
\end{quote}

It is true that both our result and theirs shows that there is a certain redundancy in the postulates, even though our result is completely different from theirs. So, following the Referee's suggestion we have emphasized this.

\begin{quote}However, I think there is an important difference between
the tensor product postulate and the measurement postulate. The
measurement postulate is the central focus of the notorious
measurement problem of quantum theory, which has been addressed in
many books and papers and which many would argue is the root of all
the interpretational difficulties. (Indeed, Ref. [28] explicitly
addresses the measurement problem.) The tensor product postulate does
not have this status. So, whereas a paper deriving the measurement
postulate might well be read by any physicist with even a passing
interest in the conceptual foundations of the theory, a paper deriving
the tensor product postulate will not have the same breadth of
appeal. Also, I think it is somehow less surprising that the tensor
product postulate emerges in this way than that the measurement
postulate does. The most obvious alternative to the tensor product is
the direct sum, which one immediately sees does not capture what is
meant by 'composite system'.\end{quote}

We certainly agree with the Referee that a solution of the measurement
problem would be more interesting than our result. BUT it is quite
debatable (and actually debated) whether any of the above results
resolved the measurement problem: the general consensus is that it is
still OPEN. In contrast, our result is quite solid (and neither
Referee objects to that): we DO show that the tensor product postulate
is not a necessary postulate, and believe (more on this below) that
our result is certainly suitable for publication in PRL. Referee A
shares this belief.

\begin{quote}
As I look over the four criteria the editors of PRL have
given for determining whether a paper should be published in PRL, I do
not see that any of them clearly applies to this paper. This paper
will be read with interest by physicists who have thought carefully
about the logical structure of quantum theory, but I do not think PRL
is the right place to publish it.
\end{quote}

We believe the Referee has seen these four criteria in a too
restrictive light. We please ask the Referee to take a look at the PRL
papers that are usually accepted for publication. As a typical
example, the first paper that currently appears on the PRL web site is
titled ``Stochastic Interpolation of Sparsely Sampled Time Series via
Multipoint Fractional Brownian Bridges". We have been professional
physicists for decades, and none of us has any idea of what that even
means. (We emphasize that this is the first paper on the PRL web site
today: we did not choose a particularly cryptic one!) Instead, any
physicist (or even a physics student) that has taken a quantum
mechanics course (i.e. all physicists?) will understand our title and
abstract: they have met tensor products when dealing with multiple
systems.

In addition, we also believe that the way in which we construct
Hilbert spaces from physical statements is also relevant: we cannot
claim that this is a new result, of course (although we could not find
a paper where our same formalization is presented), but this is
typically not emphasized in existing literature. In particular, we
show how the link between events in probability and observables in
Hilbert spaces goes through the projective spaces, and how the choice
of representation, the choice of gauge (phase and modulus of the state
vectors), plays into that. So, in addition to our main result, we also
claim that popularizing this particular connection is also a strength
of our paper.

\begin{quote}
I also have a couple of specific comments. 1. I think the
authors are emphasizing too strongly the smallness of the number of
postulates they are using (e.g., in the paper's title).
\end{quote}

We used the usual quantum postulates that appear in the literature,
e.g. in the referenced literature. We have no qualms in adding
postulates if the Referee suggests that we should do it. Our title is
invariant: indeed, we could also title our paper ``The five postulates
of quantum mechanics are four" or so on for any n: ``The n postulates
of quantum mechanics are n-1". Our main result is that we can remove
one of the postulates (the tensor product one): how many we start from
is not very relevant.

\begin{quote} As I was
reading the main text, it felt as if the authors were sometimes
finding new postulates in their interpretations of the words used in
definitions, and yet they were declining to call them postulates,
possibly because to do so would contradict the paper's title. I
understand that these `smaller' assumptions can be thought of as
simply making precise the meanings of commonly used expressions-- this
fact becomes clearer in the supplementary information. But strictly speaking, these are genuine assumptions.\end{quote}

We disagree: we did not add ``new postulates". 
Let us first be precise as to what a postulate is.
If one looks carefully, all postulates of quantum mechanics link something
defined within the physics world (e.g. states, observables, composite system,
time evolution) to some mathematical object (e.g. rays in a Hilbert space, Hermitian
operators, tensor product, unitary evolution). The idea is that we are
establishing a link between two different definitions (the physical
and the mathematical) that has no a priori justification. But each
element is well defined within its realm. We are not adding new postulates in the sense that we are not adding any
new link of this kind. What we are doing is showing that the definition of 
a composite system, which is ultimately a definition (NOT a postulate) within the physics
realm, together with the links already established between state space/Hilbert
space and measurement/Born rule is enough to demonstrate that the state space of the composite
system must be the tensor product of the subsystems.


More specifically, for example, there is a fundamental difference between a statement
like ``the state of a quantum system is represented by a vector in a
Hilbert space" and something like ``the state of a composite system
will fully determine the probability distribution of the observables of
all its parts and only of its parts." The first connects two
completely distinct elements, the physical preparation of a quantum
system and the mathematical construct of a Hilbert space. There is no
a priori reason for that and we are not giving one in this paper. The
second is defining one object based on others. That is, we do not
define composite systems in some other way and than later we
independently find that they have the property of characterizing the
subsystems.  We define the composite system to be that. There is no
assumption being made: it is just a definition, and it is the natural
definition that comes from our physical intuition.

In any case, the Referee's observations were very useful to us: we realized that the connection between postulates and physics should have been explained much better and we did so in the new version of the paper.

\begin{quote} For
example, Definition I.4, part 2, which says essentially that no new
orthogonal states arise when one combines two systems, is a
substantial postulate. After all, it is not the case that no new
states at all arise when one combines two systems-- entangled states
arise, and they are new. One can imagine a world in which not only
entangled states but also new orthogonal states arise. We do not live
in that world, but one needs a postulate to rule out that world. For
the authors, that postulate is Definition I.4, part 2.
\end{quote}

That ``world" would simply correspond to a different definition of a composite system. Whereas we used the natural one: the (set) union of the subsystems. 
In the standard definition, the
composite system must still define all the properties of the subsystems. A new orthogonal state would not be part of the set of eigenstates of the
operators corresponding to the observables of the subsystem.

Again, the Referee's comment made us understand that our paper was definitely unclear on this and we have
clarified it in the new version. 

Of course, entangled states arise and that is a
trivial consequence of linearity: we start from FACTORIZED states,
those are all we need to obtain tensor products, but once tensor
products are established, then ENTANGLED states arise immediately from
linearity in the composite system's Hilbert space.

\begin{quote}
(Note that the authors' informal definition of a composite
system, which is that it consists of nothing but the two systems,
could alternatively be read as excluding entangled states. Indeed, a
reader not familiar with quantum theory might well read it that way. I
mention this just to point out the ambiguity associated with an
informal definition and to highlight the importance of the actual
mathematical postulate.)
\end{quote}

We do agree that the mathematical definitions are important to precisely
address the issue: that is why we have included all the mathematical
details in the supplementary material. However, the mathematical definitions
are not as arbitrary as the referee implies.

As the Referee correctly points out, our definition certainly does not
exclude entangled states. However, we see the Referee's point and we
agree that a reader unfamiliar with quantum theory might get confused,
so we added an explicit mention that our construction can be extended
to include entangled states, which are the ones that do not satisfy
statistical independence. We added this at the end of the paragraph,
rather than at the end of that sentence, because we did not want to
ruin the flow of the narration by introducing the concept of ``states"
in that sentence (which refers to systems).

\begin{quote}
2. On page 2, postulate (a) begins, `The state of a system
is described by a ray...'. As expressed here, this postulate actually
contradicts the authors' ultimate conclusion, which allows entangled
states of the combined system. As the authors certainly know very
well, if the combined system is in an entangled state, the state of
each component system is not described by a ray in that system's
Hilbert space. I'm not sure why the authors didn't say, `A pure state
of a system is described by a ray...'. (The word 'pure' is missing
also in the statement of this postulate in the supplementary
information.)
\end{quote}

Definitely: the word `pure' is missing! We apologize for this
oversight and thank the Referee for pointing it out. We certainly
should have added the specification that the postulate (as expressed
in our paper) refers to pure states. Also Referee A implicitly pointed
this out. We have now corrected it.

In conclusion, we thank Referee B for his/her insightful
observations and helpful comments. We believe that his/her assessment
on the suitability of our paper for PRL stems for an overly
restrictive interpretation of the publishing criteria, as discussed
above, and from giving more weight to the supplementary information
than to the paper. The paper mainly contains the physical ideas (which
we believe are our main contribution) whereas the supplementary
information contains only the mathematical details, as the Referee
also pointed out.

In this respect, before it was possible to attach supplementary
material to PRLs, we used to write a PRL to give the main idea and
then an adjoining PRA or PRD paper to provide the technical
details. If the Referee suggests so, we can do it also in this case
(by transforming the supplementary material into a PRA
paper). However, we would rather not do that.

--------------------


In closing, we were very pleasantly surprised by the quality of both
reports and we thank both Referees for the time they spent in
evaluating our paper.

Sincerely,

The Authors

\end{document}