\documentclass[letterpaper]{article}

\begin{document}

\title{TBD}
\author{TBD}

\date{\today}

\maketitle

\begin{abstract}
	TBD
\end{abstract}


\section{Introduction}

%% testing testing! 

TODO: Lagrangian assumption vs kinematic assumption. Investigate what happens when the transformation is non-linear.

\subsection{Define Lagrangian}

Talk about disippative systems.

\subsection{Fundamentality}

Different notions of fundamentality.

Joint carving, perfectly natural properties. Ontological notion of fundamental. Clearly state why this is not "good" for this context. We are agnostic. One may use following points to argue whatever they want.

Joint carving? What notion of fundamentality? Hamiltonian mechanics is "fundamental" because allows to show that densities are coordinate independent quantities and therefore physical quantities. Canonical pairs (position and conjugate momentum) allows a simpler expression/comparison of densities over space. That is why statistical mechanics works best on phase space (we have a uniform measure over q/p).

Hamiltonian mechanics makes fewer assumptions on the physical system being studied than Lagrangian mechanics. Lagrangian is a subset than Hamiltonian. Makes "amount of structure" more rigorous. Hamiltonian defines the relationship between conjugate momentum and velocity ($dq/dt = \partial H / \partial p$), so kinematic assumption constrains the Hamiltonian.

The action principle can be understood geometrically in the extended phase space as a consequence of "state density" conservation. It also holds when the action cannot be expressed as a function of just position and velocity, since it can always be expressed as a function of position and conjugate momentum. Also shows that Lagrangian is "unphysical" and depends on gauge.
 



\bibliography{bibliography}


\end{document}
