\documentclass[letterpaper]{article}

\usepackage{amsmath}

\begin{document}

%% testing from desktop

\title{TBD}
\author{TBD}

\date{\today}

\maketitle

\begin{abstract}
	TBD
\end{abstract}


\section{Introduction}

%% testing testing! 
%% testing twice 

TODO: Lagrangian assumption vs kinematic assumption. Investigate what happens when the transformation is non-linear.

\subsection{Define Lagrangian \& Hamiltonian mechanics}

Talk about disippative systems.

\subsubsection{Strong vs weak Kinematic assumption}

Problem: changing variables from state variables $(q^i, p_j)$ to kinematic variables $(x^i, v^j)$. We set $q^i=x^i$ because $x^i$ defines the units for $v_j$. The change of name helps us remember that $\frac{\partial}{\partial x^i} \neq \frac{\partial}{\partial q^i}$ as one is taken at constant $v$ while the other is taken at constant $p$.

Weak assumptions: $v^i = v^i(q^j, p^k)$ invertible. Only particles trajectories are trasported. Strong assumptions: $m v^i = g^{ij} (p_j - q A_j(q^k))$ linear in $p_j$. This considers the distribution as well. The unit for the density only depends on units of $q^i$.

We want to transport the density from state variables to initial conditions: $\rho(q^i, p_j) = |J| \rho(x^i, v^j)$. We have:
\begin{equation}
	|J| = \begin{bmatrix}
	\frac{\partial x^i}{\partial q^j} & \frac{\partial v^i}{\partial q^j} \\
	\frac{\partial x^i}{\partial p_j} & \frac{\partial v^i}{\partial p_j}
	\end{bmatrix}
	= \begin{bmatrix}
	\delta^i_j & \frac{\partial v^i}{\partial q^j} \\
	0 & \frac{\partial v^i}{\partial p_j}
	\end{bmatrix}
	= \left|\frac{\partial v^i}{\partial p_j}\right|
\end{equation}
Under the strong assumption, $\left|\frac{\partial v^i}{\partial p_j}\right| = \frac{1}{m} |g^{ij}|$.

\subsection{Fundamentality}

Different notions of fundamentality, perfectly natural properties, naturalness, etc. 

Joint carving, perfectly natural properties. Ontological notion of fundamental. Clearly state why this is not "good" for this context. We are agnostic. One may use following points to argue whatever they want.

Joint carving? What notion of fundamentality? 

\section{Argument for privileging Hamiltonian mechanics}

\subsection{Relevant formal background from assumptions of physics framework}

Hamiltonian mechanics makes fewer assumptions on the physical system being studied than Lagrangian mechanics. Lagrangian is a subset than Hamiltonian. Makes "amount of structure" more rigorous. Hamiltonian defines the relationship between conjugate momentum and velocity ($dq/dt = \partial H / \partial p$), so kinematic assumption constrains the Hamiltonian.

Hamiltonian mechanics is "fundamental" because allows to show that densities are coordinate independent quantities and therefore physical quantities. Canonical pairs (position and conjugate momentum) allows a simpler expression/comparison of densities over space. That is why statistical mechanics works best on phase space (we have a uniform measure over q/p).

Can characterize this argument as physically or epistemically privileging Hamiltonian mechanics 

\section{Interpreting Lagrangian mechanics}

Somewhere in this section, should reply to Curiel's privileging of Lagrangian mechanics. Curiel frequently appeals to naturalness and a distinction between natural vs. ad hoc/artificial constructions [figure out what his definition of ``naturalness" is, if he provides one].

For instance, Curiel argues that Hamiltonian mechanics is less natural and thus less physically significant/privileged:  ``Hamiltonian mechanics represents abstract classical systems only in so far as we restrict ourselves to a subfamily of all the formally acceptable Hamiltonians by the \textit{ad hoc} use of conditions foreign to Hamiltonian mechanics itself. Its structures do not provide the appropriate concepts and tools to formulate in their terms the required structures that treat configurative quantities differently from momental, nor do they provide any natural justification for the restriction to Hamiltonians of the form (6.2)" (306). 

Response: symplectic geometry indeed does not care about the difference between q's and p's. Classical mechanics does. Misindentification of the math (symplectic geometry) with the physics (Classical Hamiltonian mechanics).

Curiel's main point: the representation of a classical mechanical system in the Lagrangian framework is \textit{natural} in a way that its representation in a Hamiltonian formulation is not. The former representation depends on intrinsic structures of the classical mechanical system rather than an artificial construction (270).

Part of our response: The action principle can be understood geometrically in the extended phase space as a consequence of ``state density" conservation. It also holds when the action cannot be expressed as a function of just position and velocity, since it can always be expressed as a function of position and conjugate momentum. Also shows that Lagrangian is ``unphysical" and depends on gauge.


Upshot: can better understand Lagrangian mechanics as a special case of Hamiltonian mechanics. Explain how the geometrical understanding of the stationary action in extended phase space should meet any criterion of ``naturalness" or ``natural construction." Lagrangian mechanics is actually less natural because contingent on having a bijection between velocity and conjugate momentum.

%

\bibliography{bibliography}


\end{document}
