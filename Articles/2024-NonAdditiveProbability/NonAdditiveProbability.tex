\documentclass[10pt,twocolumn, nofootinbib]{revtex4-2}

\usepackage{assumptionsofphysics}
\usepackage{tikz}
\usepackage{breakurl}

\newcommand\hull{\mathrm{hull}}
\newcommand\stcap{\mathrm{scap}}
\newcommand\fraction{\mathrm{frac}}
\newcommand\frcap{\mathrm{fcap}}

\newcommand{\ens}[1][e] {\mathsf{#1}} % Ensemble
\newcommand{\Ens}[1][E] {\mathcal{#1}} % Ensemble space


\def\>{\rangle}
\def\<{\langle}
\DeclareMathOperator{\erf}{erf}

\begin{document}

\title{A non-additive generalization of probability theory \\for quantum mechanics and beyond}
\author{Gabriele Carcassi}
\affiliation{Physics Department, University of Michigan, Ann Arbor, MI 48109}
\author{Christine A. Aidala}
\affiliation{Physics Department, University of Michigan, Ann Arbor, MI 48109}

\date{\today}


\begin{abstract}
	We present a physically motivated generalization of probability theory that is suitable for classical mechanics, quantum mechanics and any future physical theory that allows a statistical description. The goal is to put the use of classical and quantum probability in a broader context, and show how the current mathematical structures are likely not suitable to solve the open problems in the foundations of physics. For the more mathematically inclined, we will point to areas where new math or generalization of established math are needed. For the more philosophically incline, we will point to areas where further conceptual work is needed.
	
	
	%Given a generic space of ensembles, we can define the fraction capacity as the maximum fraction of a particular ensemble that can be understood as a mixture of ensembles from a given set. This gives a non-additive (i.e. fuzzy) measure that reduces to a probability (i.e. additive) measure for classical spaces and for quantum measurements. We can also define the state capacity as the exponential of the maximum entropy reachable by a mixture of ensembles from a given set. This is also a non-additive measure and reduces to the Liouville measure in classical mechanics and the Hilbert space dimensions for subspaces of quantum mechanics. Conceptually, it gives us a notion of probability that is both theory and interpretation independent. Mathematically, it gives us a measure theoretic generalization of probability. Physically, it may allow to find new way to understand current theory and tool to investigate new ones. The purpose of this paper is to show the core ideas and present questions that may be developed on the mathematical, physical and philosophical side.
\end{abstract}

\maketitle


\section{Introduction}

Our current understanding of probability is given in terms of the measure theoretic formalization due to Kolmogorov, and its various philosophical interpretations. However, probability in quantum mechanics, as it is widely known, is not Kolmogorovian, in the sense that it provides results that are not reproducible in standard probability spaces. It follows that our mathematical axioms for probability and their interpretations are insufficient for physical theories.

An alternative measure theoretic strategy, that has been successful is quantum mechanics, is the use of quasi-probability distributions: measures that are still unitary over the whole space, but can have negative values in small regions. However, these haven't resulted in an alternative axiomatic approach with suitable interpretations for at least, in our mind, three reasons. First, it is not clear how negative probability should be interpreted. Second, it is not clear how to characterize the space of valid quasi-probability distributions (e.g. there is no simple rule to specify which $\rho(q,p)$ are valid Wigner functions). Third, there are multiple options for quasi-probability distributions (e.g. Wigner function, Husimi distribution, Sudarshan–Glauber distribution). This lack of foundational clarity presents an additional problem: how do we know whether future physical theories will not require yet another extension?

The goal of this paper is to show that, in the same way that classical probability can be axiomatized through standard measure theory, quantum probability, and in fact any probability used a physical theory, can be axiomatized through non-additive (i.e. fuzzy) measure theory. We will define a generalized concept of probability that is ultimately based on physically motivated axioms, giving the physicist a clear operational motivation to our definitions, the mathematicians a formally well-defined mathematical structure and the philosopher a unified framework in which probability can be interpreted in the context of physical theories. We are not going to be able to develop a full theory here, as there are still a number of conceptual and technical challenges to be solved that will require work in physics, mathematics and philosophy. However, we will show that there are enough results that point in a good direction, and therefore make it reasonable to invest time and effort to work on the open problems. For those interested, the current state of the research will always be available in an open document that will be routinely updated as the theory matures.

In a nutshell, we will start by arguing that physical theory must at least be able to talk about statistical ensembles, and that ensembles must allow the preparation of statistical mixture. Given a target ensemble $\ens$ and a set of ensemble $A$, we define the fraction capacity of $A$ for $\ens$ as the biggest fraction of $\ens$ that can be constructed from another set of ensembles $A$. That is, the highest $p$ such that we can express our ensemble as $\ens = p \ens[a] + (1-p) \ens[b]$, where $\ens[a]$ is a mixture of elements of $A$. For example, if $\ens$ is the uniform distributions for a roll of a six-faced die, and $A$ includes only two ensembles with $100\%$ probability for the outcomes $3$ and $4$ respectively, then the faction capacity of $A$ for $\ens$ will be $\frac{1}{3}$. Note that this coincides with the probability of the event ``$3$ or $4$''. The fraction capacity is a non-negative, unit bounded, sub-additive and continuous set function, which is therefore like a probability measure except for the additivity, which is recovered for classical mechanics and over quantum measurements.

\section{Quick review of probability theory}


\section{Ensemble spaces}

As mentioned in the introduction, our goal is to construct a probability theory that is guaranteed to work in any physical theory. Therefore our fundamental axioms will not be about probability, but about physical theories. This will help us place probability in a bigger context, and provide a minimal interpretation. As it is common in our larger project, we want axioms to describe necessary requirements of the scientific practice. That is, they will describe features of physical practice, not feature of the systems being described.

We are looking to find a minimal set of axioms required by a physical theory that allows statistical descriptions. Such a theory must, at the very least, provide us with a notion of ensemble, and the space of ensembles that are allowed in the theory. An ensemble can be understood as the infinite collection of the outputs of a particular preparation procedure. The ensemble space of classical mechanics, for example, is the space of continuous probability distributions of phase space. In quantum mechanics, instead, it is the space of density operators. While the ensemble space for each theory will be in principle different, some will have many common features. In all cases we must be able to talk about conservative or dissipative processes, measurable properties of the system, probability distributions for outcomes of said properties and so on. We are looking for necessary minimal basic requirements upon which these common tools can be built.

At least two arguments can be made to show that any physical theory must provide a set of ensembles. First, note that physical laws are never about specific instances but rather regularities. They are statements of the type ``whenever we prepare this, we can measure that.'' This means that, ultimately, they are about ensembles. Second, if a physical theory is to be testable experimentally, and repeatedly so, it must be in terms of ensembles because this is what we prepare in practice and this is what, in the end, we characterized experimentally. It seems, then, that the approach can be given firm conceptual grounds, which we leave as open but solvable philosophical problem.

We have identified three basic requirements on ensembles: first, they have to be identifiable experimentally; second, they need to allow statistical mixtures; lastly, they need a well-defined entropy. Let us go through these items one by one. We leave the full details to the working document (TODO: decide how to call it).

\subsection{Experimental verifiability}
The first requirement is that ensemble must be connected to experimental verification. That is, we must have enough \textbf{verifiable statements} at our disposal to define ensembles and tell them apart. By verifiable statement we mean a statement for which an experimental test is available that terminates in finite time if and only if the statement is true. For example, the statement ``the mass of the photon is less than $10^{-13} \, eV$'' is verifiable, while ``the mass of the photon is exactly $0 \, eV$'' is not verifiable due to its infinite precision. Based on previous work, this mathematically requires an ensemble space to be a $T_0$ second countable topological space, where open sets represent verifiable statements and Borel sets represent statements associated with a test, but without guarantee of termination.

The topological structure, then, represent the most foundational structure for a physical theory. It tells us why functions are ``well-behaved,'' which simply means they have to be topologically continuous in the ``natural topology'' induced by this requirement. It tells why probability is assigned to Borel sets (i.e. all statements associated with tests) and why the Banach-Tarski paradox does not apply in physics (i.e. non-Borel sets are physically ill-defined as they are not connected to experimental verification). Similarly, it tells us that sets with cardinality greater than that of the continuum are physically not relevant, as they cannot be given a $T_0$ second countable topology.

\subsection{Statistical mixtures}
Another basic requirement is the ability to prepare a \textbf{mixture} of two ensembles. If $\ens[a], \ens[b] \in \Ens$ are two ensembles, and $p \in [0,1]$ a weight, then $\ens = p \ens[a] + (1-p) \ens[b] \in \Ens$ that describes a process that selects the first ensemble over the second $p$ percent of the times. Mathematically, the ensemble space is endowed with a \textbf{convex structure}.

Ultimately, the convex structure is responsible for all linear structures we have in physics. Both in classical and quantum mechanics, the ensemble space is not just a convex set, but a convex subset of a vector space. In those cases, in fact, the convex structure is, in a sense, invertible. If we fix $\ens$, $\ens[a]$ and $p$ then, if it exists, there is only one $\ens[b]$ such that $\ens = p \ens[a] + (1-p) \ens[b]$. The ensemble space is \textbf{complemented}, meaning that there is only one complement to a component of an ensemble.

This perspective allows to understand what negative probability is. Suppose we have a real vector space of dimensions $n$, then $n$ linearly independent vectors are enough to express every point as a linear combination. If we add another point, we can now express every point as an affine combination, that is a linear combination where the coefficients sum to one, though some can still be negative. Since the space of mixed space in quantum mechanics is a real vector space, we can choose a linearly dependent set of states space span the whole space, and every mixed state can be expressed as an affine combination, as a pseudo-probability.

Pseudo-probability distributions, then, require the additional vector space structure. However, we found no physical justification to require the additional structure, and we suspect that it may need to be relaxed in future physical theories.  Therefore it is important that our fundamental definitions require only the convex structure, and not the vector space structure.

\subsection{Entropy}
The last requirement is that each ensemble must have a well defined entropy. That is, there is a scalar function $S : \Ens \to \mathbb{R}$ that satisfies the following
\begin{enumerate}
	\item strictly convex: $S(p \ens[a] + (1-p) \ens[b]) - (p S(\ens[a]) + (1-p) S(\ens[b]) ) \geq 0$
	\item upper bounded: $S(p \ens[a] + (1-p) \ens[b]) - (p S(\ens[a]) + (1-p) S(\ens[b]) ) \leq - p \log p - (1-p) \log(1-p)$
\end{enumerate}
The first tells us that, during mixing, the entropy cannot decrease, and it stays the same if and only we are mixing an ensemble with itself. The second tells us that the most the entropy can increase is given by the choice between the two ensembles.\footnote{The use of the Shannon entropy for the upper bound can likely be recovered.}

The entropy is ultimately responsible for all geometrical structure in physics. In classical mechanics, the geometry is essentially defined by volumes in phase space. The entropy of uniform distributions is given by the logarithm of the volume, meaning that given the volumes we are able to calculate the entropy and given the entropy we are able to reconstruct the volume. In quantum mechanics, the inner product (i.e. the Born rule) allows us to calculate the entropy of the mixture of two pure states, and given that entropy we can recover the Born rule.

An interesting insight is that the entropy, being strictly concave, has a negative defined Hessian. The negation of the Hessian, then, is a symmetric positive defined function of two infinitesimal variations, and can serve as a metric tensor. In both classical and quantum mechanics this recovers the Fisher-Rao metric.

Another important feature is that if two ensembles saturate the upper bound, then they are orthogonal in both classical and quantum mechanics. Therefore we define two ensembles to be \textbf{orthogonal} if their mixture lead to a maximal entropy increase. If two states are orthogonal, on physical grounds, the must have no common component. However, the reverse is not case: two pure states in quantum mechanics have no components to have in common, but are not necessarily orthogonal. However, classical probability spaces are exactly those space in which no common component implies orthogonality. This is the main source of the ``weirdness'' of quantum mechanics.

\subsection{Interplay between structures}

Each structure is important, but the balance between them is even more important. The key is to understand, on physical grounds, where each problem lies and therefore use the appropriate tool. For example, one may focus on the convex/linear structure and may be tempted to close the space under infinite convex/linear combinations. Mathematically, it would be very convenient. However, not all infinite mixtures correspond to physically meaningful systems, such as a particle with an infinite average position. It is the connection to experimental verification, the topology, that will tell us which limits are allowed and which are not. This type of conceptual clarity has to come first. The math follows. A full examination of these issues is beyond the scope of this article, and we refer to our larger, and future, work. However, there is one key issue that we need to discuss as, we found, it was the main conceptual obstacle to progress.

On the convex structure we can define \textbf{pure states} as the ones that cannot be decomposed into other ensembles. For example, the pure states for a six-sided die are the ensemble perfectly prepared respectively on the six possible outcomes. All ensembles are convex combinations $\sum_{i+1}^{6} p_i = 1$ of those pure states. Mathematically, the pure states are the extreme points of the space and all other points can be understood as convex combinations over them. This is exactly how classical discrete spaces work. Most physicists, including are older selves, think that that's how everything works. For example, in classical mechanics the pure states are perfect preparations of position and momentum, and ensembles are probability distributions. This does not work.

We cannot consistently prepare classical particles perfectly, therefore the points of phase space are not ensemble. The mathematical facts that delta Dirac are not proper functions, have support with empty interior, have minus infinite entropy, and so on, are simply a consequence of our conceptual problem. A bit better is to think them as the limit of infinitesimal decomposition: classical ensembles can be decomposed into more and more precise ones, and the points of are the limit of this decomposition. Probability distributions are defined on these limits. This works in classical mechanics, but in general.

In quantum mechanics we have pure states, and these are ensembles that cannot be further decomposed into finer ensembles. Therefore an eigenstate of position is not a limit of decomposition, a limit in the convex structure. It is the limit where we are preparing a pure state on a smaller and smaller region of space. It is the limit of a sequence of pure states, a limit in the topology. Physically, more than mathematically, it is a completely different thing, therefore it is not a good route to generalization.\footnote{The fact that pure states can be seen, in some sense, as superposition of eigenstates of position, makes the matter even more confusing.}

The insight is that probability distributions for a quantity are not defined by statistical mixtures of perfect preparations (i.e. a convex combination of extreme points), but by assigning a probability over each experimentally verifiable (i.e. open) interval of the possible values of said quantity (its spectra). The two coincide in classical discrete space, which is the least physically interesting case as it does not even allow for classical particle mechanics. 

To recap, the topological structure captures experimental verifiability and is responsible for the limits and the connection to measureable quantities. The convex structure captures statistical mixing and is responsible for all linear structures of physical theories. The entropic structure captures the variability within an ensemble and is ultimately responsible for all geometric structure. We now concentrate on recovering the measure theoretic structures associated to probability and state counting.

\section{Fraction capacity}

Note that the mixing coefficient $p$ is connected to probability but it is not itself probability. It represents a parameter for the mixing states, mixing preparations, while probability is typically understood as the chance of something happening, the likelihood of a final event. Mathematically, the probability measure is a function from an event, which can be understood as a collection of fully distinct outcomes, to a number from zero to one. If we are mixing ensemble that represent fully distinct outcomes (i.e. they are orthogonal), then the mixing $p$ is a probability. If we are mixing ensemble that do not represent fully distinct outcomes (e.g. spin up and spin left), the probability of getting one of the outcomes is not the mixing coefficient $p$.

This is exactly what happens in quantum mechanics: probability is defined only on a set of orthogonal states. If we focus only on preparation, then, we only talk about mixing coefficients across non-orthogonal elements and define a more general tool that, when restricted on orthogonal sets, will recover probability.

We concentrate on the following question: given an ensemble $\ens \in \Ens$ and a (Borel) set of ensembles $A \subset \Ens$, what fraction of $\ens$ can be constructed by a mixture of $A$? How much of $\ens$ can be explained as coming from preparations corresponding to $A$? For example, if $\ens$ is a uniform distribution over a fair six-sided die, and $A$ contains four ensemble, each fully on ${1,2}$ and uniform distribution over ${5,6}$ two and four, than $5/6$ of $\ens$ can be constructed from $A$, which is the probability that we d

Given a target ensemble $\ens \in \Ens$ and an arbitrary ensemble $\ens[a] \in \Ens$ we will be able to write $\ens = p \ens[a] + (1-p) \ens[b]$ for some $p \in [0,1]$ and some $\ens[b] \in \Ens$. At the very least, we can use $p=0$ and $\ens[b] = \ens$. Given that $p$ is bounded, there will be a biggest $p$ possible which we call the \textbf{fraction} of $\ens[a]$ in $\ens$. Mathematically
\begin{equation}
	\fraction_{\ens}(\ens[a]) = \sup(\{ p \in [0,1] \, | \, \exists \, \ens_1 \in \Ens \text{ s.t. }  \ens = p \ens[a] + \bar{p} \ens_1 \}).
\end{equation}
This quantity tells us how much of ensemble $\ens$ can be constructed, can be characterized, by $\ens[a]$. 

We now extend this idea from a single ensemble $\ens[a] \in \Ens$ to a set of ensembles $A \subset \Ens$., meaning we want to characterize how much of a target ensemble $\ens$ can be constructed as a mixture of ensembles from $A$. Given a set of ensembles $A$, its \textbf{hull}, noted $\hull(A)$ is the set of all mixtures that can be constructed from $A$. That is, is the set of all convex combinations $\sum_i^n p_i \ens[a]_i$ such that $\{a_i\}_1^n \subseteq A$. The \textbf{fraction capacity} is the biggest fraction among all the elements of the hull. Mathematically
\begin{equation}
	\frcap_{\ens}(A) = \sup(\fraction_{\ens}(\hull(A))\cup\{0\}).
\end{equation}

If $\Ens$ is a topological space, we can imagine the fraction capacity to be defined on the sigma algebra $\Sigma_{\Ens}$. We can then show that the fraction capacity $\frcap_{\ens} : \Sigma_{\Ens} \to [0,1]$ is a set function that satisfies the following:
\begin{enumerate}
	\item non-negative and unit bounded - $0 \leq \frcap_{\ens}(A) \leq 1$
	\item monotone - $A \subseteq B \implies \frcap_{\ens}(A) \leq \frcap_{\ens}(B)$
	\item sub-additive - $\frcap_{\ens}(A \cup B) \leq \frcap_{\ens}(A) + \frcap_{\ens}(B)$.
	\item continuous from below and above - $\frcap_{\ens}(\lim\limits_{i \to \infty} A_i) = \lim\limits_{i \to \infty} \frcap_{\ens}(A_i)$ for any increasing or decreasing sequence $\{A_i\}$.
\end{enumerate}



\section{Statistical properties}

\section{Beyond real valued quantities}


\section{State capacity}

\section{Quantization}
* 3 pick 2

\section{Quantizing space-time}
* we need a non-additive measure on degrees of freedom
*

\section{Conclusion}



\section*{Acknowledgments}
This paper is part of the ongoing \textit{Assumptions of Physics} project \cite{aop-book}, which aims to identify a handful of physical principles from which the basic laws can be rigorously derived. This article was made possible through the support of grant \#62847 from the John Templeton Foundation.


\bibliography{bibliography}

\newcommand{\pj}[1] {\underbar{$#1$}}


\end{document}
