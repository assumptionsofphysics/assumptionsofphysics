\documentclass[10pt,twocolumn, nofootinbib]{revtex4-2}

\usepackage{assumptionsofphysics}
\usepackage{tikz}
\usepackage{breakurl}
\usepackage{tcolorbox}

\newcommand\hull{\mathrm{hull}}
\newcommand\stcap{\mathrm{scap}}
\newcommand\fraction{\mathrm{frac}}
\newcommand\frcap{\mathrm{fcap}}

\newcommand{\ens}[1][e] {\mathsf{#1}} % Ensemble
\newcommand{\Ens}[1][E] {\mathcal{#1}} % Ensemble space

\def\ortho{\perp}
\def\northo{\nperp}
\def\separate{\downmodels}
\def\nseparate{\ndownmodels}

\def\>{\rangle}
\def\<{\langle}

\begin{document}

\title{A non-additive generalization of probability theory \\for quantum mechanics and beyond}
\author{Gabriele Carcassi}
\affiliation{Physics Department, University of Michigan, Ann Arbor, MI 48109}
\author{Christine A. Aidala}
\affiliation{Physics Department, University of Michigan, Ann Arbor, MI 48109}

\date{\today}


\begin{abstract}
	We present a physically motivated generalization of probability theory that is suitable for classical mechanics, quantum mechanics and any future physical theory that allows a statistical description. The goal is to put the use of classical and quantum probability in a broader context, and show how the current mathematical structures are likely not suitable to solve the open problems in the foundations of physics. For the more mathematically inclined, we will point to areas where new math or generalization of established math are needed. For the more philosophically incline, we will point to areas where further conceptual work is needed.
	
	
	%Given a generic space of ensembles, we can define the fraction capacity as the maximum fraction of a particular ensemble that can be understood as a mixture of ensembles from a given set. This gives a non-additive (i.e. fuzzy) measure that reduces to a probability (i.e. additive) measure for classical spaces and for quantum measurements. We can also define the state capacity as the exponential of the maximum entropy reachable by a mixture of ensembles from a given set. This is also a non-additive measure and reduces to the Liouville measure in classical mechanics and the Hilbert space dimensions for subspaces of quantum mechanics. Conceptually, it gives us a notion of probability that is both theory and interpretation independent. Mathematically, it gives us a measure theoretic generalization of probability. Physically, it may allow to find new way to understand current theory and tool to investigate new ones. The purpose of this paper is to show the core ideas and present questions that may be developed on the mathematical, physical and philosophical side.
\end{abstract}

\maketitle


\section{Introduction}

Our current understanding of probability is given in terms of the measure theoretic formalization due to Kolmogorov, and its various philosophical interpretations. However, probability in quantum mechanics, as it is widely known, is not Kolmogorovian, in the sense that it provides results that are not reproducible in standard probability spaces. It follows that our mathematical axioms for probability and their interpretations are insufficient for physical theories.

An alternative measure theoretic strategy, that has been successful is quantum mechanics, is the use of quasi-probability distributions: measures that are still unitary over the whole space, but can have negative values in small regions. However, these haven't resulted in an alternative axiomatic approach with suitable interpretations for at least, in our mind, three reasons. First, it is not clear how negative probability should be interpreted. Second, it is not clear how to characterize the space of valid quasi-probability distributions (e.g. there is no simple rule to specify which $\rho(q,p)$ are valid Wigner functions). Third, there are multiple options for quasi-probability distributions (e.g. Wigner function, Husimi distribution, Sudarshan–Glauber distribution). This lack of foundational clarity presents an additional problem: how do we know whether future physical theories will not require yet another extension?

The goal of this paper is to show that, in the same way that classical probability can be axiomatized through standard measure theory, quantum probability, and in fact any probability used a physical theory, can be axiomatized through non-additive (i.e. fuzzy) measure theory. We will define a generalized concept of probability that is ultimately based on physically motivated axioms, giving the physicist a clear operational motivation to our definitions, the mathematicians a formally well-defined mathematical structure and the philosopher a unified framework in which probability can be interpreted in the context of all physical theories. We are not going to be able to develop a full theory here, as there are still a number of conceptual and technical challenges to be solved that will require work in physics, mathematics and philosophy. However, we will show that there are enough results that point in the same direction, and therefore make it reasonable to invest time and effort to work on the open problems.

The material presented is part of a larger work that aims to coalesce ideas and results from different branch of mathematics and physics. For those interested, the current state of the research is openly available in an open document that will be routinely updated as experts from different fields review and help us improve the overall theory.

In a nutshell, we will start by arguing that physical theory must at least be able to talk about statistical ensembles, and that ensembles must allow the preparation of statistical mixture. Given a target ensemble $\ens$ and a set of ensemble $A$, we define the fraction capacity of $A$ for $\ens$ as the biggest fraction of $\ens$ that can be constructed from another set of ensembles $A$. That is, the highest $p$ such that we can express our ensemble as $\ens = p \ens[a] + (1-p) \ens[b]$, where $\ens[a]$ is a mixture of elements of $A$. For example, if $\ens$ is the uniform distributions for a roll of a six-faced die, and $A$ includes only two ensembles with $100\%$ probability for the outcomes $3$ and $4$ respectively, then the faction capacity of $A$ for $\ens$ will be $\frac{1}{3}$. Note that this coincides with the probability of the event ``$3$ or $4$''. The fraction capacity is a non-negative, unit bounded, sub-additive and continuous set function, which is therefore like a probability measure except for the additivity, which is recovered for classical mechanics and over quantum measurements.

Similarly, given a set $A$ of ensembles we define the \textbf{state capacity} as the exponential of the highest entropy obtainable with a mixture of $A$. Note that the highest entropy of a subspace in both classical and quantum statistical mechanics is given by the uniform distribution, therefore the state capacity recovers the count of states (i.e. the Liouville volume) in classical mechanics and the dimensionality of the Hilbert space in quantum mechanics. The state capacity is also a non-negative, sub-additive and continuous set function. This points to a possible generalization of measure theoretic calculus to a sub-additive case that would be guaranteed to work on all physical theories.

\section{Quick review of probability theory}

The standard way to characterize probability is through measure theory. Given a sample space $\Omega$, which represents the possible outcomes, and a $\sigma$-algebra $\Sigma_{\Omega}$, which represents the events, we define a probability measure $p : \Sigma_{\Omega} \to [0,1]$. Note that the probability is assigned to events, not the outcomes. This is what allows to treat discrete and continuous spaces within the same framework. In physics, all sample spaces are ultimately topological spaces, and the $\sigma$-algebra is typically the Borel algebra (i.e. the smallest $\sigma$-algebra that contains all open sets).

In classical mechanics, the sample space is phase space $M$ given by all possible values of position and momentum for all particles under considerations. Phase space is equipped with another measure, the Liouville measure $\mu : \Sigma_M \to [0, +\infty]$, that returns volume of each region. In statistical mechanics, this volume represents the count of states and is another key ingredient. The probability density over states is given by the Radon-Nikodym derivative $\rho = \frac{dp}{d\mu}$ and the entropy is given by $S(\rho) = - \int_M \rho \log \rho d\mu$. In case of a uniform distribution $\rho_U$ with support $U$, the entropy is given by $S(\rho_U) = \log \mu(U)$.\footnote{Note that entropy, the count of states and the symplectic form that characterizes phase space are deeply intertwined, yet they require different mathematical framework to express them. We suspect that they are "shadows" of a more fundamental mathematical structure, which our work on ensembles space is trying to identify.}

The above definitions do not apply to quantum mechanics. In the Hilbert space formulation, mixed states are represented by density operator (i.e. positive semi-definite self-adjoint operators with trace one). Note that if $\rho(x)$ is an integrable function such that $\int_X \rho(x) dx = 1$, the map $\psi(x) \mapsto \rho(x) \psi(x)$ is a density operator. Therefore the space of probability densities over position are mixed states in quantum mechanics. However, no joint distribution between position and momentum can be given, therefore quantum states cannot be described by the space of probability measures over a single sample space.

Alternative representations over phase space do exist, and they are widely used.(CITE) The use quasi-probability measure, meaning that the measure over the full space is still unitary (i.e. $\mu(M) =1$). However, the measure can be negative for smaller regions, which means it can also be grater than $1$, and there is no simple characterization of which quasi-probability measures are allowed and which aren't. Moreover, there is no simple link between then entropy of the distribution and the Liouville measure.

\section{Ensemble spaces}

Note that our goal is not to construct an abstract probability theory that applies to everything, but rather to construct a specific probability theory that is guaranteed to apply to all physical systems. Therefore we do not to define an axiomatic theory of probability and apply it to physical systems, but rather characterize basic properties of physical states and from those build a theory of probability that fits those definitions. Whether it is applicable to decision theory, economics, or psychology is out of scope. As is common in our larger project, we start from axioms that, in our opinion, represent requirements of scientific practice. We call these \textbf{constitutive assumptions} as they are essential to proceed with the scientific endeavor. The advantage is that the realm of applicability and physical interpretation will be clear from the start. 

We start from requiring a physical theory to allow statistical descriptions: at the very least, it will provide a notion of ensemble and describe which ensembles that are allowed by the theory. An ensemble can be understood as the infinite collection of the outputs of a particular preparation procedure. The ensemble space of classical mechanics, for example, is the space of probability distributions on phase space. In quantum mechanics, instead, it is the space of density operators. While the ensemble space for each theory will in principle be different, they will all share common features: they must allow measurable properties of the system, probability distributions for outcomes of said properties, characterize reversible and irreversible processes through an entropy function, and so on. We are looking for necessary minimal basic requirements upon which these common tools can be built.

At least two arguments can be made to show that any physical theory must allow statistical descriptions, i.e.~provide an ensemble space. First, note that physical laws are never about specific instances but rather regularities. They are statements of the type ``whenever we prepare this, we can measure that.'' This means that, ultimately, they are about ensembles. Second, if a physical theory is to be testable experimentally, and repeatedly so, it must be in terms of ensembles because this is what we prepare in practice and this is what, in the end, we characterize experimentally. Moreover, repeatability requires the ability to always test ``one more time,'' which implies that ensembles are infinite collections, and therefore idealization. The approach, then, can most likely be given firm conceptual grounds, which we leave as an open but likely solvable philosophical problem.

We have identified three basic requirements on ensembles: first, they have to be identifiable experimentally; second, they need to allow statistical mixtures; lastly, they need a well-defined entropy. Let us go through these items one by one. We leave the full details to the working document (TODO: decide how to call it).

\subsection{Experimental verifiability}
The first requirement is that ensembles must be connected to experimental verification. That is, we must have enough \textbf{verifiable statements} at our disposal to define ensembles and tell them apart. By verifiable statement we mean a statement for which an experimental test is available that terminates in finite time if and only if the statement is true. For example, the statement ``the mass of the photon is less than $10^{-13} \, eV$'' is verifiable, while ``the mass of the photon is exactly $0 \, eV$'' is not verifiable due to its infinite precision. Based on previous work, (CITE) this mathematically requires an ensemble space to be a $T_0$ second-countable topological space, where open sets represent verifiable statements and Borel sets represent statements associated with a test, but without guarantee of termination.

The topological structure, then, represents the most foundational structure for a physical theory. It tells us why functions are ``well-behaved,'' which simply means they have to be topologically continuous in the ``natural topology'' induced by this requirement. It tells why probability is assigned to Borel sets (i.e. all statements associated with tests) and why the Banach-Tarski paradox does not apply in physics (i.e. non-Borel sets are physically ill-defined as they are not connected to experimental verification). Similarly, it tells us that sets with cardinality greater than that of the continuum are not physically relevant, as they cannot be given a $T_0$ second-countable topology.

In the case of ensembles, typical verifiable statements will be ``the average energy of the particle is between 3 and 4 $eV$,'' ``the probability of getting heads is between 49 and 51 percent,'' ``the probability distribution for the position is a Gaussian of mean 0 $m$ and 1 $m$ standard deviation within 1\%.'' Note that these are not results of single-shot measurements, and therefore verifiable statements are distinctly different from observables in quantum mechanics. In the same manner, these are not measurements that extract one bit of information, as there is no requirement of termination in the negative case.

\begin{tcolorbox}[colback=white, colframe=black]
	Experimental verifiability $\Rightarrow$ An ensemble space $\Ens$ is a $T_0$ second-countable topological space.
\end{tcolorbox}


\subsection{Statistical mixtures}
Another basic requirement is the ability to prepare a \textbf{mixture} of two ensembles. If $\ens[a], \ens[b] \in \Ens$ are two ensembles, and $p \in [0,1]$ a weight, then $\ens = p \ens[a] + (1-p) \ens[b] \in \Ens$ is the ensemble that describes a process that selects the first ensemble over the second $p$ percent of the times. Mathematically, the ensemble space is endowed with a \textbf{convex structure}. Moreover, since the mixing operation must be consistent with experimental verifiability, it will be topologically continuous, and the ensemble space will be a topological convex space.

Ultimately, the convex structure is responsible for all linear structures we have in physics. In face, the entropy and its bounds, defined later, forces the ensemble space to be the subset of a convex subset of a vector space. That is, the convex structure is ``invertible.''\footnote{If we fix $\ens$, $\ens[a]$ and $p$ then, if it exists, there is only one $\ens[b]$ such that $\ens = p \ens[a] + (1-p) \ens[b]$. In the context of convex spaces, this property is called ``cancellative.''} What is still not clear is whether the ensemble space embeds continuously in a topological vector space. While topological vector spaces are well established in the literature, topological convex spaces are not. 

Regardless of the topology, we are already equipped to understand what negative probability is. Suppose we have a real vector space of dimensions $n$, then $n$ linearly independent vectors are enough to express every point as a linear combination. We can also take an additional vector and express every point as an affine combination, that is a linear combination where the coefficients sum to one, though some can still be negative. Since every ensemble space, including the space of mixed states in quantum mechanics, is a subset of a real vector space, we can choose a linearly dependent set of states that span the whole space, and every mixed state can be expressed as an affine combination, as a pseudo-probability. However, the boundaries of the ensemble space will determine which affine combinations are valid states (i.e. they end up within the ensemble space), and there is no requirement that these boundary are conveniently expressible in terms of affine combinations.

The convex structure, however, allows us to ask whether two ensembles have a \textbf{common component}. That is, then can be seen as different statistical mixture that have some part in common. Two ensemble are separate, noted $\ens[a] \separate \ens[b]$, if they do not have a common component. This allows to characterize the space in terms of statistical decomposition, and see how this differs from other decompositions.

\begin{tcolorbox}[colback=white, colframe=black]
	Statistical mixture of ensembles $\Rightarrow$ An ensemble space $\Ens$ is a topological convex space.
\end{tcolorbox}

\subsection{Entropy}
The last requirement is that each ensemble must have a well defined \textbf{entropy}. That is, there is a scalar function $S : \Ens \to \mathbb{R}$ that satisfies the following:
\begin{enumerate}
	\item strictly concave: $S(p \ens[a] + (1-p) \ens[b]) - (p S(\ens[a]) + (1-p) S(\ens[b]) ) \geq 0$
	\item bounded from above: $S(p \ens[a] + (1-p) \ens[b]) - (p S(\ens[a]) + (1-p) S(\ens[b]) ) \leq - p \log p - (1-p) \log(1-p)$
\end{enumerate}
The first tells us that, during mixing, the entropy cannot decrease, and it stays the same if and only we are mixing an ensemble with itself. The second tells us that the most the entropy can increase is given by the choice between the two ensembles.\footnote{The expression for the Shannon entropy is actually derived instead of imposed axiomatically.} We define two ensembles to be \textbf{orthogonal} when they maximally increase the entropy during mixing. This recovers the standard notion of orthogonality: two classical ensembles are orthogonal when they have disjoint support, and therefore $\int_M \rho_1 \rho_2 d\mu = $; two quantum ensembles are orthogonal when they are defined on orthogonal subspaces, and therefore $\tr(\rho_1 \rho_2) = 0$.

The entropy is ultimately responsible for all geometrical structure in physics. In classical mechanics, the geometry is essentially defined by phase space volumes and areas in each degree of freedom (DOF). The entropy of uniform distributions is given by the logarithm of the volume, meaning that given the volume we are able to calculate the entropy and given the entropy we are able to reconstruct the volume. Similarly, uniform marginals will recover DOF areas. In quantum mechanics, the inner product (i.e. the Born rule) allows us to calculate the entropy of the mixture of two pure states, and given the entropy we can recover the Born rule.

An interesting insight is that the entropy, being strictly concave, has a negative defined Hessian. The negation of the Hessian, then, is a symmetric positive defined function of two infinitesimal variations, and can serve as a metric tensor on the affine structure given by the mixing coefficients. In both classical and quantum mechanics this recovers the Fisher-Rao metric.

Another crucial feature is that two ensembles maximize the entropy increase if and only if they are orthogonal: two classical ensembles maximize the entropy increase when they have disjoint support, and therefore $\int_M \rho_1 \rho_2 d\mu = 0$; two quantum ensembles when they are defined on orthogonal subspaces, and therefore $\tr(\rho_1 \rho_2) = 0$. This corresponds to the case when the two ensemble are made of instances that are mutually exclusive. Therefore we define two ensembles to be \textbf{orthogonal}, noted $\ens[a] \ortho \ens[b]$, if their mixture leads to a maximal entropy increase. The following additional axiom is valid for orthogonal ensembles
\begin{enumerate}
	\setcounter{enumi}{2}
	\item mixtures preserve orthogonality: $\ens[a] \ortho \ens[b]$ and $\ens[a] \ortho \ens[c]$ if and only if $\ens[a] \ortho p \ens[b] + \bar{p} \ens[c]$ for any $p \in (0,1)$
\end{enumerate}
as mixing elements distinguishable from those of ensemble $\ens[a]$, still gives elements distinguishable from ensemble $\ens[a]$.

It can be shows that orthogonal ensembles are separate, but the converse is not true. In quantum mechanics, for example, all pure states are separate (they cannot be decomposed and therefore cannot have components in common) but not all are orthogonal. In classical spaces, however, all separate ensembles are also orthogonal. Understanding the relationships between these two properties, then, is key in understanding the difference between classical and quantum mechanics, and what can happen in other physical theories.

\begin{tcolorbox}[colback=white, colframe=black]
	Existence of entropy $\Rightarrow$ Strictly concave function whose upper bound defines orthogonality. Statistical mixtures preserve orthogonality.
\end{tcolorbox}

\subsection{Interplay between structures}

Before introducing the actual measures, we want to note that there is a delicate balance between these structures which is important at both a conceptual level and at a more mathematical level. Part of the difficulty is to recognize which standard modes of understanding the problem should be kept, as the they generalize nicely, and which should be discarded. Let us go through some key examples.

Those that focus on the convex/linear structure may be tempted to close the space under infinite convex/linear combinations. It would be very convenient mathematically, and, in fact, this is what happens in Hilbert spaces. However, this closure leads yo physically ill-defined systems, such as a particle with an infinite average position or undefined energy.(CITE Hilbert) It is the connection to experimental verification, the topology, that will tell us which limits are allowed and which are not. Therefore, the lack of infinite convex is a crucial feature of the space, and convex sets in ensemble spaces are to be understood as closed convex sets, so that closure on the topology includes those, and only those, infinite mixtures that are physically meaningful.

If we focus on probability, we are tempted to extend the intuition from classical mechanics and it is useful to understand exactly why and where it fails. On a convex structure we can define the extreme points, those that cannot be further decomposed. Therefore it is tempting to define \textbf{pure states} as those ensembles that cannot be further decomposed into other ensembles. For example, the pure states for a six-sided die are the ensembles perfectly prepared respectively on the six possible outcomes. All ensembles are convex combinations $\sum_{i=1}^{6} p_i \ens_i = 1$ of those pure states $\ens_i$. This is exactly how classical discrete spaces work and much of our intuition for probability comes from there.

This approach, however, already becomes problematic in classical mechanics. Since we can take uniform distributions over any region of phase space, we can imagine to take a sequence of more and more refined distributions around a value of position and momentum. Therefore we may be tempted to say that the points of phase space represent perfectly prepared systems, and every other probability is a distribution on these space. This is physically unsound. First of all, we cannot consistently prepare position and momentum with infinite precision, therefore those ensembles do not exist even exist as an idealization. Secondly, the entropy of those ensembles, if they existed, would be minus infinity, which is a problem for any thermodynamic treatment. To make this intuition consistent with the math would require each point in phase space to be topologically isolated, which would extend the $\sigma$-algebra to the power set of $\mathbb{R}^n$, which means there would be no consistent way to define a volume. The fact that the math breaks down is simply a symptom that our physical problem is ill-specified.

In quantum mechanics, things are even worse. Pure states are indeed extreme points of the convex set. However, we also have discrete spectra. Here the temptation is, like we ``did'' in classical mechanics, to consider wave-functions all concentrated at a single value of position. While one may think this is the analogue of classical mechanics, it is not. We can, in fact, take a wave-function that is uniformly distributed on every finite range of position, and we can imagine to shrink that range. However, this is a sequence of pure states. All pure states have zero entropy, and the reduced spread on position will mean an increased spread on momentum. That is, we are not providing a sequence of ensembles at ever greater prevision, like in the classical case. In other words, the eigenstates of position in quantum mechanics, even if we wanted to say they were actual states, they would not be the limit of infinitesimal decomposition (i.e. a limit in the convex structure) but rather the limit of a path over pure states (i.e. a topological limit).

The insight is that probability distributions for a quantity are not defined by statistical mixtures of perfect preparations (i.e. convex combinations of extreme points), but by assigning a probability over each experimentally verifiable (i.e. open) interval of the possible values of said quantity (its spectrum). The two coincide in classical discrete spaces, which are the least physically interesting case as they do not even allow classical particle mechanics. Equivalently, \textbf{mixtures are about preparations, not measurements, and probability distributions are about measurements, not preparations}. In quantum mechanics, an equal mixture of spin up and spin left, for example, does not correspond to a 50\% probability of measuring spin up: the mixing coefficients and the outcome probability are different.

To recap, the topological structure captures experimental verifiability and is responsible for the limits and the connection to measureable quantities. The convex structure captures statistical mixing and is responsible for all linear structures of physical theories. The entropic structure captures the variability within an ensemble and is ultimately responsible for all geometric structures. Since the physical problem is indivisible, there is an interplay across these structure which goes beyond this article and we refer to our larger work.

\section{Fraction capacity}

We have clarified that mixing coefficients are not the same as probabilities of outcomes. Unlike most of the literature on probability, we will concentrate on mixtures and, particularly, on the following question: given an ensemble $\ens \in \Ens$ and a (Borel) set of ensembles $A \subseteq \Ens$, what fraction of $\ens$ can be constructed by a mixture of $A$? How much of $\ens$ can be explained as coming from preparations corresponding to $A$? For example, let $\Ens$ be the space of all probability distributions for a six-sided die. Let $\ens_{123456}$ be the uniform distribution over all outcomes. Let $A_1 = \{\ens_1\}$ where $\ens_1$ represents outcome one with 100\% probability. Since we can write $\ens_{123456} = \frac{1}{6} \ens_{1} + \frac{5}{6} \ens_{23456}$, where $\ens_{23456}$ represents a uniform distributions over the five outcomes, $1/6$ of $\ens$ can be constructed from $A$, but no more. Similarly, if $A_{12} = \{\ens_{1},\ens_{2}\}$, we can write $\ens = \frac{1}{3} \left(\frac{1}{2} \ens_1 + \frac{1}{2} \ens_2 \right)  + \frac{2}{3} \ens_{3456}$, so $1/3$ of $\ens$ can be constructed from $A$, but no more. Note that, given the uniform distribution, $1/6$ is exactly the probably for the event $A_1$ and $1/3$ the probability for event $A_{12}$. This gives the basic insight for our definition.

Given a target ensemble $\ens \in \Ens$ and an arbitrary ensemble $\ens[a] \in \Ens$, we define the \textbf{fraction} of $\ens[a]$ in $\ens$ to be
\begin{equation}
	\fraction_{\ens}(\ens[a]) = \sup(\{ p \in [0,1] \, | \, \exists \, \ens_1 \in \Ens \text{ s.t. }  \ens = p \ens[a] + \bar{p} \ens_1 \}).
\end{equation}
The fraction is always well-defined because we can always write $\ens = 0 \ens[a] + 1 \ens$, therefore it must be zero or greater. This quantity tells us how much of ensemble $\ens$ can be constructed from $\ens[a]$. 

We now extend this idea from a single ensemble $\ens[a] \in \Ens$ to a Borel set of ensembles $A \subset \Ens$. Recalling that the $\hull(A)$ is the set of all possible convex combinations (i.e. mixtures), we define the\textbf{fraction capacity} of $A$ for $\ens$ to be
\begin{equation}
	\frcap_{\ens}(A) = \sup(\fraction_{\ens}(\hull(A))\cup\{0\}).
\end{equation}
This returns the biggest fraction of $\ens$ that can be achieved with a mixture of elements of $A$.\footnote{The name fraction capacity is chosen to both signify the ability of the set $A$ to contain $\ens$ (i.e. fraction capacity of one means $\ens$ is within the convex combinations of $A$) and to indicate that it will be a non-additive monotonic measure, which are called ``capacities'' in some literature. TODO \href{ https://link.springer.com/book/10.1007/978-3-319-03155-2}{cite} }

One can then that the fraction capacity $\frcap_{\ens} : \Sigma_{\Ens} \to [0,1]$ is a set function that satisfies the following:
\begin{enumerate}
	\item non-negative and unit bounded - $0 \leq \frcap_{\ens}(A) \leq 1$
	\item monotone - $A \subseteq B \implies \frcap_{\ens}(A) \leq \frcap_{\ens}(B)$
	\item sub-additive - $\frcap_{\ens}(A \cup B) \leq \frcap_{\ens}(A) + \frcap_{\ens}(B)$
	\item continuous from below and above - $\frcap_{\ens}(\lim\limits_{i \to \infty} A_i) = \lim\limits_{i \to \infty} \frcap_{\ens}(A_i)$ for any increasing or decreasing sequence $\{A_i\}$.
\end{enumerate}

The first property readily comes from the domain of the fraction. For the second, note that $\hull$ and $\sup$ are both monotone. For sub-additivity, note that the fraction of two components can at most sum during mixing. For the last, increasing and decreasing sequences strictly add or remove possible mixtures, therefore they will lead to increasing or decreasing sequences of real numbers bounded between zero and one. The limit of these sequences must agree with the fraction capacity of the limit.

In the classical case, the fraction capacity reduces to a probability measure. Note that a probability satisfies all the properties of a fraction capacity, but it also satisfy on additional property: additivity over disjoint sets. That is, if $A \cup B = \emptyset$ then $\mu(A \cup B) = \mu(A) + \mu(B)$. Adding finite additivity over disjoint sets is enough as continuity will extended the property to additivity over countably many disjoint sets. Conversely, having countable additivity over disjoint sets implies sub-additivity and continuity from below and above. 



While additivity over disjoint set holds in classical probability, it does not hold in quantum mechanics

 This is not true in general, as we can see in a simple example from quantum mechanics. Let $\Ens$ be the ensemble space of a qubit, that is the space of density matrices of a two dimensional Hilbert space. Geometrically, this is the convex set represented by a three dimensional ball. Let $\ens_{\psi} = |\psi\>\<\psi|$ be a pure state and let $\ens_{\phi} = |\phi\>\<\phi|$ be its orthogonal. Geometrically, these are two opposite points on the surface of the ball. The maximally mixed state can be expressed as $\ens = \frac{1}{2} \ens_0 + \frac{1}{2} \ens_1$. This is the center point of the ball and is in fact the midpoint between two opposite points. Since this is valid for any point, we have $\frcap_{\ens}(\{\ens_{\psi}\})=\frac{1}{2}$ for any $\ens_{\psi} \in \Ens$ and $\frcap_{\ens}(\{\ens_{\psi}, \ens_{\phi}\})=1$ for any pair of orthogonal pure states. If $A$ is a set of two orthogonal pure states, and $B$ is state different from the two, we have $\frcap_{\ens}(A \cup B) = 1 < \frcap_{\ens}(A) + \frcap_{\ens}(B) = \frac{3}{2}$.

If we restrict ourselves to lattice of subsets of $\Ens$ we can recover additivity. For example, if we take a family $A_i$ of orthogonal sets (i.e. all the elements of one set are orthogonal to all the elements of any other set) and take an element that is a mixture of elements from that family, then the fraction capacity will be additive over the lattice formed by closing the family under union and intersection. In quantum mechanics, this would correspond to all probability distributions that are outcomes of the projective measurement that distinguishes between $A_i$. Part of the task, then, will be finding a set of necessary and sufficient conditions under which the fraction capacity for a particular ensemble is additive over a particular lattice of subsets

What is important here is that 


\section{Statistical properties}

\section{Beyond real valued quantities}


\section{State capacity}

\section{Quantization}
* 3 pick 2

\section{Quantizing space-time}
* we need a non-additive measure on degrees of freedom
*

\section{Conclusion}



\section*{Acknowledgments}
This paper is part of the ongoing \textit{Assumptions of Physics} project \cite{aop-book}, which aims to identify a handful of physical principles from which the basic laws can be rigorously derived. This article was made possible through the support of grant \#62847 from the John Templeton Foundation.


\bibliography{bibliography}

\newcommand{\pj}[1] {\underbar{$#1$}}


\end{document}
