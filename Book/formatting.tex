\usepackage[top=3cm,bottom=3cm,left=3cm,right=3cm,headsep=10pt,letterpaper]{geometry} % Page margins

% Theorem definitions using amsthm

\usepackage{amsthm}
\usepackage{amsmath}
\usepackage{amssymb}

% Remove line spaces between items of enumerate and itemize
\usepackage{enumitem}
\setlist{noitemsep}


% Adds double bracket symbols
\usepackage{stmaryrd}

% Latex symbol guide at http://mirrors.ibiblio.org/CTAN/info/symbols/comprehensive/symbols-letter.pdf

% LOGIC symbols
% -------------

% Allows to create negation symbols
\usepackage{MnSymbol,centernot}

% Symbol for "implies" and "not implies"
\def\imp{\Rightarrow}
\def\nimp{\nRightarrow}

% Symbol for "compatibility" and "incompatibility"
\def\comp{\doublefrown}
\def\ncomp{\ndoublefrown}


% Symbol for "mutually exclusive"
\def\me{\asymp}

% Symbol for "not mutually exclusive"
\def\nme{\nasymp}

% Symbol for "independent"
\def\indep{\upmodels}

% Symbol for "not mutually exclusive"
\def\nindep{\nupmodels}

% Aliases for logical operations
\def\AND{\wedge}
\def\bigAND{\bigwedge}
\def\OR{\vee}
\def\bigOR{\bigvee}
\def\NOT{\neg}

% Formatting for statements
\newcommand{\stmt}[1] {\mathsf{#1}}
% Formatting for experimental tests
\newcommand{\expt}[1] {\mathsf{#1}}
% Formatting for observations
\newcommand{\obs}[1] {\mathsf{#1}}


\usepackage{xcolor} % Required for specifying colors by name
\definecolor{sectionNumbers}{RGB}{44, 103, 0}


\renewcommand\thesubsection{\thesection.\Alph{subsection}}
\renewcommand{\theequation}{\thechapter.\arabic{equation}}

\newtheorem{assump}{Assumption}
\renewcommand*{\theassump}{\Roman{assump}}

\newtheorem{defn}[equation]{Definition}
\newtheorem{prop}[equation]{Proposition}

%\theoremstyle{definition}

\newenvironment{rationale}{\emph{Rationale}.}{\qed}
\newenvironment{justification}{\emph{Justification}.}{\qed}
\renewenvironment{proof}{\emph{Proof}.}{\qed}

% Style for math section
\RequirePackage[framemethod=default]{mdframed} % Required for creating the theorem, definition, exercise and corollary boxes
\newmdenv[skipabove=7pt,
skipbelow=7pt,
rightline=false,
leftline=true,
topline=false,
bottomline=false,
linecolor=sectionNumbers,
backgroundcolor=black!2,
innerleftmargin=5pt,
innerrightmargin=5pt,
innertopmargin=5pt,
leftmargin=0cm,
rightmargin=0cm,
linewidth=4pt,
innerbottommargin=5pt]{mathSection}


%----------------------------------------------------------------------------------------
%	SECTION NUMBERING IN THE MARGIN
%----------------------------------------------------------------------------------------

\makeatletter
\renewcommand{\@seccntformat}[1]{\llap{\textcolor{sectionNumbers}{\csname the#1\endcsname}\hspace{1em}}}                    
\renewcommand{\section}{\@startsection{section}{1}{\z@}
	{-4ex \@plus -1ex \@minus -.4ex}
	{1ex \@plus.2ex }
	{\normalfont\large\sffamily\bfseries}}
\renewcommand{\subsection}{\@startsection {subsection}{2}{\z@}
	{-3ex \@plus -0.1ex \@minus -.4ex}
	{0.5ex \@plus.2ex }
	{\normalfont\sffamily\bfseries}}
\renewcommand{\subsubsection}{\@startsection {subsubsection}{3}{\z@}
	{-2ex \@plus -0.1ex \@minus -.2ex}
	{.2ex \@plus.2ex }
	{\normalfont\small\sffamily\bfseries}}                        
\renewcommand\paragraph{\@startsection{paragraph}{4}{\z@}
	{-2ex \@plus-.2ex \@minus .2ex}
	{.1ex}
	{\normalfont\small\sffamily\bfseries}}