\documentclass{article}

\usepackage{amsthm}
\usepackage{amsmath}
\usepackage{amssymb}
% Remove line spaces between items of enumerate and itemize
\usepackage{enumitem}
\setlist{noitemsep}

\newtheorem{assump}{Assumption}
\renewcommand*{\theassump}{\Roman{assump}}

\newtheorem{axiom}[equation]{Axiom}
\newtheorem{defn}[equation]{Definition}
\newtheorem{prop}[equation]{Proposition}
\newtheorem{coro}[equation]{Corollary}
\newtheorem{thrm}[equation]{Theorem}

%\theoremstyle{definition}

\newenvironment{remark}{\noindent \emph{Remark}.}{\medskip}
\newenvironment{rationale}{\emph{Rationale}.}{\qed}
\newenvironment{justification}{\emph{Justification}.}{\qed}
\renewenvironment{proof}{\emph{Proof}.}{\qed}


\title{Just the math: symplectic geometry}

\date{\vspace{-5ex}}
\begin{document}

\maketitle


\begin{abstract}
Short summary of symplectic geometry.
\end{abstract}

\section{General definitions and coordinates}

\begin{defn}
	Let $M$ be a smooth manifold. A \textbf{symplectic form} $\omega$ on $M$ is a differential two-form that is:
	\begin{itemize}
		\item \textbf{closed}: the exterior derivative $d \omega = 0$ is zero
		\item \textbf{non-degenerate}: at every point $p \in M$ $\omega|_p$ we have $v=0 \in T_PM$ if and only if $\omega(v, w)=0$ for all $w \in T_pM$
	\end{itemize}
\end{defn}

\begin{prop}
	Let $\omega$ be a symplectic form on $M$. Then $M$ is even dimensional, that is $\dim(M)=2n$.
\end{prop}

\begin{defn}
	A \textbf{symplectic manifold} is a pair $(M, \omega)$ where $M$ is a smooth manifold and $\omega$ is a symplectic form.
\end{defn}

\begin{thrm}[Darboux's theorem]
	Given a point $p \in M$ we can always find a neighborhood $U$ with local coordinates $\xi^a = \{q^i, p_j\}$ such that $\omega = e^{q^i} \wedge e^{p_i}$.
\end{thrm}

\begin{coro}
	Locally, we have $\omega = - d\theta$ where $\theta = p_i e^{q^i}$.
\end{coro}

\begin{remark}
	Globally this can fail (e.g a two dimensional sphere).
\end{remark}

\textbf{Index notations}:

\textit{Unified coordinates}. $\xi^a = \{\xi^1, \xi^2, ..., \xi^{2n}\}$.

\textit{Conjugate coordinates}. $\{q^i, p_j\} = \{q^1, q^2, ..., q^n, p_1, p_2, ..., p_n\}$.

\textit{Mixed index}. $\xi^{\alpha i } = \{\xi^{q1}, \xi^{q2}, ..., \xi^{qn}, \xi^{p1}, \xi^{p2}, ..., \xi^{pn}\}$.

\textbf{Components with different notations}:
\begin{equation}
\begin{aligned}
\theta &= \theta_a e^a = p_i e^{q^i} = \theta_{\alpha i} e^{\alpha i} \\
\theta_a &= \left[ \begin{matrix}
p_i & 0 \end{matrix} \right], \;  \theta_{qi} = p_i, \; \theta_{pi} = 0\\
\omega &= \omega_{ab} e^a \otimes e^b = \omega_{\alpha i \beta j} e^{\alpha i} \otimes e^{\beta j} \\
&= e^{q^i} \otimes e^{p_i} - e^{p_i} \otimes e^{q^i} = e^{q^i} \wedge e^{p_i} \\
\omega &= - d \theta = - \partial_b \theta_a e^b \wedge e^a = - e^{p_\alpha} \wedge e^{q^\alpha} = e^{q^\alpha} \wedge e^{p_\alpha} \\
\omega_{ab} &= \left[ \begin{matrix}
0 & I_n \\[2.2ex]
- I_n & 0 \end{matrix} \right] , \; 
\omega^{ab} = \left[ \begin{matrix}
0 & - I_n \\[2.2ex]
I_n & 0 \end{matrix} \right] , \;
\omega_{ab} \omega^{bc} = \delta_a^c \\
w_{\alpha \beta} &= \left[ \begin{matrix}
0 & 1 \\[2.2ex]
- 1 & 0 \end{matrix} \right], \;
w^{\alpha \beta} = \left[ \begin{matrix}
0 & -1 \\[2.2ex]
1 & 0 \end{matrix} \right], \; w_{\alpha\beta}w^{\beta\gamma} = \delta_\alpha^\gamma \\
\omega_{\alpha i \beta j} &= w_{\alpha \beta} \delta_{ij}, \;
\omega^{\alpha i \beta j} = w^{\alpha \beta} \delta^{ij} \\
w_{qp} &= 1, w_{pq} = -1
\end{aligned}
\end{equation}


%\bibliographystyle{plain}
%\bibliography{bibliography}

\end{document}