\documentclass[aps,pra,10pt,floatfix,nofootinbib]{revtex4-1}

\usepackage{bbm}
\usepackage{amsmath}
\usepackage{amssymb}
\usepackage{graphicx}
\usepackage{amsfonts}
\usepackage{amsthm}
\usepackage{tikz}

\newtheorem{thm}{Theorem}[section]
\newtheorem{cor}[thm]{Corollary}
\newtheorem{lem}[thm]{Lemma}
\newtheorem{prop}[thm]{Proposition}

\theoremstyle{definition}
\newtheorem{defn}[thm]{Definition}
\newtheorem*{assump1}{Classical assumption}
\newtheorem*{assump2}{Determinism and Reversibility assumption}

\begin{document}

\section{What are Poisson brackets?}

TL;DR The Poisson bracket tells how a quantity changes under a transformation generated by another. It also tells us the state count of a cell of phase space identified by the two variables.

Another operator of particular importance in Hamiltonian mechanics is the Poisson bracket: $\{f,g\}=\frac{\partial f}{\partial x}\frac{\partial g}{\partial p} - \frac{\partial f}{\partial p} \frac{\partial g}{\partial x}$. Its quantum equivalent is the commutator $\frac{[F, G]}{\imath \hbar} = \frac{FG - GF}{\imath \hbar}$. But what does the Poisson bracket signify geometically and physically?

\section{Poisson bracket and change under a generated transformation}

Recall that given $H(x,p)$, Hamilton's equations are:
\begin{equation}
\begin{aligned}
\frac{dx}{dt} &= \frac{\partial H}{\partial p}  \\
\frac{dp}{dt} &= - \frac{\partial H}{\partial x}
\end{aligned}
\label{Hamilton}
\end{equation}

We can generalize these equations so that we can use them for a generic transformation that preserves areas in phase space, not just time evolution. We use a generic parameter $a$ and generic function $A(x,p)$ and we say that $A$ generates the transformation parametrized by $a$ if we have:
\begin{equation}
\begin{aligned}
\frac{dx}{da} &= \frac{\partial A}{\partial p}  \\
\frac{dp}{da} &= - \frac{\partial A}{\partial x}
\end{aligned}
\label{GeneratedTransformation}
\end{equation}
For example, if we have $A=p$ we have:
\begin{equation}
\begin{aligned}
\frac{dx}{da} &= 1  \\
\frac{dp}{da} &= 0
\end{aligned}
\label{GeneratedTranslation}
\end{equation}
That is, the generated transformation is really the translation and the parameter $a$ is the travelled distance. Note that any quantity that is a function of position and momentum is the generator of some area preserving transformation (i.e. it can be used in Hamilton's equations in place of the Hamiltonian) and any infinitesimal transformation that preserves area in phase space has a generator that is a function of position and momentum.

Now we ask: how does a generic function $f(x,p)$ change during this transformation? We have:
\begin{equation}
\begin{aligned}
\frac{df(x(a),p(a))}{da} &= \frac{\partial f}{\partial x}\frac{dx}{da} + \frac{\partial f}{\partial p}\frac{dp}{da} \\
&=\frac{\partial f}{\partial x} \frac{\partial A}{\partial p} - \frac{\partial f}{\partial p}\frac{\partial A}{\partial x}
&=\{f,A\}
\end{aligned}
\label{PoissonGenerator}
\end{equation}
That is: the derivative of a quantity $f$ with respect to the trajectory parameter $a$ is the Poisson bracket of the quantity $f$ with the generator $A$.

With that in mind, the relationship $\{x, p\} = 1 = \frac{dx}{dx}$ means that $p$ generates the translation along $x$ since the change of $x$ along the transformation is $1$. The equation $\frac{df}{dt} = \{f, H\}$ makes sense because $H$ generates time evolution. Moreover, since $\{A, A\} = 0$, a quantity is always constant during a transformation it generates. For example, energy is conserved in time and momentum does not change under translation.

These relationships are identical in quantum mechanics, except the commutator takes the role of the Poisson braket.

\section{Poisson brackets and areas of phase space}

There is also another more geometrical way to understand the Poisson bracket. This works for classical mechanics and I do not know whether it has an analogue in quantum mechanics.

The idea is the following: suppose you have two functions $f(x,p)$ and $g(x,p)$. Suppose you create the area $df dg$. How does it compare to the area $dx dp$? To answer the question, we simply make a coordinate transformation. The Jacobian will relate the two areas in the following way:
\begin{equation}
\begin{aligned}
df dg &= | J | dx dp = \left| \begin{matrix}
\dfrac{\partial f}{\partial x} & \dfrac{\partial f}{\partial p} \\[2.2ex]
\dfrac{\partial g}{\partial x} & \dfrac{\partial g}{\partial p} \end{matrix} \right| dx dp \\
&= \left( \frac{\partial f}{\partial x} \frac{\partial g}{\partial p} - \frac{\partial f}{\partial p}\frac{\partial g}{\partial x} \right) dx dp = \{f,g\} dx dp
\end{aligned}
\label{PoissonJacobian}
\end{equation}
In other words, the Poisson bracket is the Jacobian of the transformation and it tells us how much the area is bigger or smaller.

Physically, the quantity $dxdp$ tells us how many states we have in that region. The Poisson bracket tells us how many more (or less) states we have in the infinitesimal region $dfdg$.

\section{Conclusion}

We can understand the Poisson bracket $\{f,g\}$ in two geometrical ways. It tells us how the quantity $f$ changes under a transformation generated by $g$, that is a transformation where we are using $g$ in place of the Hamiltonian. It also tells us how many possible states we have in an infinitesimal area $df dg$ in units of $dx dp$.

\end{document}
