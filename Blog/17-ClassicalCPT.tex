\documentclass[aps,pra,10pt,floatfix,nofootinbib]{revtex4-1}

\usepackage{bbm}
\usepackage{amsmath}
\usepackage{amssymb}
\usepackage{graphicx}
\usepackage{amsfonts}
\usepackage{amsthm}
\usepackage{tikz}

\newtheorem{thm}{Theorem}[section]
\newtheorem{cor}[thm]{Corollary}
\newtheorem{lem}[thm]{Lemma}
\newtheorem{prop}[thm]{Proposition}

\theoremstyle{definition}
\newtheorem{defn}[thm]{Definition}
\newtheorem*{assump1}{Classical assumption}
\newtheorem*{assump2}{Determinism and Reversibility assumption}

\begin{document}

\section{CPT-like theorem for classical particles}

TL;DR Inverting the direction of time and space is equivalent to inverting the direction of the parametrization.

In previous posts we have seen the classical equivalent for anti-particles. Here we see something that looks similar to the CPT theorem. Naturally, it's not a complete analogue because we are not really looking at fields at all (just the state of a single particle moving in phase-space), we don't have spin, etc... But the symmetry we get has the same flavor.

As we saw, the classical analogue for anti-particle are simply those states for which the parametrization of the phase-space trajectory is anti-aligned with time. If we flip both space and time, the vectors tangent to the evolution will flip sign and this is equivalent to flipping the sign of the parametrization.

\section{Reflections in space, time and evolution parameter}

Suppose we have a state in the extend phase space $(x^i,t,p_j,E)$. It's change during the evolution will be given by the vector $\frac{d}{ds}(x^i,t,p_j,E) = (\frac{dx^i}{ds},\frac{dt}{ds},\frac{dp_j}{ds},\frac{dE}{ds})$.

Now suppose we perform a spatial reflection. This will send $x^i$ to $-x^i$, multiplying the coordinate by the constant $-1$. The momentum transforms as the components of a covector, so it will change as the inverse of that constant: momentum also reflects. Therefore $p_j$ will become $-p_j$. The change during the evolution becomes $(-\frac{dx^i}{ds},\frac{dt}{ds},-\frac{dp_j}{ds},\frac{dE}{ds})$.

Suppose we perform a time reflection on top of the spatial reflection. This will send $t$ to $-t$. And since energy transforms as the time component of a covector, $E$ will become $-E$. The change during the evolution becomes $(-\frac{dx^i}{ds},-\frac{dt}{ds},-\frac{dp_j}{ds},-\frac{dE}{ds})$.

The change is now exactly the opposite on all components. The same change could have been achieved by changing the parametrization from $s$ to $-s$. But changing the sign of the parametrization means changing what we label as particle to what we label as anti-particle. If we call this our analogue for charge conjugation, we see that performing all three changes gets us back where we started.


\section{Conclusion}

Note that I would not considered an exact analogue for CPT. The other analogies we have found so far between classical/quantum/relativity are a lot stronger. Still, I think it points our intuition in the right direction and I don't think is just a coincidence either.

\end{document}
