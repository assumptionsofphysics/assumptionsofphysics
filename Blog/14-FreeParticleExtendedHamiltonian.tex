\documentclass[aps,pra,10pt,floatfix,nofootinbib]{revtex4-1}

\usepackage{bbm}
\usepackage{amsmath}
\usepackage{amssymb}
\usepackage{graphicx}
\usepackage{amsfonts}
\usepackage{amsthm}
\usepackage{tikz}

\newtheorem{thm}{Theorem}[section]
\newtheorem{cor}[thm]{Corollary}
\newtheorem{lem}[thm]{Lemma}
\newtheorem{prop}[thm]{Proposition}

\theoremstyle{definition}
\newtheorem{defn}[thm]{Definition}
\newtheorem*{assump1}{Classical assumption}
\newtheorem*{assump2}{Determinism and Reversibility assumption}

\begin{document}

\section{Classical Klein-Gordon}

TL;DR The extended Hamiltonian for a classical free particle is the classical version of the Klein-Gordon equation.

After having seen the classical version of antiparticles we will see the classical version of the Klein-Gordon equation. This is the essentially the extended Hamiltonian for the free particle.

Let's look at the details.

\section{Extended Hamiltonian for the free particle}

The extended Hamiltonian for a free particle is:
\begin{equation}
\begin{aligned}
\mathcal{H}(x^i,t,p_j,E) = \frac{1}{2m} \left(p^2 - \left(\frac{E^2}{c^2}\right)\right)  + \frac{1}{2} mc^2\\
\end{aligned}
\label{ExtendedHamiltonFree}
\end{equation}

Remember that, for valid physical states, we have $\mathcal{H}=0$. Therefore:
\begin{equation}
\begin{aligned}
\left(p^2 - \left(\frac{E^2}{c^2}\right)\right) = - m^2c^2\\
\end{aligned}
\label{FourMomentum}
\end{equation}
If you studied special relativity, you recognized the norm of the four-momentum. Moreover:
\begin{equation}
\begin{aligned}
\frac{E^2}{c^2} &= p^2 + m^2c^2\\
E &= \pm c \sqrt{p^2 + m^2c^2} \\
\end{aligned}
\label{HamiltonFree}
\end{equation}
This is the energy as a function of momentum: it the relativistic Hamiltonian for a free particle. But in this case the energy can both be positive or negative. Note that:
\begin{equation}
\begin{aligned}
\frac{dt}{ds} = - \frac{\partial \mathcal{H}}{\partial E} = \frac{E}{mc^2}
\end{aligned}
\label{TimeEvolution}
\end{equation}
So if the energy is negative, then the parametrization is anti-aligned to the time: $t$ decreases as $s$ increases. As we saw in the previous post, these correspond to anti-particle states.

So, not only the extended Hamiltonian recovers relativistic free-particles, it also recovers anti-particles and it is has much nicer form than the relativistic Hamiltonian (i.e. it is quadratic and does not have any square root).


\section{Relation to Klein-Gordon}

If you use the standard classical to quantum substitutions $\left( p_i, -E \right)$ to $\left(-\imath \hbar \frac{\partial}{\partial x^i}, -\imath \hbar \frac{\partial}{\partial t} \right)$ we have:
\begin{equation}
\begin{aligned}
\mathcal{H} = &\frac{1}{2m} \left(- \hbar^2 \nabla^2 + \hbar^2 \left(\frac{\partial^2}{c^2\partial t^2}\right)\right)  + \frac{1}{2} mc^2 = 0\\
&\frac{1}{c^2}\frac{\partial^2}{\partial t^2}- \nabla^2 + \frac{m^2 c^2}{\hbar^2}  = 0\\
\end{aligned}
\label{KleinGordon}
\end{equation}
If you studied quantum field theory, that should look familiar: if we add a $\psi$ it's the Klein-Gordon equation. (Note: the mass term is positive as we are using the relativity metric convention $(-,+,+,+)$ instead of the particle physics metric convention $(+, -, -, -)$).

So why should we add $\psi$? There is good reason to do this, even in the classical setting.

Suppose you have a phase space distribution $\rho(x^i,t,p_j,E)$. This would correspond to a distribution of matter over space, time, momentum and energy. Consider $\mathcal{H} \rho$, the product between the extended Hamiltonian and the distribution. As we saw, $\mathcal{H}=0$ is a constraint between energy and the other variables so that only states that satisfy that equation are physical. Therefore $\rho$ can only be different than zero for those states where $\mathcal{H}$ is zero. But this means that $\mathcal{H} \rho = 0$ everywhere. In other words: $\mathcal{H} \rho = 0$ can be used as a constraint to find physically meaningful distributions over the extended phase space.

Now, in the quantum case states are always ensembles: they are always distributions. Therefore we use the constraint $\mathcal{H}\psi = 0$. Which, in the case of the free particle, corresponds to the Klein-Gordon equation.

\section{Conclusion}

The extended Hamiltonian for a free particle has a much nicer form than the relativistic Hamiltonian one finds in standard textbooks, and it ties in more with quantum concepts. The Klein-Gordon equation is the extended Hamiltonian for a free particle acting on the distribution. The only thing it does is constrain the state so that the norm of the four-momentum is $-m^2c^2$.

\end{document}
