\documentclass[aps,pra,10pt,floatfix,nofootinbib]{revtex4-1}

\usepackage{bbm}
\usepackage{amsmath}
\usepackage{amssymb}
\usepackage{graphicx}
\usepackage{amsfonts}
\usepackage{amsthm}
\usepackage{tikz}

\newtheorem{thm}{Theorem}[section]
\newtheorem{cor}[thm]{Corollary}
\newtheorem{lem}[thm]{Lemma}
\newtheorem{prop}[thm]{Proposition}

\theoremstyle{definition}
\newtheorem{defn}[thm]{Definition}
\newtheorem*{assump1}{Classical assumption}
\newtheorem*{assump2}{Determinism and Reversibility assumption}

\begin{document}

\section{What are commutators?}

TL;DR Commutators, like Poisson bracket, tells how a quantity changes under a transformation generated by another.

Commutators play a fundamental role in quantum mechanics. What we want to show is what they are in relation to infinitesimal transformation, which gives a better understanding of why they are related to Poisson brackets and why they must give the relationships they give.

\section{Conjugate quantities}

In classical mechanics, conjugate quantities are pair of variables that form an independent degree of freedom. If you integrate over a pair, you get a quantity that no longer depends on the unit and coordinate used for those quantities.

In quantum mechanics, things work a bit differently mathematically. In quantum mechanics you have operators and parameters. If a quantity is an operator it means you can write the state as a wave-function over the quantity. If a quantity is an operator it means you can make and infinitesimal transformation along it. The conjugate relationship is between operators and parameters: for each operator $A$ you have one parameter $\alpha$ and we say that $A$ generates the infinitesimal transformation $(1 + \frac{A d\alpha}{\imath \hbar})$, which is a unitary transformation.

For example, spin component $S_z$ is the operator while the angle $\theta_{xy}$ is the parameter, and $S_z$ generates the rotation $(1 + \frac{S_z d\theta_{xy}}{\imath \hbar})$. The confusing part is that some quantities, like position and momentum, can both be operators and parameters. So we have the $P$ operator and the $x$ parameter and $P$ generates the translation $(1 + \frac{P dx}{\imath \hbar})$, but also $X$ operator and $p$ parameter and $X$ generates the change in momentum $(1 + \frac{X dp}{\imath \hbar})$.

\section{Commutators}

Now we ask: suppose we have two operators $A$ and $B$. Suppose you perform the infinitesimal transformation generated by the second, over the parameter $\beta$. How does the first change?

The infinitesimal transformation corresponds to the unitary operator $U=(1 + \frac{B d\beta}{\imath \hbar})$. An operator under a unitary transformation changes as $\hat{A} = U^T A U$. Therefore we have:

\begin{equation}
\begin{aligned}
\hat{A}&=U^\dagger A U = (1 - \frac{B d\beta}{\imath \hbar}) A (1 + \frac{B d\beta}{\imath \hbar}) \\
&=A + A \frac{B d\beta}{\imath \hbar} - \frac{B d\beta}{\imath \hbar} A - \frac{B d\beta}{\imath \hbar} \frac{B d\beta}{\imath \hbar} \\
&=A + \frac{(AB - BA)}{\imath \hbar}d\beta + \frac{B^2}{\hbar^2}d\beta^2
\end{aligned}
\end{equation}

The first order change of $A$ is therefore:

\begin{equation}
\begin{aligned}
\frac{dA}{d\beta}&=\frac{(AB - BA)}{\imath \hbar} \\
&= \frac{[A,B]}{\imath \hbar}
\end{aligned}
\end{equation}

Therefore the commutator between $A$ and $B$ tells us how $A$ changes under a transformation generated by $B$. In a previous post we saw that the Poisson brackets in classical Hamiltonian mechanics had the same role. That is why we can do the formal substitution: they describe the same physical relationship in the different mathematical framework.

We can also understand why the commutators must be what they are. For example, the conjugate of the operator momentum $P_j$ is the translation over the position parameter $x_j$. So how will the operator position $X_i$ change under the transformation generated by $P_j$?

\begin{equation}
\begin{aligned}
\frac{[X_i,P_j]}{\imath \hbar} &= \frac{dX_i}{dx_j}=I \delta_{ij}
\end{aligned}
\end{equation}

If $X_i$ and $P_j$ are along the same direction, then position operator will change of the same amount of the position parameter: they are physically the same thing. If they are along different directions there is no change.

Now, consider the operators $S_x$, $S_y$ and $S_z$ for the spin components. How will they change under the transformation generated by $S_z$, which is the rotation over $\theta_{xy}$?

\begin{equation}
\begin{aligned}
\frac{[S_x,S_z]}{\imath \hbar} &= \frac{dS_x}{d\theta_{xy}} = S_y \\
\frac{[S_y,S_z]}{\imath \hbar} &= \frac{dS_x}{d\theta_{xy}} = - S_x \\
\frac{[S_z,S_z]}{\imath \hbar} &= \frac{dS_z}{d\theta_{xy}} = 0
\end{aligned}
\end{equation}

The last one is the easiest: the $z$ component of spin does not change under rotation over the $xy$ plane. For the other two, just remember that the $x$ component is moved in the $y$ direction and the $y$ direction is moved in the $-y$ direction.

These are indeed the commutation relationship one has in quantum mechanics. Given the physical meaning of the operators and their transformations, the commutation relationship can't be anything else.

\section{Conclusion}

We have seen that commutators have a well defined physical/geometrical meaning, which is the same one for Poisson brackets in classical Hamiltonian mechanics. I personally not only find it insightful, but it helps me remember the correct sign for the commutation relationships.


\end{document}
