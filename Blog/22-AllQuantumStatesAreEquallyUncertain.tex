\documentclass[aps,pra,10pt,floatfix,nofootinbib]{revtex4-1}

\usepackage{bbm}
\usepackage{amsmath}
\usepackage{amssymb}
\usepackage{graphicx}
\usepackage{amsfonts}
\usepackage{amsthm}
\usepackage{tikz}

\newtheorem{thm}{Theorem}[section]
\newtheorem{cor}[thm]{Corollary}
\newtheorem{lem}[thm]{Lemma}
\newtheorem{prop}[thm]{Proposition}

\theoremstyle{definition}
\newtheorem{defn}[thm]{Definition}
\newtheorem*{assump1}{Classical assumption}
\newtheorem*{assump2}{Determinism and Reversibility assumption}

\begin{document}

\section{All quantum states are equally defined}
TL;DR - No quantum state is more uncertain than the other. All states can be identified by a set of perfectly prepared quantities.

When studying quantum mechanics, you may get the (wrong) impression that some states are better defined than others. Some are eigenstates while others are just a superposition. Or that the gaussian packets, are more determined than the other states, since some satisfy the uncertainty principle with an equal. Well, that's just not the case.

The confusion stems from the idea that quantum states are like classical statistical distribution, in which you have some well defined elements (which are the objects that are well defined) upon which you assign probabilities. Quantum states are not like this at all. We are going to see that any state is always the eigenstate of some Hermitian operator, and therefore they always have a quantity that is perfectly prepared. And they also have a symmetry under the transformation generated by that operator, so there is always another quantity of which no value is more likely than the other.

\section{Prepared quantities and unprepared conjugates}

Suppose that you have a quantum state for which the position is perfectly prepared. This corresponds to the eigenstate $|x\rangle$ of the operator $X$. The transformation generated by $X$ is $1+\frac{Xdp}{\imath\hbar}$ which corresponds to increasing momentum by $dp$. Now, since $|x\rangle$ is an eigenstate of $X$, it is also an eigenstate of the transformation generated by $X$. If we imagine the distribution of $|x\rangle$ over momentum, then, this has to be a distribution that does not change if we increase momentum. But the only distribution that is symmetric under that change is the one that has the same value for all possible values of momentum: the eigen state $|x\rangle$ is uniformly distributed in momentum. This can be generalized to any quantity. For example, an eigenstate of spin in the $z$ direction will be symmetric along the angle on the $(x,y)$ plane. This will also hold for any function of position and momentum and indeed for any Hermitian operator.

So, for a particular space of quantum states, we can imagine that, instead of specifying a wavefunction, we can give a set of operators and eigenvalues. Given that information, we can identify the corresponding eigenstate. This would not give the value for all quantities since the conjugate quantities will be left completely unspecified. The question is: can all states can be specified in this way? Which is equivalent to asking: are all states eigenstates of some Hermitian operator?

Suppose we have a state $|\psi\rangle$. We can construct the operator $O=|\psi\rangle a \langle \psi |$ where $a$ is a real number. $O$ is Hermitian and $|\psi\rangle$ is the eigenstate corresponding to the eigenvalue $a$. Yes, the operator is trivial, but we can indeed construct it. Which shows that for any state there exists Hermitian operators that allow that state as an eigenstate. In fact, there are infinitely many.

Overall, a state always has some well specified quantities and it also has some unspecified ones. For example, suppose we have an eigenstate for the operator $X + a P$, a linear combination of position and momentum. This means that the quantity $x + ap$ is perfectly prepared while the conjugate quantity, $\frac{1}{2a}(x-ap)$, is a uniform distribution. We can verify the two quantities are conjugates by calculating the commutator.

\begin{equation}
\begin{aligned}
\frac{\left[\frac{1}{2a}(X-aP), X+aP \right]}{\imath \hbar} &= \frac{1}{2a \imath \hbar} \left[X-aP, X+aP\right] \\
 &= \frac{1}{2a \imath \hbar} \left( \left[X, X\right] + \left[-aP, X\right] + \left[X, aP\right] + \left[-aP, aP\right] \right) \\
&= \frac{1}{2a \imath \hbar} \left( 0 + a \imath \hbar + a \imath \hbar + 0 \right) \\
&= 1
\end{aligned}
\end{equation}

Now consider the two distributions for that state over position and momentum. Since all the values of $\frac{1}{2a}(x-ap)$ are equally likely, then all values of $x$ and $p$ are also equally likely. That is: the distribution is uniform for both quantities. The variance for both $x$ and $p$ is infinite. If we only know that, we would think that state to be infinitely less defined than an eigetnstate of position. What happens is that for that state the uniform distribution in $x$ and $p$ are strongly correlated, so much so that the distribution over $x+ap$ admits a single value. Therefore this state is no less defined of an eigenestate of $x$: what changes is what quantities are well defined and which aren't.

\section{Conclusion}

What is significantly different in quantum mechanics is that all states are distributions but no distribution is more defined than the other. Even if a state has an extremely large spread in a pair of conjugate variables, it will have a quantity somewhere that will be perfectly defined. There is no base that is mathematically better then the other: all states can be part of a basis and all states are superpositions in some other basis. Whenever we are superposing two states, we are always adding some other correlation such that the new states is as well defined as the other two.

This actually makes a lot of sense conceptually. It's telling us that each quantum state is a distribution that cannot be decomposed into smaller independent ones. Each quantum system is an irreducible unit. This is the main difference from classical mechanics, where each distribution over phase space can be divided into smaller pieces that can be studied independently.

\end{document}
