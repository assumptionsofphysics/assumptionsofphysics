\documentclass[10pt, onecolumn, longbibliography, nofootinbib]{revtex4-2}

\usepackage{assumptionsofphysics}

\usepackage{tikz}
\usepackage{hyperref}
\hypersetup{
	colorlinks=true,
	citecolor=blue,
	urlcolor=blue,
	linkcolor=blue
}
\frenchspacing

% Usual (decimal) numbering
\renewcommand{\thesection}{\arabic{section}}
\renewcommand{\thesubsection}{\thesection.\arabic{subsection}}
\renewcommand{\thesubsubsection}{\thesubsection.\arabic{subsubsection}}

% Fix references
\makeatletter
\renewcommand{\p@subsection}{}
\renewcommand{\p@subsubsection}{}
\makeatother

\begin{document}

\title{Physical motivation for differentiability}
%\author{Gabriele Carcassi}
%\email{carcassi@umich.edu}
%\affiliation{Physics Department, University of Michigan, Ann Arbor, MI 48109}
%\author{Christine Aidala}
%\affiliation{Physics Department, University of Michigan, Ann Arbor, MI 48109}
\date{\today}

\begin{abstract}
	This paper presents ideas and progress towards a more direct physical justification for differentiability. The general idea that the differentiable structure on a manifold exists to allow us to express a distribution as a density over the coordinates.
\end{abstract}

\maketitle

\section{Introduction}

\emph{This paper presents the goal and the current status of our research. It is intended to gather early feedback, including pointers to other relevant work. Feedback is welcome and encouraged.}

As part of our larger project, Assumptions of Physics, we want to rederive the mathematical structures we use in physics from precise physical starting points. In previous works we have shown the following:
\begin{description}
	\item[Experimentally distinguishable cases] $\to$ topological space. Each open set corresponds to an experimentally testable hypothesis.
	\item[Cases identifiable by continuous quantities] $\to$ manifold. Coordinates represent measurable quantities.
\end{description}
Before we establish what Riemannian and symplectic manifolds correspond to, we need to establish what differentiability corresponds to.

The general idea is that differentiability allows us to express a distribution as a continuous function $\rho(x^i)$ of the coordinates. That is, among all the possible coordinates of the manifold we restrict ourselves to those that allow us to express $\rho$ in a nice way.

Conceptually, we will want to start from statements that are directly testable. As we do not measure densities directly, but finite values in finite regions, the density distribution $\rho$ cannot be our starting point. Given a set $X$ of possible cases, our verifiable statements are of the form ``in region $U$ the value of the quantity is within the range $V$'', where $U$ is an open set of $X$ and $V$ is an open set of $\mathbb{R}$. We need to identify the exact conditions for which we will recover differentiability in the case that $X$ is a manifold.

\section{Mathematical problem}

\subsection{Deep approaches}

These approaches try to formulate differentiability in terms of physical significance. The status of differentiable coordinates is that they are needed to validly express the physics problem.

\subsubsection{Smoothness as coordinates with continuous Radon-Nikodym derivate}

\begin{defn}
	Let $X$ be a manifold, $\textsf{T}_X$ its topology and $\Sigma_X$ its Borel algebra. A \textbf{distributed quantity} $\mu$ is a measure $\mu : \Sigma_X \to \mathbb{R}$ such that $\mu(U) = \mu(\interior(U))$. 
\end{defn}

\begin{defn}
	Given a manifold $X$ and a distributed quantity $\mu$, we say a coordinate $q : U \to \mathbb{R}$ is \textbf{smooth with respect to $\mu$} if the Radon-Nikodym derivative is continuous.
\end{defn}

\begin{desid}
	The set of all smooth coordinates with respect to a distributed quantity $\mu$ identifies a differentiable structure for $X$.
\end{desid}

The idea is that continuity of the derivative is needed to express the dependence of the density from the coordinates in a meaningful way (from previous work, only continuous function can expressed experimentally meaningful quantities).

The issue is that, in general, a single measure is not going to be enough: regions where the measure is zero would allow for non-differentiable coordinates (the derivative would be zero, which is always continuous).

More physical would be to consider the space of all measures, which would correspond to all possible ways a distributive quantity can be distributed (i.e. the states of a composite system). This, however, does move the problem of differentiability to the space of measures. In general, the derivative will be measurable and not continuous. Therefore one would need a way to constrain the space to only the ones that are continuous.

General issue: more thought needs to be put into how densities are actually measured, and what would be the correct topology on the space.

\subsection{Shallow approaches}

These approaches do not try to assign physical significance to differentiability. The status of differentiable coordinates is that they are mathematically convenient.

\subsubsection{Smoothness as convenient to express approximations of linear functionals}

\begin{defn}
	Let $X$ be a manifold of dimension $n$. A \textbf{k-surface} is a subset $U \subseteq X$ that is a $k$-dimensional manifold (with boundary) with the subspace topology. We note $S^k(X)$, or simply $S^k$, the set of all $k$-surfaces and $S(X) = \bigcup_{i=0}^{n} S^k$ the set of all possible $k$-surfaces of all dimensions.
\end{defn}

\begin{remark}
	Note that $S^k = \emptyset$ if $k < 0$ or $k > n$.
\end{remark}

\begin{defn}
	The \textbf{boundary operator} $\partial : S(X) \to S(X)$ is a map from a $k$-surface to its boundary (as a manifold).
\end{defn}

\begin{prop}
	For every $\sigma^k \in S^k(X)$ with $k > 0$, we have $\partial \sigma^k \in S^{k-1}$. For every $\sigma^0 \in S^0$, $\partial \sigma^0 = \emptyset$. For every $\sigma \in S$, we have $\partial \partial \sigma = \emptyset$.
\end{prop}

\begin{defn}
	A \textbf{k-functional} is a function $f_k : S^k \to \mathbb{R}$ with the following properties:
	\begin{description}
		\item[Linear] $f_k(\sigma_1 \cup \sigma_2) = f_k(\sigma_1) + f_k(\sigma_2)$ for every $\sigma_1, \sigma_2 \in S^k$ such that $\sigma_1 \cap \sigma_2 = \emptyset$
		\item[No contribution from boundary] $f_k(\sigma) = f_k(\sigma \setminus \partial \sigma)$ for every $\sigma \in S^k$
		\item[Commutes with the limit] $\lim\limits_{i \to \infty} f_k(\sigma_i^k) = f_k(\lim\limits_{i \to \infty}\sigma_i^k)$
	\end{description}
\end{defn}

\begin{remark}
	Linearity means that the contribution from each surface is independent from the rest (reducibility). If boundaries gave contribution, the functional would not really be a function of the $k$-surface, since the boundary is of lower dimension.
	
	Note the operator $length : S^1 \to \mathbb{R}$ that returns the length of a line is not a $1$-functional as it does not commute with the limit (e.g. diagonal is the limit of a perpendicular zig-zag, but the length does not converge).
\end{remark}

\begin{desid}
	Let $X$ be a differentiable manifold. For every $k$-functional $f_k$ there exists a suitable $k$-form $\omega_k$ whose integral $\int_U \omega_k$ in a region $U$ approximates $f_k(U)$ at a desired level of precision.
\end{desid}

\begin{remark}
	The simplest strategy may be to fix an interval for the value, and use standard criteria of convergence. However, this may be far too restrictive with respect to what we need physically.
	
	Given that we measure the quantity $f_k(\sigma^k)$ in a finite region $\sigma_k$ with finite precision, there are two things to fix: the confidence interval $\epsilon \in \mathbb{R}$ for the quantity itself and the smallest regions we can measure. As we impose a cutoff in the size of the regions, we do not care whether the approximation holds well at smaller scale (e.g. anything ``higher frequency'' can be ignored, and maybe it doesn't even matter whether higher frequencies diverge).
	
	We want to define those without reference to a metric/measure on $X$.
\end{remark}

\begin{defn}
	A \textbf{coarsening} is a map $c : S^k \to S^k$ with the following properties:
	\begin{description}
		\item[monotonic] $c(\sigma) \supseteq \sigma$ for all $\sigma \in S^k$
		\item[idempotent] $c(c(\sigma)) = c(\sigma$) for all $\sigma \in S^k$
		\item[bounded] for all $\sigma \in S^k$, there exists a $\hat{\sigma} \in S^k$ such that $\hat{\sigma} \subseteq \sigma$ and $c(\hat{\sigma}) \neq \hat{\sigma}$
	\end{description}
	A \textbf{coarsened region} is a $k$-surface $\sigma_k$ such that $c(\sigma_k) = \sigma_k$. 
\end{defn}

\begin{remark}
	The idea is that the coarsening substitutes the measuring region to one bigger. Monotony guarantees that the region is bigger. Boundedness guarantees that there is a cutoff in size at a certain point. Idempotency guarantees that there is a desired resolution.
	
	A simple example of coarsening: consider the set of all surfaces with integer areas in some coordinates; given any surface, grow it to match one of those.
\end{remark}

\begin{desid}
	Given a $k$-functional $f_k : S^k \to \mathbb{R}$, a coarsening $c : S^k \to S^k$ and an $\epsilon \in \mathbb{R}$, we can always find a $k$-form $\omega_k$ such that $| f_k(\sigma) - \int_\sigma \omega_k | < \epsilon$ for all coarsened regions $\sigma$.
\end{desid}

%\section{Examples}

%\section{Insightful failures}


\bibliography{bibliography}

\end{document}
