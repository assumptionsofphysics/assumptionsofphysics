\documentclass[10pt, onecolumn, longbibliography, nofootinbib]{revtex4-2}

\usepackage{assumptionsofphysics}

\usepackage{tikz}
\usepackage{hyperref}
\hypersetup{
	colorlinks=true,
	citecolor=blue,
	urlcolor=blue,
	linkcolor=blue
}
\frenchspacing

% Usual (decimal) numbering
\renewcommand{\thesection}{\arabic{section}}
\renewcommand{\thesubsection}{\thesection.\arabic{subsection}}
\renewcommand{\thesubsubsection}{\thesubsection.\arabic{subsubsection}}

% Fix references
\makeatletter
\renewcommand{\p@subsection}{}
\renewcommand{\p@subsubsection}{}
\makeatother

\begin{document}

\title{Physical motivation for differentiability}
%\author{Gabriele Carcassi}
%\email{carcassi@umich.edu}
%\affiliation{Physics Department, University of Michigan, Ann Arbor, MI 48109}
%\author{Christine Aidala}
%\affiliation{Physics Department, University of Michigan, Ann Arbor, MI 48109}
\date{\today}

\begin{abstract}
	This paper presents ideas and progress towards a more direct physical justification for differentiability. The general idea that the differentiable structure on a manifold exists to allow us to express a distribution as a density over the coordinates.
\end{abstract}

\maketitle

\section{Introduction}

\emph{This paper presents the goal and the current status of our research. It is intended to gather early feedback, including pointers to other relevant work. Feedback is welcome and encouraged.}

As part of our larger project, Assumptions of Physics, we want to rederive the mathematical structures we use in physics from precise physical starting points. In previous works we have shown the following:
\begin{description}
	\item[Experimentally distinguishable cases] $\to$ topological space. Each open set corresponds to an experimentally testable hypothesis.
	\item[Cases identifiable by continuous quantities] $\to$ manifold. Coordinates represent measurable quantities.
\end{description}
Before we establish what Riemannian and symplectic manifolds correspond to, we need to establish what differentiability corresponds to.

The general idea is that differentiability allows us to express a distribution as a continuous function $\rho(x^i)$ of the coordinates. That is, among all the possible coordinates of the manifold we restrict ourselves to those that allow us to express $\rho$ in a nice way.

Conceptually, we will want to start from statements that are directly testable. As we do not measure densities directly, but finite values in finite regions, the density distribution $\rho$ cannot be our starting point. Given a set $X$ of possible cases, our verifiable statements are of the form ``in region $U$ the value of the quantity is within the range $V$'', where $U$ is an open set of $X$ and $V$ is an open set of $\mathbb{R}$. We need to identify the exact conditions for which we will recover differentiability in the case that $X$ is a manifold.

\section{Mathematical problem}

\subsection{General definitions}

\begin{defn}
	Let $X$ be a manifold of dimension $n$. A \textbf{k-surface} is a subset $U \subseteq X$ that is a $k$-dimensional manifold (with boundary) with the subspace topology (i.e. regular/embedded submanifold). We note $S^k(X)$, or simply $S^k$, the set of all $k$-surfaces and $S(X) = \bigcup_{i=0}^{n} S^k$ the set of all possible $k$-surfaces of all dimensions.
\end{defn}

\begin{remark}
	Note that $S^k = \emptyset$ if $k < 0$ or $k > n$.
\end{remark}

\begin{defn}
	The \textbf{boundary operator} $\partial : S(X) \to S(X)$ is a map from a $k$-surface to its boundary (as a manifold).
\end{defn}

\begin{prop}
	For every $\sigma^k \in S^k(X)$ with $k > 0$, we have $\partial \sigma^k \in S^{k-1}$. For every $\sigma^0 \in S^0$, $\partial \sigma^0 = \emptyset$. For every $\sigma \in S$, we have $\partial \partial \sigma = \emptyset$.
\end{prop}

\begin{defn}
	A \textbf{k-functional} is a function $f_k : S^k \to \mathbb{R}$ with the following properties:
	\begin{description}
		\item[Linear] $f_k(\sigma_1 \cup \sigma_2) = f_k(\sigma_1) + f_k(\sigma_2)$ for every $\sigma_1, \sigma_2 \in S^k$ such that $\sigma_1 \cap \sigma_2 = \emptyset$
		\item[No contribution from boundary] $f_k(\sigma) = f_k(\sigma \setminus \partial \sigma)$ for every $\sigma \in S^k$
		\item[Commutes with the limit] $\lim\limits_{i \to \infty} f_k(\sigma_i^k) = f_k(\lim\limits_{i \to \infty}\sigma_i^k)$ (limit of a sequence of submain)
	\end{description}
\end{defn}

\begin{remark}
	Linearity means that the contribution from each surface is independent from the rest (reducibility). If boundaries gave contribution, the functional would not really be a function of the $k$-surface, since the boundary is of lower dimension.
	
	Note the operator $length : S^1 \to \mathbb{R}$ that returns the length of a line is not a $1$-functional as it does not commute with the limit (e.g. diagonal is the limit of a perpendicular zig-zag, but the length does not converge).
\end{remark}

\subsubsection{Continuity for functionals}

The requirements of commutation with the limit is similar to a definition of continuity in terms of limits. Can we turn that into a topological notion of continuity? Suppose we had a topology on $S^k$, could we simply require the functional to be continuous? How that would constrain $f_k$?

This forms a nice parallel between a map $f : X \to Y$ and the functional $f_k : S^k \to \mathbb{R}$. For both, the open sets are associated to the measurement precision. For $f_k$, the precision of the quantity is an open interval (finite precision) while the precision of the argument is a set of possible $k$ surfaces (i.e. we have an uncertainty on the shape/area).

For example, let $U, V \in S^k$ be two $k$-surfaces without borders such that $U \subset V$. The set $R(U,V) = \{ W \in S^k \, | \, U \subseteq W \subseteq V\}$ is called the \textbf{surface range} between $U$ and $V$. That should be an open set in the topology of $S^k$.

\subsection{Deep approaches}

These approaches try to formulate differentiability in terms of physical significance. The status of differentiable coordinates is that they are needed to validly express the physics problem.

\subsubsection{Smoothness as coordinates with continuous Radon-Nikodym derivate}

\begin{defn}
	Let $X$ be a manifold, $\textsf{T}_X$ its topology and $\Sigma_X$ its Borel algebra. A \textbf{distributed quantity} $\mu$ is a measure $\mu : \Sigma_X \to \mathbb{R}$ such that $\mu(U) = \mu(\interior(U))$. 
\end{defn}

\begin{defn}
	Given a manifold $X$ and a distributed quantity $\mu$, we say a coordinate $q : U \to \mathbb{R}$ is \textbf{smooth with respect to $\mu$} if the Radon-Nikodym derivative is continuous.
\end{defn}

\begin{desid}
	The set of all smooth coordinates with respect to a distributed quantity $\mu$ identifies a differentiable structure for $X$.
\end{desid}

The idea is that continuity of the derivative is needed to express the dependence of the density from the coordinates in a meaningful way (from previous work, only continuous function can expressed experimentally meaningful quantities).

The issue is that, in general, a single measure is not going to be enough: regions where the measure is zero would allow for non-differentiable coordinates (the derivative would be zero, which is always continuous).

More physical would be to consider the space of all measures, which would correspond to all possible ways a distributive quantity can be distributed (i.e. the states of a composite system). This, however, does move the problem of differentiability to the space of measures. In general, the derivative will be measurable and not continuous. Therefore one would need a way to constrain the space to only the ones that are continuous.

General issue: more thought needs to be put into how densities are actually measured, and what would be the correct topology on the space.

\subsubsection{title}

\subsection{Shallow approaches}

These approaches do not try to assign physical significance to differentiability. The status of differentiable coordinates is that they are mathematically convenient.

\subsubsection{Convergence at scale}

We need to understand what is the proper criteria for convergence that correctly reflect the physical approximations we are looking for. The general idea is that we want an approximation that works from a given level of scale upwards, leaving errors at lower scale. This is not how typically things work with standard types of convergence.

Suppose we have a function $f(x)$. Consider now $\hat{f}(x) = f(x) + \epsilon$, where $\epsilon$ can be understood as a small constant error over the whole space. The difference in integral $\int_U \hat{f}(x) - \int_U f(x)dx = \epsilon \Delta x$ increases with the size of the region. In the case the values $x$ and $f(x)$ represent directly the value measured (e.g. $x$ is temperature and $f(x)$ is the length of a metal bar), $\hat{f}$ represents a good approximation. If what we measure directly, however, is the integral (e.g. $U$ is a region and $\int_U f(x)dx$ is the mass in that region), then $\hat{f}$ would give a poor approximation at large scale. For this second case, we need a different method of convergence.

A first attempt could be the following.

\begin{defn}
	Let $X$ be a differentiable manifold. Let $d : X \times X \to [0, +\infty)$ be a metric that induces the topology of $X$. Let $B^k(X, \delta)$ be the set of all $k$-dimensional balls with radius greater than $\delta$. Let $\omega$ be a $k$-form. A sequence of $k$-forms $\{\omega_i\}$ \textbf{converges at scale} to $\omega$ if given $\epsilon, \delta \in \mathbb{R}^+$ we can find $N$ such that $\| \int_U \omega_n - \int_U \omega \| < \epsilon$ for all $n \geq N$ and $U \in B^k(X, \delta)$.
\end{defn}

\begin{defn}
	Let $X$ be a differentiable manifold. Let $d : X \times X \to [0, +\infty)$ be a metric that induces the topology of $X$. Let $B^k(X, \delta)$ be the set of all $k$-dimensional balls with radius greater than $\delta$. Let $f_k$ be a $k$-functional. A sequence of $k$-forms $\{\omega_i\}$ \textbf{converges at scale} to $f_k$ if given $\epsilon, \delta \in \mathbb{R}^+$ we can find $N$ such that $\| \int_U \omega_n - f_k(U) \| < \epsilon$ for all $n \geq N$ and $U \in B^k(X, \delta)$.
\end{defn}

Then we would like to prove the following:

\begin{desid}
	Let $X$ be a differentiable manifold. For every $k$-functional $f_k$ there exists a sequence of $k$-forms $\{\omega_i$\} that converges at scale to $f_k$.
\end{desid}

\section{Graveyard}

\subsection{Coarsening}

This was an attempt to define how to coarsen integrand regions in a way that doesn't depend on a metric.

\begin{defn}
	A \textbf{coarsening} is a map $c : S^k \to S^k$ with the following properties:
	\begin{description}
		\item[monotonic] $c(\sigma) \supseteq \sigma$ for all $\sigma \in S^k$
		\item[idempotent] $c(c(\sigma)) = c(\sigma$) for all $\sigma \in S^k$
		\item[bounded] for all $\sigma \in S^k$, there exists a $\hat{\sigma} \in S^k$ such that $\hat{\sigma} \subseteq \sigma$ and $c(\hat{\sigma}) \neq \hat{\sigma}$
	\end{description}
	A \textbf{coarsened region} is a $k$-surface $\sigma_k$ such that $c(\sigma_k) = \sigma_k$. 
\end{defn}

\begin{remark}
	The idea is that the coarsening substitutes the measuring region to one bigger. Monotony guarantees that the region is bigger. Boundedness guarantees that there is a cutoff in size at a certain point. Idempotency guarantees that there is a desired resolution.
	
	A simple example of coarsening: consider the set of all surfaces with integer areas in some coordinates; given any surface, grow it to match one of those.
\end{remark}

\begin{desid}
	Given a $k$-functional $f_k : S^k \to \mathbb{R}$, a coarsening $c : S^k \to S^k$ and an $\epsilon \in \mathbb{R}$, we can always find a $k$-form $\omega_k$ such that $| f_k(\sigma) - \int_\sigma \omega_k | < \epsilon$ for all coarsened regions $\sigma$.
\end{desid}

%\section{Examples}

\section{Differentiability as reducibility}

%Given a topological embedding of a closed segment and a sequence into the segment with dense image, let for every $n$ $\lambda_n$ be the polyline defined by the first $n$ elements in the sequence. We say that the polyline converges to the embedded if for every open neighbourhood of the image of the embedding there is an $N$ such that all polylines beyond $N$ are in the neighbourhood.


%Let X be a manifold and \psi be a chart (a choice of coordinate {x^i}). Let \lambda be a curve and {P_i} be a dense subset \lambda. Let \lambda_k be the polyline consisting of line segments (straightness is defined by the coordinates {x^i}) identified by the first k points in {P_i}. We say that \lambda is smooth if, for any dense subset the sequence of the \lambda_k convergence to \lambda.

\textbf{Conjecture:} a curve is differentiable if it can be seen as a limit of a polyline approximation that is independent on the details of the approximation.

\textbf{Lines and slopes.} Let $U \subseteq \mathbb{R}^n$. Let $L$ be the set of all straight lines. Given two points $P, Q \in U$, we note $l(P,Q) \in L$ the line that passes through $P$ and $Q$. That is, $\{P, Q\} \subset l(P,Q)$. We note $l_1 \parallel l_2$ when two lines $l_1, l_2 \in L$ are parallel. Parallelism is an equivalence relationship. We call the quotient set $S = L_{/\parallel}$ the set of slopes. Given a point $P \in U$ and a slope $s \in S$, we note $l(P, s) \in L$ the line that passes through $P$ with the given slope. That is, $P \in l(P, s) \in s$. It can be shown that $Q \in l(P, s)$ if and only if $P \in l(Q, s)$. Note that the topology of $U$ endows a topology on both $L$ and $Q$ which is Hausdorff and second countable, and therefore $L$ and $Q$ possess a notion of limit. 

\textbf{Oriented segments.} An oriented segment (or simply a segment) is a triplet $\langle P, s, Q \rangle$ with $P, Q \in U$ and $s \in S$ such that $P \in l(Q,s)$. If $P \neq Q$ the segment is said finite; if not, it is said infinitesimal. In can be shown that given two distinct points $P, Q$, there is only one oriented segment $\langle P, s, Q \rangle$. It can also be shown that $\langle P, s, P \rangle$ is a segment for all $s \in S$. We say a sequence of oriented segments $\{\langle P_i, s_i, Q_i \rangle\}_{i=1}^{\infty}$ converges if the limit $\lim\limits_{i \to \infty} \langle P_i, s_i, Q_i \rangle\ = \langle \lim\limits_{i \to \infty} P_i, \lim\limits_{i \to \infty} s_i, \lim\limits_{i \to \infty} Q_i \rangle$ exists.

\textbf{Polylines and refinement.} Two segments $\langle P_1, s_1, Q_1 \rangle$ and $\langle P_2, s_2, Q_2 \rangle$ are consecutive if $Q_1 = P_2$..  A polyline is a finite sequence of finitely many consecutive segments. That is, a polyline is a sequence $\{\langle P_i, s_i, Q_i \rangle\}_{i=1}^{n}$ such that $Q_i = P_{i+1}$. A point $P$ belongs to a polyline if there is some $1 \leq i \leq n$ such that $P=P_i$ or $P=Q_i$. A polyline $\{\langle P^a_i, s^a_i, Q^a_i \rangle\}_{i=1}^{n}$ is said to refine another polyline $\{\langle P^b_j, s^b_j, Q^b_j \rangle\}_{j=1}^{m}$ if $n > m$ and all the points that belong to the second belongs to the first as well without changing their order. That is, for all $1 \leq j \leq m$ we can find $1 \leq i \leq n$ such that $P^a_i = P^b_j$ and also find $1 \leq k \leq n$ such that $Q^a_k = Q^b_j$. Additionally, we say an oriented segment $\langle P^a_i, s^a_i, Q^a_i \rangle$ refines a segment $\langle P^b_j, s^b_j, Q^b_j \rangle$ if we can find $k$ and $l$ such that $k \leq i \leq l$, $P^b_j = P^a_k$ and $Q^b_j = Q^a_l$.

\textbf{Polyline approximating sequence (PAS).} A polyline approximating sequence (PAS) is a sequence of polylines such that each one refines the previous. We say a point $P \in U$ belongs to the PAS if it is one of the boundary points of a segment in a polyline of the sequence. Given a PAS, a segment approximating sequence (SAS) is a sequence of segments such that the $i$-th segment is a segment of the $i$-th polyline, and each segment in the sequence refines the previous segment in the sequence. We say a PAS converges if all the SAS converge.

\textbf{Polyline approximation and differentiability.} Let $\lambda \subset U$ be a curve (i.e. a topologically embedded interval). A polyline approximation for $\lambda$ is a PAS such that the set of points that belong to the PAS is a dense set of $\lambda$. It can be shown that if a SAS of a polyline approximation converges, it does so to an infinitesimal segment which we call limit segment. Moreover, it can be shown that there will be a limit segment for each point of $\lambda$. We say that a curve is differentiable is the limit segment for each point exists and it the same for all polyline approximations.

\bibliography{bibliography}

\end{document}
