\thispagestyle{empty}

\noindent
\textbf{Overview:}\\
The goal of the proposed work is to develop a common axiomatic basis for the information-theoretic structure underlying all physical theories, classical and quantum. The immediate aim is to generalize results of both classical and quantum information theory to a common, more abstract theory that can also handle the infinite-dimensional case. It takes the notion of statistical ensemble as a fundamental concept, as any physical theory must \emph{at least} provide us with a space of statistical ensembles. The relationships among the four mathematical structures required by a space of ensembles---the topology, the convex structure, the proposed entropic structure and the Lie algebra---will be investigated. \\

\noindent
\textbf{Intellectual Merit:}  \\
While the state spaces of classical mechanics and quantum mechanics have different structures, symplectic manifolds and projective spaces of complex Hilbert spaces, respectively, the spaces of classical and quantum \emph{statistical ensembles} have common features.  They are both convex subsets of a locally convex topological vector space whose metric is induced by the information entropy, or more specifically, by the Jensen-Shannon divergence.  Moreover, classical and quantum information theory have numerous corresponding features.  The primary advancement of the research will be to take those common and corresponding features and generalize them, providing a common, mathematically rigorous, foundational structure to both classical and quantum theories.  Such a generalization will provide further insight into the relationships among established physical theories and place constraints on the allowed mathematical structures in potential future physical theories. All axioms will be physically motivated and stated as properties of ensemble spaces.  Preliminary work indicates that the symplectic form of classical mechanics and the inner product (i.e.~Born rule) of quantum mechanics can be recovered by the entropy, tightly linking the geometric and entropic structures, and that the states defined by mechanics can be understood as special cases of ensembles with the lowest possible entropy or the extreme points of the convex set. 


	
\textcolor{blue}{
} 
	

\noindent
\textbf{Broader Impacts:} \\
The proposed activities include an annual online summer school and significant outreach through social media, primarily through YouTube.  The summer school will take place over three days, with six hours of lectures and three hours of discussion total, presenting ideas and results from the project at a level appropriate to advanced undergraduates and above.  For the outreach through social media, videos will be produced to disseminate the results of the research, with some videos aimed at the level of an advanced undergraduate in mathematics and/or physics.  Other videos will be aimed at active researchers in various fields of mathematics and physics, providing a new mathematical framework as well as a new perspective linking traditionally disparate realms of physics that may impact tools or thinking in their own research.  In the spirit of an open-research approach that moreover leverages ``crowd'' contributions, YouTube videos aimed at researchers will additionally be used to advertise open problems that the project personnel may not be able to expediently resolve themselves. Further dissemination of the open-research approach may inspire researchers in completely different areas to adopt a similar approach for their own projects, with potential acceleration of research progress in particular within interdisciplinary areas, where not all necessary expertise may be available within a single research community. 

