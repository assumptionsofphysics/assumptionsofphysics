\thispagestyle{empty}
\noindent
\textbf{Overview:}\\
\textcolor{blue}{[Provide brief overview]} \\

\noindent
\textbf{Intellectual Merit:}  \\
We want to mathematically formalize the intuition that specifying the geometry in physics is equivalent to specifying the entropy. To do so, we need to give a precise, physically motivated, axiomatic definition of what an entropic structure is for a general physical theory. The fundamental grounding concept will be the notion of statistical ensemble, as we can argue that any physical theory must \emph{at least} provide us with a space of statistical ensembles. All axioms will be stated as properties of ensemble spaces. Then we must show that the entropic structure characterizes the geometry of the general space in a way that recovers the geometry of classical and quantum mechanics in these specific cases. While pursuing this work, we will gain insight on how to handle topological issues properly, and hopefully show that ensemble spaces must embed into a second-countable metrizable locally convex vector space, so that the notion of differentiability can be extended to the infinite-dimensional case. We will also keep an eye open to connections to measure theory and Lie algebra, so that more features of physical theories can be generalized in the future. \\

\noindent
\textbf{Broader Impacts:} 
The proposed activities additionally include an annual online summer school and significant outreach through social media, primarily through YouTube.  Videos will be produced to disseminate the results of the research, with some videos aimed at the level of an advanced undergraduate in mathematics and/or physics, furthering education.  Other videos will be aimed at active researchers in various fields of mathematics and physics, providing a new mathematical framework as well as a new perspective linking traditionally disparate realms of physics that may impact tools or thinking in their own research.  In the spirit of an open-research approach that moreover leverages ``crowd'' contributions, YouTube videos aimed at researchers will additionally be used to advertise open problems that the project personnel may not be able to expediently resolve themselves. Further dissemination of the open-research approach may inspire researchers in completely different areas to adopt a similar approach for their own projects, with potential acceleration of research progress in particular within interdisciplinary areas, where not all necessary expertise may be available within a single research community. 

