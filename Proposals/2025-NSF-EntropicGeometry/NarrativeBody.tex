\begin{center}
	\textbf{A generalized axiomatic basis for entropic structures in physical theories} \\
	Christine A. Aidala, University of Michigan
\end{center}
%\vspace{-8mm}

\section{Introduction}

Notes:

Generalized ensemble spaces + geometry defined from entropy; full set of axioms

-Develop axiomatics for the entropy such that they’re physically motivated.  Not taking starting points because they’re convenient mathematically, nor because we think one physical theory (e.g. quantum mechanics) is the foundational one.

-See if this type of framework will help constrain the problem in infinite dimensions (motivation, not goal).  Need to generalize the Poisson bracket structure in infinite dimensions—relationship between entropy and Poisson brackets used as Lie brackets.  Well-known open research problem that there’s no general theory for infinite-dimensional analysis/infinite-dimensional Lie groups.

Motivation (but not goal): Eventual framework to generalize classical and quantum information theory

1. To answer fundamental questions.

2. To do classical and quantum info in a more generalized setting

3. Give a structure for infinite-dimensional analysis that’s constrained enough that mathematicians can prove theorems that are useful for physics

Have: 1. Topological structure; 2. Convex structure; 3. Geometric structure.  

Open problems: 1. Entropic open balls can generate topology? 2. Are all axioms independent?   …

To determine which fundamental structures/properties are needed for all physical theories.

Open with: One of the difficulties with mathematical physics is that physicists aren’t good at specifying the mathematical requirements such that mathematicians can work effectively on the problems.

Mention economists evidently aware of topological issues more than physicists.



\section{Proposed research program}

\subsection{Intellectual merit}



\section{Summary of proposed research program}


\noindent
\textbf{Assessment of intellectual merit:} \\

\noindent
\textbf{Relationship to other current funding:} The PI's current funding from the NSF Division of Physics is for work on LHCb experiment at CERN, and from the NSF AccelNet program is to increase collaboration on quantum chromodynamics (QCD) research between the U.S.~and Latin America.  The PI's funding from the Department of Energy is for work at the Relativistic Heavy Ion Collider (RHIC) on the PHENIX and sPHENIX experiments.  These grants are for a completely separate line of research. Her grant from the John Templeton Foundation, titled ``Why classical probability and classical information theory are incompatible with quantum mechanics and quantum contextuality,'' supports the development of novel non-additive measures (set functions) suitable for quantifying the number of states in quantum mechanics~\cite{aop-nonaddmeas}.  There is no direct overlap with the activities proposed here.  \\

\noindent
\textbf{Relationship to other commitments of PI:} PI Aidala has been continuously involved in high-energy experimental QCD since 2001.  Her separate line of theoretical research on the mathematical foundations of physics is relatively new, with her first publication in 2018~\cite{aop-phys-blueprint}. She has successfully split her efforts between her different research lines for the past six years, with her work related to the Assumptions of Physics project producing 14 peer-reviewed publications, 1 more submitted paper, and a published technical book since 2018.  She is moreover in the process of reducing the number of personnel she supervises on QCD research in order to have more time for her theoretical work---four of her Ph.D.~students in QCD completed their degrees in the last 18 months, and two more are expected to graduate before the start of the proposed project.  




\section{Personnel}
The proposed research will be performed by PI Christine Aidala, her collaborator Gabriele Carcassi, and Ph.D.~student Tobias Thrien, all part of the Physics Department at the University of Michigan.  See also the section on ``Qualifications of the PI and Research Team'' below.  \\
TODO: Add Alejandro assuming he agrees.

Aidala as PI will be responsible for coordinating all of the work, and she will serve as the academic advisor for Thrien.  She will identify suitable journals for publication and conferences at which to disseminate the results, and oversee the direction of the work.  

Carcassi will serve as the technical lead.  He is (currently part-time) staff in the U-M Department of Physics, a Research Area Specialist Senior, with his salary fully supported through grants.  He will identify the most promising technical directions, perform a significant fraction of the calculations, serve as the lead writer for publications until Thrien is at a level to do so, and connect the results of the project to broader results so that the impact of the work is maximized.  With training in engineering and mathematics, he will additionally serve as the primary contact to interface with others in these disciplines who may be interested in the work.

Thrien entered the U-M Physics Ph.D.~program in Fall 2024.  While Carcassi will be the primary individual responsible for ``outward-facing'' technical work, Thrien will focus primarily on inward-facing technical work.  \\
TODO: Be more specific about Thrien's work, per reviews from last year.



%\begin{figure*}[htb]
%	\centering
%	\includegraphics[scale=.6]{figures/PersonnelTimelineNSF2022.png}
%	\caption{Estimated timeline of research activities.}
%	\label{fig:personnelTimeline}
%\end{figure*}



\section{Broader impacts}
In addition to disseminating the goals and results of the project through traditional means such as peer-reviewed publications and presentations at workshops and conferences, the goals and results of the project will be disseminated to wider audiences by leveraging existing social media and other outreach infrastructure that has already been developed for the broader Assumptions of Physics project, in particular, Carcassi's two YouTube channels, as described below in the section on qualifications of the research team.  One of them is aimed at the level of individuals with upper-level undergraduate training in physics and/or mathematics, and the second, more technical channel is aimed at professional researchers, for which he creates videos presenting detailed technical results or describing open technical questions, leveraging the expertise of others to help advance research goals.  Graduate student Tobias Thrien originally discovered the AoP project through Carcassi's YouTube channel while a Bachelor's student in Germany several years ago.  He has since completed a Master's in theoretical cosmology in Germany and is currently a first-year Ph.D.~student at U-M, having come to Michigan specifically to work on the AoP project with the PI and Carcassi.  Regarding the impact of posting open technical problems, our first collaborative paper that resulted from someone reaching out to us after seeing a video, on classical mechanics as the high-entropy limit of quantum mechanics, is currently under review at a journal~\cite{aop-classicallimit}.  We were previously completely unaware of the researcher and his work at the University of Innsbruck, Austria. Carcassi additionally writes monthly blog-style essays related to the AoP project and advertised through social media platforms such as Facebook and LinkedIn.  Thus, the results of the proposed research will reach a much wider audience than through the publications and conference talks alone.  Further dissemination of the open-research approach may inspire researchers in completely different areas to adopt a similar approach for their own projects, with potential acceleration of research progress in particular within interdisciplinary areas, where not all necessary expertise may be available within a single research community.  

TODO: Paragraph about impact of unifying disparate branches of physics, and eventual impact on physics education.\\

\noindent
\textbf{Assessment of broader impacts:}  The reach and impact of videos related to the proposed work will be assessed by the number of views they receive and by other measures of audience engagement such as comments posted or subsequent email contacts.  The impact of unifying disparate branches of physics will need to be assessed on a longer-term timescale than that of the grant, but early indicators could include talk invitations and paper citations. \\
TODO: Consider ways to assess engagement in more detail, per last year's reviews.


\section{Qualifications of the PI and research team}
PI Christine Aidala is a highly recognized leader within physics, having received an NSF CAREER Award, a Sloan Fellowship, and a Presidential Early Career Award for Scientists and Engineers. She was a 2019-20 U.S.~Fulbright Scholar and is a Fellow of the American Physical Society (APS).  She served on the National Academy of Sciences Committee on a U.S.-Based Electron-Ion Collider Science Assessment 2016-18 and is currently serving on the U.S.~Nuclear Science Advisory Committee.  She served as an elected member of the Executive Committee of the APS Division of Nuclear Physics (DNP) 2021-23, and since 2021 she has served as a DNP Ally to ensure that DNP meetings are welcoming and inclusive environments.  Much of her work has been in quantum chromodynamics; however, in 2015 she initiated the Assumptions of Physics project with her collaborator and husband Gabriele Carcassi, and since 2018 they have so far published together 14 peer-reviewed articles~\cite{aop-phys-blueprint,aop-topExpDisting,aop-HamQuantInfo,Carcassi:2021,aop-spacetimeStruct,aop-HamConsInfoEnt,Carcassi2021four,Carcassi:2022bpm,aop-HamPriv,aop-commonLogical,aop-action,aop-nogo,aop-nonaddmeas,aop-unphysHilbert} as well as a technical book on the mathematical foundations of physics~\cite{aop-book}, with one additional article currently under review and publicly available as a preprint~\cite{aop-classicallimit}.  She is the PI on a grant from the John Templeton Foundation that supports the development of novel measure theoretic structures for quantum mechanics. \\

TODO: Add info for Alejandro.

Carcassi is the technical lead for the Assumptions of Physics project.  He is the Co-PI on the Templeton Foundation grant with Aidala.  Related to the AoP project, he has given seminars to audiences of physicists, mathematicians, and philosophers at U-M, Harvard, Caltech, the University of Pavia, Italy, the Munich Center for Mathematical Philosophy, Germany, and the Wolfram Institute, and he and Aidala gave a joint public talk for the Michigan State University Advanced Studies Gateway.  His training in engineering allows him to bring ideas from fields not typically studied by individuals trained primarily in physics, for example from theoretical computer science, information theory, systems theory, signal processing, and operations research.  His position as U-M Physics Department staff without teaching responsibilities provides him with time and flexibility to dedicate to research related to the AoP project such as the work proposed here. \\
TODO: Add Kansas, Saturday Morning Physics, invited talk at AMS meeting, also earlier Topology conference talk?

Together, Aidala and Carcassi are uniquely positioned to advance the mathematical foundations of physics, with Aidala's traditional training in physics and traditional academic position complementing Carcassi's formal training in engineering and his university staff position, which for example gave him the time and flexibility to audit multiple graduate Mathematics courses at U-M.  For the AoP project, their published work so far has covered a wide range of topics.  The ``physical mathematics'' approach within their work is an approach to the mathematical foundations of physics that seeks to construct mathematical structures strictly from axioms and definitions that can be rigorously justified from physical requirements~\cite{aop-book}.  For example, they have shown that experimental distinguishability is mathematically captured by Hausdorff topologies~\cite{aop-book,aop-topExpDisting}, and they have argued that at the smallest space-time scales, at which geometric information becomes in principle experimentally inaccessible, there should still in principle be a regime in which topological structure remains experimentally accessible~\cite{aop-spacetimeStruct}.  Their ``reverse physics'' approach instead is an approach to the foundations of physics that analyzes known theories to identify those physical principles and assumptions that can be taken as their conceptual foundation~\cite{aop-book,Carcassi:2022bpm}. Their reverse physics approach has led for example to a geometric and physical interpretation of the principle of least action~\cite{aop-action} and to a demonstration that the uncertainty principle in quantum mechanics emerges solely from the lower limit on the entropy~\cite{Carcassi:2022bpm}. In other work, they have for example reduced the number of postulates of quantum mechanics, proving that the tensor product postulate for composite systems can be derived from the state and measurement postulates~\cite{Carcassi2021four}, and they have offered a clearer characterization of Shannon information entropy~\cite{Carcassi:2021}. 

Regarding success in mentoring and education, in 2019 Aidala received the U-M Imes and Moore Mentorship Award for exceptional contributions toward recruiting and mentoring graduate students in the sciences from disadvantaged and nontraditional backgrounds, and in 2025 the U-M Rackham Distinguished Graduate Mentor Award.  She has served as the dissertation chair for 10 graduate students in Physics and Applied Physics who have already completed their doctorates, as co-chair for one in Applied and Indisciplinary Mathematics and another in Applied Physics, and is currently the dissertation chair for an additional 4 Ph.D.~students in Physics.  Aidala and Carcassi have also successfully published with former U-M Ph.D.~students in Mathematics~\cite{aop-spacetimeStruct} and Philosophy~\cite{aop-HamPriv} whose primary dissertation work was with other faculty members.  In 2020 Aidala received the U-M John Dewey Award for long-term commitment to the education of undergraduate students.  Between Aidala and Carcassi, they have mentored more than 50 undergraduate students in research since coming to U-M in 2012.  \\

TODO: Add info on summer schools here? 

TODO: Add info for Alejandro.

Regarding the qualifications of the research team to achieve the broader impacts beyond academia as described above, Carcassi has a YouTube channel (@gcarcassi) focused on physics and mathematics with more than 17,500 subscribers currently.  He posts a new video monthly, generally targeted at individuals with an advanced undergraduate background in physics and/or mathematics.  Collectively, his videos on this channel have received over 350k views [TODO: update]. He additionally has a second, more technical YouTube channel (@AssumptionsofPhysicsResearch) aimed at professional researchers, with more than 1600 subscribers.  He uses this channel as a venue to disseminate research results at a more technical level, including video versions of our preprints or published papers as well as conference presentations.  He moreover uses the technical channel as an innovative approach to try to ``crowd-source'' technical expertise to advance the research more efficiently.  Given the breadth of the broader AoP project, there is no single community with the appropriate technical background across all relevant and potentially relevant areas of mathematics and physics.  So far this ``crowd-sourcing'' approach has led to multiple instances of viewers with whom we have had no previous contact pointing us to valuable reference materials, and it recently led to our first coauthored paper with a researcher who reached out to us after seeing an open problem video, on ``Classical mechanics as the high-entropy limit of quantum mechanics''~\cite{aop-classicallimit}. We were previously completely unaware of the researcher and his work in Innsbruck, Austria.  In other forms of reach beyond academia, Carcassi was interviewed in 2024 by Curt Jaimungal for the ``Theories of Everything'' podcast~\cite{Carcassi-ToEInterview}, which has 500k subscribers.  Carcassi additionally writes monthly blog-style essays related to the AoP project and advertised through social media platforms such as Facebook and LinkedIn.  In more creative outreach work, in April 2024 at U-M he performed an original musical comedy lecture, ``An Engineer in the Foundations of Physics,'' which drew approximately 100 attendees. 

Graduate student Tobias Thrien, for whom partial support is requested, originally discovered the Assumptions of Physics project through Carcassi's YouTube channel while a Bachelor's student in Germany several years ago.  He has since completed a Master's in theoretical cosmology in Germany and is currently a second-year Ph.D.~student at U-M.  He attended Aidala and Carcassi's inaugural online summer school on the AoP project in June 2024 and is already highly familiar with the broader project, its goals, and its approaches.  



\section{Results from prior NSF support}
\subsection{Intellectual merit of prior NSF support}

Aidala has been a successful steward of NSF funds since receiving a CAREER Award in 2015.  Her prior research awards have been through the Physics Division, Nuclear Physics - Experiment program: \emph{CAREER: Valence and sea quark dynamics at Fermilab}, 2015-21, and \emph{Studying quantum chromodynamics at LHCb}, 2020-present (renewed in 2023).  She is additionally a Co-PI on an NSF AccelNet award, \emph{AccelNet-Design: Inter-American Network of Networks on Quantum Chromodynamics Challenges}, 2021-present, aimed at further developing collaboration in quantum chromodynamics research between the U.S.~and Latin America.  Her QCD research has focused on advancing the multidimensional quark-gluon structure of the proton and on studying the processes by which scattered quarks or gluons form new bound states (hadrons).  Her group's NSF-funded work includes results published in Nature~\cite{SeaQuest:2021zxb}, Physical Review Letters~\cite{LHCb:2022tbc,LHCb:2019qoc}, Physical Review D Letters~\cite{LHCb:2022rky}, and Physical Review C~\cite{SeaQuest:2022xdu}, among other journals.  Regarding the specific scientific advancements of some of the work, Aidala and her collaborators have studied antimatter in the proton and found surprising evidence that there is an excess of down ``flavored'' antiquarks relative to up flavored antiquarks in the proton for antiquarks carrying as much as 45\% of the proton's total momentum~\cite{SeaQuest:2021zxb,SeaQuest:2022xdu}.  Her group has also led the first-ever measurements of identified hadron species production by high-energy light-mass quarks in proton-proton collisions, multidifferential in multiple kinematic variables simultaneously, showing that existing models poorly predict the relative production rates of some species~\cite{LHCb:2022rky}. [TODO: Add dead cone effect] This proposal is the second time she is submitting to the Applied Mathematics program, based on the research direction in the mathematical foundations of physics that she initiated more recently, which has so far produced 14 peer-reviewed papers since 2018 as well as a technical book, as described above in the section on qualifications of the PI and research team.  \\

TODO: Prior NSF support for Alejandro?



\subsection{Broader impacts of prior NSF support}
Aidala's group has been leading the open data efforts of the LHCb experiment at the Large Hadron Collider at CERN.  Of particular note, they have developed a user-friendly web-based application to securely access and provide analysis documentation for large-scale data sets in high-energy physics, the ``NTuple Wizard''~\cite{Aidala:2023dai}.  It is currently in beta-release to provide access to open data sets from LHCb, and individuals from other experimental collaborations have expressed interest in potentially adapting it to create a similar application to facilitate access to their own open data sets and corresponding documentation.

[TODO: Update and rewrite to avoid DEI language.] In addition, a variety of activities under Aidala's prior NSF grants have contributed directly to training junior scientists and to diversity in the STEM workforce.  A woman postdoctoral scholar was involved 2019-22. Including students with support from on and off of the grant, 8 Ph.D. students and 4 undergrads have been involved in the project since July 2020. An additional 3 distinct Ph.D. students and 7 undergraduates worked on the projects supported by Aidala's CAREER grant 2015-20.  Of the grad students, 2 have been African-American men, 2 Caucasian men, 2 Latino men, 1 an Arabic man, and 4 women, of which 2 Latina.  Former African-American grad student Bryan Ramson is now a permanent staff scientist at Fermilab; former Latina grad student Catherine Ayuso is now Head of Special Projects and Research Manager at Entanglement, Inc.  The others are in postdoctoral positions or still students in the group.  The undergrads included 1 Latino, 1 man who was openly autistic, and 5 women from Mt.~Holyoke College, among which were international students from the Philippines, Bangladesh, and China. Four of the grad/undergrad students openly identify as LGBT.  All Mt.~Holyoke students have continued in STEM.  One received an M.S. from Stanford in Statistics and now works for Amazon, the other four are now Ph.D. students at Harvard, Michigan, Virginia Tech, and Vanderbilt in Applied Physics, Physics, or Electrical Engineering.  \\

TODO: Prior NSF support for Alejandro?




\newpage






