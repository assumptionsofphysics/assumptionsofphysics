\begin{center}
	\textbf{A generalized axiomatic basis for entropic structures in physical theories} \\
	Christine A. Aidala, University of Michigan
\end{center}
%\vspace{-8mm}

\section{Introduction}

\iffalse
Physics is currently a patchwork of different theories with no common underlying foundation. As part of our larger project, Assumptions of Physics, we aim to develop an axiomatic basis for states and processes that can be equally applied to different theories, such as classical and quantum mechanics. Different theories are then recovered as specializations of the more general structure.


While the state spaces of classical (i.e.~symplectic manifolds) and quantum mechanics (i.e.~projective Hilbert spaces) are rather different, the space of ensembles of both have a similar structure. They are both convex subsets of a topological vector space, equipped with a strictly concave entropy function. The core idea, then, is to use ensembles as a way to generalize the notion of physical states, and the entropy to characterize the geometry in a way that is independent of the particulars of each physical theory. In the past year, we have done significant preliminary work that shows that not only is this approach possible, but it is extremely productive. The proposed work is to turn those preliminary insights in an axiomatic specification that is fully physically motivated. If successful, such an approach will provide the following benefits:
\begin{enumerate}
	\item clarify exactly what physical premises guarantee the basic geometric structures of physical theories, allowing a more principled formulation of current physical theories as well as the search for new ones;
	\item provide a generalized structure for information theory, possibly allowing future generalization of results from both classical and quantum information theory
	\item provide useful constraints to more precisely formulate physical theories in the infinite-dimensional case.
\end{enumerate}
\fi

\section{Proposed research program}

The proposed work starts from the observation that the space of statistical ensembles of both classical mechanics have a lot in common. They are both convex subsets of a locally convex topological vector spaces whose metric is induced by the entropy. More specifically, by the Jensen-Shannon divergence. This is in contrast of the different structure of classical and quantum mechanics, which rely on symplectic manifolds and projective spaces of complex Hilbert spaces respectively. Even more notably, our preliminary work shows that the symplectic form of classical mechanics and the inner product (i.e. Born rule) of quantum mechanics can be indeed recovered by the entropy. This makes sense as the states defined by mechanics can be understood as special cases of ensembles with the lowest possible entropy or the extreme points of the convex set.

The goal is to find minimal requirements for an entropic structure that is common to all physical theories in such a way that results from classical and quantum theories, in areas like mechanics and information theory, can be generalized. In particular, it needs to be understood the interaction between the four mathematical structure required by a space of ensembles: the topology, the convex structure, the proposed entropic structure and the Lie algebra. Moreover, it is crucial that these premises are well founded on physical requirements, so that the generalization is of value to physics.



\subsection{Intellectual merit}

The goal of this proposal is to develop a common axiomatic basis for the information theoretic structure underlying to all physical theory, classical and quantum. The immediate aim is to generalize results of both classical and quantum information theory to a common more abstract theory that can also handle the infinite dimensional case. Longer term, we aim to use the axioms developed as part of a larger effort to give unified base theory for physical theories and clean up some mathematical issues.

While the geometric structure of both classical and quantum mechanics are rather different, their space of ensembles have a rather similar structure: they consist of a bounded convex subset of a metrizable and second countable locally convex topological vector space, where the metric can be derived from the entropy (e.g. trough the square root of the Jensen-Shannon divergence). There are, therefore, three common structures at play: a topology (which physically represents experimental verifiability), a convex structure (which physically represents the ability to make statistical mixtures) and an entropic structure (which physically represents the variability of the elements within each ensemble). While the axioms for the topology and the convex structure are readily available, to our knowledge, the correct axioms for the entropic structure still needs to be identified. Moreover, the interplay between the topology, the convex structure and the entropic structure opens additional mathematical questions which have not been investigated.

As a guide to the reader, let us recall how ensembles are defined in both classical and quantum mechanics. For a discrete classical system, the state space $X$ is a countable set with the discrete topology, the space of ensembles consists of all probability measure over the Borel algebra of $X$ and the entropy is given by $S(p_i) = - \sum_i p_i \log p_i$. Note that all logs here are assumed base 2. In classical mechanics, instead, the state space $X$ is a symplectic manifold, which is therefore equipped with the  Liouville measure $\mu(U) = \int_U \omega^n$, where $\omega$ is the symplectic form. The space of statistical ensembles is the space of all probability measures over the Borel algebra of $X$ that are absolutely continuous with respect to the Liouville measure. For these measures, the Radon-Nikodym derivative $\rho = \frac{dp}{d\mu}$ exists and the entropy is defined as $S(\rho) = - \int \rho \log \rho d\mu$. In particular, for a uniform distribution with support $U$, the entropy reduces to $S(\rho_U) = \log \mu(U)$. In quantum mechanics, the state space is given by the projective space $P(\mathcal{H})$ of a complex Hilbert space $\mathcal{H}$, the space of ensembles is given by the density operators (i.e. positive semi-definite, self-adjoint trace-one operators) and the entropy is given by $S(\rho) = -\tr(\rho \log \rho)$.

Note that in all three cases the definition of the entropy is tightly bound to the definition of the space, therefore the biggest problem is understanding how one can give a single characterization that is general enough to encompass the three cases, while constrained enough to recover the correct expression in each cases. Fortunately, in our preliminary work we already have found the solution. In all cases, the entropy presents an upper bound to its increase under convex combinations: $S(p\ens[a] + \bar{p} \ens[b]) \leq I(p) + p S(\ens[a]) + \bar{p} S(\ens[b])$ where $I(p) = -p \log p - (1-p) \log (1-p)$. This maximum increase is given precisely when the two ensembles are orthogonal with respect to the inner product of the respective spaces. In the classical case, this means that the measures have disjoint support and in the quantum case that the eigenspaces of non-zero eigenvalues are orthogonal. In other words, characterizing the interaction of the entropy function with the topology and the convex structure is enough to characterize the entropic structure. We will see that, in fact, the form of $I(p)$, the Shannon entropy formula, can be recovered.

Another important insight from our preliminary work is that the entropy is not only responsible for the geometry of the ensemble space, but it is also intimately related to the geometry of the state space of both classical and quantum mechanics. That is, the entropic structure of the ensemble space is equivalent to the geometric structure of the state space. As we saw in classical mechanics, the symplectic form gives us the Liouville measure which in turn is used to define the entropy. Therefore, the entropy of ensemble is defined from the geometry. However, note that the entropy for a uniform distribution $\rho_U$ over a set $U$ is given by $S(\rho_U) = \log \mu(U)$. Therefore, if we are given the entropy for all probability measures, we can recover the Liouville measure by setting $\mu(U) = 2^{S(\rho_U)}$. While this does not recover the symplectic form directly, note that the symplectic form can also be used to define the entropy of the marginal distributions along each degree of freedom. Around a point, in fact, we can define Darboux coordinates and define a joint probability distribution that is the product of independent marginals along each pair. One finds a similar relationships between the entropy of uniform marginal distributions and the integral of the symplectic form over the support of the marginal.

In quantum mechanics, the geometry is given by the Born rule
\begin{equation}
	p(\phi|\psi) = \frac{\< \phi | \psi \>\< \psi | \phi \>}{\< \psi | \psi \>\< \phi | \phi \>}.
\end{equation}
Given two states $\psi$ and $\phi$, the ensemble given by their equal mixture, mathematically $\rho_{\psi\phi} = \frac{1}{2}|\psi\>\<\psi|+\frac{1}{2}|\phi\>\<\phi|$,  has von Neumann entropy $S(\rho_{\psi\phi}) = I\left(\frac{1 + \sqrt{p(\phi|\psi)}}{2} \right)$ where $I$, as before, is the Shannon entropy for two elements. As before, the entropy is defined in terms of the inner product. However, the relationship between $S(\rho_{\psi\phi})$ and $p(\phi|\psi)$ is invertible. This means that, given the entropy of the equal mixtures of all pairs of states we can reconstruct the Born rule and, therefore, the state space.

These insights from our preliminary work shows that the entropic structure plays a crucial role in all physical theories and merits to be studied. Moreover, while typically information geometry is casted in finite dimensional spaces, there is nothing preventing its generalization to infinite dimensions. If the entropic structure, in fact, imposes a distance function whose open balls are convex, then this provides us with a locally convex topology which is exactly what is needed to define calculus in infinite dimensions. Moreover, the Lie algebras of both classical (i.e. Poisson brackets) and quantum (i.e. commutators) have a special relationship with the geometric and entropic structure, which may provide us with a template for generalizes those structure as well. 

Given these very promising preliminary findings, the goal is to find a minimal set of axioms to characterize the entropic structure. Minimal in the sense that, in the spirit of analogous work on the foundations of set theory or reverse mathematics, the axioms can be shown to be satisfied by the different theories (which now become model of the larger theory) but also allow to rederive the entropy formula for each specific theory. Let us, then, go through the current state of our preliminary work, and show the type of questions that the overall approach touches.


%The goal is to find a very limited set of axioms to define the entropic structure of a physical theory that can be justified from physical considerations, show that this set of axioms imposes a geometric structure on a generic space of statistical ensembles and that it recovers the standard geometric structures of classical and quantum mechanics. Furthermore, the same structure should be connected to topological and measure theoretic structures used in physics.

%and Other that are required by any physical theory to define its entropic structure, which  and construct a general connection to geometry, calculus and topology, thus giving a generalized precise mathematical characterization of physical theories. In our numerous conversations with mathematicians, physicists and philosophers, it is clear that such work is sorely needed, and, since it lies at the intersection of different fields, it requires an interdisciplinary approach.
 
%We want to stress that finding the correct starting point is not just a mathematical issue, but also a physics problem. Let us explore this problem through an example. The space $M_1([0,1])$ of probability measures defined over the $[0,1]$ interval is an infinite-dimensional vector space. As Aliprantis and Border note in the context of their own field of economics [add reference],  ``Since there is more than one topology of interest on an infinite dimensional space, the choice of topology is a key modeling decision that can have economic as well as technical consequences.'' Unbeknowst to many physicists, this is true for physics as well. If we choose a weak topology, we are saying the convergence of expectation values of all continuous functions is enough to determine convergence of ensembles. While this may seem plausible at first glance, we can find a weakly convergent sequence for which the entropy does not converge. If the entropic structure of a physical theory is really equivalent to its geometric structure, not only should the entropy be continuous, but it is likely that the topology should be the one generated by the entropic structure. Ultimately, answering these questions means understanding both the physical and mathematical consequences of those decisions.

The central focus is the intuitive notion of statistical ensemble (i.e.~the collection of all outputs of a preparation procedure), as this seems to be particularly suitable for generalization for a number of both practical and conceptual considerations. In terms of the math, as we have seen, the structure of statistical ensembles of both classical and quantum mechanics can be made to look very similar. From the experimental side, statistical ensembles are what we produce and measure experimentally. On the conceptual side, the reproducibility requirement of scientific investigation means that physical laws are of the form ``whenever I prepare a system according to this procedure, I will find this range of outcomes.'' That is, physical laws are relationships between ensembles. Therefore, it makes sense to require that any physical theory must \emph{at least} posit a space of ensembles. This provides us with a strong conceptual starting point and it is properties of ensembles that needs to be specified through axioms.

The first axiom establishes that an ensemble space $\Ens$ is a $\mathsf{T}_0$ second-countable topological space. This is based on our previous work, that established a link between the experimental verifiability of the theory and its topology. The open sets of the topology correspond to experimentally verifiable statements (i.e.~statements associated with an experimental test that terminates successfully in finite time if and only if the statement is true) while Borel sets correspond to statements associated with an experimental test without termination guarantees~\cite{aop-book,aop-topExpDisting}. Real and integer numbers, for example, are recovered from an idealized model representing references of a reference system. The key insight is that the set-theoretic operations represent a two-valued logic system (i.e.~truth or falsehood of statements) while the topology keeps track of a three-valued logic system (i.e.~interior, exterior and boundary represent respectively successful termination, unsuccessful termination and non-termination). The topology, therefore, plays an important role.

The second axiom imposes a convex structure which represent the ability to perform statistical mixtures. Given two ensembles $\ens[a], \ens[b] \in \Ens$, we can always choose a probability $p \in [0,1]$ and select $\ens[a]$ $p$ percent of the times and $\ens[b]$ $\bar{p} = 1-p$ percent of the times. This gives the ensemble space a convex structure as $p \ens[a] + \bar{p} \ens[b] \in \Ens$ for all $\ens[a], \ens[b] \in \Ens$ and $p \in [0,1]$. This insight is already used in the literature, particularly in Generalized Probabilistic Theories (GPTs) [add refs]. The main innovation in this respect is that we do not limit ourselves to finite combinations and we do not simply allow all infinite combinations (which can be shown to include physically untenable cases). Rather, we let the topology do the work. That is, infinite convex combinations exist only if they are the topological limit of finite convex combinations. On physical grounds, the mixing operation must be continuous in the natural topology defined by experimental verifiability, much like addition and scalar multiplication are continuous in a topological vector space, leading to the idea of a topological convex space.

Surprisingly, topological convex spaces do not seem to have been studied, and this structure alone presents interesting sets of questions.  For example, it is established that a convex space embeds into a vector space if and only if it is cancellative. That is, if $p \ens[a] + \bar{p} \ens = p \ens[b] + \bar{p} \ens$ for some $p \in (0,1)$ implies $\ens[a] = \ens[b]$. The question now is what are necessary and sufficient conditions for a topological convex space to embed continuously into a topological vector space? Can we have, for example, an Hausdorff second countable topological convex space that is not a continuous embedding of a topological vector space? Topological vector space present a number of nice results which may or may not hold in topological convex spaces. For example, any $\mathsf{T}_1$ second countable topological vector spaces is metrizable. Does that same result hold for topological convex spaces? Another interesting aspect is the notion of closure. In a topological vector space, the topological closure of a convex set is convex. We found that this is true in topological convex spaces as well. Additionally, there is another type of closure that may be more interesting physically. We can define the $\sigma$-hull of a set $A$ the set of possible infinite convex combinations, which can also be shown to be a convex set. This will be a superset of the hull of $A$, because it will include all finite convex combinations, and it will be a subset of the closed hull of $A$, as all infinite convex combinations are topological limits. However, it can be shown that, in some case, the hull, the $\sigma$-hull and the closed hull are all different. From the physics side, the $\sigma$-hull may be a more physically meaningful closure as it represents the true limit process of statistical mixing, the operation the axiom is characterizing. Therefore, there are already interesting topics to study even without imposing the entropic structure.

The third axiom, central to the proposed work, imposes an entropy function that quantifies the variability of elements within a statistical ensemble. Though we imagine ensembles to be a collection of instances, these instances are never formalized as they need to be left unspecified to leave the theory general. However, the variability of the elements within an ensemble is physically significant as it will influence, for example, the variability of statistical measurements. As we showed in previous work~\cite{Carcassi:2021}, entropy quantifies the variability of the elements within a distribution, which gives us a physically meaningful starting point. The primary goal of the proposed work, then, is to find what axiomatic properties this variability indicator must have, and to understand their conceptual, physical and mathematical implications.

In our preliminary work, for example, the axiom equips each ensemble space with an entropy function $S : \Ens \to \mathbb{R}$ currently has the following properties:
\begin{itemize}
	\item Continuity
	\item Strict concavity: $S(p\ens[a] + \bar{p} \ens[b]) \geq p S(\ens[a]) + \bar{p} S(\ens[b])$ with the equality holding if and only if $\ens[a] = \ens[b]$
	\item Upper variability bound: there exists a universal function $I : [0,1] \to \mathbb{R}$ (i.e.~the same for all ensemble spaces) such that $S(p\ens[a] + \bar{p} \ens[b]) \leq I(p) + p S(\ens[a]) + \bar{p} S(\ens[b])$; if the equality holds, $\ens[a]$ and $\ens[b]$ are orthogonal, noted $\ens[a] \ortho \ens[b]$
	\item Mixtures preserve orthogonality: $\ens[a] \ortho \ens[b]$ and $\ens[a] \ortho \ens[c]$ if and only if $\ens[a] \ortho p \ens[b] + \bar{p} \ens[c]$ for any $p \in (0,1)$
\end{itemize}
These properties can be justified by the physics. For example, if we are performing a statistical mixture between two ensembles, the average variability cannot decrease, which justifies the concavity. The goal is to have a full justification for each property to make sure that we are not leaving out any possibly interesting physical theories. Similarly to one does in foundations of mathematics, we typically construct different models of the various formulations of the axioms to check that each property is truly independent from the other, and that is physically required.

Though we do not believe the axiom lists all necessary properties of the entropic structure, it does list enough of them to already prove interesting results. For example, we can show that $I(p)$, the maximum entropy increase during mixing, must be, up to a multiplicative constant, the Shannon entropy~\cite{Shannon} $I(p)= - p \log p - \bar{p} \log \bar{p}$. The proof is general, meaning that if we are right that those axioms must hold for any physical theory, then the Shannon entropy is of truly general applicability. Moreover, in the case of both classical and quantum mechanics, states can be decomposed into orthogonal components, which means the upper variability is an equality and can be used to calculate the entropy from that decomposition. This will allows us to recover the expressions for the entropy in the three different cases, classical discete, classical continuous, and quantum. Similarly, we will be able to show that, from this structure alone, the geometric structure of both classical and quantum mechanics can be recovered.

Our preliminary work also shows that an entropy satisfying those properties forces the convex space to be cancellative, meaning that it embeds into a vector space. That is, a convex space that does not embed into a vector space cannot posses an entropic structure. We can, in fact, produce a model of that case and see exactly why the entropic structure fails. We can also show ensemble spaces must be ``directionally bounded,'' meaning its intersection with any affine line is a finite interval. It is an open question as to whether the ensemble must be a bounded set. We can also show that some models of GPTs (i.e. Generalized Probabilistic Theories) are inconsistent with an entropic structure and therefore cannot represent a physically meaningful space of ensembles. In other words, the entropic structure constrain the type of spaces possible in surprising and interesting ways.

Another line of inquiry is the relationship between the entropic structure and a metric ensemble space. For example, we can define the mixing entropy as $MS(\ens[a], \ens[b]) = S\left(\frac{1}{2}\ens[a] + \frac{1}{2} \ens[b]\right) - \left(\frac{1}{2} S(\ens[a]) + \frac{1}{2} S(\ens[b])\right)$. If we express the entropy in bits, the mixing entropy satisfies the following:
\begin{itemize}
	\item non-negativity: $MS(\ens[a], \ens[b]) \geq 0$
	\item identity of indiscernibles: $MS(\ens[a], \ens[b]) = 0 \iff \ens[a]=\ens[b]$
	\item unit boundedness: $MS(\ens[a], \ens[b]) \leq 1$
	\item maximality of orthogonals: $MS(\ens[a], \ens[b]) = 1 \iff \ens[a] \ortho \ens[b]$
	\item symmetry: $MS(\ens[a], \ens[b]) = MS(\ens[b], \ens[a])$
\end{itemize}
It is almost a distance function, as it does not satisfy the triangle inequality, but it recovers the Jensen-Shannon divergence (JSD) in both classical [add ref] and quantum [add ref] mechanics. While the JSD is the square of a distance function in both classical and quantum information theory, our current axioms do not guarantee $MS$ to have this property in general. An interesting question, then, is whether enough physical constraints exist for that to be the case. The $MS$ can be used to create open balls and show that the topology must be Hausdorff. It is not yet clear whether this topology is metrizable and whether it must coincide with the topology of the ensemble space.

A related line of questions relates the entropic structure to elements of differential geometry. First, is the relationship to differentiability. We are aware that the current axioms do not guarantee the entropic open balls to be convex, even though they are in both the classical and quantum case. We suspect that to be the case in general, even though we haven't yet found a physically satisfying premise. If the convexity of the open ball is established, then the topology generated by entropic open balls is locally convex, which give us enough structure to define calculus in infinite dimensional spaces. That is, the entropic structure would be responsible for the ability to define derivatives.

In the final dimensional case, and if the entropy is twice differentiable with respect to the convex coefficients, the strict concavity of the entropy means that the Hessian is negative definite, which allows us to define a metric tensor. Let $V_{\ens[a]}$ be the embedding vector space with $\ens[a] \in \Ens$ as origin. This will be a differentiable manifold. On a linear coordinate chart we can define $g_{\ens}( v, w ) = - \frac{\partial^2 S}{\partial \ens ^2}( v, w )$ where $v$ and $w$ are elements of the tangent space at $\ens \in \Ens \subset V_{\ens[a]}$. Since all Hessians are symmetric, $g_{\ens}$ will be bilinear, symmetric and positive definite: it is a metric tensor and $V_{\ens[a]}$ is a Hessian manifold. In classical and quantum mechanics it recovers the Fisher information metric. The immediate goal, then, is to see how much of information geometry can be generalized to this case. If a locally convex topology can be established, the further goal is to extend this to the infinite dimensional case.

Another goal is to recover the geometric structure of the state spaces of both classical and quantum mechanics from the entropic structure, to formally show the equivalence. For quantum mechanics, one can simply define the Born rule from the entropy as we have seen before. For classical mechanics, the situation is a bit more complex as we need to recover first the Liouville measure. This can be done by noting two things: for a uniform distribution with support $U \subseteq X$, $\mu(U) = 2^{S(\rho_U)}$ and the uniform distribution over $U$ has the highest entropy of all distributions with support $U$. Given a set $A \subseteq \Ens$ of ensemble, we can then define the state capacity as supremum of the exponential of the hull of $A$, which physically means the highest entropy achievable with a statistical mixture of elements of $A$. This set function has nice properties: it is non-negative, monotonic, subadditive in general and additive on orthogonal sets of ensembles. Note that if $A_U \subseteq \Ens$ is the subspace of all probability measures with support $U \subseteq X$, $A_U$ is orthogonal to $A_V$ if and only if $U$ and $V$ are disjoint. This means that the state capacity restricted to the lattice of those subspaces is an additive measure. Interestingly enough, one can follow the same construction in the quantum case, and the measure will return the dimensionality of the subspace. In the areas of non-additive measure, there are more interesting questions and one can create a non-additive version of probability, which may point to a non-additive version of probability theory. We have already looked at this problem, and, unfortunately, the standard techniques that have been developed in that area are not sufficient.

A possible fourth axiom maybe require to conclude all the requirements from physical theories. In both classical and quantum mechanics, there a fundamental Lie algebra on the statistical variables. In classical mechanics, this is captured by the Poisson bracket while in quantum mechanics is captured by the commutator. Since the set of the probability measures defined on an odd dimensional manifold satisfies the previous axioms, at least in the current form, it may be that the above structure is not sufficient. However, there is an intimate relationship between this Lie algebraic structure and the entropic structure since, in both cases, they describe the relationship between the generators of transformations that leave the geometric structure (i.e.~symplectic form and inner product respectively) unchanged, which means, as outlined before, leaving the entropic structure unchanged. That is, it describes the group of deterministic and reversible motions, those that leave the entropy unchanged~\cite{aop-HamConsInfoEnt,aop-HamQuantInfo,aop-book}. Part of the investigation for the correct specification of the entropic structure, then, will need to consider its impact on the definition of the Lie algebraic structure. For example, in our reverse physics approach to classical mechanics we were able to show that a symplectic structure is necessary and sufficient to define an entropy that is invariant under frame transformations. If such requirement could be specified in general, the principle of relativity could be shown to be equivalent to this structure.

As we have seen, the program is rich and connects to many areas of physics, information theory and mathematics. To not get lost, let us reiterate the primary goals. We want to mathematically formalize the intuition that specifying the geometry in physics is equivalent to specify the entropy. To do so, we need to give a precise, physically motivated, axiomatic definition of what an entropic structure is for a general theory. Then show that this characterized the geometry of the general space in a way that recovers the geometry of classical and quantum mechanics in the specific case. While doing that, we will gain insight on how to handle topological issues properly, and hopefully show that ensemble spaces must embed into a second countable metrizable locally convex vector space, so that notion of differentiability can be extended to the infinite dimensional case. We will also keep an eye open to connection to measure theory and Lie algebra, so that more features of physical theories can be generalize in the future.


Notes:

-Develop axiomatics for the entropy such that they’re physically motivated.  Not taking starting points because they’re convenient mathematically, nor because we think one physical theory (e.g. quantum mechanics) is the foundational one.

-See if this type of framework will help constrain the problem in infinite dimensions (motivation, not goal).  Need to generalize the Poisson bracket structure in infinite dimensions—relationship between entropy and Poisson brackets used as Lie brackets.  Well-known open research problem that there’s no general theory for infinite-dimensional analysis/infinite-dimensional Lie groups.

Motivation (but not goal): Eventual framework to generalize classical and quantum information theory

1. To answer fundamental questions.

2. To do classical and quantum info in a more generalized setting

3. Give a structure for infinite-dimensional analysis that’s constrained enough that mathematicians can prove theorems that are useful for physics

To determine which fundamental structures/properties are needed for all physical theories.

Open with: One of the difficulties with mathematical physics is that physicists aren’t good at specifying the mathematical requirements such that mathematicians can work effectively on the problems.



\section{Summary of proposed research program}
[Add brief recap]

\noindent
\textbf{Assessment of intellectual merit:} \\

\noindent
\textbf{Relationship to other current funding:} The PI's current funding from the NSF Division of Physics is for work on LHCb experiment at CERN, and from the NSF AccelNet program is to increase collaboration on quantum chromodynamics (QCD) research between the U.S.~and Latin America.  The PI's funding from the Department of Energy is for work at the Relativistic Heavy Ion Collider (RHIC) on the PHENIX and sPHENIX experiments.  These grants are for a completely separate line of research. Her grant from the John Templeton Foundation, titled ``Why classical probability and classical information theory are incompatible with quantum mechanics and quantum contextuality,'' supports the development of novel non-additive measures (set functions) suitable for quantifying the number of states in quantum mechanics~\cite{aop-nonaddmeas}.  There is no direct overlap with the activities proposed here.  \\

\noindent
\textbf{Relationship to other commitments of PI:} PI Aidala has been continuously involved in high-energy experimental QCD since 2001.  Her separate line of theoretical research on the mathematical foundations of physics is relatively new, with her first publication in 2018~\cite{aop-phys-blueprint}. She has successfully split her efforts between her different research lines for the past seven years, with her work related to the Assumptions of Physics project producing 15 peer-reviewed publications, 1 more submitted paper, and a published technical book since 2018.  She is moreover in the process of reducing the number of personnel she supervises on QCD research in order to have more time for her theoretical work---four of her Ph.D.~students in QCD completed their degrees in the last 18 months, and two more are expected to graduate before the start of the proposed project.  




\section{Personnel}
The proposed research will be performed by PI Christine Aidala, her collaborator Gabriele Carcassi, and Ph.D.~student Tobias Thrien, all part of the Physics Department at the University of Michigan.  See also the section on ``Qualifications of the PI and Research Team'' below.  \\
TODO: Add Alejandro if he agrees.

Aidala as PI will be responsible for coordinating all of the work, and she will serve as the academic advisor for Thrien.  She will identify suitable journals for publication and conferences at which to disseminate the results, and oversee the direction of the work.  

Carcassi will serve as the technical lead.  He is (currently part-time) staff in the U-M Department of Physics, a Research Area Specialist Senior, with his salary fully supported through grants.  He will identify the most promising technical directions, perform a significant fraction of the calculations, serve as the lead writer for publications until Thrien is at a level to do so, and connect the results of the project to broader results so that the impact of the work is maximized.  With training in engineering and mathematics, he will additionally serve as the primary contact to interface with others in these disciplines who may be interested in the work.

Thrien entered the U-M Physics Ph.D.~program in Fall 2024.  While Carcassi will be the primary individual responsible for ``outward-facing'' technical work, Thrien will focus primarily on inward-facing technical work.  \\
TODO: Be more specific about Thrien's work, per reviews from last year.



%\begin{figure*}[htb]
%	\centering
%	\includegraphics[scale=.6]{figures/PersonnelTimelineNSF2022.png}
%	\caption{Estimated timeline of research activities.}
%	\label{fig:personnelTimeline}
%\end{figure*}



\subsection{Broader impacts}
In addition to disseminating the goals and results of the project through traditional means such as peer-reviewed publications and presentations at workshops and conferences, the goals and results of the project will be disseminated to wider audiences by leveraging existing social media and other outreach infrastructure that has already been developed for the broader Assumptions of Physics project, in particular, Carcassi's two YouTube channels, as described below in the section on qualifications of the research team.  One of them is aimed at the level of individuals with upper-level undergraduate training in physics and/or mathematics, and the second, more technical channel is aimed at professional researchers, for which he creates videos presenting detailed technical results or describing open technical questions, leveraging the expertise of others to help advance research goals.  Graduate student Tobias Thrien originally discovered the AoP project through Carcassi's YouTube channel while a Bachelor's student in Germany several years ago.  He has since completed a Master's in theoretical cosmology in Germany and is currently a second-year Ph.D.~student at U-M, having come to Michigan specifically to work on the AoP project with the PI and Carcassi.  Regarding the impact of posting open technical problems, our first collaborative paper that resulted from someone reaching out to us after seeing a video, on classical mechanics as the high-entropy limit of quantum mechanics, is currently under review at a journal~\cite{aop-classicallimit}.  We were previously completely unaware of the researcher and his work at the University of Innsbruck, Austria. Carcassi's August 2025 video on ``The need for physical mathematics'' has already received 49k views and led directly to an invitation to speak at the American Mathematical Society conference in Washington, DC in January 2026. Carcassi additionally writes monthly blog-style essays related to the AoP project and advertised through social media platforms such as Facebook and LinkedIn.  Thus, the results of the proposed research will reach a much wider audience than through the publications and conference talks alone.  Further dissemination of the open-research approach may inspire researchers in completely different areas to adopt a similar approach for their own projects, with potential acceleration of research progress in particular within interdisciplinary areas, where not all necessary expertise may be available within a single research community.  

Carcassi and Aidala will host an annual online school that includes an introduction to generalized ensemble spaces and the mathematical relationships between geometric and entropic structures, showcasing various results from the proposed work in a didactic manner.  The school will consist of six hour-long lectures, two per day over three days, each followed by 30 minutes of questions and discussion, from 10 am - 1 pm U.S. Eastern Time, enabling individuals from across the Americas, Europe, Africa, and parts of Asia to participate at a reasonable hour.  The school will be advertised primarily through social media.  Registrants will be given a Zoom link for participation, and those who choose not to register will be able to follow a livestream on YouTube (on Carcassi's technical channel) and ask questions through the livestream chat.  There will be no fee to participate.  All lectures and discussion will also be recorded and available through YouTube later. With 27 registrants for the inaugural Assumptions of Physics online school in June 2024, this grew to 167 registrants in June 2025, mainly undergraduate and graduate students and recent graduates in physics and/or mathematics, but also a handful of professors and other senior researchers.  Among the registrants were individuals from India, Bangladesh, Pakistan, Ethiopia, Egypt, Sudan, Bolivia, Brazil, and Singapore, in addition to North America and Europe. During the period of proposed work, we expect the school to reach several hundred individuals across similar demographics.    

Beyond the immediate research goals, the proposed work will serve to unify what are currently disparate branches of physics---classical and quantum, and aspects of statistical mechanics.  Developing a common framework will pave the way for eventual new approaches to physics education, with less compartmentalized teaching across the core advanced undergraduate courses.  Finding general mathematical structures that any physical theory must have will also help constrain the search for new physical theories.  \\

\noindent
\textbf{Assessment of broader impacts:}  The reach and impact of videos related to the proposed work will be assessed by the number of views they receive and by other measures of audience engagement such as comments posted or subsequent email contacts, as well as potential talk invitations or proposals for collaboration.  The reach and impact of the school will be assessed by the number of registrants and their demographics, as well as the subsequent views of the recorded lectures.  The impact of unifying disparate branches of physics will need to be assessed on a longer-term timescale than that of the grant, but early indicators could include talk invitations and paper citations. \\



\section{Qualifications of the PI and research team}
PI Christine Aidala is a highly recognized leader within physics, having received an NSF CAREER Award, a Sloan Fellowship, and a Presidential Early Career Award for Scientists and Engineers. She was a 2019-20 U.S.~Fulbright Scholar and is a Fellow of the American Physical Society (APS).  She served on the National Academy of Sciences Committee on a U.S.-Based Electron-Ion Collider Science Assessment 2016-18 and on the U.S.~Nuclear Science Advisory Committee 2023-25.  She served as an elected member of the Executive Committee of the APS Division of Nuclear Physics (DNP) 2021-23, and since 2021 she has served as a DNP Ally to ensure that DNP meetings are welcoming and inclusive environments.  Much of her work has been in quantum chromodynamics; however, in 2015 she initiated the Assumptions of Physics project with her collaborator and husband Gabriele Carcassi, and since 2018 they have so far published together 15 peer-reviewed articles [one more to add]~\cite{aop-phys-blueprint,aop-topExpDisting,aop-HamQuantInfo,Carcassi:2021,aop-spacetimeStruct,aop-HamConsInfoEnt,Carcassi2021four,Carcassi:2022bpm,aop-HamPriv,aop-commonLogical,aop-action,aop-nogo,aop-nonaddmeas,aop-unphysHilbert} as well as a technical book on the mathematical foundations of physics~\cite{aop-book}, with one additional article currently under review and publicly available as a preprint~\cite{aop-classicallimit}.  She is the PI on a grant from the John Templeton Foundation that supports the development of novel measure theoretic structures for quantum mechanics. \\

TODO: Add info for Alejandro if agrees.

Carcassi is the technical lead for the Assumptions of Physics project.  He is the Co-PI on the Templeton Foundation grant with Aidala.  Related to the AoP project, he has given seminars to audiences of physicists, mathematicians, and philosophers at places including Harvard, Caltech, Oxford (UK) the University of Pavia (Italy) the Munich Center for Mathematical Philosophy (Germany) and the Wolfram Institute.  He and Aidala gave a joint public talk for the Michigan State University Advanced Studies Gateway, and he will give a U-M ``Saturday Morning Physics'' public talk in November 2025.  His training in engineering allows him to bring ideas from fields not typically studied by individuals trained primarily in physics, for example from theoretical computer science, information theory, systems theory, signal processing, and operations research.  His position as U-M Physics Department staff without teaching responsibilities provides him with time and flexibility to dedicate to research related to the AoP project such as the work proposed here. \\

Together, Aidala and Carcassi [add Alejandro if agrees] are uniquely positioned to advance the mathematical foundations of physics, with Aidala's traditional training in physics and traditional academic position complementing Carcassi's formal training in engineering and his university staff position, which for example gave him the time and flexibility to audit multiple graduate Mathematics courses at U-M.  For the AoP project, their published work so far has covered a wide range of topics.  The ``physical mathematics'' approach within their work is an approach to the mathematical foundations of physics that seeks to construct mathematical structures strictly from axioms and definitions that can be rigorously justified from physical requirements~\cite{aop-book}.  For example, they have shown that experimental distinguishability is mathematically captured by Hausdorff topologies~\cite{aop-book,aop-topExpDisting}, and they have argued that at the smallest space-time scales, at which geometric information becomes in principle experimentally inaccessible, there should still in principle be a regime in which topological structure remains experimentally accessible~\cite{aop-spacetimeStruct}.  Their ``reverse physics'' approach instead is an approach to the foundations of physics that analyzes known theories to identify those physical principles and assumptions that can be taken as their conceptual foundation~\cite{aop-book,Carcassi:2022bpm}. Their reverse physics approach has led for example to a geometric and physical interpretation of the principle of least action~\cite{aop-action} and to a demonstration that the uncertainty principle in quantum mechanics emerges solely from the lower limit on the entropy~\cite{Carcassi:2022bpm}. In other work, they have for example reduced the number of postulates of quantum mechanics, proving that the tensor product postulate for composite systems can be derived from the state and measurement postulates~\cite{Carcassi2021four}, and they have offered a clearer characterization of Shannon information entropy~\cite{Carcassi:2021}. 

Regarding success in mentoring and education, in 2019 Aidala received the U-M Imes and Moore Mentorship Award for exceptional contributions toward recruiting and mentoring graduate students in the sciences from disadvantaged and nontraditional backgrounds, and in 2025 the U-M Rackham Distinguished Graduate Mentor Award.  She has served as the dissertation chair for 10 graduate students in Physics and Applied Physics who have already completed their doctorates, as co-chair with Uribe for one in Applied and Indisciplinary Mathematics and another in Applied Physics, and is currently the dissertation chair for an additional 4 Ph.D.~students in Physics.  Aidala and Carcassi have also successfully published with former U-M Ph.D.~students in Mathematics~\cite{aop-spacetimeStruct} and Philosophy~\cite{aop-HamPriv} whose primary dissertation work was with other faculty members.  In 2020 Aidala received the U-M John Dewey Award for long-term commitment to the education of undergraduate students.  Between Aidala and Carcassi, they have mentored more than 50 undergraduate students in research since coming to U-M in 2012.  \\

TODO: Add info for Alejandro.

Regarding the qualifications of the research team to achieve the broader impacts beyond academia as described above, Carcassi has a YouTube channel (@gcarcassi) focused on physics and mathematics with more than 17,600 subscribers currently.  He posts a new video monthly, generally targeted at individuals with an advanced undergraduate background in physics and/or mathematics.  Collectively, his videos on this channel have received over 350k views [TODO: update]. He additionally has a second, more technical YouTube channel (@AssumptionsofPhysicsResearch) aimed at professional researchers, with more than 1600 subscribers.  He uses this channel as a venue to disseminate research results at a more technical level, including video versions of our preprints or published papers as well as conference presentations.  He moreover uses the technical channel as an innovative approach to try to ``crowd-source'' technical expertise to advance the research more efficiently.  Given the breadth of the broader AoP project, there is no single community with the appropriate technical background across all relevant and potentially relevant areas of mathematics and physics.  So far this ``crowd-sourcing'' approach has led to multiple instances of viewers with whom we have had no previous contact pointing us to valuable reference materials, and in 2024 it led to our first coauthored paper with a researcher who reached out to us after seeing an open problem video, on ``Classical mechanics as the high-entropy limit of quantum mechanics''~\cite{aop-classicallimit}. We were previously completely unaware of the researcher and his work in Innsbruck, Austria.  In other forms of reach beyond academia, Carcassi was interviewed in 2024 by Curt Jaimungal for the ``Theories of Everything'' podcast~\cite{Carcassi-ToEInterview}, which has 500k subscribers.  Carcassi additionally writes monthly blog-style essays related to the AoP project and advertised through social media platforms such as Facebook and LinkedIn.  In more creative outreach work, in April 2024 at U-M he performed an original musical comedy lecture, ``An Engineer in the Foundations of Physics,'' which drew approximately 100 attendees. 

Graduate student Tobias Thrien, for whom partial support is requested, originally discovered the Assumptions of Physics project through Carcassi's YouTube channel while a Bachelor's student in Germany several years ago.  He has since completed a Master's in theoretical cosmology in Germany and is currently a second-year Ph.D.~student at U-M.  He attended Aidala and Carcassi's inaugural online summer school on the AoP project in June 2024 and is already highly familiar with the broader project, its goals, and its approaches.  



\section{Results from prior NSF support}
\subsection{Intellectual merit of prior NSF support}

Aidala has been a successful steward of NSF funds since receiving a CAREER Award in 2015.  Her prior research awards have been through the Physics Division, Nuclear Physics - Experiment program: \emph{CAREER: Valence and sea quark dynamics at Fermilab}, 2015-21, and \emph{Studying quantum chromodynamics at LHCb}, 2020-present (renewed in 2023).  She is additionally a Co-PI on an NSF AccelNet award, \emph{AccelNet-Design: Inter-American Network of Networks on Quantum Chromodynamics Challenges}, 2021-present, aimed at further developing collaboration in quantum chromodynamics research between the U.S.~and Latin America.  Her QCD research has focused on advancing the multidimensional quark-gluon structure of the proton and on studying the processes by which scattered quarks or gluons form new bound states (hadrons).  Her group's NSF-funded work includes results published in Nature~\cite{SeaQuest:2021zxb}, Physical Review Letters~\cite{LHCb:2022tbc,LHCb:2019qoc}, Physical Review D Letters~\cite{LHCb:2022rky}, and Physical Review C~\cite{SeaQuest:2022xdu}, among other journals.  Regarding the specific scientific advancements of some of the work, Aidala and her collaborators have studied antimatter in the proton and found surprising evidence that there is an excess of down ``flavored'' antiquarks relative to up flavored antiquarks in the proton for antiquarks carrying as much as 45\% of the proton's total momentum~\cite{SeaQuest:2021zxb,SeaQuest:2022xdu}.  Her group has also led the first-ever measurements of identified hadron species production by high-energy light-mass quarks in proton-proton collisions, multidifferential in multiple kinematic variables simultaneously, showing that existing models poorly predict the relative production rates of some species~\cite{LHCb:2022rky}. [TODO: Add dead cone effect] This proposal is the second time she is submitting to the Applied Mathematics program, based on the research direction in the mathematical foundations of physics that she initiated more recently, which has so far produced 14 peer-reviewed papers since 2018 as well as a technical book, as described above in the section on qualifications of the PI and research team.  \\

TODO: Prior NSF support for Alejandro?



\subsection{Broader impacts of prior NSF support}
Aidala's group has been leading the open data efforts of the LHCb experiment at the Large Hadron Collider at CERN.  Of particular note, they have developed a user-friendly web-based application to securely access and provide analysis documentation for large-scale data sets in high-energy physics, the ``NTuple Wizard''~\cite{Aidala:2023dai}.  It is currently in beta-release to provide access to open data sets from LHCb, and individuals from other experimental collaborations have expressed interest in potentially adapting it to create a similar application to facilitate access to their own open data sets and corresponding documentation.

[TODO: Update and rewrite to avoid DEI language.] In addition, a variety of activities under Aidala's prior NSF grants have contributed directly to training junior scientists and to diversity in the STEM workforce.  A woman postdoctoral scholar was involved 2019-22. Including students with support from on and off of the grant, 8 Ph.D. students and 4 undergrads have been involved in the project since July 2020. An additional 3 distinct Ph.D. students and 7 undergraduates worked on the projects supported by Aidala's CAREER grant 2015-20.  Of the grad students, 2 have been African-American men, 2 Caucasian men, 2 Latino men, 1 an Arabic man, and 4 women, of which 2 Latina.  Former African-American grad student Bryan Ramson is now a permanent staff scientist at Fermilab; former Latina grad student Catherine Ayuso is now Head of Special Projects and Research Manager at Entanglement, Inc.  The others are in postdoctoral positions or still students in the group.  The undergrads included 1 Latino, 1 man who was openly autistic, and 5 women from Mt.~Holyoke College, among which were international students from the Philippines, Bangladesh, and China. Four of the grad/undergrad students openly identify as LGBT.  All Mt.~Holyoke students have continued in STEM.  One received an M.S. from Stanford in Statistics and now works for Amazon, the other four are now Ph.D. students at Harvard, Michigan, Virginia Tech, and Vanderbilt in Applied Physics, Physics, or Electrical Engineering.  \\

TODO: Prior NSF support for Alejandro?




\newpage






