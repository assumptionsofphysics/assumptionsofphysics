\begin{center}
	\textbf{A generalized axiomatic basis for entropic structures in physical theories} \\
	Christine A. Aidala, University of Michigan
\end{center}
%\vspace{-8mm}

\section{Introduction}

Physics is currently a patchwork of different theories with no common underlying foundation. As part of our larger project, Assumptions of Physics, we aim to develop an axiomatic basis for states and processes that can be equally applied to different theories, such as classical and quantum mechanics. Different theory are then recovered as specializations of the more general structure.

A core insight that is guiding this enterprise is that geometric structure in both classical and quantum mechanics are equivalent to entropic structures. In classical mechanics, a single degree of freedom is a two dimensional symplectic manifold. In this case, symplectic form $\omega$ is a volume form and therefore specifying $\omega$ is equivalent to specifying the Liouville measure $\mu(U) = \int_U \omega$. In classical statistical mechanics, the entropy of a statistical ensemble is given by $S(\rho) = - \int \rho \log \rho d\mu$ where $\rho$ is the probability density associated to the ensemble. For a uniform distribution with support $U$, the entropy reduce to $S(\rho_U) = \log \mu(U)$. Since the logarithm is an invertible function, one can recover the measure from the entropy. That is, specifying the entropy associated to all uniform distributions allows us to reconstruct the Liouville measure, and therefore the symplectic form. In the general case, Darboux coordinates allow, at least locally, to factorize the space into multiple independent degrees of freedom, where the volume is the product of the corresponding areas. The corresponding entropic property is that the entropy for the joint distribution is the sum of the marginals if they are independently distributed. Therefore, if the distribution of all possible marginals are specified, the symplectic form can be recovered.

In quantum mechanics, the geometry is given by the Born rule
\begin{equation}
	p(\phi|\psi) = \frac{\< \phi | \psi \>\< \psi | \phi \>}{\< \psi | \psi \>\< \phi | \phi \>}.
\end{equation}
Given two states $\psi$ and $\phi$, the ensemble $\rho_{\psi\phi}$ given by their equal mixture has entropy $S(\rho_{\psi\phi}) = I\left(\frac{1 + \sqrt{p(\phi,\psi)}}{2},\frac{1 - \sqrt{p(\phi,\psi)}}{2} \right)$ where $I(x,y) = -x \log x -y \log y$ is the Shannon entropy. The entropy as a function of $p(\psi,\phi)$ is invertible. Therefore, given the entropy of the equal mixtures of all pairs of states we can reconstruct the Born rule and, therefore, the state space.

While the states spaces of classical (i.e. symplectic manifolds) and quantum mechanics (i.e. projective Hilbert spaces) are rather different, the space of ensembles of both have a similar structure. They are both convex subsets of a topological vector space, equipped with a strictly convex entropy function. The core idea, then, is to use ensembles as a way to generalize the notion of physical states, and the entropy to characterize the geometry in a way that is independent of the particulars of each physical theory. In the past year we have done significant preliminary work that shows that not only this approach is possible, but is extremely productive. The proposed work is to turn those preliminary insight in an axiomatic specification that is fully physically motivated. Such an approach, if successful, would at least provide the following benefits:
\begin{enumerate}
	\item clarify exactly what physical premises guarantee the basic geometric structures of physical theories, allowing a more principled formulation of current physical theories and search for new ones;
	\item provide a generalized structure for information theory, possibly generalizing results from both classical and quantum information theory
	\item provide useful constrains to more precisely formulate physical theories in the infinite dimensional case.
\end{enumerate}

\section{Proposed research program}

In a nutshell, the goal is to find a very limited set of axioms that can be guaranteed to be valid in any physical theory, and that allow us to generalize a good number of results from all physical theories. The notion of statistical ensemble (i.e. the collection of all outputs of a preparation procedure) is particularly suitable for generalization for a number of both practical and conceptual considerations. Our preliminary work shows that there are three, possible four, distinct set of requirements for an ensemble space $\Ens$, the set of ensemble given by a physical theory.

The first requirement is that of experimental verifiability of the theory. This has been thoroughly investigated in our previous work that established a precise correspondence between experimental verifiable statements (i.e. statement associated with an experimental test that terminates successfully in finite time if and only if the statement is true) and the open set of a ``natural'' topology, and between theoretical statements (i.e. statement associated with an experimental test without termination guarantees) and Borel sets. Real and integer numbers, for example, are recovered from an idealized model representing references of a reference system. The key insight is that the set theoretic operations represent a two-valued logic system (i.e. truth or falsehood of statements) while the topology keeps track of a three-valued logic system (i.e. interior, exterior and boundary represent respectively successful termination, unsuccessful termination and non-termination not terminate). This work gives us a solid foundation on which to build, and guarantees that an ensemble space $\Ens$ is, at least, a $\mathsf{T}_0$ second countable topological space.

The second requirement is the ability to perform statistical mixtures. Given two ensembles $\ens[a], \ens[b] \in \Ens$, we can always choose a probability $p \in [0,1]$ and select $\ens[a]$ $p$ percent of the times and $\ens[b]$ $\bar{p} = 1-p$ percent of the times. This gives the ensemble space a convex structure as $p \ens[a] + \bar{p} \ens[b] \in \Ens$ for all $\ens[a], \ens[b] \in \Ens$ and $p \in [0,1]$. This is insight is already used in the literature, particularly in Generalized Probabilistic Theories (GPTs). The main innovation in this respect is that we do not limit ourselves to finite combinations and we do not simply allow all infinite combinations (which can be shown to physically untenable cases). Rather, we let the topology do the work. That is, infinite convex combinations exist only if they are the topological limit of finite convex combinations. On physical grounds, the mixing operation must be continuous in the natural topology defined by experimental verifiability, much like addition and scalar multiplication are continuous in a topological vector space, leading to the idea of a topological convex space. Unfortunately, we haven't found such a structure already studied in the literature. While it would be extremely useful for us to study these spaces on their own merit, we choose to concentrate on problems where our expertise in the mathematical foundations of physical theories is more critical, as in the next requirement.

The third requirement is characterizing the variability of elements within a statistical ensemble. Though the instance within ensembles are never formalized, the variability within an ensemble is physically significant as it will influence, for example, the variability of statistical measurements. As we showed in previous work, entropy quantifies the variability of the elements within a distribution, and which gives a physically meaningful starting point. The proposed work is to find the exact specification of the axiom to see whether it can also guarantee that ensemble spaces must be metric spaces.

The hardest conceptual problem is that entropy is defined differently in each space. In a classical discrete space, the entropy is $-\sum p_i \log p_i$; in classical statistical mechanics is $- \int \rho \log \rho d\mu$ where $\mu$ is the Liouville measure; in quantum mechanics is $-\tr(\rho \log \rho)$. How can we give a single definition for these very different expressions? Note that the expressions are different precisely because they are in terms of the specific representation of ensembles in specific theory. Our approach is to simply characterize the entropy by its properties with respect to the topology and the convex structure. In our preliminary properties, for example, the entropy $S : \Ens \to \mathbb{R}$ has the following properties:
\begin{itemize}
	\item Continuity
	\item Strict concavity: $S(p\ens[a] + \bar{p} \ens[b]) \geq p S(\ens[a]) + \bar{p} S(\ens[b])$ with the equality holding if and only if $\ens[a] = \ens[b]$
	\item Upper variability bound: there exists a universal function $I(p_1, p_2)$ (i.e. the same for all ensemble spaces) such that $S(p\ens[a] + \bar{p} \ens[b]) \leq I(p, \bar{p}) + p S(\ens[a]) + \bar{p} S(\ens[b])$; if the equality holds, $\ens[a]$ and $\ens[b]$ are orthogonal, noted $\ens[a] \ortho \ens[b]$
	\item Mixtures preserve orthogonality: $\ens[a] \ortho \ens[b]$ and $\ens[a] \ortho \ens[c]$ if and only if $\ens[a] \ortho p \ens[b] + \bar{p} \ens[c]$ for any $p \in (0,1)$
\end{itemize}
These properties are justified by the physics. For example, if we are performing a statistical mixture between two ensemble, the average variability cannot decrease, which justifies the concavity. These relatively simple properties allow us, for example, to show that $I(p,\bar{p})$, the maximum entropy increase during mixing, must be $ - \kappa \left(p \log p + \bar{p} \log \bar{p}\right)$, the Shannon entropy. The proof is general, meaning that if we are right that those axioms must hold for any physical theory, then the Shannon entropy is of truly general applicability. Moreover, in the case of both classical and quantum mechanics, states can be decomposed into orthogonal components, which means the upper variability is valid with an equal. That allows to recover the expression for the entropy in the three different cases.

Our preliminary work shows that an entropy satisfying those axioms forces the convex space to be cancellative, meaning that it embeds into a vector space. That is, if we pick an internal point $\ens \in \Ens$, we can define a vector space $V_{\ens}$ such that $\Ens \subset V_{\ens}$. If we pick to different internal points $\ens \in \Ens$, we can show that there is an invertible affine map between $V_{\ens[a]}$ is $V_{\ens[b]}$. That is, the ensemble space embeds in an affine space. We can also show that it is ``directionally bounded,'' meaning its intersection with any affine line is finite. It is an open question whether the axioms are enough to prove that the embedding is continuous in a topological vector space, and whether the set is bounded.

We can also show that the entropy can be used to define geometric structures directly in the ensemble space. For example, we can define the mixing entropy as $MS(\ens[a], \ens[b]) = S\left(\frac{1}{2}\ens[a] + \frac{1}{2} \ens[b]\right) - \left(\frac{1}{2} S(\ens[a]) + \frac{1}{2} S(\ens[b])\right)$. If we express the entropy in bits, the mixing entropy satisfies the following:
\begin{enumerate}
	\item non-negativity: $MS(\ens[a], \ens[b]) \geq 0$
	\item identity of indiscernibles: $MS(\ens[a], \ens[b]) = 0 \iff \ens[a]=\ens[b]$
	\item unit boundedness: $MS(\ens[a], \ens[b]) \leq 1$
	\item maximality of orthogonals: $MS(\ens[a], \ens[b]) = 1 \iff \ens[a] \ortho \ens[b]$
	\item symmetry: $MS(\ens[a], \ens[b]) = MS(\ens[b], \ens[a])$
\end{enumerate}
It is almost a distance function, as it does not satisfy the triangle inequality, but it recovers the Jensen-Shannon divergence (JSD) in both classical and quantum mechanics. The JSD is the square of a distance function in both classical and quantum information theory, though our current axioms do not guarantee $MS$ to be so. The $MS$ can be used to create open balls, and show that the topology must be Hausdorff. The current axioms do not guarantee the open balls to be convex.

If we assume we are in finite dimensions and that the entropy is twice differentiable with respect to the convex coefficients, the strict concavity of the entropy means that the Hessian is negative definite which allows us to define a metric tensor. Let $V_{\ens[a]}$ be the embedding vector space with $\ens[a] \in \Ens$ as origin. This will be a differentiable manifold. On a chart whose coordinates are linear with respect to the affine structure we can define $g_{\ens}( v, w ) = - \frac{\partial^2 S}{\partial \ens ^2}( v, w )$ where $v$ and $w$ are elements of the tangent space at $\ens \in \Ens \subset V_{\ens[a]}$. Since all Hessian are symmetric, $g_{\ens}$ will be bilinear, symmetric and positive definite: it is a metric tensor. In classical and quantum mechanics it recovers the Fisher information metric. Note that, if were to show continuous embedding into a locally convex topological vector space, this construction can be extended to the infinite dimensional case.

understand whether the existence of a physically meaningful entropy function is enough to guarantee that the space of statistical ensembles of any physical theory is a convex subset of a locally convex topological vector space equipped with a metric. That is, we should not simply ``declare'' it to be so, but make sure that the mathematical starting points are justified by physical arguments, so that we can 


Notes:

Generalized ensemble spaces + geometry defined from entropy; full set of axioms

-Develop axiomatics for the entropy such that they’re physically motivated.  Not taking starting points because they’re convenient mathematically, nor because we think one physical theory (e.g. quantum mechanics) is the foundational one.

-See if this type of framework will help constrain the problem in infinite dimensions (motivation, not goal).  Need to generalize the Poisson bracket structure in infinite dimensions—relationship between entropy and Poisson brackets used as Lie brackets.  Well-known open research problem that there’s no general theory for infinite-dimensional analysis/infinite-dimensional Lie groups.

Motivation (but not goal): Eventual framework to generalize classical and quantum information theory

1. To answer fundamental questions.

2. To do classical and quantum info in a more generalized setting

3. Give a structure for infinite-dimensional analysis that’s constrained enough that mathematicians can prove theorems that are useful for physics

Have: 1. Topological structure; 2. Convex structure; 3. Geometric structure.  

Open problems: 1. Entropic open balls can generate topology? 2. Are all axioms independent?   …

To determine which fundamental structures/properties are needed for all physical theories.

Open with: One of the difficulties with mathematical physics is that physicists aren’t good at specifying the mathematical requirements such that mathematicians can work effectively on the problems.

Mention economists evidently aware of topological issues more than physicists.



\section{Proposed research program}

\subsection{Intellectual merit}



\section{Summary of proposed research program}


\noindent
\textbf{Assessment of intellectual merit:} \\

\noindent
\textbf{Relationship to other current funding:} The PI's current funding from the NSF Division of Physics is for work on LHCb experiment at CERN, and from the NSF AccelNet program is to increase collaboration on quantum chromodynamics (QCD) research between the U.S.~and Latin America.  The PI's funding from the Department of Energy is for work at the Relativistic Heavy Ion Collider (RHIC) on the PHENIX and sPHENIX experiments.  These grants are for a completely separate line of research. Her grant from the John Templeton Foundation, titled ``Why classical probability and classical information theory are incompatible with quantum mechanics and quantum contextuality,'' supports the development of novel non-additive measures (set functions) suitable for quantifying the number of states in quantum mechanics~\cite{aop-nonaddmeas}.  There is no direct overlap with the activities proposed here.  \\

\noindent
\textbf{Relationship to other commitments of PI:} PI Aidala has been continuously involved in high-energy experimental QCD since 2001.  Her separate line of theoretical research on the mathematical foundations of physics is relatively new, with her first publication in 2018~\cite{aop-phys-blueprint}. She has successfully split her efforts between her different research lines for the past six years, with her work related to the Assumptions of Physics project producing 14 peer-reviewed publications, 1 more submitted paper, and a published technical book since 2018.  She is moreover in the process of reducing the number of personnel she supervises on QCD research in order to have more time for her theoretical work---four of her Ph.D.~students in QCD completed their degrees in the last 18 months, and two more are expected to graduate before the start of the proposed project.  




\section{Personnel}
The proposed research will be performed by PI Christine Aidala, her collaborator Gabriele Carcassi, and Ph.D.~student Tobias Thrien, all part of the Physics Department at the University of Michigan.  See also the section on ``Qualifications of the PI and Research Team'' below.  \\
TODO: Add Alejandro assuming he agrees.

Aidala as PI will be responsible for coordinating all of the work, and she will serve as the academic advisor for Thrien.  She will identify suitable journals for publication and conferences at which to disseminate the results, and oversee the direction of the work.  

Carcassi will serve as the technical lead.  He is (currently part-time) staff in the U-M Department of Physics, a Research Area Specialist Senior, with his salary fully supported through grants.  He will identify the most promising technical directions, perform a significant fraction of the calculations, serve as the lead writer for publications until Thrien is at a level to do so, and connect the results of the project to broader results so that the impact of the work is maximized.  With training in engineering and mathematics, he will additionally serve as the primary contact to interface with others in these disciplines who may be interested in the work.

Thrien entered the U-M Physics Ph.D.~program in Fall 2024.  While Carcassi will be the primary individual responsible for ``outward-facing'' technical work, Thrien will focus primarily on inward-facing technical work.  \\
TODO: Be more specific about Thrien's work, per reviews from last year.



%\begin{figure*}[htb]
%	\centering
%	\includegraphics[scale=.6]{figures/PersonnelTimelineNSF2022.png}
%	\caption{Estimated timeline of research activities.}
%	\label{fig:personnelTimeline}
%\end{figure*}



\section{Broader impacts}
In addition to disseminating the goals and results of the project through traditional means such as peer-reviewed publications and presentations at workshops and conferences, the goals and results of the project will be disseminated to wider audiences by leveraging existing social media and other outreach infrastructure that has already been developed for the broader Assumptions of Physics project, in particular, Carcassi's two YouTube channels, as described below in the section on qualifications of the research team.  One of them is aimed at the level of individuals with upper-level undergraduate training in physics and/or mathematics, and the second, more technical channel is aimed at professional researchers, for which he creates videos presenting detailed technical results or describing open technical questions, leveraging the expertise of others to help advance research goals.  Graduate student Tobias Thrien originally discovered the AoP project through Carcassi's YouTube channel while a Bachelor's student in Germany several years ago.  He has since completed a Master's in theoretical cosmology in Germany and is currently a first-year Ph.D.~student at U-M, having come to Michigan specifically to work on the AoP project with the PI and Carcassi.  Regarding the impact of posting open technical problems, our first collaborative paper that resulted from someone reaching out to us after seeing a video, on classical mechanics as the high-entropy limit of quantum mechanics, is currently under review at a journal~\cite{aop-classicallimit}.  We were previously completely unaware of the researcher and his work at the University of Innsbruck, Austria. Carcassi additionally writes monthly blog-style essays related to the AoP project and advertised through social media platforms such as Facebook and LinkedIn.  Thus, the results of the proposed research will reach a much wider audience than through the publications and conference talks alone.  Further dissemination of the open-research approach may inspire researchers in completely different areas to adopt a similar approach for their own projects, with potential acceleration of research progress in particular within interdisciplinary areas, where not all necessary expertise may be available within a single research community.  

TODO: Paragraph about impact of unifying disparate branches of physics, and eventual impact on physics education.\\

\noindent
\textbf{Assessment of broader impacts:}  The reach and impact of videos related to the proposed work will be assessed by the number of views they receive and by other measures of audience engagement such as comments posted or subsequent email contacts.  The impact of unifying disparate branches of physics will need to be assessed on a longer-term timescale than that of the grant, but early indicators could include talk invitations and paper citations. \\
TODO: Consider ways to assess engagement in more detail, per last year's reviews.


\section{Qualifications of the PI and research team}
PI Christine Aidala is a highly recognized leader within physics, having received an NSF CAREER Award, a Sloan Fellowship, and a Presidential Early Career Award for Scientists and Engineers. She was a 2019-20 U.S.~Fulbright Scholar and is a Fellow of the American Physical Society (APS).  She served on the National Academy of Sciences Committee on a U.S.-Based Electron-Ion Collider Science Assessment 2016-18 and is currently serving on the U.S.~Nuclear Science Advisory Committee.  She served as an elected member of the Executive Committee of the APS Division of Nuclear Physics (DNP) 2021-23, and since 2021 she has served as a DNP Ally to ensure that DNP meetings are welcoming and inclusive environments.  Much of her work has been in quantum chromodynamics; however, in 2015 she initiated the Assumptions of Physics project with her collaborator and husband Gabriele Carcassi, and since 2018 they have so far published together 14 peer-reviewed articles~\cite{aop-phys-blueprint,aop-topExpDisting,aop-HamQuantInfo,Carcassi:2021,aop-spacetimeStruct,aop-HamConsInfoEnt,Carcassi2021four,Carcassi:2022bpm,aop-HamPriv,aop-commonLogical,aop-action,aop-nogo,aop-nonaddmeas,aop-unphysHilbert} as well as a technical book on the mathematical foundations of physics~\cite{aop-book}, with one additional article currently under review and publicly available as a preprint~\cite{aop-classicallimit}.  She is the PI on a grant from the John Templeton Foundation that supports the development of novel measure theoretic structures for quantum mechanics. \\

TODO: Add info for Alejandro.

Carcassi is the technical lead for the Assumptions of Physics project.  He is the Co-PI on the Templeton Foundation grant with Aidala.  Related to the AoP project, he has given seminars to audiences of physicists, mathematicians, and philosophers at U-M, Harvard, Caltech, the University of Pavia, Italy, the Munich Center for Mathematical Philosophy, Germany, and the Wolfram Institute, and he and Aidala gave a joint public talk for the Michigan State University Advanced Studies Gateway.  His training in engineering allows him to bring ideas from fields not typically studied by individuals trained primarily in physics, for example from theoretical computer science, information theory, systems theory, signal processing, and operations research.  His position as U-M Physics Department staff without teaching responsibilities provides him with time and flexibility to dedicate to research related to the AoP project such as the work proposed here. \\
TODO: Add Kansas, Saturday Morning Physics, invited talk at AMS meeting, also earlier Topology conference talk?

Together, Aidala and Carcassi are uniquely positioned to advance the mathematical foundations of physics, with Aidala's traditional training in physics and traditional academic position complementing Carcassi's formal training in engineering and his university staff position, which for example gave him the time and flexibility to audit multiple graduate Mathematics courses at U-M.  For the AoP project, their published work so far has covered a wide range of topics.  The ``physical mathematics'' approach within their work is an approach to the mathematical foundations of physics that seeks to construct mathematical structures strictly from axioms and definitions that can be rigorously justified from physical requirements~\cite{aop-book}.  For example, they have shown that experimental distinguishability is mathematically captured by Hausdorff topologies~\cite{aop-book,aop-topExpDisting}, and they have argued that at the smallest space-time scales, at which geometric information becomes in principle experimentally inaccessible, there should still in principle be a regime in which topological structure remains experimentally accessible~\cite{aop-spacetimeStruct}.  Their ``reverse physics'' approach instead is an approach to the foundations of physics that analyzes known theories to identify those physical principles and assumptions that can be taken as their conceptual foundation~\cite{aop-book,Carcassi:2022bpm}. Their reverse physics approach has led for example to a geometric and physical interpretation of the principle of least action~\cite{aop-action} and to a demonstration that the uncertainty principle in quantum mechanics emerges solely from the lower limit on the entropy~\cite{Carcassi:2022bpm}. In other work, they have for example reduced the number of postulates of quantum mechanics, proving that the tensor product postulate for composite systems can be derived from the state and measurement postulates~\cite{Carcassi2021four}, and they have offered a clearer characterization of Shannon information entropy~\cite{Carcassi:2021}. 

Regarding success in mentoring and education, in 2019 Aidala received the U-M Imes and Moore Mentorship Award for exceptional contributions toward recruiting and mentoring graduate students in the sciences from disadvantaged and nontraditional backgrounds, and in 2025 the U-M Rackham Distinguished Graduate Mentor Award.  She has served as the dissertation chair for 10 graduate students in Physics and Applied Physics who have already completed their doctorates, as co-chair for one in Applied and Indisciplinary Mathematics and another in Applied Physics, and is currently the dissertation chair for an additional 4 Ph.D.~students in Physics.  Aidala and Carcassi have also successfully published with former U-M Ph.D.~students in Mathematics~\cite{aop-spacetimeStruct} and Philosophy~\cite{aop-HamPriv} whose primary dissertation work was with other faculty members.  In 2020 Aidala received the U-M John Dewey Award for long-term commitment to the education of undergraduate students.  Between Aidala and Carcassi, they have mentored more than 50 undergraduate students in research since coming to U-M in 2012.  \\

TODO: Add info on summer schools here? 

TODO: Add info for Alejandro.

Regarding the qualifications of the research team to achieve the broader impacts beyond academia as described above, Carcassi has a YouTube channel (@gcarcassi) focused on physics and mathematics with more than 17,500 subscribers currently.  He posts a new video monthly, generally targeted at individuals with an advanced undergraduate background in physics and/or mathematics.  Collectively, his videos on this channel have received over 350k views [TODO: update]. He additionally has a second, more technical YouTube channel (@AssumptionsofPhysicsResearch) aimed at professional researchers, with more than 1600 subscribers.  He uses this channel as a venue to disseminate research results at a more technical level, including video versions of our preprints or published papers as well as conference presentations.  He moreover uses the technical channel as an innovative approach to try to ``crowd-source'' technical expertise to advance the research more efficiently.  Given the breadth of the broader AoP project, there is no single community with the appropriate technical background across all relevant and potentially relevant areas of mathematics and physics.  So far this ``crowd-sourcing'' approach has led to multiple instances of viewers with whom we have had no previous contact pointing us to valuable reference materials, and it recently led to our first coauthored paper with a researcher who reached out to us after seeing an open problem video, on ``Classical mechanics as the high-entropy limit of quantum mechanics''~\cite{aop-classicallimit}. We were previously completely unaware of the researcher and his work in Innsbruck, Austria.  In other forms of reach beyond academia, Carcassi was interviewed in 2024 by Curt Jaimungal for the ``Theories of Everything'' podcast~\cite{Carcassi-ToEInterview}, which has 500k subscribers.  Carcassi additionally writes monthly blog-style essays related to the AoP project and advertised through social media platforms such as Facebook and LinkedIn.  In more creative outreach work, in April 2024 at U-M he performed an original musical comedy lecture, ``An Engineer in the Foundations of Physics,'' which drew approximately 100 attendees. 

Graduate student Tobias Thrien, for whom partial support is requested, originally discovered the Assumptions of Physics project through Carcassi's YouTube channel while a Bachelor's student in Germany several years ago.  He has since completed a Master's in theoretical cosmology in Germany and is currently a second-year Ph.D.~student at U-M.  He attended Aidala and Carcassi's inaugural online summer school on the AoP project in June 2024 and is already highly familiar with the broader project, its goals, and its approaches.  



\section{Results from prior NSF support}
\subsection{Intellectual merit of prior NSF support}

Aidala has been a successful steward of NSF funds since receiving a CAREER Award in 2015.  Her prior research awards have been through the Physics Division, Nuclear Physics - Experiment program: \emph{CAREER: Valence and sea quark dynamics at Fermilab}, 2015-21, and \emph{Studying quantum chromodynamics at LHCb}, 2020-present (renewed in 2023).  She is additionally a Co-PI on an NSF AccelNet award, \emph{AccelNet-Design: Inter-American Network of Networks on Quantum Chromodynamics Challenges}, 2021-present, aimed at further developing collaboration in quantum chromodynamics research between the U.S.~and Latin America.  Her QCD research has focused on advancing the multidimensional quark-gluon structure of the proton and on studying the processes by which scattered quarks or gluons form new bound states (hadrons).  Her group's NSF-funded work includes results published in Nature~\cite{SeaQuest:2021zxb}, Physical Review Letters~\cite{LHCb:2022tbc,LHCb:2019qoc}, Physical Review D Letters~\cite{LHCb:2022rky}, and Physical Review C~\cite{SeaQuest:2022xdu}, among other journals.  Regarding the specific scientific advancements of some of the work, Aidala and her collaborators have studied antimatter in the proton and found surprising evidence that there is an excess of down ``flavored'' antiquarks relative to up flavored antiquarks in the proton for antiquarks carrying as much as 45\% of the proton's total momentum~\cite{SeaQuest:2021zxb,SeaQuest:2022xdu}.  Her group has also led the first-ever measurements of identified hadron species production by high-energy light-mass quarks in proton-proton collisions, multidifferential in multiple kinematic variables simultaneously, showing that existing models poorly predict the relative production rates of some species~\cite{LHCb:2022rky}. [TODO: Add dead cone effect] This proposal is the second time she is submitting to the Applied Mathematics program, based on the research direction in the mathematical foundations of physics that she initiated more recently, which has so far produced 14 peer-reviewed papers since 2018 as well as a technical book, as described above in the section on qualifications of the PI and research team.  \\

TODO: Prior NSF support for Alejandro?



\subsection{Broader impacts of prior NSF support}
Aidala's group has been leading the open data efforts of the LHCb experiment at the Large Hadron Collider at CERN.  Of particular note, they have developed a user-friendly web-based application to securely access and provide analysis documentation for large-scale data sets in high-energy physics, the ``NTuple Wizard''~\cite{Aidala:2023dai}.  It is currently in beta-release to provide access to open data sets from LHCb, and individuals from other experimental collaborations have expressed interest in potentially adapting it to create a similar application to facilitate access to their own open data sets and corresponding documentation.

[TODO: Update and rewrite to avoid DEI language.] In addition, a variety of activities under Aidala's prior NSF grants have contributed directly to training junior scientists and to diversity in the STEM workforce.  A woman postdoctoral scholar was involved 2019-22. Including students with support from on and off of the grant, 8 Ph.D. students and 4 undergrads have been involved in the project since July 2020. An additional 3 distinct Ph.D. students and 7 undergraduates worked on the projects supported by Aidala's CAREER grant 2015-20.  Of the grad students, 2 have been African-American men, 2 Caucasian men, 2 Latino men, 1 an Arabic man, and 4 women, of which 2 Latina.  Former African-American grad student Bryan Ramson is now a permanent staff scientist at Fermilab; former Latina grad student Catherine Ayuso is now Head of Special Projects and Research Manager at Entanglement, Inc.  The others are in postdoctoral positions or still students in the group.  The undergrads included 1 Latino, 1 man who was openly autistic, and 5 women from Mt.~Holyoke College, among which were international students from the Philippines, Bangladesh, and China. Four of the grad/undergrad students openly identify as LGBT.  All Mt.~Holyoke students have continued in STEM.  One received an M.S. from Stanford in Statistics and now works for Amazon, the other four are now Ph.D. students at Harvard, Michigan, Virginia Tech, and Vanderbilt in Applied Physics, Physics, or Electrical Engineering.  \\

TODO: Prior NSF support for Alejandro?




\newpage






