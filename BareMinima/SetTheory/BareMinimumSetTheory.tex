\documentclass{article}

\usepackage{assumptionsofphysics}
\usepackage{hyperref}
\hypersetup{
	colorlinks=true,
	citecolor=blue,
	urlcolor=blue,
	linkcolor=blue
}
\frenchspacing

\title{Bare minimum: set theory}

\date{\vspace{-5ex}}
\begin{document}

\maketitle


\begin{abstract}
This note presents a condensed summary of set theory.

Bare minima are not meant to be either didactic or complete. They are meant to condense areas of knowledge in a few pages and function as a knowledge check, refresher and/or reference. They collect what we feel is the bare minimum of a subject an intellectually curious mind should be aware of, particularly in the context of foundational questions. This work is part of Assumptions of Physics (\url{https://assumptionsofphysics.org}), a project that aims to identify a handful of physical principles from which the basic laws can be rigorously derived.
\end{abstract}

\section{Introduction}

Set theory is important for those working on the Assumptions of Physics for at least three reasons. First, it is a foundational framework in mathematics. Second, it showcases a successful attempt at concept generalization. Third, it shows how a formal system is bootstrapped.

There are two rough branches of set theory: naive and axiomatic. Naive set theory is built on intuitive concepts, and as such is not fully formalized and is open to potential paradoxes. Axiomatic set theory provides a fully formalized axiomatic system that aims to close those problems. We will start with naive set theory, which is the best setting for defining all the basic notions and getting a sense of how the framework works. Then we will turn our attention to axiomatic set theory to get a sense of what problems it solves and how, and whether those solutions are aligned with the aims of the foundations of physics.

TODO: cite https://www.math24.net/topics-set-theory and set theory book

\section{Naive set theory}

We take sets and elements as primitive objects, with no formal definition. A \textbf{set} $A$ is a collection of elements. If an element $e$ belongs to the set we say $e$ \textbf{is in} $A$, noted $ e 
\in A$. A set $A$ can be defined by listing the elements, noted $A = \{ e_1, e_2, e_3 \}$. A set $A$ can be defined by specifying how to build the elements through a rule (i.e. predicate) $P(e)$, and is noted $A = \{e \, | \,  P(e) \}$. The rule may include symbols like for all $\forall$, exists $\exists$, logical connectors AND $\AND$, OR $\OR$ and $\NOT$.

\begin{defn}[Common sets] The \textbf{empty set} $\emptyset$ is the set with no elements. The \textbf{set of natural numbers} $\mathbb{N} = \{ 0, 1, 2, ... \}$. The \textbf{set of positive natural numbers} $\mathbb{N}^+ = \{n \in \mathbb{N} \, | \, n > 0\}$. The \textbf{set of integer numbers} $\mathbb{Z} = \{ ... , -2, -1, 0, +1, +2, ...\}$. The \textbf{set of rational numbers} $\mathbb{Q}$. The \textbf{set of real numbers} $\mathbb{R}$. The \textbf{set of complex numbers} $\mathbb{C}$.
\end{defn}

\subsection{Basic definitions}

\begin{defn}[Set relationships]
	Let $A$ and $B$ be two sets. If all elements in $A$ are also contained in $B$, we say $A$ is a \textbf{subset} of $B$, noted $A \subseteq B$. In this case, we also say $B$ is a \textbf{superset} of $A$, noted $B \supseteq A$. Additionally, if $B$ contains elements that $A$ does not contain, we say $A$ is a \textbf{strict subset} of $B$, noted $A \subset B$, and $B$ is a \textbf{strict superset} of $A$, noted $B \supset A$. If the two sets contain exactly the same elements, then we say the sets are \textbf{equal}, noted $A = B$. If the two sets have no element in common they are said to be \textbf{disjoint}.
\end{defn}

\begin{defn}[Power set]
	Given a set $A$, its \textbf{power set}, noted $\mathcal{P}(A)$ or $2^{A}$, is the set of all subsets of $A$. That is, $\mathcal{P}(A) = \{B \, | \, B \subset A \}$.
\end{defn}

\begin{defn}[Set operations]
	We define the following operations between two sets $A$ and $B$:
	\begin{description}
		\item[Union.] Noted $A \cup B$, the union of $A$ and $B$ is the set of all elements that are contained by $A$, $B$ or both.
		\item[Intersection.] Noted $A \cap B$, the intersection of $A$ and $B$ is the set of all elements that are contained by both.
		\item[Set difference (or relative complement).] Noted $A \setminus B$, $A$ minus $B$ is the set of all elements in $A$ that are not contained by $B$.
		\item[Complement (or absolute complement).] Noted $A^{\complement}$, represents all elements that are not contained in $A$. Note that his operation is context specific: we need to know that, within a certain context, $U$ is the set of all elements under study; then the complement of $A \subseteq U$ is $A^{\complement} = U \setminus A$.
	\end{description}
\end{defn}

We give the following informal definition. Given two elements $a$ and $b$, the \textbf{ordered pair} $(a, b)$ specifies two objects in that order. That is $(a, b) \neq (b, a)$ and $(a, a) \neq \{a\}$.

\begin{defn}[Cartesian product]
	Given two sets $A$ and $B$, the \textbf{Cartesian product} $A \times B = \{ (a,b) \, | \, \forall a \in A, b \in B \}$ is the set of all possible ordered pairs between the elements of $A$ and $B$.
\end{defn}

\subsection{Relations and function}

\begin{defn}[Binary relation]
	Given a set $A$, called \textbf{domain}, and a set $B$, called \textbf{codomain}, a \textbf{binary relation} is a set $R \subseteq A \times B$ of ordered pairs. We say $a$ is \textbf{$R$-related} to $b$, noted $aRb$ if $(a,b) \in R$. If $A=B$ the relation is said \textbf{homogeneous}, \textbf{heterogeneous} if not.
\end{defn}
	
\begin{defn}[Image and preimage]
	If $X \subseteq A$, the \textbf{image of $X$ under $R$} is the set of all elements in $B$ that are related to at least one element in $X$. The \textbf{image of $R$} is set of all elements in $B$ that are related to at least one element in $A$ (i.e. the image of the full set $A$).
	
	If $Y \subseteq B$, the \textbf{preimage of $Y$ under $R$} is the set of all elements in $A$ that are related to at least one element in $Y$. The \textbf{preimage of $R$} is set of all elements in $A$ that are related to at least one element in $B$ (i.e. the preimage of the full set $B$).
\end{defn}

\begin{defn}[Relation properties]
	Let $R$ be a binary relation between $A$ and $B$. We define the following properties:
	\begin{description}
		\item[Functional or right-unique.] An element $a \in A$ is related to at most one element $b \in B$. That is, if $aRb$ and $cRb$ then $a = c$.
		\item[Injective or left-unique.] An element $b \in B$ is related to at most one element $a \in A$. That is, if $aRb$ and $aRc$ then $b = c$.
		\item[Serial or left-total.] Every element $a \in A$ is related to at least one element $b \in B$. That is, for each $a \in A$ there exists at least one $b \in B$ such that $aRb$. In this case, the preimage of $R$ is the whole $A$.
		\item[Surjective or right-total.] Every element $b \in B$ is related to at least one element $a \in B$. That is, for each $b \in B$ there exists at least one $a \in A$ such that $aRb$. In this case, the image of $R$ is the whole $B$.
	\end{description}
	From those basic properties, we define the following:
	\begin{description}
		\item[One-to-one.] Injective and functional (i.e. left-unique and right-unique).
		\item[One-to-many.] Injective and not functional (i.e. left-unique and not right-unique).
		\item[Many-to-one.] Not injective and functional (i.e. not left-unique and right-unique).
		\item[Many-to-many.] Not injective nor functional (i.e. not left-unique nor right-unique).
	\end{description}	
\end{defn}

\begin{defn}[Function]
	A \textbf{partial function} $f : A \to B$ is a binary relationship between $A$ and $B$ that is functional (i.e. right-unique). The \textbf{graph} of the function $G \subset A \times B$ is the function expressed as ordered pairs. A \textbf{total function}, or simply a \textbf{function}, is a partial function that is also serial (i.e. left-total, defined on the whole domain).
\end{defn}

\begin{defn}[Composition]
	Let $A$, $B$ and $C$ be three sets. Let $R$ be a binary relationship between $A$ and $B$ and $S$ be a binary relationship between $B$ and $C$. Then the \textbf{composition of $R$ and $S$}, noted $S \circ R$, is the binary relationship between $A$ and $C$ such that $(a, c) \in S \circ R$ if we can find $b \in B$ such that $aRb$ and $bSc$.
\end{defn}

\begin{remark}
	The order of composition follows those of functions: if $f : A \to B$ and $g : B \to C$, composition leads to $c = g(f(a)) = (g \circ f)(a)$.
\end{remark}

\begin{prop}[Composition properties]
	Relation composition follows the following properties:
	\begin{itemize}
		\item Composition is associative: $R \circ (T \circ S) = (R \circ T) \circ S$
		\item Composition of functional/injective/serial/surjective relationship is functional/injective/serial/surjective.
		\item Specifically, if $f : A \to B$ and $g : B \to C$ are (partial) functions, then $g \circ f$ is a (partial) function.
	\end{itemize}
\end{prop}

\begin{defn}[Families]

\end{defn}

\begin{itemize}
	\item families (and operations on families)
	\item properties (boolean algebra, lattice)
	\item Ordered pairs and Cartesian product
	\item distribution rules of cartesian product
	\item power set
	\item cardinality
	\item Notable sets (empty, universal, natural, integers, rationals, reals, complex)
\end{itemize}

Functions
\begin{itemize}
	\item ...
\end{itemize}

\section{Axiomatic set theory}

Classes, elements, sets and proper classes.

%\bibliographystyle{plain}
%\bibliography{bibliography}

\end{document}