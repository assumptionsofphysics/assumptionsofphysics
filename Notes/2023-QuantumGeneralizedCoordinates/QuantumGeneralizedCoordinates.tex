\documentclass[11pt]{article}

\usepackage[margin=1.25in]{geometry}

\usepackage{amsmath}
\usepackage{amssymb}
\usepackage{graphicx}
\usepackage{amsfonts}
\usepackage{enumitem}


\def\>{\rangle}
\def\<{\langle}


\begin{document}

\title{Notes on QM in generalized coordinates}
\author{Gabriele Carcassi}

\date{\today}

\maketitle

\begin{abstract}
\end{abstract}

\section{Change of coordinates in manifolds}
Transformation rules for differentials on a generic manifold. Only differentiable structure is required.
\begin{equation}
	\begin{aligned}
		\hat{x}^i &= \hat{x}^i(x^j) \\
		d\hat{x}^i &= \partial_j \hat{x}^i dx^j \\
		d\hat{x}^n &= | \partial_j \hat{x}^i | dx^n = |J| dx^n \\
	\end{aligned}
\end{equation}

Transformation rules for a density $\rho(x^i)$. The requirement is that the integral is invariant.
\begin{equation}
	\begin{aligned}
		\int \rho(x^j) dx^n &= \int | \partial_j \hat{x}^i | \hat{\rho}(\hat{x}^i) dx^n = \int \hat{\rho}(\hat{x}^i) d\hat{x}^n \\
		|J| \hat{\rho}(\hat{x}^i) &= | \partial_j \hat{x}^i | \hat{\rho}(\hat{x}^i)  = \rho(x^j) \\
	\end{aligned}
\end{equation}

Mathematically, $\rho$ can be understood as the (only) component of a volume form. In that case, $\rho$ changes sign under change of orientation. Alternatively, $\rho$ can be understood as the (only) component of a density.\footnote{Effectively, the density is like a volume form that keeps track of orientation.} In this case, $\rho$ does not change sign under change of orientation and the transformation rule should use the absolute value of the Jacobian determinant.

\section{Change of coordinates for wave-function}

Deducing transformation rules for wave-function based on the fact that the modulues of a wave-function is a density over the manifold.
\begin{equation}
	\begin{aligned}
		\rho(x^j) &= \psi^\dagger(x^j) \psi(x^j)\\
		||J|| \hat{\rho}(\hat{x}^i) &= || \partial_j \hat{x}^i || \hat{\psi}^\dagger(\hat{x}^i) \hat{\psi}(\hat{x}^i) = \psi^\dagger(x^j) \psi(x^j) = \rho(x^j) \\
		\sqrt{|| \partial_j \hat{x}^i ||} \hat{\psi}(\hat{x}^i) &= \psi(x^j) \\
		\hat{\psi}(\hat{x}^i) &= \sqrt{|| \partial_i x^j ||} \psi(x^j) \\
	\end{aligned}
\end{equation}
Note that $\rho$ can't be negative. Mathematically,  $\rho$ should be the component of a density. Therefore $||J||$ represents the absolute value of the Jacobian determinant.

\section{Change of coordinates for momentum}

These are the change of coordinates for momentum. They are derived by making an arbitrary change on the position coordinates and inducing a coordinate change on momentum such that the symplectic form $\omega = dx^i \wedge dp_i$ is invariant. Therefore this depends on the symplectic structure. This makes conjugate momentum vary as a covector.

\begin{equation}
	\begin{aligned}
		\hat{p}_i &= \partial_i x^j p_j \\
		d \hat{p}_i &= \partial_i x^j d p_j \\
		d\hat{p}^n &= | \partial_i x^i | dp^n = |J^{-1}| dx^n \\
	\end{aligned}
\end{equation}

Since in quantum mechanics we do not have a symplectic structure, we can find the same transformation rule based on the fact that $P_i = -\imath \hbar \partial_i$
\begin{equation}
	\begin{aligned}
		\hat{P}_i &= -\imath \hbar \partial_i = -\imath \hbar \partial_i x^j \partial_j = \partial_i x^j P_j \\
	\end{aligned}
\end{equation}

\section{Properties of Fourier transform}

Definition of Fourier transform
\begin{equation}
	\begin{aligned}
		\phi(p_j) &= F(\psi(x^i)) = \frac{1}{\sqrt{2 \pi \hbar}} \int e^{\frac{p_j x^j}{\imath \hbar}} \psi(x^j) dx^n \\
	\end{aligned}
\end{equation}

Definition of convolution
\begin{equation}
	\begin{aligned}
		(g * h)(t) &= \int_{0}^{t} g(\tau) h(t-\tau) d\tau\\
	\end{aligned}
\end{equation}

Convolution theorems
\begin{equation}
	\begin{aligned}
		F((g * h)(t)) &= F(g(t)) F(h(t))\\
		F((g h)(t)) &= F(g(t)) * F(h(t))\\
	\end{aligned}
\end{equation}
TODO: find explicit confirmation for complex functions

\section{Change of Fourier transform}
\begin{equation}
	\begin{aligned}
		F(\hat{\psi}(\hat{x}^i)) &= F\left(\sqrt{|| \partial_i x^j ||} \psi(x^j)\right) = F\left(\sqrt{|| \partial_i x^j ||}\right) * F \left( \psi(x^j)\right)  \\
	\end{aligned}
\end{equation}
This expression depends only on the definition and properties of the Fourier transform and the transformation rules of densities of manifolds.

TODO: Need a cross check with the transformation rule of momentum. For example, show that the inner product in momentum basis is invariant:
\begin{equation}
	\begin{aligned}
		\int \phi^\dagger(p_j) \phi(p_j) dp^n &= \int \hat{\phi}^\dagger(\hat{p}_i) \phi(\hat{p}_i) d\hat{p}^n  \\
	\end{aligned}
\end{equation}
Another idea, show that applying the momentum operator in the new coordinates is equal to applying the the one in the old coordinates and then rescaling by the square root of the Jacobian.

\end{document}
