\documentclass[11pt]{article}

\usepackage[margin=1.25in]{geometry}
\usepackage{assumptionsofphysics}
\usepackage{hyperref}

\def\>{\rangle}
\def\<{\langle}

\DeclareMathOperator{\mix}{mix}
\DeclareMathOperator{\component}{comp}
\DeclareMathOperator{\cospan}{cospan}

\begin{document}

\title{Generalized Ensemble Spaces}

\maketitle

\begin{abstract}
The goal is to develop a generalized mathematical framework of statistical ensemble that can accommodate both the classical and quantum case, as well theories that are yet to be developed. The idea is that statistical mixtures impose a convex structure on the set of ensembles which define its geometry. This structure is tightly connected to physical entropy. All definitions must be justified on physical ground. 
\end{abstract}

\section{Convex spaces and ensembles}

\begin{remark}
	Basic definitions are taken from \url{https://ncatlab.org/nlab/show/convex+space} and \url{https://arxiv.org/abs/0903.5522}
\end{remark}

\begin{defn}
	For a real number $\lambda \in [0,1]$, $\bar{\lambda} = 1-\lambda$.
\end{defn}

\begin{defn}
	A \textbf{convex space} is a set $X$ equipped with a family of operations $c_\lambda : X \times X \to X$ where $\lambda \in [0,1]$ such that:
	\begin{itemize}
		\item $c_0(x,y)=x$
		\item $c_\lambda(x,x)=x$ for all $\lambda \in [0,1]$
		\item $c_\lambda(x,y)=c_{\bar{\lambda}}(y,x)$ for all $\lambda \in [0,1]$
		\item $c_\lambda(c_\mu(x,y), z)=c_{\lambda\mu}(x, c_{\frac{\lambda\bar{\mu}}{\overline{\lambda\mu}}}(y,z))$ for all $\lambda,\mu \in (0,1).$
	\end{itemize}
\end{defn}

\begin{remark}
	The above definition is such that, given a finite set of elements $\{x_i\}$ and a finite set of weights $\{p_i\} \in [0,1]$ such that $\sum p_i = 1$, we can define
	$$\sum p_i x_i = c_{\lambda_1}(x_1, c_{\lambda_2}(x_2, c_{\lambda_3}(x_3, \dots))$$
	and find that the sum is commutative. Such sum is called \textbf{convex combination}. The definition avoids having to define a map for each arity. However, the physical justification is more easily given in terms of the sum.
	
	TODO: Work out the relationship between $p_i$ and $\lambda_i$. It seems to proceed like:
	\begin{equation}
		\begin{aligned}
			\lambda_1 &= \overline{p_1} \\
			\lambda_2 &= \overline{\left(\frac{p_2}{\overline{p_1}}\right)} \\
			\lambda_2 &= \overline{\left(\frac{p_3}{\overline{p_1+p_2}}\right)}
		\end{aligned}
	\end{equation}
\end{remark}

\begin{defn}
	Let $X$ be a convex space. Let $y = \sum_i p_i x_i$ where $y, \{x_i\} \in X$ and $\lambda_i \in [0,1]$ such that $\sum \lambda_i = 1$. We say that $y$ is a \textbf{mixture} of $\{x_i\}$ and each $x_i$ is a \textbf{component} of $y$.
\end{defn}


\begin{prop}
	A convex space embeds into a real vector space with $c_\lambda(x,y) = \lambda x + \bar{\lambda}y$ if and only if
	$$ c_\lambda(x,y) = c_\lambda(x,z) \; \forall \lambda \in (0,1) \implies y = z. $$
\end{prop}
\begin{proof}
	See Theorem 4 in \url{https://arxiv.org/abs/1105.1270}.
\end{proof}

\begin{defn}
	Given a physical system, its \textbf{ensemble space} is the set of all the ensembles that characterize its state. 
\end{defn}

\begin{prop}
	Given a physical system, its \textbf{ensemble space} is a convex set. 
\end{prop}
\begin{justification}
	Given finite number of ensembles and a set of weights $\{p_i\} \in [0,1]$ such that $\sum p_i = 1$, we can in principle prepare a new ensemble that mixes the previous in the given proportions. The mixture $1x+0y$ will always give $x$, no matter the $y$. The mixture of any $x$ with itself will always return itself, no matter the weight. Mixing is commutative, so a permutation of the components corresponds to a permutation of the weights.
\end{justification}

\begin{remark}
	TODO: It is not yet clear an ensemble space must embed into a real vector space. Find a justification that proves or disproves.
\end{remark}

\begin{defn}
	Given a convex space $X$, an extreme point is an element $x \in X$ that cannot be written as convex combination of other elements of $X$.
\end{defn}

\begin{defn}
	Physically, the extreme points correspond to the \textbf{pure states}. The existence of extreme points is a major difference between convex spaces and vector spaces. In vector spaces, any element can be chose as basis for linear combination. In convex spaces, this is not the case.
\end{defn}

\subsection{Physical examples}

These are the examples that we need to keep in mind, which will need to be fully characterized in terms of convex spaces and their properties.

\subsubsection{Classical case - discrete}

The discrete classical case corresponds to a symplex (\url{https://en.wikipedia.org/wiki/Simplex}). In the finite case, the pure states $\mathcal{G} = \{g_i\}_{i=1}^{n} \subset \mathcal{E}$ are finitely many and each ensemble $e = \sum p_i g_i$ is uniquely identified by a decomposition of pure states. Effectively, each ensemble is a probability distribution over the pure states. Mathematically, each point of the space is a convex combination of the vertices. The symplex has a center point, which corresponds to the maximally mixed state, a uniform distribution over all pure states.

The countably infinite case is a bit different because not all infinite sequences $\{p_i\} \in [0,1]$ converge, and therefore cannot be normalized. As we cannot create a uniform distribution over infinitely many cases, there is no center point. Effectively, there is ``hole'' in the middle. More precisely, the space is not closed in the sense it does not contain all the limit points.

TODO: understand whether the maximally mixed state is the only limit point missing.

The uncountably infinite case is not physically relevant (i.e. no second countable discrete topology, cases are not experimentally decidable).

\subsubsection{Classical case - continuous}

In the continuous case, the space of ensembles is the space of integrable functions over a symplectic manifold (e.g.  over phase space) that integrate to one. That is, if $X$ is a symplectic manifold, then $\mathcal{E} = L^1(X, \mu)$ where $\mu=\int \omega^n$ is the Liouville measure. The pure states are the space of delta distributions $\mathcal{G} = \{\delta_x\}_{x \in X}$, but they are not in  $\mathcal{E}$: they are limit points. If we imagine a uniform distribution over a region of the symplectic manifold, we now have two limits that are not part of the ensemble space: the limit with infinite support, and the limit of zero support. In some sense, a distribution can be understood as a convex integral of the zero support limits: $\rho(x) = \int_X \rho(y) \delta_x(y) d\mu$. Even in this case, each ensemble can be understood as a probability distribution (density in this case) over all pure states.

TODO: The existence of a Liouville measure is connected to the ability to define probability densities that depend on the point, and not on the coordinate. The convex space defined finite sum. There should be an argument that shows how the existence of the Liouville measure is connected to the existence of probability density and extreme points.

TODO: The existence of the symplectic form is connected to the independence of degrees of freedom. In this context, there should be a notion of product of convex spaces, and the symplectic form should be connected to that. Find that connection.


\subsubsection{Quantum case - finite dimensional}

For a qubit, the Bloch ball is the space of ensembles $\mathcal{E}$. That is, let $\mathcal{H}$ be a two dimensional Hilbert space. The interior of the block sphere correspond to mixtures (i.e. the space of positive semi-definite Hermitian operators with trace one $M(\mathcal{H})$) while the surface corresponds to the pure states $\mathcal{G}$ (i.e. $\{ |\psi\> \<\psi| \}_{\psi \in \mathcal{H}}$). In this case, there is no single decomposition. The general finite dimensional case works similarly.

A crucial difference is that each mixed state has multiple decompositions in terms of pure states. Defining the multiple decomposition effectively defines the state space. These multiple decompositions makes the ensemble space behave in a way that is a hybrid of the classical discrete and continuous. Pure states are properly a part of the state, as in the discrete case, and we can describe each mixture in terms of finitely many states. However, the pure state are a continuum, therefore we can also define probability densities over the space, convex integrals. For example, for a single qubit, the maximally mixed state (the center of the sphere) can be equally described as the equal mixture of two opposite states (e.g. spin up and spin down, or spin left and spin right). However, it can also be described as the equal mixture of the whole sphere.

TODO: Note that complex projective are symplectic. Again, there should be an argument that the ability to define convex integrals in a coordinate invariant way requires the space to be symplectic. This would radically constrain the possible convex spaces.

\subsubsection{Quantum case - infinite dimensional}

The infinite dimensional can be both understood as the countably infinite dimensional version of the discrete case, or as the space of complex square integrable functions. As in the  (i.e. $L^2$) should work the same, except that there is no upper bound on the entropy, and therefore there is no maximally mixed state.

As before, the maximally mixed state is not the convex space.

TODO: understand the difference between Hilbert space and Schwartz space in terms of both extreme points and convex combinations/integrals.

\section{Entropy}

\begin{defn}
	Given a set of weights $\{p_i\} \in [0,1]$ such that $\sum p_i = 1$, the \textbf{entropy of the weights} is defined as $I(\{p_i\}) = - \sum p_i \log p_1 $.
\end{defn}

\begin{defn}
	Given a convex set $X$, we call \textbf{entropy} a function $S : X \to \mathbb{R}$ such that
	\begin{itemize}
		\item the entropy is strictly concave: $S(p_1\rho_1 + p_2 \rho_2) \geq p_1 S(\rho_1) + p_2 S(\rho_2)$ with the equality holding if and only if $\rho_1 = \rho_2$
		\item the entropy of a mixture is bounded by the entropy of the weights: $S(p_1\rho_1 + p_2 \rho_2) \leq I(p_1, p_2) + p_1 S(\rho_1) + p_2 S(\rho_2)$
	\end{itemize}
\end{defn}

\begin{remark}
	Note that the bounds for entropy can also be written as
	$$ 0 \leq S(p_1\rho_1 + p_2 \rho_2) - (p_1 S(\rho_1) + p_2 S(\rho_2)) \leq I(p_1, p_2).$$
	The term in the middle compares the entropy of the mixture as the average entropy of the components. It essentially tells us that the entropy can only increase during the mixture, and it can only increase up to the entropy.
	
	TODO: This makes the above conditions necessary; it is not clear whether they are sufficient.
	
	Note that if an ensemble can be decomposed in multiple ways, the decomposition such that the entropy of the weights is minimal provides the least upper bound.
\end{remark}

\begin{defn}
	Two mixtures $x, y \in X$ are \textbf{disjoint} if $S(p_1\rho_1 + p_2 \rho_2) - (p_1 S(\rho_1) + p_2 S(\rho_2)) = I(p_1, p_2)$.
\end{defn}

\begin{remark}
	Two classical distributions are disjoint if and only if their support is disjoint. Two quantum states are disjoint if and only if they are the mixture of states that belong to disjoint subspaces.
	
	Physically, two disjoint ensembles are distinct in the sense that they will never produce overlapping results.
\end{remark}

\begin{prop}
	The thermodynamic entropy is an entropy over an ensemble space.
\end{prop}
\begin{justification}
	TODO This can be easily done in terms of information theory and statistical mechanics. An argument based on thermodynamics would be better.
\end{justification}

\subsection{Physical examples}

We review the physical examples in terms of the entropy.

\subsubsection{Classical case - discrete}

The entropy is given by $-\sum p_i \log p_i$ for both finite and countable cases. Therefore the entropy for each pure state is assumed to be zero. The entropy of the maximally mixed state is $\log n$ where $n$ is the number of pure states. The entropy increase as we go from pure states to the maximally mixed state. The level sets (i.e. the fibers) of the entropy form a series of concentric "shells".

TODO: Check that the entropy is finite for all convex combinations. Check that it diverges as we go to a limit point that is not in the ensemble space.

TODO: Is it important that all pure states have zero entropy?

\subsubsection{Classical case - continuous}

The entropy is given by $- \int \rho \log \rho d\mu$ where $\mu$ is the Liouville measure and $\rho$ is the probability density over canonical coordinates. If a different measure is used, or if the coordinates are not canonical, the formula gives the wrong result.

The entropy can be used to check whether two ensembles are overlapping and define the size of the support. That is: take $\rho \in X$; construct the set $\text{dis}(\rho)$ of all the ensembles that are disjoint from $\rho$; now construct the set $\text{con}(\rho)$ of all ensembles that are disjoint from all elements of $\text{dis}(\rho)$. These will be the set of ensembles that share the supports. Find the supremum of $\text{dis}(\rho)$ in terms of the entropy. If it doesn't exists, the support is infinite. If it does, the support is finite and the measure is $2^{\sup S(\text{con}(\rho))}$.

TODO: Study the shell of zero entropy states. For example, it should not be path connected: all uniform distributions with support of the same finite size (in terms of Liouville measure) will have the same entropy. The region, however, need not be contiguous. Since we cannot continuously transform a single region into two disjoint regions, there will be different distributions at zero entropy that cannot be transformed continuously.

\subsubsection{Quantum case}

In quantum mechanics, the entropy is the von Neumann entropy for both finite and infinite dimension. The entropy for the maximally mixed state is $\log n$ where $n$ is the dimensionality of the Hilbert space.

TODO: How exactly do the bounds of entropy interact with multiple decomposition?

\subsection{Convex subspaces}

We need to develop a theory of subspaces that map to vector subspaces in the case that the convex space is also a projective space.

\begin{defn}
	Let $X$ be a convex space and $Y \subseteq X$. The \textbf{mix} of $Y$, noted $\mix(Y)$, is the set of all the mixtures of $Y$. The \textbf{components} of $Y$, noted $\component(Y)$, is the set of all the components of all the elements of $Y$.
\end{defn}

\begin{defn}
	Let $X$ be a convex space and $Y \subseteq X$. Then $Y$ is a \textbf{convex subspace} of $X$ if it contains all its mixtures and components. That is, $Y = \mix(Y) = \component(Y)$.
\end{defn}

\begin{remark}
	Note that the cospan of two elements is not simply the segment that connects them. That is just the mix of the two elements. Now we need to consider all the elements that, when mixed, can give those elements. If we extend the segment, the cospan will at least include those.
	
	The cospan can even be bigger. Consider a ball, which is a convex space, and suppose we want to construct the cospan of a single point $A$. The mix of a single point is simply that point, so we have to look at the components. If it is on the surface, it is an extreme point, and therefore the only component is itself. Therefore every surface point by itself is a subspace. If the point is not on the surface, then take a point on the surface $B$. Now connect $A$ and $B$ with a line. This will intersect the sphere on another point $C$. Therefore $A$ is a convex combination of $B$ and $C$, and therefore the whole segment $BC$ is part of the cospan of $A$. But the same argument works with any point on the surface, therefore the cospan of any interior point of the ball is the whole ball. 
	
	TODO: The embedding of the convex space into a vector space may be related to features the cospan

	TODO: The definition mimics definitions on other algebraic structure so that it can work in the infinite case. Does it?
\end{remark}

\begin{defn}
	Let $X$ be a convex space and $Y \subseteq X$. The \textbf{cospan} of $Y$, noted $\cospan(Y)$ is the smallest convex subspace that contains $Y$.
\end{defn}


\subsection{Physical examples}

We review the physical examples in terms of subspaces.

\subsubsection{Classical case - discrete}

For a symplex, all subspaces will be symplexes with fewer extereme points. In fact, these will be the sides of simplexes with more points. In particular, if we take two extreme points, the cospan is the segment connecting the two.

TODO We should find a characterization of the classical discrete case in terms of subspaces. Requiring the cospan of two extreme points to be a line is necessary but not sufficient. Is the converse true? For example, take a square. There are four extreme points. The cospan of two consecutive vertices is a segment. However, the cospan of two non consecutive vertices is the whole square. This would seem to exclude all possible flat polygons.

\subsubsection{Classical case - continuous}

The continuous case is trickier because the extreme points are not in the convex space. However, the cospan should give the set of elements that have the support in the union of the original elements. Note that this is related to the constuction in the previous section.

\subsubsection{Quantum case}

In quantum mechanics, the cospan coincides with the projective space of the subspaces of the complex vector space.

TODO: Is there a relationship between entropy and cospan? 


\end{document}
