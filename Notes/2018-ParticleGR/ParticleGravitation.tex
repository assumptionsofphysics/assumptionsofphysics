\documentclass[11pt]{article}

\usepackage[margin=1.25in]{geometry}

\usepackage{amsmath}
\usepackage{amssymb}
\usepackage{graphicx}
\usepackage{amsfonts}
\usepackage{enumitem}




\begin{document}

\title{Relativity and Particle  Mechanics}
\author{Gabriele Carcassi, Christine A. Aidala}

\date{\today}

\maketitle

\begin{abstract}
	In this note we want to explore how much of relativity is already contained in Hamiltonian particle mechanics. One central idea is that the metric tensor emerges as the linear relationship between conjugate momentum and velocity. When expressing the symplectic form in terms of position and velocity, the metric tensor and the force tensor appear, therefore these are really characterizations of the geometry of the tangent bundle. We want to re-understand those object in this setting, and in particular understanding whether the the properties of the symplectic form impose conditions among those elements.
\end{abstract}

\section{General properties}
We start with a summary of the main relationships so that we can bridge the tensor notation used in physics (i.e. index notation of general relativity) with the one used in symplectic geometry.

We use roman letters $i,j,k$ to span spatial indexes (i.e. $\{x,y,z\}$), greek letters $\alpha, \beta, \gamma$ to span space-time indexes (i.e. $\{t, q^i\}$ or $\{-E, p_i\})$, roman letters $a,b,c$ to span phase space variables (i.e. $\{q^\alpha, p_\alpha\}$).

\begin{equation}
\begin{aligned}
a \wedge b &= a \otimes b - b \otimes a \\
g_{\alpha\beta} e^\alpha \wedge e^\beta &= g_{\alpha\beta} e^\alpha \otimes e^\beta - g_{\alpha\beta} e^\beta \otimes e^\alpha = g_{\alpha\beta} e^\alpha \otimes e^\beta - g_{\beta\alpha} e^\alpha \otimes e^\beta \\
&= (g_{\alpha\beta} - g_{\beta\alpha}) e^\alpha \otimes e^\beta \\
\end{aligned}
\end{equation}
\begin{equation}
\begin{aligned}
\partial_\alpha \delta^\beta_\gamma &= 0 = \partial_\alpha ( g^{\beta \delta} g_{\delta \gamma} ) = \partial_\alpha g^{\beta \delta} g_{\delta \gamma} + g^{\beta \delta} \partial_\alpha g_{\delta \gamma} \\
\partial_\alpha g^{\beta \gamma} &= - g^{\beta \delta} g^{\gamma \epsilon} \partial_\alpha g_{\delta \epsilon} \\
\Gamma_{\alpha \beta \gamma} &= \frac{1}{2} (\partial_\gamma g_{\alpha\beta} + \partial_\beta g_{\alpha\gamma} - \partial_\alpha g_{\beta\gamma}) \\
\Gamma^\alpha_{\beta\gamma} &= g^{\alpha \delta} \Gamma_{\delta\beta\gamma}
\end{aligned}
\end{equation}

\section{Position and conjugate momentum}

We review the canonical one-form and two-form in canonical coordinates. The expression is independent of the Hamiltonian.

\begin{equation}
\begin{aligned}
\theta &= \left[ \begin{matrix}
p_\alpha & 0 \end{matrix} \right] \\
\theta &= \theta_a e^a = p_\alpha e^{q^\alpha} \\
\omega &= - \partial \theta = - \partial_b \theta_a e^b \wedge e^a = - e^{p_\alpha} \wedge e^{q^\alpha}  \\
\omega &= \left[ \begin{matrix}
0 & I_n \\[2.2ex]
- I_n & 0 \end{matrix} \right] \\
\omega &= \omega_{ab} e^a \otimes e^b = e^{q^\alpha} \otimes e^{p_\alpha} - e^{p_\alpha} \otimes e^{q^\alpha} = e^{q^\alpha} \wedge e^{p_\alpha} \\
\end{aligned}
\end{equation}

\section{Position and kinetic momentum}

We now switch non-canonical coordinate position and kinetic momentum. This depends on the choice of the Hamiltonian, which is taken to be the one under vector potential forces (i.e. EM). Note that we use the extended phase space formulation: $\mathcal{H}$ is both the generator of the affine parameter $s$ and the constraint on the mass.

\begin{equation}
\begin{aligned}
	d\xi^a \omega_{ab} d\xi^b &= dq^\alpha dp_\alpha = dx^\alpha g_{\alpha\beta} du^\beta \\
	q^\alpha &= x^\alpha \\
	p_\alpha &= m g_{\alpha \beta} u^\beta + q A_\alpha \\
	&= \Pi_\alpha + q A_\alpha \\
	\Pi_\alpha &= m g_{\alpha \beta} u^\beta \\
	&= dq^i dp_i - dt dE \\
	\mathcal{H} &= \frac{1}{2m} \Pi_\alpha g^{\alpha\beta} \Pi_\beta 
\end{aligned}
\end{equation}
\begin{equation}
\begin{aligned}
	e^{q^\alpha} &= \frac{\partial q^\alpha}{\partial x^\beta}e^{x^\beta} + \frac{\partial q^\alpha}{\partial \Pi_\gamma}e^{\Pi_\gamma} = e^{x^\alpha} \\
	e^{p_\alpha} &= \frac{\partial p_\alpha}{\partial x^\beta}e^{x^\beta} + \frac{\partial p_\alpha}{\partial \Pi_\gamma}e^{\Pi_\gamma} \\
	&= q \frac{\partial A_\alpha}{\partial x^\beta} e^{x^\beta} +e^{\Pi_\alpha} \\
\end{aligned}
\end{equation}
\begin{equation}
\begin{aligned}
	\theta &= (\Pi_\alpha + q A_\alpha) e^{x^\alpha} \\
	\omega &= e^{x^\alpha} \wedge \left( q \frac{\partial A_\alpha}{\partial x^\beta} e^{x^\beta} +e^{\Pi_\alpha} \right) \\
	&= q \frac{\partial A_\alpha}{\partial x^\beta} e^{x^\alpha} \wedge e^{x^\beta} + e^{x^\alpha} \wedge e^{\Pi_\alpha} \\
	&=  q \left( \frac{\partial A_\alpha}{\partial x^\beta} - \frac{\partial A_\beta}{\partial x^\alpha}\right)  e^{x^\alpha} \otimes e^{x^\beta} \\
	&+ e^{x^\alpha} \otimes e^{\Pi_\gamma} - e^{\Pi_\gamma} \otimes e^{x^\alpha} \\
	F_{\alpha\beta} &= \partial_\alpha A_\beta - \partial_\beta A_\alpha \\
	\omega_{ab} &= \left[ \begin{matrix}
		- q F_{\alpha\beta} & I_n \\[2.2ex]
		- I_n & 0 \end{matrix} \right] \\
\end{aligned}
\end{equation}
Note how the force tensor appears as the spatial component of the symplectic form.
\begin{equation}
\begin{aligned}
e_{q^\alpha} &= \frac{\partial x^\beta }{\partial q^\alpha}e_{x^\beta} + \frac{\partial \Pi_\beta}{\partial q^\alpha}e_{\Pi_\beta} \\
&= e_{x^\alpha} - q \frac{\partial A_\beta}{\partial q^\alpha}e_{\Pi_\beta} \\
e_{p_\alpha} &= \frac{\partial x^\beta}{\partial p_\alpha}e_{x^\beta} + \frac{\partial \Pi_\gamma}{\partial p_\alpha}e_{\Pi_\gamma} \\
&= e_{\Pi_\alpha} \\
\end{aligned}
\end{equation}
\begin{equation}
\begin{aligned}
\{x^\alpha, x^\beta\} &= \{q^\alpha, q^\beta\} = 0 \\
\{x^\alpha, \Pi_\beta\} &= \{q^\alpha, p_\beta - q A_\beta\} = \{q^\alpha, p_\beta\} - \{q^\alpha, q A_\beta\} \\
\{f(x^\alpha), \Pi_\beta\} &= \{f(q^\alpha), p_\beta - q A_\beta\} = \{f(q^\alpha), p_\beta\} - \{f(q^\alpha), q A_\beta\} \\
&= \partial_\beta f(q^\alpha) \\
\{ \Pi_\alpha, \Pi_\beta\} &= \{p_\alpha - q A_\alpha, p_\beta - q A_\beta\} \\
&= \{p_\alpha, p_\beta\} - \{p_\alpha, q A_\beta\} - \{q A_\alpha, p_\beta\} + \{q A_\alpha , q A_\beta\} \\
&= \{q A_\beta, p_\alpha\} - \{q A_\alpha, p_\beta\} = q (\partial_\alpha A_\beta - \partial_\beta A_\alpha) \\
&= q F_{\alpha \beta} \\
\end{aligned}
\end{equation}
Note how the force tensor appears as the Poisson bracket between kinetic momentum.
\begin{equation}
\begin{aligned}
d_s x^\alpha = \{x^\alpha, \mathcal{H}\} &= \frac{1}{2m} \{x^\alpha, \Pi_\beta g^{\beta\gamma} \Pi_\gamma\} = \frac{1}{m} g^{\alpha \beta} \Pi_\beta \\
m g_{\alpha\beta} d_s x^\beta &= \Pi_\alpha \\
d_s \Pi_\alpha = \{ \Pi_\alpha , \mathcal{H} \} &= \frac{1}{2m} \{\Pi_\alpha, \Pi_\beta g^{\beta\gamma} \Pi_\gamma\} \\
&= \frac{1}{2m} \left( \{\Pi_\alpha, \Pi_\beta\}  g^{\beta\gamma} \Pi_\gamma + \{\Pi_\alpha,  g^{\beta\gamma} \} \Pi_\beta \Pi_\gamma + \{\Pi_\alpha,  \Pi_\gamma\} \Pi_\beta g^{\beta\gamma} \right)  \\
&= \frac{1}{2m} \left( q F_{\alpha \beta}  g^{\beta\gamma} \Pi_\gamma - \partial_\alpha g^{\beta\gamma} \Pi_\beta \Pi_\gamma + q F_{\alpha \gamma} \Pi_\beta g^{\beta\gamma} \right)  \\
&= - \frac{1}{2m} \partial_\alpha g^{\beta\gamma} \Pi_\beta \Pi_\gamma + \frac{q}{m} F_{\alpha\beta} g^{\beta\gamma} \Pi_\gamma \\
&= \frac{1}{2m} g^{\beta\delta} g^{\gamma\epsilon} \partial_\alpha g_{\delta\epsilon} \Pi_\beta \Pi_\gamma + \frac{q}{m} F_{\alpha\beta} g^{\beta\gamma} \Pi_\gamma \\
&= \frac{1}{2m} g^{\beta\delta} g^{\gamma\epsilon}(\partial_\epsilon g_{\delta\alpha} - \partial_\epsilon g_{\delta\alpha} + \partial_\alpha g_{\delta\epsilon} ) \Pi_\beta \Pi_\gamma + \frac{q}{m} F_{\alpha\beta} g^{\beta\gamma} \Pi_\gamma \\
&= \frac{1}{2m} g^{\beta\delta} g^{\gamma\epsilon}(\partial_\delta g_{\epsilon\alpha} - \partial_\epsilon g_{\delta\alpha} + \partial_\alpha g_{\delta\epsilon} ) \Pi_\beta \Pi_\gamma + \frac{q}{m} F_{\alpha\beta} g^{\beta\gamma} \Pi_\gamma \\
&= \frac{1}{2m} g^{\beta\delta} g^{\gamma\epsilon}(\partial_\delta g_{\epsilon\alpha} + \partial_\alpha g_{\delta\epsilon} - \partial_\epsilon g_{\alpha\delta}  ) \Pi_\beta \Pi_\gamma + \frac{q}{m} F_{\alpha\beta} g^{\beta\gamma} \Pi_\gamma \\
&= \frac{1}{m} g^{\beta\delta} \Gamma^\gamma_{\delta\alpha} \Pi_\beta \Pi_\gamma + \frac{q}{m} F_{\alpha\beta} g^{\beta\gamma} \Pi_\gamma \\
&= d_sx^\delta \Gamma^\gamma_{\delta\alpha} \Pi_\gamma + \frac{q}{m} F_{\alpha\beta} g^{\beta\gamma} \Pi_\gamma \\
D_s \Pi_\alpha &= d_s \Pi_\alpha - d_sx^\delta \Gamma^\gamma_{\delta\alpha} \Pi_\gamma = \frac{q}{m} F_{\alpha\beta} g^{\beta\gamma} \Pi_\gamma \\
d_s f(x^\alpha) =\{f(x^\alpha), \mathcal{H}\} &= \frac{1}{2m} \{f(x^\alpha), \Pi_\beta g^{\beta\gamma} \Pi_\gamma\} \\
&= \frac{1}{m} \Pi_\beta g^{\beta\gamma} \{f(x^\alpha), \Pi_\gamma\} = g_{\beta\delta} d_s x^\delta g^{\beta\gamma} \partial_\gamma f(x^\alpha) \\
&= d_s x^\delta \partial_\delta f(x^\alpha) = d_s f(x^\alpha) \\
\end{aligned}
\end{equation}
Note how we recover the equation of a geodesic modified by the force tensor.

\section{Position and velocity}
We now switch to position and velocity.

\begin{equation}
\begin{aligned}
q^\alpha &= x^\alpha \\
p_\alpha &= m g_{\alpha \beta} u^\beta + q A_\alpha \\
\mathcal{H} &= \frac{1}{2} m u^\alpha g_{\alpha\beta} u^\beta \\
e^{q^\alpha} &= \frac{\partial q^\alpha}{\partial x^\beta}e^{x^\beta} + \frac{\partial q^\alpha}{\partial u^\gamma}e^{u^\gamma} = e^{x^\alpha} \\
e^{p_\alpha} &= \frac{\partial p_\alpha}{\partial x^\beta}e^{x^\beta} + \frac{\partial p_\alpha}{\partial u^\gamma}e^{u^\gamma} \\
&= \left(m \frac{\partial g_{\alpha\gamma}}{\partial x^\beta} u^\gamma + q \frac{\partial A_\alpha}{\partial x^\beta} \right)e^{x^\beta} + m g_{\alpha \gamma} e^{u^\gamma} \\
\theta &= (m g_{\alpha \beta} u^\beta + q A_\alpha) e^{x^\alpha} \\
\omega &= e^{x^\alpha} \wedge \left[ \left(m \frac{\partial g_{\alpha\gamma}}{\partial x^\beta} u^\gamma + q \frac{\partial A_\alpha}{\partial x^\beta} \right)e^{x^\beta} + m g_{\alpha \gamma} e^{u^\gamma} \right] \\
&= \left(m \frac{\partial g_{\alpha\gamma}}{\partial x^\beta} u^\gamma + q \frac{\partial A_\alpha}{\partial x^\beta} \right) e^{x^\alpha} \wedge e^{x^\beta} + m g_{\alpha \gamma} e^{x^\alpha} \wedge e^{u^\gamma} \\
&= \left(m u^\gamma \left( \frac{\partial g_{\alpha\gamma}}{\partial x^\beta} -  \frac{\partial g_{\beta\gamma}}{\partial x^\alpha}\right) + q \left( \frac{\partial A_\alpha}{\partial x^\beta} - \frac{\partial A_\beta}{\partial x^\alpha}\right) \right) e^{x^\alpha} \otimes e^{x^\beta} \\
&+ m g_{\alpha \gamma} e^{x^\alpha} \otimes e^{u^\gamma} - m g_{\alpha \gamma} e^{u^\gamma} \otimes e^{x^\alpha} \\
F_{\alpha\beta} &= \partial_\alpha A_\beta - \partial_\beta A_\alpha \\
G_{\alpha\beta\gamma} &= \partial_\alpha g_{\beta \gamma} - \partial_\beta g_{\alpha \gamma} = \frac{1}{2} (\partial_\alpha g_{\beta \gamma} - \partial_\beta g_{\alpha \gamma}) - \frac{1}{2} (\partial_\beta g_{\alpha \gamma} - \partial_\alpha g_{\beta \gamma}) \\
&= \frac{1}{2} (\partial_\gamma g_{\alpha \beta} + \partial_\alpha g_{\beta \gamma} - \partial_\beta g_{\alpha \gamma}) - \frac{1}{2} (\partial_\gamma g_{\alpha \beta} + \partial_\beta g_{\alpha \gamma} - \partial_\alpha g_{\beta \gamma}) \\
&= \Gamma_{\beta\alpha\gamma} - \Gamma_{\alpha\beta\gamma} \\
\omega_{ab} &= \left[ \begin{matrix}
- m G_{\alpha\beta\gamma} u^\gamma - q F_{\alpha\beta} & mg_{\alpha \beta} \\[2.2ex]
- mg_{\alpha \beta} & 0 \end{matrix} \right] \\
\end{aligned}
\end{equation}
Note the appearance of the new symbol $G_{\alpha\beta\gamma}$. Note it is different from the torsion as it anti-symmetrizes the first two indexes of the Christoffel symbols (the torsion uses the second two).
\begin{equation}
\begin{aligned}
G^{\alpha\beta\gamma} &= g^{\alpha\delta}g^{\beta\epsilon}g^{\gamma\zeta} G_{\delta\epsilon\zeta} = g^{\alpha\delta}g^{\beta\epsilon}g^{\gamma\zeta} \partial_\delta g_{\epsilon\zeta} - g^{\alpha\delta}g^{\beta\epsilon}g^{\gamma\zeta} \partial_\epsilon g_{\delta\zeta}\\
&= - g^{\alpha\delta} \partial_\alpha g^{\beta\gamma} + g^{\beta\epsilon} \partial_\beta g^{\alpha\gamma} \\
&= \partial^\beta g^{\alpha\gamma} - \partial^\alpha g^{\beta\gamma} \\
F^{\alpha\beta} &= g^{\alpha\gamma}g^{\beta\delta} F_{\gamma\delta} = g^{\alpha\gamma}g^{\beta\delta} \partial_\gamma A_{\delta} - g^{\alpha\gamma}g^{\beta\delta} \partial_\delta A_{\gamma}  \\
&= g^{\alpha\gamma} \partial_\gamma (g^{\beta\delta} A_{\delta}) - g^{\beta\delta} \partial_\delta ( g^{\alpha\gamma}A_{\gamma}) - g^{\alpha\gamma} \partial_\gamma g^{\beta\delta} A_{\delta} + g^{\beta\delta} \partial_\delta g^{\alpha\gamma}A_{\gamma}\\
&= \partial^\alpha A^\beta - \partial^\beta A^\alpha + \partial ^\beta g^{\alpha \gamma} A_\gamma - \partial^\alpha g^{\beta \delta} A_\delta\\
&= \partial^\alpha A^\beta - \partial^\beta A^\alpha +G^{\alpha\beta\gamma} A_\gamma
\end{aligned}
\end{equation}
\begin{equation}
\begin{aligned}
\partial \omega &= \partial \left(\left(m \partial_\beta g_{\alpha \gamma}u^\gamma + q \partial_\beta A_\alpha\right)e^{x^\alpha}\wedge e^{x^\beta} + mg_{\alpha\beta} e^{x^\alpha}\wedge e^{u^\beta}\right) \\ 
&= \partial_{x^\delta} \left(m \partial_\beta g_{\alpha \gamma}u^\gamma + q \partial_\beta A_\alpha\right)e^{x^\delta}\wedge e^{x^\alpha}\wedge e^{x^\beta} + m \partial_{x^\delta}  g_{\alpha\beta} e^{x^\delta}\wedge e^{x^\alpha}\wedge e^{u^\beta} \\
&+ \partial_{u^\delta} \left(m \partial_\beta g_{\alpha \gamma}u^\gamma + q \partial_\beta A_\alpha\right)e^{u^\delta}\wedge e^{x^\alpha}\wedge e^{x^\beta} + m \partial_{u^\delta} g_{\alpha\beta} e^{u^\delta}\wedge  e^{x^\alpha}\wedge e^{u^\beta} \\
&= \left(m \partial_\delta \partial_\beta g_{\alpha \gamma}u^\gamma + q \partial_\delta \partial_\beta A_\alpha\right)e^{x^\delta}\wedge e^{x^\alpha}\wedge e^{x^\beta} + m \partial_\delta  g_{\alpha\beta} e^{x^\delta}\wedge e^{x^\alpha}\wedge e^{u^\beta} \\
&+ m \partial_\beta g_{\alpha \gamma}\delta^\gamma_\delta e^{u^\delta}\wedge e^{x^\alpha}\wedge e^{x^\beta}  \\
&= \left(m \partial_\delta \partial_\beta g_{\alpha \gamma}u^\gamma + q \partial_\delta \partial_\beta A_\alpha\right)e^{x^\delta}\wedge e^{x^\alpha}\wedge e^{x^\beta} + m \partial_\delta  g_{\alpha\beta} e^{x^\delta}\wedge e^{x^\alpha}\wedge e^{u^\beta} \\
&+ m \partial_\delta g_{\alpha \beta} e^{u^\beta}\wedge e^{x^\alpha}\wedge e^{x^\delta}  \\
&= \left(m \partial_\delta \partial_\beta g_{\alpha \gamma}u^\gamma + q \partial_\delta \partial_\beta A_\alpha\right)e^{x^\delta}\wedge e^{x^\alpha}\wedge e^{x^\beta} + m \partial_\delta  g_{\alpha\beta} e^{x^\delta}\wedge e^{x^\alpha}\wedge e^{u^\beta} \\
&- m \partial_\delta g_{\alpha \beta} e^{x^\delta}\wedge e^{x^\alpha}\wedge e^{u^\beta} \\
\end{aligned}
\end{equation}
\begin{equation}
\begin{aligned}
\{x^\alpha, x^\beta\} &= \{q^\alpha, q^\beta\} = 0 \\
\{x^\alpha, u^\beta\} &= \{q^\alpha, \frac{1}{m} g^{\beta\gamma}(p_\gamma - q A_\gamma)\} = \{q^\alpha, \frac{1}{m} g^{\beta\gamma} p_\gamma\} - \{q^\alpha, \frac{q}{m} g^{\beta\gamma} A_\gamma\} \\
&= \frac{1}{m} g^{\alpha\beta} \\
\{u^\alpha, u^\beta\} &= \frac{1}{m^2} \{g^{\alpha \gamma} p_\gamma - q A^\alpha, g^{\beta \delta} p_\delta - q A^\beta\} \\
&= \frac{1}{m^2}\left[ \{g^{\alpha \gamma} p_\gamma, g^{\beta \delta} p_\delta\} - \{g^{\alpha \gamma} p_\gamma, q A^\beta\} - \{q A^\alpha , g^{\beta \delta} p_\delta \} + \{q A^\alpha , q A^\beta\} \right]\\
&= \frac{1}{m^2}\left[ g^{\alpha \gamma} \{p_\gamma, p_\delta\} g^{\beta \delta} + p_\gamma \{g^{\alpha \gamma}, p_\delta\} g^{\beta \delta} + g^{\alpha \gamma} \{p_\gamma, g^{\beta \delta}\} p_\delta + p_\gamma \{g^{\alpha \gamma}, g^{\beta \delta}\} p_\delta \right.\\
&\left. - g^{\alpha \gamma} \{p_\gamma, q A^\beta\} - p_\gamma \{g^{\alpha \gamma} , q A^\beta\} - \{q A^\alpha , p_\delta \} g^{\beta \delta} - \{q A^\alpha , g^{\beta \delta} \} p_\delta \right]\\
&= \frac{1}{m^2}\left[ G^{\alpha\beta\gamma} p_\gamma  + q (\partial^\alpha A^\beta - \partial^\beta A^\alpha) \right] \\
&= \frac{1}{m^2}\left[ G^{\alpha \beta \gamma} m ( g_{\gamma \delta} u^\delta + q A_\gamma ) + q (\partial^\alpha A^\beta - \partial^\beta A^\alpha) \right] \\
&= \frac{1}{m^2}\left[ G^{\alpha \beta \gamma} m g_{\gamma \delta} u^\delta + q (G^{\alpha \beta \gamma} A_\gamma + \partial^\alpha A^\beta - \partial^\beta A^\alpha) \right] \\
&= \frac{1}{m^2}\left[ G^{\alpha \beta \gamma} m g_{\gamma \delta} u^\delta +  q F^{\alpha \beta} \right] \\
\end{aligned}
\end{equation}
\begin{equation}
\begin{aligned}
d_s x^\alpha = \{x^\alpha, \mathcal{H}\} &= \frac{1}{2} m  \{x^\alpha, u^\beta g_{\beta\gamma} u^\gamma\} = m u^\beta g_{\beta \gamma} \{x^\alpha, u^\gamma\}\\
&= m u^\beta g_{\beta\gamma} \frac{1}{m} g^{\alpha\gamma} = u^\alpha \\
d_s u^\alpha = \{ u^\alpha , \mathcal{H} \} &= \frac{1}{2} m \{u^\alpha, u^\beta g_{\beta\gamma} u^\gamma\} \\
&=  m u^\beta g_{\beta\gamma} \{u^\alpha,  u^\gamma\} + \frac{1}{2} m u^\beta u^\gamma \{u^\alpha,  g_{\beta\gamma} \} \\
&= m u^\beta g_{\beta\gamma} \frac{1}{m^2} (G^{\alpha \gamma \delta} m g_{\delta \epsilon} u^\epsilon +  q F^{\alpha \gamma}) - \frac{1}{2} m u^\beta u^\gamma \frac{1}{m} g^{\alpha \delta} \partial_\delta g_{\beta\gamma} \\
&= u^\beta u^\epsilon g_{\beta\gamma} g_{\epsilon \delta} G^{\alpha \gamma \delta} - \frac{1}{2} u^\beta u^\gamma g^{\alpha \delta} \partial_\delta g_{\beta\gamma} + \frac{q}{m} F^{\alpha \gamma}g_{\gamma\beta}u^\beta \\
&= u^\beta u^\epsilon g^{\alpha\gamma} G_{\gamma \beta \epsilon} - \frac{1}{2} u^\beta u^\gamma g^{\alpha \delta} (\partial_\beta g_{\gamma\delta} - \partial_\beta g_{\gamma\delta} + \partial_\delta g_{\beta\gamma}) + \frac{q}{m} F^{\alpha \gamma}g_{\gamma\beta}u^\beta \\
&= u^\beta u^\gamma g^{\alpha\delta} G_{\delta \beta \gamma} -  u^\beta u^\gamma g^{\alpha \delta} \frac{1}{2} (\partial_\gamma g_{\delta\beta} + \partial_\delta g_{\beta\gamma} - \partial_\beta g_{\gamma\delta}) + \frac{q}{m} F^{\alpha \gamma}g_{\gamma\beta}u^\beta \\
&= u^\beta u^\gamma g^{\alpha\delta} (\Gamma_{\beta \delta \gamma} - \Gamma_{\delta \beta \gamma} - \Gamma_{\beta \delta \gamma}) + \frac{q}{m} F^{\alpha \gamma}g_{\gamma\beta}u^\beta \\
&= - u^\beta u^\gamma g^{\alpha\delta} \Gamma_{\delta \beta \gamma} + \frac{q}{m} F^{\alpha \gamma}g_{\gamma\beta}u^\beta \\
D_s u^\alpha &= d_s u^\alpha + u^\beta d_sx^\gamma \Gamma^\alpha_{\beta\gamma} = \frac{q}{m} F^{\alpha \gamma}g_{\gamma\beta}u^\beta \\
\end{aligned}
\end{equation}

Again, note how the geodesic equation is recovered.

\section{More on $G_{\alpha\beta\gamma}$}
To better understand $G_{\alpha\beta\gamma}$, we study how the anti-symmetrization of the covariant derivative works.

\begin{equation}
\begin{aligned}
\nabla_\alpha A_\beta &= \partial_\alpha A_\beta - \Gamma^\gamma_{\beta\alpha} A_\gamma  \\
\nabla_\alpha A_\beta - \nabla_\beta A_\alpha &= \partial_\alpha A_\beta - \Gamma^\gamma_{\beta\alpha} A_\gamma -\partial_\beta A_\alpha + \Gamma^\gamma_{\alpha\beta} A_\gamma  \\
&= \partial_\alpha A_\beta -\partial_\beta A_\alpha  \\
\nabla_\alpha A^\beta &= \partial_\alpha A^\beta + \Gamma^\beta_{\gamma\alpha} A^\gamma  \\
\nabla_\alpha A_\beta = \nabla_\alpha ( g_{\beta\gamma} A^\gamma) &= g_{\beta\gamma} \nabla_\alpha A^\gamma = g_{\beta\gamma} \partial_\alpha A^\gamma + g_{\beta\gamma} \Gamma^\gamma_{\delta\alpha} A^\delta \\
&= \partial_\alpha (g_{\beta\gamma} A^\gamma) - \partial_\alpha g_{\beta\gamma} A^\gamma + g_{\beta\gamma} \Gamma^\gamma_{\delta\alpha} A^\delta \\
&= \partial_\alpha A_\beta - \partial_\alpha g_{\beta\gamma} A^\gamma +  \Gamma_{\beta\gamma\alpha} A^\gamma \\
&= \partial_\alpha A_\beta - \partial_\alpha g_{\beta\gamma} A^\gamma +  \frac{1}{2} (\partial_\alpha g_{\beta\gamma} + \partial_\gamma g_{\beta\alpha} - \partial_\beta g_{\alpha\gamma}) A^\gamma \\
&= \partial_\alpha A_\beta +  \frac{1}{2} ( - \partial_\alpha g_{\beta\gamma} + \partial_\gamma g_{\beta\alpha} - \partial_\beta g_{\alpha\gamma}) A^\gamma \\
&= \partial_\alpha A_\beta - \Gamma_{\gamma\beta\alpha} A^\gamma \\
&= \partial_\alpha A_\beta - \Gamma^\gamma_{\beta\alpha} A_\gamma = \nabla_\alpha A_\beta \\
\nabla^\alpha A^\beta - \nabla^\beta A^\alpha &= g^{\alpha\gamma} \nabla_\gamma A^\beta - g^{\beta\gamma} \nabla_\gamma A^\alpha \\
&= g^{\alpha\gamma} (\partial_\gamma A^\beta + \Gamma^\beta_{\delta\gamma} A^\delta) - g^{\beta\gamma} (\partial_\gamma A^\alpha + \Gamma^\alpha_{\delta\gamma} A^\delta) \\
&= \partial^\alpha A^\beta - \partial^\beta A^\alpha + g^{\alpha\delta} g^{\beta\epsilon} A^\gamma( \Gamma_{\epsilon\delta\gamma} - \Gamma_{\delta\epsilon\gamma}) \\
&= \partial^\alpha A^\beta - \partial^\beta A^\alpha + G^{\alpha\beta\gamma} A_\gamma \\
\end{aligned}
\end{equation}
Note how the symmetrization of the covariant derivative is equal to the symmetrization of the partial derivative. However, this is not true with the controvariant indexes. In this case, $G$ can be used to express the difference.

If we look at the transformation rules, $G$ is not a tensor.
\begin{equation*}
	\begin{aligned}
		g_{\alpha'\beta'}&=\Lambda_{\alpha'}^{\alpha}\Lambda_{\beta'}^{\beta}g_{\alpha\beta} \\
		\partial_{\alpha'}g_{\beta'\gamma'}&=\Lambda_{\alpha'}^{\alpha}\partial_{\alpha}\left(\Lambda_{\beta'}^{\beta}\Lambda_{\gamma'}^{\gamma}g_{\beta\gamma}\right) \\
		&=\Lambda_{\alpha'}^{\alpha}\partial_{\alpha}\left(\Lambda_{\beta'}^{\beta}\right)\Lambda_{\gamma'}^{\gamma}g_{\beta\gamma} + \Lambda_{\alpha'}^{\alpha}\Lambda_{\beta'}^{\beta}\partial_{\alpha}\left(\Lambda_{\gamma'}^{\gamma}\right)g_{\beta\gamma} + \Lambda_{\alpha'}^{\alpha}\Lambda_{\beta'}^{\beta}\Lambda_{\gamma'}^{\gamma}\partial_{\alpha}g_{\beta\gamma}\\
		G_{\alpha'\beta'\gamma'} &= \partial_{\alpha'}g_{\beta'\gamma'} - \partial_{\beta'}g_{\alpha'\gamma'}\\
		&= \Lambda_{\alpha'}^{\alpha}\partial_{\alpha}\left(\Lambda_{\beta'}^{\beta}\right)\Lambda_{\gamma'}^{\gamma}g_{\beta\gamma} + \Lambda_{\alpha'}^{\alpha}\Lambda_{\beta'}^{\beta}\partial_{\alpha}\left(\Lambda_{\gamma'}^{\gamma}\right)g_{\beta\gamma} + \Lambda_{\alpha'}^{\alpha}\Lambda_{\beta'}^{\beta}\Lambda_{\gamma'}^{\gamma}\partial_{\alpha}g_{\beta\gamma}\\
		&-\Lambda_{\beta'}^{\beta}\partial_{\beta}\left(\Lambda_{\alpha'}^{\alpha}\right)\Lambda_{\gamma'}^{\gamma}g_{\alpha\gamma} - \Lambda_{\beta'}^{\beta}\Lambda_{\alpha'}^{\alpha}\partial_{\beta}\left(\Lambda_{\gamma'}^{\gamma}\right)g_{\alpha\gamma} - \Lambda_{\beta'}^{\beta}\Lambda_{\alpha'}^{\alpha}\Lambda_{\gamma'}^{\gamma}\partial_{\beta}g_{\alpha\gamma}\\
		&=\Lambda_{\alpha'}^{\alpha}\partial_{\alpha}\left(\Lambda_{\beta'}^{\beta}\right)\Lambda_{\gamma'}^{\gamma}g_{\beta\gamma} + \Lambda_{\alpha'}^{\alpha}\Lambda_{\beta'}^{\beta}\partial_{\alpha}\left(\Lambda_{\gamma'}^{\gamma}\right)g_{\beta\gamma}\\
		&-\Lambda_{\beta'}^{\beta}\partial_{\beta}\left(\Lambda_{\alpha'}^{\alpha}\right)\Lambda_{\gamma'}^{\gamma}g_{\alpha\gamma} - \Lambda_{\beta'}^{\beta}\Lambda_{\alpha'}^{\alpha}\partial_{\beta}\left(\Lambda_{\gamma'}^{\gamma}\right)g_{\alpha\gamma}\\
		&+\Lambda_{\alpha'}^{\alpha}\Lambda_{\beta'}^{\beta}\Lambda_{\gamma'} G_{\alpha\beta\gamma} \\
		&=\Lambda_{\alpha'}^{\alpha}\partial_{\alpha}\left(\Lambda_{\beta'}^{\beta}\Lambda_{\gamma'}^{\gamma}\right)g_{\beta\gamma} -\Lambda_{\beta'}^{\beta}\partial_{\beta}\left(\Lambda_{\alpha'}^{\alpha}\Lambda_{\gamma'}^{\gamma}\right)g_{\alpha\gamma} \\
		&+\Lambda_{\alpha'}^{\alpha}\Lambda_{\beta'}^{\beta}\Lambda_{\gamma'} G_{\alpha\beta\gamma} \\
	\end{aligned}
\end{equation*}

If $G_{\alpha\beta\gamma}=0$, then $g_{\alpha\beta} = \partial_\alpha \partial_\beta S$ where $S$ is a scalar function. In fact
\begin{equation}
	\begin{aligned}
		G_{\alpha\beta\gamma}&=0=\partial_\alpha g_{\beta \gamma} - \partial_\beta g_{\alpha \gamma} \\
		g_{\alpha \beta} &= \partial_\alpha a_\beta \\
		g_{\alpha \beta} - g_{\beta \alpha} &= 0 = \partial_\alpha a_\beta - \partial_\beta a_\alpha \\
		a_\alpha &= \partial_\alpha S \\
		g_{\alpha \beta} &= \partial_\alpha \partial_\beta S
	\end{aligned}
\end{equation}

If we are in an inertial Cartesian frame, then $G_{\alpha\beta\gamma}=0$. What is the characteristic of reference frames in which $G_{\alpha\beta\gamma}=0$? Do they exist in curved space-time?


\begin{thebibliography}{0}
	
\end{thebibliography}

\end{document}
